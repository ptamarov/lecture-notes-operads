\section{Symmetric modules and algebraic operads}\label{lecture:thebasics}

\textbf{Goals.}
We will define
some related gadgets (symmetric collections,
algebras, modules, endomorphism operads)
necessary to introduce operads. 
Then, we define what an operad is (topological,
algebraic, symmetric, non-symmetric). 
We will then give some
(not so) well known examples of topological
and algebraic operads.

\subsection{Basic definitions}
\emph{What is an operad?} A group is a model of
$\operatorname{Aut}(X)$ for $X$ a set, an algebra
over a field
is a model of $\End(V)$ for $V$
a vector space. Equivalently, groups are the
gadgets that act on objects by automorphisms,
and algebras are the gadgets that act
on objects by their (linear) endomorphisms. 
Operads are the gadgets that act on
objects through operations with many 
inputs (and one output), and at the same
time keep track of symmetries when
the inputs are permuted.

The underlying objects to operads are known as
\emph{symmetric sequences}: a symmetric sequence
(also known as a $\Sigma$-module or combinatorial 
species) is a sequence of vector spaces
$\XX = (\XX(n))_{n\geqslant 0}$ such that for
each $n\in\NN_0$ there is a right action of
$S_n$ on $\XX(n)$. We usually consider \emph{reduced}
$\Sigma$-modules, those for which $\XX(0)=0$
(or, in other cases, the terminal object of the
category). They were introduced in~\cite{Joyal1981}
to provide a ``combinatorial theory of formal
power series''; see Exercise~\ref{ex:operations-on-species} and 
the book~\cite{Bergeron1997} for more information.

A map of $\Sigma$-modules is a collection of maps
$(f_n : \XX_1(n) \longrightarrow \XX_2(n))_{n\geqslant 0}$,
each equivariant for the corresponding group action. 
This defines the category $\SMod$ of symmetric
sequences, and whenever we think of symmetric sequences
using this definition, we will say we are considering  
a biased or skeletal approach to them.

In parallel, it is convenient to consider the 
category $\GSet$ of finite sets and bijections.
An object in this category is a finite set $I$,
and a morphism $\sigma : I\longrightarrow J$ is a
bijection. Since every finite set $I$ with $n$
elements is (non-canonically) isomorphic to 
$\underline{n} =\{1,\ldots,n\}$, the following holds:

\begin{lemma} The skeleton of $\GSet$ is
equal to the category with objects the finite sets
$\underline{n}$ for $n\geqslant 0$ and with morphisms the
bijections $\underline{n}\longrightarrow \underline{n}$ (and no morphism
between $\underline{n}$ and $\underline{m}$ if $m\neq n$).
\end{lemma}

\begin{proof}
This is Exercise~\ref{ex:skeleton}.
\end{proof}

We set ${}_\Sigma\Mod  = 
\Fun(\GSet,\mathsf{Vect}^{\mathrm{op}})$,
so that a $\Sigma$-module is a pre-sheaf of vector
spaces $I\longmapsto \XX(I)$ assigning to each
isomorphism $\tau : I\longrightarrow J$ an isomorphism
$\XX(\tau): \XX(J)\longrightarrow \XX(I)$. When we
think of $\Sigma$-modules as pre-sheaves, we will say we 
are taking an unbiased approach, if we specify only
its values on natural numbers, we will say we are taking the
biased or skeletal approach; we will come back to this later.

With this at hand, 
we can in turn define the \emph{Cauchy product}
of two $\Sigma$-modules $\XX$ and $\YY$
\[ (\XX\otimes_\Sigma \YY)(I) = 
 	\bigoplus_{S\sqcup T= I}
 		 \XX(S)\otimes \YY(T)\] 
where the right-hand is the usual tensor product of
vector spaces
and the sum runs through ordered 
partitions of $I$ into
two disjoint sets. The symmetric product is then
defined by 
\[ (\XX\circ_\Sigma \YY)(I) 
 	= \bigoplus_{\substack{k\geqslant 1 }} \XX(k) 
 		\otimes_{S_k} \YY^{\otimes k}(I)\] 
as the sum runs through (ordered) partitions of $I$, where the tensor product in the exponent is
the Cauchy product. Writing things down 
explicitly, we see that 
\[ (\XX\circ_\Sigma \YY)(I) 
 	= \bigoplus_{\substack{k\geqslant 1}}\XX(k) 
 		\otimes_{S_k}
 		\left[
 		\bigoplus_{\pi\vdash_k I}\YY(\pi_1)\otimes\cdots\otimes
 			\YY(\pi_k) \right]\]
where for a partition $\pi$ of size $k$ of $I$,
the action of $S_k$ on $X(\pi)$ is by permutation
of the factors, and the action of $S_k$
on the big summand on the right is by permutation
of the individual factors. 
These two products will be central in what follows.

\begin{lemma}
The category ${}_\Sigma\Mod$ with $\circ_\Sigma$ is
monoidal with unit the $\Sigma$-module taking the value $\kk e_x$ at 
the singleton sets $\{x\}$ and zero everywhere else. The same
category is monoidal for $\otimes_\Sigma$ with unit
the species taking the value $\kk$ at $\varnothing$
and zero everywhere else.
\end{lemma}

We will use the notation $\kk$ for the base field but
also for the unit for the composition product $\circ_\Sigma$,
hoping it will not cause much confusion. It will be useful 
later to think of $\kk$ as a twig or ``stick''.

Observe that the associator for $\circ_\Sigma$ is
not too simple and involves reordering certain
factors of tensor products in $\mathsf{Vect}$. In
particular, replacing vector spaces by graded vector
spaces or complexes will create signs in the
associator.

We are now ready to define the prototypical symmetric
 sequence that carries the structure of an algebraic 
 operad. 
 
\begin{definition}
The \emph{endomorphism operad} of a space $V$ is the symmetric sequence $\End_V$
where for each $n\in\NN$ we set $\End_V(n) = \End(V^{\otimes n}, V)$. 
The symmetric group $S_n$ acts on the right
on $\End_V(n)$ 
so that $(f\sigma)(v) = f(\sigma v)$ for
$v\in V^{\otimes n}$, where $S_n$ acts on
the left on $V^{\otimes n}$ by $(\sigma v)_i
= v_{\sigma i}$. The composition maps are defined
by $\gamma(f;g_1,\ldots,g_n) = f\circ (g_1\otimes\cdots \otimes g_n)$. 
If $f\in\End_V(n)$, we say that $f$ has arity $n$ and write
it $\ari{f}$.
\end{definition}


The following two operations on permutations 
will streamline our definition of (algebraic)
operads.

\medskip

\textbf{Two useful maps.} For each $k\geqslant 1$
and each tuple $\lambda = (n_1,\ldots,n_k)$ 
with sum $n$ there is a map $\lambda : S_k
 \longrightarrow S_{n_1+\cdots+n_k} $
that sends $\sigma\in S_k$ to the permutation
$\lambda(\sigma)$ of $\underline{n}$ that permutes the blocks 
$\pi_i = \{n_1+\cdots+n_{i-1}+1,\ldots
			n_1+\cdots+n_{i-1}+n_i\}$
			according to $\sigma$.
There is also a map
$S_{n_1}\times \cdots \times  S_{n_k} 
	\longrightarrow S_{n_1+\cdots+n_k}$
	that sends a tuple of permutations 
	$(\sigma_1,\ldots,\sigma_k)$ to the
	permutation $\sigma_1\#\cdots \# \sigma_k$
	that acts like $\sigma_i$ on the block $\pi_i$
	as above. These operations are illustrated
	in Figure~\ref{fig:1}. With these at hand, 
one can check that these composition maps
satisfy the following axioms:

\begin{figure}
$$\lambda = (2,1,2), \quad \sigma = 312
	\quad \leadsto \quad \lambda(\sigma) =  34512 \in S_5
	$$
	$$
	(213,213,132)\in S_3\times S_3\times S_3 \quad \leadsto \quad 213546798\in S_9 $$
\caption{The useful operations}
\label{fig:1}
\end{figure}

\begin{tenumerate}
\item \emph{Associativity}: let $f\in \End_V(n)$,
and consider $g_1,\ldots,g_n \in \End_V$ and
for each $i\in \underline{n}$ a tuple $h_i= (h_{i1},\ldots,h_{i n_i})$
were $n_i= \mathrm{ar}(g_i)$. Then for
$f_i = \gamma(g_i; h_{i1},\ldots,h_{in_i})$ 
and $g= \gamma(f; g_1,\ldots,g_n)$ we have
that
\[ \gamma(f;f_1,\ldots,f_n) = 
	\gamma(g; h_1,\ldots,h_n).\]
\item \emph{Intrinsic equivariance}: for
each $\sigma\in S_k$ and $\lambda = (\ari(g_1),\ldots,\ari(g_k))$ we have that
\[ \gamma(f\sigma; g_1,\ldots,g_k) = 	
	\gamma(f; g_{\sigma 1} ,\ldots, 
		g_{\sigma k})\lambda(\sigma),\]
	
\item \emph{Extrinsic equivariance}: for each
tuple of permutations $(\sigma_1,\ldots,\sigma_k) \in S_{n_1} \times
\cdots \times S_{n_k}$, if $\sigma = \sigma_1\#\cdots\#\sigma_k$, we have that
\[\gamma(f,g_1\sigma_1,\ldots,g_k\sigma_k) = 
	\gamma(f; g_1,\ldots,g_k)\sigma.\]
\item \emph{Unitality:} the identity $1\in\End_V(1)$
satisfies $\gamma(1;g) = g$ and $\gamma(g;1,\ldots,1) = g$ for every $g\in\End_V$.
\end{tenumerate}

\begin{definition} 
A symmetric operad (in vector spaces) is a
$\Sigma$-module $\PP$ along with a composition
map $\gamma : \PP\circ \PP \longrightarrow \PP$
of signature
\[\gamma : \PP(k)\otimes 
	\PP(n_1) \otimes \cdots \otimes \PP(n_k)
	 	\longrightarrow \PP(n_1+\cdots+n_k)\]
along with a unit $1\in \PP(1)$, that satisfy
the axioms above. 
\end{definition} 

\begin{variante} A non-symmetric operad is
an operad whose underlying object is a collection
(with no symmetric group actions). Operads in
topological spaces or chain complexes require
the composition maps to be morphisms (that is,
continuous maps or maps of chain complexes,
respectively) and, more generally, operads 
defined on a symmetric monoidal category
require, naturally, that the composition
maps be morphisms in that category. 
\end{variante}

\textbf{Pseudo-operads.}
One can define operads through \emph{partial 
composition maps}, modeling the honest partial
composition map
\[ f\circ_i g = f(1,\ldots,1,g,1,\ldots,1)\] 
in $\End_V$. These composition maps satisfy the
following properties:

\begin{tenumerate}
\item \emph{Associativity}: for
each $f,g,h\in\End_V$, and $\delta = i-j+1$,
we have
\[ 
(f \circ_j g) \circ_i h  = 
 	\begin{cases} 
 		 (f \circ_i h) \circ_{\ari(f)+j-1} g
 		  	& \delta \leqslant 0  \\
 		  	f\circ_j (g \circ_\delta h) &
 		  	\delta\in [1,\ari(g)] \\
 		  	(f \circ_\delta h) \circ_j g & \delta > \ari(g)
 		   \end{cases}
 		 \]
\item \emph{Intrinsic equivariance}: for
each $\sigma\in S_k$, we have that
\[  (f\sigma) \circ_i g  = (f\circ_{\sigma i} g)\sigma'\]
where $\sigma'$ is the same permutation as $\sigma$
that treats the block $\{i,i+1,\ldots,i+\ari(g)-1\}$
as a single element. 
\item \emph{Extrinsic equivariance}: 
 for each $\sigma\in S_k$, we have that
\[  f \circ_i (g\sigma)  = (f\circ_i g)\sigma''\]
where $\sigma''$ acts by only permuting the
block $\{i,\ldots,i+\ari(g)-1\}$ according
to $\sigma$.
\item \emph{Unitality:} the identity $1\in\End_V(1)$
satisfies $1 \circ_1 g = g$ and $g\circ_i 1 = g$ for every $g\in\End_V$ and $1\leqslant i\leqslant \ari(g)$.
\end{tenumerate}

\begin{definition}
A symmetric operad (in vector spaces) is a
$\Sigma$-module $\PP$ along with partial composition
map of signature
\[ -\circ_i -  : \PP(m)\otimes \PP(n) 
	\longrightarrow \PP(m+n-1) \]
and a unit $1\in\PP(1)$ satisfying the axioms above.
\end{definition}

It is not hard to see (but must be checked at least once)
that an operad with $\PP(n) = 0$ for $n\neq 1$ is
exactly the same as an associative algebra. 

\textbf{Warning!} If one does not require
the existence of a unit, the notion of a
\emph{pseudo-operad} by Markl (defined by partial
compositions) does not coincide with the
notion of an operad as defined by May.


\subsection{Constructing operads by hand}

One can define operads in various ways. For example,
one can define the underlying collection explicitly,
and give the composition maps directly:
\begin{tenumerate}
\item \emph{Commutative operad.} The reduced symmetric topological (or set)
operad with $\mathsf{Com}(n)$ a single point for each
$n\in \NN$, and composition maps the unique
map from a point to a point.
\item \emph{Associative operad.} 
The reduced set operad with
$\mathsf{As}(n) = S_n$
the regular representation and composition maps
\[ S_k \times S_{n_1} \times
\cdots \times S_{n_k} \longrightarrow S_{n_1+\cdots+ n_k} \]
the unique equivariant map that sends the tuple
of identities to the identity.
\item  \emph{Stasheff operad.}
Let $K_{n+2}$ be the subset of $I^n$ (the
product of $n$ copies of $I=[0,1]$) 
consisting of tuples $(t_1,\ldots,t_{n+2})$
such that $t_1\cdots t_k\leqslant 2^{-k}$
for $j\in \underline{n+2}$. The boundary of 
$K_{n+2}$ consists of those points such
that for some $j\in \underline{n+2}$ we have
either $t_j$ or $t_1\cdots t_j = 2^{-j}$.
It is tedious (but otherwise doable)
to show that for each pair $(r,s)$ of
natural numbers and each $i\in \underline{r}$
there exists an inclusion
\[ \circ_i : K_{r+1} \times K_{s+1} \longrightarrow
 	K_{r+s+1} \] 
that defines on the sequence of
spaces $\{K_{n+2}\}_{n\geqslant 0}$
the structure of a non-symmetric operad.

\item If $M$ is a monoid, there is an
operad $\mathbb W_M$ with $\mathbb{W}_M(n) =
M^n$ such that 
\[(m_1,\ldots,m_s) \circ_i (m_1',\ldots,m_t') = 
 	(m_1,\ldots,m_{i-1}, m_im_1',\ldots,m_im_t',m_{i+1},\ldots, m_s).\] We call it the
 \emph{word operad of $M$}. Its underlying
 symmetric collection is $\mathsf{As}\circ M$. 
 
 \item Write $\operatorname{Aff}(\mathbb C) = \mathbb{C}\times \mathbb{C}^\times$ for the group of affine transformations of
$\mathbb C$ with group law $(z,\lambda)(w,\mu) = (z+\lambda w,\lambda\mu)$. In turn, define for each finite set $I$ the topological space
\[ \mathcal{C}(I) = \{ (z_i,\lambda_i)\in \operatorname{Aff}(\mathbb C)^I  : |z_i-z_j|>|\lambda_i|+|\lambda_j| \}.\] 
The group law of $\operatorname{Aff}(\mathbb C)$ allows us to
define an operad structure on $\mathcal{C}(I)$ using the
exact same definition as in the word operad of a monoid. 
The subspaces $\mathcal{D}_2^{\mathrm{fr}}(I)
	\subseteq \mathcal{C}(I)$
where $|z_i|+|\lambda_i|\leqslant 1$ for all $i\in I$, and 
where the inequality is strict unless $z_i=0$ is called
the \emph{framed little disks operad}. The little disks operad
is the suboperad where $\lambda_i = 1$ for all $i\in I$, and 
we write it $\mathcal{D}_2(I)$.

 \item The operad of rooted trees $\mathsf{RT}$ has
 $\mathsf{RT}(n)$ the collection of rooted trees with $n$
 vertices labelled by $\underline{n}$, and the composition $T \circ_j T'$
  is obtained by inserting $T'$ at the $j$th vertex of $T$
  and reattaching the children of that vertex to $T'$ in
  all possible ways. For example, if
\[
\mathsymbol{.6}{S=} \begin{tikzpicture}
	\begin{pgfonlayer}{main}
		\node [style=inner] (0) at (0, 0) {$1$};
		\node [style=inner] (2) at (0, 1) {$2$};
	\end{pgfonlayer}
	\begin{pgfonlayer}{bg}
		\draw (2.center) to (0.center);
	\end{pgfonlayer}
\end{tikzpicture}\quad
\mathsymbol{.6}{T=}
\begin{tikzpicture}
	\begin{pgfonlayer}{main}
		\node [style=inner] (0) at (0, 0) {$1$};
		\node [style=inner] (1) at (-.75, 1) {$2$};
		\node [style=inner] (3) at (.75, 1) {$3$};
	\end{pgfonlayer}
	\begin{pgfonlayer}{bg}
		\draw (1.center) to (0.center);
		\draw (0.center) to (3.center);
	\end{pgfonlayer}
\end{tikzpicture}
\]
then we have that  
\[
\mathsymbol{.6}{T\circ_1 S \; =} \;
\begin{tikzpicture}
	\begin{pgfonlayer}{main}
		\node [style=inner] (0) at (0, 0) {$1$};
		\node [style=inner] (1) at (-1, 1) {$2$};
		\node [style=inner] (2) at (0, 1) {$3$};
		\node [style=inner] (3) at (1, 1) {$4$};
	\end{pgfonlayer}
	\begin{pgfonlayer}{bg}
		\draw (1.center) to (0.center);
		\draw (2.center) to (0.center);
		\draw (0.center) to (3.center);
	\end{pgfonlayer}
\end{tikzpicture}
\;\mathsymbol{.6}{+}\quad
\begin{tikzpicture}
	\begin{pgfonlayer}{main}
		\node [style=inner] (0) at (0, 0) {$1$};
		\node [style=inner] (1) at (0, 1) {$2$};
		\node [style=inner] (2) at (0, 2) {$3$};
		\node [style=inner] (3) at (1, 2) {$4$};
	\end{pgfonlayer}
	\begin{pgfonlayer}{bg}
		\draw (2.center) to (1.center);
		\draw (1.center) to (0.center);
		\draw (1.center) to (3.center);
	\end{pgfonlayer}
\end{tikzpicture}
\;\mathsymbol{.6}{+}\quad
\begin{tikzpicture}
	\begin{pgfonlayer}{main}
		\node [style=inner] (0) at (0, 0) {$1$};
		\node [style=inner] (1) at (0, 1) {$2$};
		\node [style=inner] (2) at (0, 2) {$3$};
		\node [style=inner] (3) at (1, 1) {$4$};
	\end{pgfonlayer}
	\begin{pgfonlayer}{bg}
		\draw (2.center) to (1.center);
		\draw (1.center) to (0.center);
		\draw (0.center) to (3.center);
	\end{pgfonlayer}
\end{tikzpicture}
\;\mathsymbol{.6}{+}\quad
\begin{tikzpicture}
	\begin{pgfonlayer}{main}
		\node [style=inner] (0) at (0, 0) {$1$};
		\node [style=inner] (1) at (0, 1) {$2$};
		\node [style=inner] (2) at (0, 2) {$4$};
		\node [style=inner] (3) at (1, 1) {$3$};
	\end{pgfonlayer}
	\begin{pgfonlayer}{bg}
		\draw (2.center) to (1.center);
		\draw (1.center) to (0.center);
		\draw (0.center) to (3.center);
	\end{pgfonlayer}
\end{tikzpicture}\mathsymbol{.6}{.}
\]
\end{tenumerate}
\subsection{Exercises}

 \begin{question}
 Our definition of the composition product of
 $\Sigma$-modules is not completely unbiased.
 Show that for each finite set $I$,
 the space $(\XX\circ_\Sigma \YY)(I)$ is isomorphic
 to
\[
\bigoplus_{\pi\vdash I} \XX(\pi)\otimes
 \YY[\pi] \quad \text{ where }\quad
\YY[\pi]  = \bigotimes_{p\in\pi}\YY(p).
\]
 The action of $\sigma\in\Aut(I)$ is as
 follows: for each ordered partition $\pi$ of $I$,
 the set ${}^\sigma\pi = \{ \sigma(p) : p\in \pi\}$
 is a partition of $I$, and we have an induced
 map $f_\pi :  \XX(\pi)\otimes \YY[\pi]
 \longrightarrow \XX({}^\sigma\pi)\otimes \YY[{}^\sigma\pi]$. The resulting map is obtained
 by summing these. Make sure to explain why
 in our original definition there is are tensor
 products taken over $S_k$ for $k\geqslant 1$.
  \end{question}
 
 \begin{question}
Define the category of collections
in $\mathsf{Vect}$
using the biased approach and the 
unbiased approach (this requires considering
\emph{totally ordered} sets instead of sets,
and their order preserving bijections. We will
write them with calligraphic letters but
use subscripts, so $\mathcal X$ has ns
components $\{\mathcal X_n\}_{n\geqslant 1}$.

\begin{enumerate}
\item Show
that it supports a non-symmetric Cauchy
product given by 
\[ (\mathcal{X}\otimes\mathcal{Y})_n =
  \bigoplus_{i+j=n} \mathcal{X}_i\otimes
   	\mathcal{Y}_j.\]
   	
   	\item Use this and the unbiased approach to 
   	argue that the ns counterpart of a
   	`subset of $I$' is an interval:
   	a totally ordered subset of $I$ of the
   	form $[i,j] = \{ x\in I : i\leqslant x \leqslant j\}$.
 \item Use the previous item to define the
 non-symmetric composition $\circ_{ns}$
  of ns collections.
 Define the generating function associated
 to a collection, and show it behaves
 well with respect to the products above. 
\end{enumerate}
\end{question}

\begin{question}
Since every finite totally ordered set
is, in particular, a finite set (and
every order preserving function is a
fortiori a function) there is a 
map of categories 
$ \mathsf{FinOrd}^\times \longrightarrow
 	\mathsf{FinSet}^\times$
which induces a map that `forgets the
symmetries' ${}_\Sigma\mathsf{Mod}	\longrightarrow\mathsf{Coll}$. 
Show that there is a functor that assigns
a ns sequence $\mathcal{X}$ to the 
sequence $\mathcal{X}_\Sigma(n) =
\kk S_n\otimes \mathcal{X}_n$.  Show that it is
left adjoint to the forgetful functor and that it is monoidal. 
\end{question}
 
\begin{question}
Describe the associator for $\circ_\Sigma$ in the
category of differential graded collections. 
In particular,
write down the signs explicitly. Explain
how this is related to the signs
in the parallel composition axiom
for \emph{graded operads} that read as follows:
for elements $f,g$ and $h$
in an operad (of homogeneous arities)
and $\delta = i-j+1$, we have that
\[ 
(f \circ_j g) \circ_i h  = 
 	\begin{cases} 
 		(-1)^{|g||h|}
 		 (f \circ_i h) \circ_{\ari(f)+j-1} g
 		  	& \delta \leqslant 0  \\
 		 \phantom{(-1)^{|g||h|}(} 	f\circ_j (g \circ_\delta h) &
 		  	\delta\in [1,\ari(g)] \\
 		(-1)^{|g||h|}
   	(f \circ_\delta h) \circ_j g
   		 & \delta > \ari(g).
 		   \end{cases}
 		 \]
\end{question}

\begin{question}
A (unital associative) monoid $x$ in a monoidal category 
$(\mathcal C,\otimes,\alpha,\rho,\lambda,1)$ is an object
along with maps $\mu: x\otimes x\to x$ and $\eta : 1
 \longrightarrow x$ such that $\mu$ is associative, 
 that is $\mu (\mu\otimes 1) = \mu(1\otimes \mu)
 \alpha_{x,x,x}$, and unital for
$\eta$, that is $\mu(\eta\otimes 1)=\rho_x$
and $\mu(1\otimes \eta) = \lambda_x$.
Show that a $\Sigma$-operad is exactly the
same as a monoid in $({}_\Sigma\mathsf{Mod},\circ_\Sigma)$.
\end{question}

\begin{question}
We write $\mathsf{End}$ for
 category of endofunctors of $\mathsf{Vect}$. Show
 that there is a \emph{monoidal}
 functor $S:{}_\Sigma\mathsf{Mod}
 \longrightarrow \mathsf{End}$ that assigns
 $\mathcal{X}$ to $V\longmapsto \bigoplus_{n\geqslant 0} \mathcal{X}(n)\otimes_{\Sigma_n} V^{\otimes n}$.
 That is, 
  you have to show that $S_\XX\circ
 S_\YY$ is naturally isomorphic to $S_{\XX\circ_\Sigma \YY}$.
 It is called the \emph{Schur functor} associated
 to $\mathcal{X}$. The endofunctors in the essential
 image of $S$ are called \emph{analytic}.
\end{question}

\begin{question}
	 If $\mathcal{X}$
	 is a symmetric sequence, describe the $\Sigma_n$
	 action on $\mathcal{X}^{\otimes n}$ where
	 $\otimes$ is the Cauchy product. Observe that
	 it commutes with the $\Aut(I)$ action on 
	 $\mathcal{X}^{\otimes n}(I)$.
\end{question}
	 
\begin{question}
Define ${}_\Sigma\mathsf{Mod}(\CC)$
for any symmetric monoidal category $(\CC,\otimes,1)$
(such as the category of sets, or topological spaces,
or chain complexes, among others) along with
its \emph{symmetric composition product}
 $-\circ_\Sigma - $.
 \end{question}
 
\begin{question} Prove that non-unital Markl operads
and non-unital May operads differ.
To do this, consider the non-unital ns operad
$\PP$ such that $\PP(2)$ and $\PP(4)$
are its only non-zero components, and are
both one dimensional, and define
\[ \gamma : \PP(2)\otimes \PP(2)\otimes \PP(2)
 	\longrightarrow \PP(4) \]
to be an isomorphism, and set all other maps to zero. 
Check that $\PP$ is a May operad, and
show that $\PP$ is not a Markl
operad by exploring the consequences
of the equality
\[ \mu(\mu,\mu) = (\mu \circ_2\mu)\circ_1 \mu \]
in any Markl operad.
\end{question}

\begin{question}
 Check that examples (1), (2), (4), (5) are indeed all operads.
\end{question}
%
%\begin{question}
%Follow these steps to construct the
%Stasheff operad as a sequence of 
%convex polytopes $K_2',K_3',\ldots$
%for which the boundary of $K_{n+1}'$
%is a union of products $K_{r+1}'\times
%K_{s+1}'$ with $r+s=n$ indexes by
%planar rooted trees with two internal
%vertices.
%\end{question}
%\begin{enumerate}
%\item Let us write $T_n$ for the collection
%of planar rooted \emph{binary} 
%trees with $n+1$ leaves, which we 
%order from left to right. Explain how
%this gives a total order on the vertices,
%which we will thus call $1,\ldots,n$.
%\item  For
%each $t\in T_n$ and each vertex $i$ of
%$t$, let $L(i)$ denote the number of paths
%from $i$ to a leaf of $t$ going through
%its left child, and let $R(i)$ denote
%the the number of paths
%from $i$ to a leaf of $t$ going through
%its right child. We define
%\[ x(t) = (L(1)R(1),\ldots,L(n)R(n))
%	\in \mathbb N^n. \]
%Show that $x(t)$ always lies in the
%hyperplane $x_1+\cdots x_n = \binom{n}{2}$.	
%We write $K_{n+2}'$ for the convex hull of 
%the points $\{ x(t) : t\in T_n \}$.
%\emph{Hint.} Any planar binary rooted tree
%$t$ decomposes into a left tree $L_t$
%and a right tree $R_t$ by looking at the
%children of the unique child of the root.
%Express $W(t) = \sum_{i=1}^n x(t)$ in terms of $W(R_t)$ and $W(L_t)$.
%
%\item 
%Show that the polytope $K_{n+2}'$ 
%is of dimension $n$, 
%and its $k$-cells for $k\in \underline{n}$
%are in bijection with planar rooted
%trees with $n-k+1$ internal vertices
%and $n+2$ leaves. Conclude, in particular,
%that its codimension one faces are
%in bijection with planar rooted trees
%with $2$ internal vertices and $n+2$
%leaves.
%
%\item Suppose that $t$ has $r+1$
%leaves and that $t'$ has $s+1$ leaves,
%and consider the grafting $t\circ_i t'$.
%We define $x(t)\circ_i x(t')$ by
%$x(t\circ_i t')$. Show that this
%defines a map
%\[\circ_i :  K_{r+1}'\times K_{s+1}' 
%	\longrightarrow K_{r+s+1}'.\]
%\item Show the maps above give the
%collection $\{K_{n+2}'\}_{n\geqslant 0}$
%the structure of a ns operad.
%\end{enumerate}

\begin{question}
Suppose that $T\in\mathsf{RT}(n)$ and
that $T'\in \mathsf{RT}(m)$, where $\mathsf{RT}$
is the symmetric collection
 of rooted trees of Lecture 1,
and let $\mathrm{In}(T,i)$ denote the
set of incoming edges of $T$ at the
vertex labeled $i$. For each function
$f: \mathrm{In}(T,i)\longrightarrow \underline{m}$,
define the tree $T\circ_i^f T'$ by
replacing vertex $i$ of $T$ by $T'$ and
attaching the loose incoming edges of 
vertex $i$ to the vertices of $T'$
according to the map $f$: the edge
$e\in \mathrm{In}(T,i)$ is attached
to vertex $f(e)\in T'$. Finally,
define $T\circ_i T'$ by taking the
sum through all possible functions
$f$. Show that this gives $\mathsf{RT}$
the structure of a unital pseudo-operad,
and thus of a usual operad, with unit
the tree with no edges and one vertex.
\end{question}


\begin{question}
 Describe the operation $T\star T' = S(T,T')$ where
$S$ is the rooted tree 
\[
\mathsymbol{.6}{S=} \begin{tikzpicture}
	\begin{pgfonlayer}{main}
		\node [style=inner] (0) at (0, 0) {$1$};
		\node [style=inner] (2) at (0, 1) {$2$};
	\end{pgfonlayer}
	\begin{pgfonlayer}{bg}
		\draw (2.center) to (0.center);
	\end{pgfonlayer}
\end{tikzpicture}
\]
above in terms of 
insertions of $T'$ in $T$ and regrafting of incoming 
edges. Show that it satisfies the following \emph{pre-Lie
identity}:
\[
  (T\star T')\star T'' -  T \star (T' \star T'' ) =
    (T\star T'')\star T' -  T \star (T'' \star T' ) 
 	\]
 	by explicitly interpreting the left hand side in
 	terms of certain insertions of $T'$ and $T''$ in $T$,
 	and showing the resulting sum of trees is symmetric
 	in $T'$ and $T''$.

\end{question}

\begin{question}
Suppose that $\mathcal P$ is an operad
and that $\mathcal X\subseteq \mathcal P$
is a symmetric subsequence. We say
$\mathcal X$ generates $\mathcal P$
if every element of $\mathcal P$ is
an iterated composition of elements of
$\mathcal X$. 
Show that the rooted trees operad 
$\mathrm{RT}$ 
is generated by the symmetric subsequence
generated by the rooted tree $S$ of the
previous exercise, which spans
the regular representation of
$S_2$. Follow these steps:
\end{question}
\begin{tenumerate}
\item Suppose that $T$ is an $n$-rooted tree
and let $J$ be a subset of $\underline{n}$ corresponding
to leaves of $T$ that are the children of a
vertex $i\in T$. Let $T'$ be the
tree obtained by erasing all these leaves
and replacing the vertex label by a new
symbol $\ast$, and let $T''$ be the rooted
tree with root $i$ and children labeled
by $J$. Show that $T'\star_\ast T'' = T$.
\item Use the above and induction on the
number of vertices to show it suffices to prove
the claim for the corollas, that is, trees with
one internal root vertex.
\item Let us write $T_n$
for the operation in $\mathsf{RT}(n)$
corresponding to a corolla with root $1$,
so in particular $T_2 = S$.
Show that 
\[ 
T_n = 
 T_2\star T_{n-1} - 
  	\sum_{i=1}^{n-1} (T_{n-1}\star T_i)\sigma_i
\]
where $\sigma_i = (i+1,i+2,\ldots,n)\in S_n$
is a cycle, and use this to conclude.
\end{tenumerate}

\emph{Note.} The operation $T_n$ is usually
denoted $\{x_1; x_2,\ldots,x_n\}$ and is called
a \emph{symmetric brace}, and the equation
above is usually written in the form
\[ 
\{x_1; x_2,\ldots,x_n\} = 
 \{\{x_1; x_2,\ldots,x_{n-1}\}; x_n\}
 - \sum_{i=1}^{n-1} \{ x_1; x_2,
 \ldots, x_{i-1}, \{x_i ; x_n \},
 x_{i+1},\ldots,x_{n-1}\}.
 \]
 
 \begin{question}\label{ex:operations-on-species}
 Let $\XX$ and $\YY$ be a symmetric sequences.
 We define \[f_\XX(z) = \sum_{n\geqslant 0} \dim \XX(n) \frac{z^n}{n!}\]
 and call it the Hilbert series of $\XX$
 (or dimension series, if we would like to be
 faithful to the original name). The derivative $\partial\mathcal X$ 
 of $\mathcal X$ is the symmetric sequence
 with $(\partial\mathcal X)(I) = \mathcal X(I^*)$
 where $I^* = I \sqcup \{ I \}$.
 \begin{tenumerate}
 \item Show that there is an equality of power series $f_{\XX\otimes \YY}(z) = f_\XX(z)\cdot f_\YY(z)$.
 \item Show that $(\partial \mathcal X)(n)$ 
 is isomorphic to the restriction of $\mathcal{X}(n+1)$ to $S_n = \mathrm{Fix}(n+1)$.
 \item  Show  that there is an equality of power series $f_{\mathcal X}'(z) = 
  			f_{\partial\mathcal X}(z)$.
  \item Suppose that $\YY$ is reduced. Show that $f_{\XX}(f_\YY(z))$ is defined and that it is equal to 
  $f_{\XX\circ \YY}(z)$.
  \end{tenumerate}
 \end{question}
