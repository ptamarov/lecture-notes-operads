

\section{Methods to prove an operad is Koszul II}\label{lecture:methods2}

\subsection{Filtrations and rewriting}

In this section, we will focus on examples where
$\PP$ is a binary quadratic operad and one can
compute the homology of either the left or
right Koszul complex explicitly. By this, we
mean that one can pin down the result directly
by inspection, or use some mild homological tools,
like filtrations, to simplify the problem.

\begin{example}
Let us consider the associative operad $\As$,
which we study as a ns operad, as sending a ns
to the corresponding symmetric one does not
break Koszulness. In this case, we can compute
the Koszul complex 
$K =\As^\antishriek\circ_\tau \As$ explicitly. 

Namely, $\As^\antishriek(n)$ is one dimensional
for each $n\geqslant 1$ generated by a cooperation
$\mu_n^c$, and so is $\As(n)$, generated by the
iterated associative product $\mu_n$,
so that $K(n)$ is spanned by elements of the form
\[
[n_1,\ldots,n_k] = 
	\mu_k^c(\mu_{n_1},\ldots,\mu_{n_k}).
\]
Moreover, the decomposition map of $\As^\antishriek$
is dual up to one suspension to the (infinitesimal)
composition map of $\otimes\As$, that is, we can
explicitly compute that
\[
\Delta_{(1)}(\mu_k^c) = 
 \sum_{k_1+k_2 = k+1}
 	\sum_{i=1}^{k_1}(-1)^{(k_2-1)(i-1)}
 		 \mu_{k_1}^c\circ_i \mu_{k_2}^c.
\]
To compute the differential on a generic basis element
$B(k;n_1,\ldots,n_k)$, we need only restrict to the
case when $k_2=2$, since in this case the weight
one elements are precisely those of arity $2$.
Computing the corresponding composition, we arrive
at the following formula:
\[
\partial [n_1,\ldots,n_k] = 
\sum_{i=1}^{k-1} 
(-1)^{i-1}[n_1,\ldots,n_i+n_{i+1},\ldots,n_k].
\]
Thus, we see that $K(n)$ is isomorphic to the
complex computing the simplicial homology
of the $n-2$ simplex $\Delta^{n-2}$ spanned
by vertices $v_1,\ldots,v_{n-1}$ if we consider
sending the simplicial basis element
$[v_{i_1},\ldots,v_{i_k}]$ where
$i_1 < \cdots < i_k$ to the basis element
in $K(n)$ given by
$[i_1,i_2-i_1,\ldots,i_k-i_{k-1},n-i_k]$.

 \end{example}

%\textbf{Monomial quadratic operads.}
%A symmetric operad $\PP$ is said to be 
%monomial if it is generated by an
%alphabet $\XX$ subject to relations
%$\RR$ that are tree monomials. We will show
%in this section that every monomial quadratic
%operad is Koszul. Passing to the corresponding
%shuffle operad, it is clear in this case that
%$\PP^\f$ has $\RR^\f$ as a Gr\"obner basis  for any
%choice of order $\prec$, and that a linear
%basis for $\PP^\f$ consists of those shuffle tree
%monomials not divisible by any tree monomial in
%$\RR^\f$.
%
%First, let us begin by observing that the bar
%complex $\B{\PP}$ of a monomial operad is
%graded by the underlying tree monomial of a
%bar tree monomial, as the differential either
%sends a bar tree monomial to zero (if a divisor
%appears), or simply merges two ``large vertices''
%(perhaps introducing a sign) if no divisor appears.
%Thus, when computing the bar homology of $\PP$,
%we can assume that $\XX$ is finite and restrict
%only to those divisors of a fixed underlying
%tree monomial $T$, of which there are finitely many.
%
%Restricting to the case $\XX$ (and thus the
%quadratic collection of monomials $\RR$) is finite
%we notice that the dual is given by the complementary

Let us now consider a common homological technique to 
prove that complexes are acyclic. Let $(C,d)$ be
a chain complex, and suppose that $F$ is a filtration
of $C$, that is, a family 
$\{ F_p \}_{p\in\mathbb Z}$ of subcomplexes
such that $F_{p-1}\subseteq F_p$ for all
 $p\in \mathbb Z$ and such that their union is
 $C$; although this is a requirement for us, some
 people call such filtrations \emph{exhaustive}. For 
 reasons that will become apparent below,
 we will restrict ourselves to filtrations that are
 bounded below, that is, we will assume that $F_p=0$
 if $p<0$.\footnote{Note that we use $p$ for the
 indexing letter here: this is customary, to avoid
 confusing it with homological degrees.}
 
 If $(C,d,F)$ is a complex with a non-negative
 filtration, we define $\mathrm{gr}_F(C)$ to be
 the graded complex with $p$th graded piece given by
 the quotient $\mathrm{gr}_F(C)_p =   F_p/F_{p-1}$.
 This is indeed a complex, since $d$ preserves 
 each subcomplex of $F$.
 
 \begin{proposition}\label{prop:graded}
 Let $(C,d,F)$ be a non-negative chain
 complex with a filtration that is
 bounded below. If $(\mathrm{gr}_F(C),d)$ is
 acyclic, then the same is true for $C$.
\end{proposition}  
  
  \begin{proof}
  Pick $c\in C$ non-zero, 
  and suppose that $dc=0$,
  so that our aim is to show that $c$ is a boundary.
  By hypothesis, there exists a smallest $p$
  such that $c\in F_p$ but $c\notin F_{p-1}$,
  so let us proceed by induction on $p$.
  Since $c$ is a cycle in $C$, it is a cycle
  in $\mathrm{gr}_F C$. 
  By hypothesis, this complex is acyclic, 
  which means that there exists a class $[c']$
  such that $d[c'] = [c]$ in $F_p/F_{p-1}$. In
  other words, there exists $c_1\in F_p$ and
  $c'\in F_{p-1}$ such that
  $c- dc_1 = c'\in F_{p-1}$.
 Since $dc' = 0$, we know by induction that $c'=dc_2$
 for some $c_2$, and so $c = dc_1+dc_2$ is a
 boundary, too, like we wanted.
  \end{proof}
  
 Let us now consider the situation where $\PP$
 is a symmetric operad. Since the forgetful
 functor from symmetric to shuffle operads is
 monoidal, it preserves the bar construction,
 in the sense that $\B{\PP^\f}$ and
 $\B{\PP}^\f$ are naturally isomorphic,
 and we can attempt to apply filtration methods
 to the shuffle operad associated to $\PP$.
 Thus, in what follows, we will consider instead
 the case $\PP$ is a shuffle operad which is
 filtered, in the sense it admits a bounded
 below (exhaustive!) filtration by subcollections,
 such that for any shuffle composition we have
 \[
 F_p \circ_{\sigma,i} F_q \subseteq
  F_{p+q}.
 \]
 
 \begin{lemma}
 Suppose that $(\PP,F)$ is a filtered shuffle operad.
 Then declaring a shuffle tree monomial $T$ in $\PP$
 to be in filtration degree $p$ if we have that
 \[\sum_{v\in T} p(\mathsf{x}_v) \leqslant p \]
 for the filtration degrees of its decorations
 in $\PP$ defines a filtration $\B{F}$ 
 of dg conilpotent
 cooperads on the shuffle bar construction
 $\B{\PP^\f}$.
 \end{lemma}
 
 \begin{proof}
 The filtration will be one of cooperads without
 imposing any compatibility conditions with the
 shuffle compositions of $\PP$. This
 later compatibility condition implies that
 the differential of the bar construction
 preserves the resulting subcomplexes,
 so that the filtration is in fact of dg
 cooperads.  \end{proof}
 
 With this at hand, we arrive at one of the most
 useful criteria to check if a symmetric operad
 is Koszul.
 
 \begin{theorem}
 Let $\PP$ be a symmetric operad, and suppose
 that the shuffle operad $\PP^\f$ admits
 a filtration (of shuffle operads) for which
 the associated graded operad is Koszul.
 Then $\PP$ is a Koszul operad.
 \end{theorem}
 
 \begin{proof}
 The only thing we have to show is that the
 associated graded conilpotent cooperad to
 $\B{\PP^\f}$ with respect to the filtration
 $\B{F}$ is isomorphic to $\B{\mathrm{gr}_F\PP^\f}$.
 Indeed, once this is verified we conclude using
 Proposition~\ref{prop:graded} and our definition
 of Koszulness using the syzygy grading: the
 map $\PP^\antishriek \longrightarrow \B{\PP}$
 induces a quasi-isomorphism in $H^0$, so it suffices
 we prove that $\B{\PP}$ has no positive cohomology.
 
 By definition, the basis elements of
 $\mathrm{gr}_{\B{F}}\B{\PP^\f}$ of degree $p$
 are those shuffle tree monomials $T$
 whose decorations satisfy $T$
 \[
 \sum_{v\in T} p(\mathsf{x}_v) = p.
 \]
 At the same time, the shuffle operad 
 $\mathrm{gr}_F\PP^\f$  is graded and 
 thus its bar construction inherits an
 extra weight grading: the weight of a tree
 monomial $T$ is the sum of the weights of its
 decorations, so that $\mathsf{wt}(T)= p$
 precisely when
 \[
 \sum_{v\in T} \mathsf{wt}(\mathsf{x}_v) = p.
 \]
 Thus, we have a canonical map
 \[
 \mathrm{gr}_{\B{F}}\B{\PP^\f} \longrightarrow
 \B{\mathrm{gr}_F\PP^\f}
 \]
 which is an isomorphism, since the weight
 $\mathsf{wt}(\mathsf{x}_v)$ is precisely
 the least of those $p$ such that $\mathsf{x}_v
 \in F_p\PP^\f$, and if this drops, the total sum above
 will also drop. 
 \end{proof}
 
 The previous result can be used together with
 rewriting methods, as follows. If $\PP$ is
 a shuffle operad generated by a collection
 $\XX$, and we have chosen an admissible 
 monomial ordering $\prec $
 on $\FF_\XX$, the projection
 $\FF_\XX\longrightarrow \PP$ defines a filtration
 on $\PP$, by declaring that the class of a 
 monomial is in degree at most $p$ if it
 is represented by a monomial in the free shuffle
 operad of weight $p$: notice that we are being
 slightly imprecise with our notation, since now
 $p$ does not belong to the natural numbers,
 but rather to a totally order set (determined by
 our choice of monomial order).
 
 The resulting filtration $F_\prec$ in $\PP$
 defines a graded shuffle operad $\mathrm{gr}_\prec
 \PP$, and at the same time we have an associated
 operad $\PP_{\mathsf{mon}} = \FF_\XX / (\lead{\RR})$
 obtained by keeping only the leading terms
 of relations of $\PP$. Moreover, we have a map
 \[
 \pi : \PP_{\mathsf{mon}} 
  	\longrightarrow 
  	\mathrm{gr}_\prec \PP
 \]
 since the leading terms of $\RR$ define
 relations in the codomain, and in fact this
 map is an isomorphism in weights one and two.
 
 \begin{theorem}
 Let $\PP$ be a quadratic operad and suppose
 that the shuffle operad $\PP^\f$ admits a
 quadratic Gr\"obner basis with respect to some
 monomial order $\prec$. Then $\PP$ is
 Koszul. 
 \end{theorem}
 
 \begin{proof}
 By the previous theorem, it suffices to show
 that $\mathrm{gr}_\prec \PP$ is Koszul. We
 see that this operad is isomorphic to the
 quotient of $\PP$ by the initial ideal associated
 to $\RR$. Thus, $\pi$ is an isomorphism 
 precisely when $\RR$ is a Gr\"obner basis
 for $(\RR)$ with respect to $\prec$, and
 it suffices to show
 a shuffle operad with quadratic monomial
 relations is Koszul, which we have
 already done. Thus,
 the claim follows for
 $\PP^\f$, which is what we wanted. 
 \end{proof}
 

\begin{example}
We have checked (or rather, suggested as Exercise~\ref{ex:SpolyLie})
that the shuffle operad $\mathsf{Lie}^\f$ has a quadratic Gr\"obner
basis where the Jacobi relation is the only element. It follows
that the Lie operad is Koszul and, hence, that the commutative
operad is Koszul, too. We will prove in Exercise~\ref{ex:quadratic-GB-dual} that in fact the Koszul dual of a shuffle operad with
a quadratic Gr\"obner basis also admits a Gr\"obner basis.
\end{example}

The following figure tabulates some well-known
quadratic operads, their shuffle generators,
and a choice of order that yields a quadratic
Gr\"obner basis. 


\begin{center}
\begin{tabular}{@{}lccc@{}} \toprule
Operad & Type & $\Sha$-generators & Order \\
\midrule 
PreLie & $\Sigma$ & $\circ,\bar\circ$ & 
	\texttt{deglex perm} for $\bar\circ \prec \circ$.\\
Perm &   $\Sigma$ & $\cdot,\bar\cdot$ &
	\texttt{deglex perm} for $\cdot \prec \bar \cdot$  \\
Zinb &  $\Sigma$  & $\cdot,\bar\cdot$ &
	\texttt{deglex perm} for $\cdot \prec \bar \cdot$ \\

Leib &  $\Sigma$  & $\cdot,\bar\cdot$ & 
	\texttt{deglexr perm} for $\bar\cdot \prec \cdot$\\
Dias &   $\Sigma$  & \\
Poiss &   $\Sigma$ & \\
Dend &  ns  & $\prec,\succ$ & \\
\bottomrule
\end{tabular}
\end{center}




\subsection{Distributive laws}

If $V$ and $W$ are associative algebras, and if we
are given a map $\tau : W\otimes V\longrightarrow V\otimes W$,
we can consider defining a new associative algebra structure
on $V\otimes W$ by first using the map
\[
V\otimes W\otimes V\otimes W 
	\xrightarrow{1\otimes \tau\otimes x}
	 V\otimes V\otimes W\otimes W
	 \]
and then the products of $V$ and $W$. This map will be
associative precisely when $\tau$ satisfies a certain
compatibility condition with $\mu_V$ and $\mu_W$. The upshot
is that one obtains a new associative algebra
$V\otimes_\tau W$ where elements of $V$ and $W$ do not
commute, but we have
\[ 1\otimes w\cdot v\otimes 1 = \tau(v\otimes w).
\]
One can produce a completely analogous formalism for
algebraic operads, as it was done by Markl in~\cite{MarklDistributive}. 

\hacer{Write section on distributive laws.}

\newpage


\subsection{Exercises}

\begin{question}\label{ex:quadratic-GB-dual} 
Let $\PP$ be a quadratic operad, generated by a \emph{set}
of variables $\XX$ subject to relations $\RR$, 
and endow $\FF_\XX^\Sha$ with some
admissible monomial order. Identify $\sus^{-1}\otimes \XX^*$
with $\XX$ through the canonical pairing, and consider
the opposite order on $\FF_{\XX^\vee}^\Sha$ 
under this identification.
\begin{tenumerate}
\item Show that if $\mathcal B$ is the linearly self-reduced
basis of $\RR$ and $\mathcal B^\perp$ the linearly self-reduced
basis of $\RR^\perp$, then $\leadm{\mathcal B}\sqcup
\leadm{\mathcal B^\perp} = \FF_\XX^{\Sha,(2)}$.
\item The set of tree monomials for which every quadratic divisor belongs to $\leadm{\mathcal B}$ spans $\PP^!$. In particular, these give
arity-wise upper bounds for the dimension of $\PP^!$.
\item This upper bound is sharp for all $n$ such that $\FF_\XX^{\Sigma,(3)}(n) \neq 0$ if and only if the operad $\PP^!$ has a quadratic 
Gr\"obner basis.
\item The operad  $\PP$ has a quadratic 
Gr\"obner basis if and only if its Koszul dual operad does. 
\end{tenumerate}
\end{question}