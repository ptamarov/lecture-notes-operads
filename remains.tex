
\subsection{Exercises}
\newcommand{\antishriek}{\text{\raisebox{\depth}{\textexclamdown}}}
\begin{definition}
A quadratic operad is Koszul if one (and hence all)
of the following equivalent conditions are
satisfied:
\begin{tenumerate}
\item The cohomology group $H^s(\mathsf{B}(\PP))$ is zero for $s>0$.
\item We have $\mathrm{rate}(\PP)=1$.
\item The inclusion
$H^0(\mathsf{B}(\PP)) \longrightarrow \mathsf{B}^*(\PP)$
is a quasi-isomorphism. 
\end{tenumerate}
We call $H^0(\mathsf{B}(\PP))$ the
\emph{Koszul dual cooperad to $\PP$} and 
write it $\PP^\antishriek$.
\end{definition}

\textcolor{trinityblue}{Add equivalence with simpler
definition using twisting cochains.}
%\item The Koszul complex $\PP\circ_\tau \PP^\antishriek$ is acyclic.
%\item The Koszul complex $\PP^\antishriek\circ_\tau \PP$ is acyclic.
%\item The canonical map $\Omega \PP^\antishriek \longrightarrow \PP$
%is a quasi-isomorphism.

\subsection{Inhomogeneous duality}\label{sec:inhom}

Suppose that $\PP$ admits a quadratic linear presentation
defined by $\XX$ and $\RR\subseteq \XX\oplus \FF_\XX^{(2)}$.
There is a projection $q : \FF_\XX \longrightarrow \FF_\XX^{(2)}$
and we define $q\PP = \FF_{\XX}/(q\RR)$, and call it 
the \emph{quadratic operad associated to $\PP$}. We say
a quadratic-linear presentation is admissible if it
satisfies the following conditions:
\begin{tenumerate}
\item There are no superfluous generators in $\XX$,
that is, have that $\RR\cap \XX = 0$.
\item No new quadratic relations can be deduced from the
quadratic-linear relations, that is
\[ (\RR\circ_{(1)} \XX + \XX\circ_{(1)} \RR) \cap \FF_\XX^{(2)}
\subseteq \RR\cap \FF_\XX^{(2)}.\]
\end{tenumerate}
When condition (1) is satisfied, there is a map 
$f: q\RR \longrightarrow \XX$ such that $\RR = \{
r - f(r) : r\in q\RR \}$ is the graph of $f$. The operad
$\PP$ is filtered by weight, and we write $\mathrm{gr}(\PP)$
for the resulting operad. There is a surjection
$ q\PP \longrightarrow \mathrm{gr}(\PP)$
that is an isomorphism in weights $0$ and $1$, but
not necessarily in weight $2$.

The map $f: q\RR \longrightarrow \XX$ induces a map
$d_f : q\PP^\antishriek \longrightarrow \FF_{s\XX}^c$
which is the unique coderivation that correstricts on
$s\XX$ to the composition
\[ \PP^\antishriek  \xrightarrow{\pi} 
 s^2 q\RR \xrightarrow{s^{-1}f} s\XX.\] 
 Moreover, the following holds:
 \begin{tenumerate}
 \item The coderivation $d_f$ maps into $q\PP^\antishriek$
 if and only if 
 $(\RR\circ_{(1)} \XX + \XX\circ_{(1)} \RR) \cap \FF_\XX^{(2)} \subseteq q\RR$.
 \item If condition (2) above is satisfied, then the previous
 condition holds, as $\RR\cap \FF_\XX^{(2)}\subseteq q\RR$,
 and $d_f^2=0$.
 \end{tenumerate}
 
\begin{definition}
A quadratic-linear presentation of $\PP$ is inhomogeneous
Koszul if and only if it satisfies conditions (1) and (2)
and if the quadratic operad $q\PP$ is Koszul. In this case,
we call $(q\PP^\antishriek,d_f)$ the Koszul dual conilpotent
dg-cooperad
of $\PP$ and write it $\PP^\antishriek$.
\end{definition}

 \textbf{Warning!} Although not every operad 
is quadratic, every operad admits
a inhomogeneous Koszul presentation (Exercise 3.8.10
in \cite{LodVal}).  The problem is finding `economical
and useful' one.
Concretely, if we choose $\XX = \#\PP$ (the symmetric 
sequence underlying $\#\PP$) and 
quadratic-linear relations $\# \mu \circ_i \#\nu = \#(\mu\circ_i \nu)$ for every $\mu,\nu\in\PP$ and $i\in [1,\ari(\mu)]$,
then $\PP^\antishriek = \mathsf{B}(\PP)$ is the
bar construction of $\PP$ with its usual differential.

The main
theorem about inhomogeneous Koszul operads is the
following:

\begin{theorem}
If $\PP$ admits an inhomogenous Koszul presentation then
the canonical morphism
 $\Omega \PP^\antishriek \longrightarrow \PP$
determined by $s^{-1}\PP^\antishriek \twoheadrightarrow
\XX \hookrightarrow \PP$ is a quasi-isomorphism of
operads.
\end{theorem}

In general, one can show that the map $\Omega\mathsf{B}(\PP)
\longrightarrow \PP$ is a quasi-isomorphism: this is the
original approach of Ginzburg--Kapranov) who instead
define a duality functor $\mathsf{D}(\PP)$ along with a
quasi-isomorphism
$\mathsf{D}\mathsf{D}(\PP) \longrightarrow \PP$,
and define the Koszul dual operad to $\PP$ as
$H_\Delta(\mathsf{D}(\PP))$. The
theorem above gives us a more economical resolution of
$\PP$ in case it is quadratic-linear Koszul. Note that
the bar-cobar construction is, more or less, obtained
as the `least economical' resolution arising from the
corresponding `least economical' quadratic-linear 
presentation of an operad $\PP$. 

\section{Prototype of inhomogeneous duality}
\textbf{Goals.}
We will define the operad governing BV algebras 
and show it admits a small inhomogeneous Koszul 
presentation. We will compute the homology of the 
corresponding dg-cooperad, and
explain how it gives rise to the Gravity operad
of E. Getzler. At the same time, we will explain
how the homotopy quotient of $\mathsf{BV}$ by
the circle action is the hypercommutative
operad of Yu. I. Manin.
\subsection{Definition and computations}
\newcommand{\BV}{\mathsf{BV}}
The BV operad is an algebraic symmetric operad, which we
write $\BV$, generated by a binary commutative associative
operation $\mu$ that we will write $x_1x_2$ 
of degree zero and a unary square-zero operation
$\Delta$ of degree $-1$ that satisfy the following 
homogeneous $7$-term
relation:
\[ 
 \Delta(x_1x_2x_3) = x_1\Delta(x_2x_3)+
 x_2\Delta(x_1x_3) + x_3\Delta(x_1x_2)
  - x_1x_2\Delta(x_3) - x_1x_3\Delta(x_2)
   	- x_2x_3\Delta(x_1).\]
Batalin--Vilkovisky algebras appear in several
areas of mathematics:
\begin{tenumerate}
\item (Algebra) Vertex operator algebras, cohomology of Lie algebras,
bar construction of $A_\infty$-algebras.
\item (Algebraic geometry) Gromov--Witten invariants and moduli spaces of curves (quantum cohomology, Frobenius manifolds), chiral algebras (geometric Langlands program),
\item (Differential geometry) The sheaf of 
polyvector fields of an orientable (resp. Poisson or
Calabi--Yau) manifold, the differential 
forms of a manifold (Hodge decomposition 
in the Riemannian case), Lie algebroids,
Lagrangian (resp. coisotropic) intersections.
\item (Noncommutative geometry) 
The Hoschchild cohomology of a symmetric 
algebra  and the cyclic Deligne conjecture,
non-commutative differential operators.
\item (Algebraic topology)
2-fold loop spaces on topological spaces 
carrying an action of the circle,
topological conformal field theories, 
Riemann surfaces, string topology. 
\item  (Mathematical physics)
BV quantization (gauge theory), BRST
cohomology, string theory, topological field theory, Renormalization theory.

\end{tenumerate}

 One can express the seven term relation by saying
 that $[\Delta,\mu] = \beta$ is a derivation
 for $\mu$, where $[f,g]$
 is the operadic commutator (\`a la Gerstenhaber)
 defined by
 \[ [f,g] = \sum_{i=1}^{\ari(f)} f\circ_i g 
 	  -(-1)^{|f||g|} \sum_{j=1}^{\ari(g)} g\circ_j f. \]
 This suggests defining $\beta = [\Delta,m]$, and presenting
 the BV-operad by quadratic-linear relations:
 \[ [\mu,\mu] = 0 ,\quad \Delta^2 =0 , \quad
  	[\Delta,\mu] = \beta, \quad
  	\beta\circ_1 \mu = \mu \circ_2 \beta +
  	 	\mu\circ_1 \beta (23).\]
  This presentation satisfies condition (1), but it
  \emph{does not} satisfy condition (2): one can
  deduce that $\beta$ is a Lie bracket of degree $-1$
 and that $\Delta$ is a derivation for $\beta$
  from the first three equations. In other words, one
  can deduce that $(\Delta,\beta)$ defines the
  datum of a dg Lie algebra purely from the
  fact that $\Delta^2=0$ and that $\mu$ is associative. 
  
  \begin{lemma}
  The BV-operad admits a quadratic-linear presentation
  satisfying conditions (1) and (2) 
  given by generators $\mu,\beta,\Delta$ of arities
  $2$, $2$ and $1$ and degrees $0$, $-1$ and $-1$,
  respectively. The operation $\mu$ is associative 
  commutative, $\beta$ is a Lie bracket, $\Delta$
  squares to zero, and 
 \[	\mathsf{Leib}(\Delta,\mu) = \beta, \quad
  	\mathsf{Leib}(\beta,\mu)=0, \quad
  	\mathsf{Leib}(\Delta,\beta) = 0.\]
  	  \end{lemma}
  	  
  	  \begin{theorem}
  	  The quadratic operad $q\BV$ is Koszul.
  	  \end{theorem}
  	  
  	  \begin{proof}
  	  We will use the distributive law criterion of
  	  Markl, adapted to the case the operads in his
  	  result have unary operators. One can first
  	  show that $\BV(n)$ and its
  	  quadratic counterpart both have dimension $2^n n!$
  	  by a Gr\"obner basis argument,
  	  and then that $q\BV$ is obtained from
  	  a distributive law between the quadratic 
  	  operads $\mathsf{Ger}$ and 
  	  $\mathsf{D} = \kk[\Delta]/(\Delta^2)$, in light of the
  	  relations
  	  \[\Delta(x_1x_2) = x_1\Delta(x_2) + \Delta(x_1)x_2, 
  	  \quad 
  	   \Delta[x_1,x_2] = [x_1,\Delta x_2] + [\Delta x_1,x_2]. \]
  	  One can prove this again by a dimension counting argument
  	  using the result above. With this at hand, we
  	  observe that $\mathsf{Ger}$ is Koszul, as it is
  	  in turn obtained from a distributive law between
  	  $\mathsf{Com}$ and $\sus\mathsf{Lie}$, which are
  	  both Koszul, and that $\mathsf{D}$ is
  	  Koszul (as any algebra with trivial multiplications
  	  is). We conclude that $q\BV$ is Koszul, and that we
  	  have isomorphisms of symmetric sequences
  	  \[ q\BV \cong \Com \circ \sus\Lie\circ \mathsf{D},
  	    	  \quad q\BV^\antishriek \cong
  	    	  T^c(\delta) \circ \Com^c \circ \sus^{-1}\Lie^c,\]
  	  which will be useful to describe the dg-cooperad
  	  $\BV^{\antishriek}$.
  	  \end{proof}

\subsection{The differential}  	  
  	 Let us note that a generic element
  	 of $\Com^c \circ \sus^{-1}\Lie^c$ consists of a
  	 corolla decorated by Lie words on a Lie bracket
  	 of degree $1$. Any Lie word can always be
  	 written uniquely as a linear combination of
  	 words in the form
  	 $\ell = [x_1, x_{\sigma 2} , \ldots , x_{\sigma n} ]
     $
     where $\sigma\in S_n$ fixes $1$, and we adopt the
     right bracketing convention:
     \[ [y_1,y_2,\ldots,y_n ] = [y_1,[y_2,\ldots,y_n]].\]
     We call $x_1$ the `head' of $\ell$. 
     We will then write an element of $\Com^c \circ \sus^{-1}\Lie^c$ generically by
  	 \[ \ell_1\odot \cdots  \odot \ell_n \]
	 where $\ell_i$ is a Lie word supported on $\pi_i$,
	 with `head' $x_j$ with $j_i = \min\pi_i$ and such that
	 $\min \pi_1 < \cdots < \min \pi_n$. 
	 	 
	 \begin{theorem}
	 A generic element of $q\BV^\antishriek$ is of the form
	     \[x =\delta^k\otimes \ell_1\odot \cdots  \odot \ell_n \]
	  and the differential $d$ of $q\BV^\antishriek$ is
	  \[ dx = \sum_{i=1}^n(-1)^{\varepsilon_i} \delta^{k-1} \otimes \ell_1\odot
	   \cdots \odot \ell_i^{(1)} \odot \ell_i^{(2)}
	    \odot \cdots \odot \ell_n,
	  	 \]	  	 
	  	 where $\ell \longmapsto 
	  	 	\sum \ell^{(1)}\otimes \ell^{(2)}$
	  	 is the binary component of the decomposition map in
	  	 $\sus^{-1}\Lie^c$.	  	 
	  	 \end{theorem}
  	  
%The operad Gerstenhaber, Gravity, Hypercomm, PreLie,
%Perm, Zinbiel, Dias, Leib, NAP (Livernet) and the 
%Connes--Kreimer Hopf algebra of
%renormalization. 
\subsection{Exercises}

%\section{Further contents}

%\subsection{Iterated integrals}
%$C^*_{\mathrm{dR}}(\Omega X)$, Chen iterated integrals,
%work of Getzler--Jones for $\Omega^2 X$. 

\bibliographystyle{alpha}
\bibliography{bibliography}


\Addresses

\end{document}

	  	 
	  	
\subsection{The gravity operad and the hypercommutative operad}
  	  
  	 There is a unique square zero derivation $\Delta$ 
	 of $\mathsf{Ger}$
	 of degree $-1$ that sends the commutative
	 product $x_1x_2$ to the Lie bracket $[x_1,x_2]$,
	 which arises from the action of $S^1$ on $D_2$
	 by turning the whole configuration.
	 This derivation is acyclic, and $\mathsf{Grav}$
	 is, by definition, $\ker\Delta   =
	 \operatorname{im}\Delta  $.	 
	 	 
	 \begin{lemma}
	 Let $M$ be an $S^1$-space with a free action, and
	 let $\Delta : H_*(M) \longrightarrow H_{*+1}(M)$
	 be the square zero operator arising from the
	 fundamental class of $S^1$. Then
	 \[ \ker\Delta \cong \Sigma H_*(M/S^1) . \]
	 \end{lemma} 

	 The action of $S^1$ on little disks is free
	 (except in arity one) 
	 with quotient homotopy equivalent to the moduli
	 space of marked curves of genus zero, so 
	 	 	  \[ \ker\Delta =  
	 	 	  \Sigma H_*(D_2/S^1) = \Sigma H_*(\mathcal{M}_{0,\bullet+1}). \]
	 
	 \begin{theorem}
	 One can realize $\mathsf{Grav}$ as the suboperad
	 of $\mathsf{Ger}$ generated by
	 \[ \{x_1,\ldots,x_n\} =  
	 \sum_{i<j} \{x_i,x_j\}
	  x_1\cdots \widehat{x_i}\cdots \widehat{x_j} \cdots x_n, 
	  	\quad n\geqslant 2.\]
	 Similarly, one can realize it as the suboperad of	 
	 $\mathsf{BV}$ generated by 
	 \[ \{x_1,\ldots,x_n\} = 
	  \Delta(x_1\cdots x_n) - \sum_{i=1}^n x_1\cdots \Delta x_i
	   	\cdots x_n, \quad n\geqslant 2.\]
	   	We have that $\Sigma H_*(\mathcal{M}_{0,n+1}) \cong
	   	\mathsf{Grav}(n)$ for $n\geqslant 2$ where the
	   	operation $\{x_1,\ldots,x_n\}$ corresponds to the
	   	suspension of the generator of 
	   	$H_0(\mathcal{M}_{0,n+1})$.
	  \end{theorem}
	  
	 \begin{theorem}
	 The homology of the dg-cooperad 
	 $\mathsf{BV}^\antishriek$ (and hence
	 the bar homology of $\mathsf{BV}$) is
	 isomorphic to
	 \[ T(\delta) \oplus \sus^{-1}\mathsf{Grav}^*,\]
	 and the  
	 homotopy quotient of $\mathsf{BV}$ by
	 $\Delta$ is quasi-isomorphic to $\mathsf{Hycomm}$,
	 the Koszul dual of $\mathsf{Grav}$. These
	 two last operads are Koszul.
	 \end{theorem}



\subsection{The homotopy quotient}

\begin{definition} 
Let $f:\PP\to \mathcal Q$ be a morphism of operads, 
and factor $f$ into a cofibration
$i: \PP\longrightarrow \FF$ followed
by a trivial fibration $j:\FF\longrightarrow \QQ$.
We define the homotopy cofibre $C_f$ of $f$
as the quotient of $\FF$ by the ideal generated
by $\PP$. We write $\PP\sslash \QQ$ for
the homology of $C_f$.
\end{definition}

Note that $\PP\sslash \mathcal Q$ is independent 
of the choice of factorization of $f$.
Usually we can take $\FF$ of the form 
$(\PP \star \FF_\XX,d)$ where $\star$ is the
coproduct in the category of operads, 
so $C_f$ is
isomorphic to $(\FF_\XX,\bar{d})$. 
If $\alpha\in\PP$, we define the homotopy
quotient of $\PP$ as the homotopy cofibre of
the inclusion
$i : \PP_\alpha \longrightarrow \PP$
where $\PP_\alpha$ is the suboperad of $\PP$ 
generated by $\alpha$, and write it more
simply by $\PP\sslash \alpha$.  
In this case, to compute $\PP\sslash\alpha$, it suffices we compute a quasi-free
model $\FF = (\FF_\XX,d)$ of $\PP$ ---which in
particular we can assume contains an 
isomorphic copy of $\FF_\alpha$, and compute
the homology of the quotient $\FF/ (\FF_\alpha).$

This algebraic homotopy quotient is related
to the more geometrical homotopy quotient
with respect to the action of a topological
group $G$. Concretely, the exact functor 
$ t:\mathsf{Top} \longrightarrow \mathsf{Top}_G$
from the category of topological spaces to 
the category of topological spaces with a 
$G$-action, that sends a space $X$ to the
same space $tX$ with the trivial action, admits
a (non-Quillen-exact) left adjoint $F$ and
the homotopy quotient is obtained
as the left derived functor $\mathbb LF$ of
$F$, which we usually write $X\longmapsto X_{hG}$. At the level of homotopy categories, it
follows that we have an adjunction isomorphism
$[X_{hG},Y] \longrightarrow [X,tY]$.

Naturally, we may pass from the geometric to
the algebraic setting through the cohomology
functor, by considering the sphere $S^1$ whose
cohomology algebra coincides with 
$\Bbbk[\Delta]$. In this way, we can consider
the category $\mathsf{Op}_\Delta$ of dg-operads
under $\Bbbk[\Delta]$, and the exact functor
$t:\mathsf{Op} \longrightarrow \mathsf{Op}_\Delta$
that assigns an operad $\PP$ to the trivial
map $\Bbbk[\Delta]\to \PP$ sending $\Delta$ to $0$. The homotopy quotient functor $\PP\longmapsto \PP\sslash \Delta$ is the left adjoint, at the level of homotopy categories,
to $t$: we have an adjunction isomorphism
$[ \PP\sslash \Delta, \mathcal Q]
 	\longrightarrow [ \PP,t\mathcal Q]$. 
 	  
