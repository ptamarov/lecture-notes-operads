\section{Computing Gr\"obner bases}\label{lecture:GB2}

\textbf{Goal.} 
Define overlapping ambiguities and $S$-polynomials of overlapping
ambiguities. State and prove Bergman's Diamond Lemma. 
Give Buchberger's algorithm for computing Gr\"obner bases.

\subsection{$S$-polynomials}

As a motivating example, consider the operad
with a single binary operation $x_1x_2$ and no
symmetries, which is \emph{anti-associative}, that is,
\[
\underline{(x_1x_2)x_3}+ x_1(x_2x_3)  = 0.
\] 
At the same time, let us choose the usual \texttt{grapathpermlex}
order so that the leading term of the relation above
is the underlined one. The following computation shows that,
among the leading terms of elements in the ideal
generated by this relation, all trees of weight three appear,
which might be at first unexpected:
\begin{align*}
\underline{((x_1x_2)x_3))x_4} + (x_1(x_2x_3))x_4 -
\underline{((x_1x_2)x_3)x_4}- ((x_1 x_2)(x_3x_4)) &=  \\
 (x_1(x_2x_3))x_4 - ((x_1 x_2)(x_3x_4)) &= \\
 (x_1(x_2x_3))x_4 - ((x_1 x_2)x_3)x_4 &=  2(x_1(x_2x_3))x_4.
\end{align*}
If the characteristic is not two, then this term is non-zero
and, up to a sign, equal to the other four tree monomials
of weight three. Thus, it follows that this quadratic operad
is in fact three dimensional! Let us introduce the device 
that captures this behaviour.

\begin{definition}
Let $g_1,g_2$ be two shuffle polynomials over an alphabet $\XX$.
We say that the monomials $\leadm{g_1}$ and $\leadm{g_2}$ for
an overlap ambiguity if they have a \emph{small common multiple},
that is, there exists tree monomial $T$ properly divisible by  
$\leadm{g_1}$ and $\leadm{g_2}$, such that $T$ is the result
of merging these along an overlap. The element
\[
S_T(g_1,g_2) = 
\repl{T_1}{T}{g_1} - \repl{T_2}{T}{g_2}	
\]
is called the $S$-polynomial of this overlapping ambiguity.
\end{definition}

Revisiting the example above, we notice that the
polynomial (non-symmetric, in this case), has an overlapping
ambiguity with \emph{itself} ---a feature that already exists
in the world of non-commutative associative algebras--- and
that the resulting polynomial is reduced with respect to
its leading term. Thus, this overlapping ambiguity is
detecting a ``hidden'' leading term in the ideal of
leading terms of this relation. 

As a second example, we could have considered the usual
associative operad, where the plus sign above becomes a
minus. In this case, one can check that the resulting 
$S$-polynomial is zero: the computation ends with two
right combs cancelling each other. We will see later
that this has very important implications for the associative
operad: it shows we obtain a quadratic Gr\"obner basis,
and hence that this operad is \emph{Koszul}.
For the moment, let us consider a useful
definition.

\begin{definition} Let $\II$ be an ideal generated by some
set $\GG$, and for an element $f\in \II$, let us consider
a representation as a linear combination of insertions of
elements of $\GG$, of the form
\[
f = \sum_{i\in I} \lambda_i \repl{T_i}{S_i}{g_i}
\]
where $T_i = \leadm{g_i}$ for all $i\in I$. We call the element
$S = \max \{ S_i : i \in I\}$ the parameter of this representation.
\end{definition}

Note that any $S$-polynomial $S_T(g_1,g_2)$ admits a representation
of parameter $T$ (although the two obvious terms carrying
where this tree monomial appears will cancel): the parameter
of a representation of $f$ need not coincide with the
leading term of $f$. We say a representation of an
$S$-polynomial $S_T(g_1,g_2)$ is non-trivial if its parameter
is smaller than $T$.

\subsection{Diamond Lemma}
The following result is one of the most useful ways one can
verify a subset of an ideal is a Gr\"obner basis.

\begin{theorem}[Diamond Lemma]
For a self-reduced set of generators $\GG$ of an ideal $\II$,
the following statements are equivalent:
\begin{tenumerate}
\item The set $\GG$ is a Gr\"obner basis of $\II$.
\item Every $S$-polynomial reduces to zero modulo $\GG$.
\item Every $S$-polynomial admits a non-trivial representation.
\item Every $f\in \II$ admits a representation with
parameter $\leadm{f}$.
\end{tenumerate}
\end{theorem}

\begin{proof}
The implications $(4)\Longrightarrow (1)\Longrightarrow (2)
\Longrightarrow (3)$ are straightforward, and we leave them
as a guided Exercise~\ref{ex:DiamondLemma}.
The hardest part of the proof is
showing that (3) implies (4), which we go through in
detail here. To prove it, let us assume that (3) holds, but
that (4) does not. Then, there exists some $f\in\II$
so that every representation of $f$ has parameter 
larger than $\leadm{f}$, and let us pick a representation
giving a counterexample with the following properties:
\begin{tenumerate}
\item The parameter $T$ of the representation is minimum.
\item Among representations with parameter $T$, we choose one
with $k=\{ i\in I : S_i=T\}$ minimum.
\end{tenumerate}
We now consider those divisors $T_1,T_2,\ldots,T_k$
of $T$ appearing in the representation and focus on the last two
divisors $T_{k-1}$
and $T_k$: notice that $k>1$, for else there is not room for
cancellations to occur so that the leading monomial of $f$ is smaller than $T$. 

We claim that we can always arrange it so that
we obtain a new representation of $f$ that breaks either
the first or the second condition. To do this, we will consider
the relative position of the divisors $T_{k-1}$ and $T_k$
in $T$.

\emph{Case 1:} one divisor is contained in the other. In this
situation, this means that $g_k = g_{k-1}$ since $\GG$ is
reduced, which implies we can merge these two terms into one.
If their coefficients sum to zero, then either we get a 
representation with smaller parameter if $k=2$, or with
smaller $k$ in general, which cannot be.

\emph{Case 2:} the divisors are disjoint.  This situation
is similarly simple to the first case. In this case, we have
a well-defined bilinear operation of ``double insertion''
\[
\repl{T_k,T_{k-1}}{T}{-,-} \FF_\XX(\ari{T_{k-1}})\otimes \FF_\XX(\ari{T_{k}}) 
\longrightarrow \FF_\XX(\ari{T})
\]
which replaces occurrences of $T_k$ and $T_{k-1}$ in 
$T$ simultaneously.
Let us write $g_k  = \lead{g_k} + g_k'$, and similarly
with the other, and let us note that the following chain of
equalities hold:
\begin{align*}
\repl{T_k}{T}{g_k} 
	&= \repl{T_k,T_{k-1}}{T}{g_k,\lead{g_{k-1}}}   \\
	&= \repl{T_k,T_{k-1}}{T}{g_k,g_{k-1}-g_{k-1}'}	\\
		&=\repl{T_k,T_{k-1}}{T}{g_k,g_{k-1}}	- \repl{T_k,T_{k-1}}{T}{g_k,g_{k-1}'}	\\
		&=\repl{T_k,T_{k-1}}{T}{\lead{g_k}+g_k',g_{k-1}}	- \repl{T_k,T_{k-1}}{T}{g_k,g_{k-1}'}	\\
			&=\repl{T_k,T_{k-1}}{T}{\lead{g_k},g_{k-1}}
			+\repl{T_k,T_{k-1}}{T}{g_k',g_{k-1}}
				- \repl{T_k,T_{k-1}}{T}{g_k,g_{k-1}'}	\\
		&=\repl{T_{k-1}}{T}{g_{k-1}}
			+\repl{T_k,T_{k-1}}{T}{g_k',g_{k-1}}
				- \repl{T_k,T_{k-1}}{T}{g_k,g_{k-1}'}.
\end{align*}
Note that this is saying there is an ``obvious'' relation
between the two possible replacements we can make into $T$,
but otherwise is not saying anything more profound. The takeaway
is that we are able to replace the sum 
$\lambda_{k-1}\repl{T_{k-1}}{T}{g_{k-1}} + 
\lambda_k\repl{T_{k}}{T}{g_{k}} $ with the sum
\[ 
(\lambda_k + \lambda_{k-1})\repl{T_{k-1}}{T}{g_{k-1}} + 	
\lambda_{k}(\repl{T_k,T_{k-1}}{T}{g_k',g_{k-1}}
				- \repl{T_k,T_{k-1}}{T}{g_k,g_{k-1}'})
\]
where the last term can be expanded, by using Lemma~\ref{lemma:repl}
and Proposition~\ref{prop:repllead}, into sums of terms
with leading monomial smaller than $T$. Thus, either
the parameter of our representation decreases, which
happens if $k=2$ and the coefficients add up to zero,
or the parameter remains unmodified but $k$ decreases.

\emph{Case 3:} the divisors overlap. This is perhaps the
most computationally heavy of the three cases. Let us
assume that $T_k$ and $T_{k-1}$ have a small common
multiple $T'$, which of course is a divisor of $T$.
Using Lemma~\ref{lemma:repl}, we can write
for $i\in \{k-1,k\}$:
\[
\lambda_i \repl{T_i}{T}{g_i}  = 
\lambda_i \repl{T'}{T}{\repl{T_i}{T}{g_i}}
\]
and with this write the sum $\lambda_{k-1}\repl{T_{k-1}}{T}{g_{k-1}} +  \lambda_k\repl{T_{k}}{T}{g_{k}}$ as follows
\begin{align*}
 {} &= 
\repl{T'}{T}{\lambda_{k-1}\repl{T_{k-1}}{T'}{g_{k-1}} + 
  \lambda_k (\repl{T_{k-1}}{T'}{g_{k-1}} - S_{T'}(g_{k-1},g_k))} \\
  &=
  	\repl{T'}{T}{(\lambda_{k-1}+\lambda_k)
  		\repl{T_{k-1}}{T'}{g_{k-1}} 
  -   \lambda_k S_{T'}(g_{k-1},g_k)} \\
  &=
  	(\lambda_{k-1}+\lambda_k)
  		\repl{T_{k-1}}{T}{g_{k-1}} 
  -   \lambda_k \repl{T'}{T}{S_{T'}(g_{k-1},g_k)}.
\end{align*}
We have assumed that all $S$-polynomials admit at least
one non-trivial representation, so we conclude that either
we obtain a representation with a smaller parameter
(which happens if $k=2$ and the coefficients cancel)
or with the same parameter, but smaller $k$. To see
this, the reader should again make use of 
Lemma~\ref{lemma:repl}
and Proposition~\ref{prop:repllead} (Exercise~\ref{ex:fillin}).
\end{proof}

As an example, let us consider the shuffle operad 
$\mathsf{Lie}^\f$, given by a single binary generator
and the Jacobi relation. Choosing the \texttt{grapathpermlex}
that picks up the leading term as follows:
\[
\smallleftc{}{}{1}{2}{3}  \mathsymbol{1}{\leadsto}
	 \smallleftc{}{}{1}{3}{2}   \mathsymbol{1}{+} \smallrightc{}{}{1}{2}{3} 
\]
and we again
obtain the following left comb with four leaves as an overlapping ambiguity:
\[
{\smallLeftc{}{}{}{1}{2}{3}{4}}
\]
The resulting $S$-polynomial is the following sum of tree monomials:
\[ \smallLeftc{}{}{}{1}{3}{2}{4} \mathsymbol{1}{+} \smallLeftr{}{}{}{2}{3}{1}{4}
\mathsymbol{1}{-}  \smallLeftc{}{}{}{1}{2}{4}{3} \mathsymbol{1}{-} \smallfork{}{}{}{1}{2}{3}{4} \mathsymbol{1}{.}\]
Instead of continuing to rewrite all four terms (all of them
are divisible by our choice of leading term) which
will create a total of eight terms, let us do this with the
first two summands first, and then with the last two summands. We will
see that the end results are the same, which means that this $S$-polynomial
rewrites to zero through the Jacobi identity. The first two terms
are divisible by occurrences of a divisor
$\raisebox{-2.5 em}{\inlineleftc{}{}{1}{2}{4}}$
 and applying the Jacobi identity we obtain the polynomial
\[ \smallLeftc{}{}{}{1}{3}{4}{2} \mathsymbol{1}{+} 
\smallfork{}{}{}{1}{3}{2}{4}  \mathsymbol{1}{+} 
\smallfork{}{}{}{1}{4}{2}{3}  \mathsymbol{1}{+} 
\smallRightl{}{}{}{1}{2}{3}{4}  \mathsymbol{1}{.} 
\]
We observe that the middle terms are reduced, and that only the last term and
first term are not, since they have corresponding divisors
\[
\smallleftc{}{}{2}{3}{4}\mathsymbol{1}{,}\quad \smallleftc{}{}{1}{3}{4}
\mathsymbol{1}{.}
\]
Applying the Jacobi identity one more time, we arrive at the six-term
reduced polynomial
\begin{align*}	
\smallLeftc{}{}{}{1}{4}{3}{2} &\mathsymbol{1}{+} 
\smallLeftr{}{}{}{3}{4}{1}{2}  \mathsymbol{1}{+} 
\smallRightl{}{}{}{1}{2}{4}{3} \mathsymbol{1}{+} 
\smallRightc{}{}{}{1}{2}{3}{4} \\
&\mathsymbol{1}{+} \smallfork{}{}{}{1}{3}{2}{4}  \mathsymbol{1}{+}
\smallfork{}{}{}{1}{4}{2}{3} 
 \mathsymbol{1}{.}
 \end{align*}
 
We leave it as Exercise~\ref{ex:SpolyLie} to carry out the
reduction algorithm on the other two terms (the ones carrying
minus signs): if all goes well, the reader will arrive at the same
six term reduced polynomial as the one above, thus confirming
that the Jacobi identity does provide us with a one element
Gr\"obner basis for $\mathsf{Lie}^\f$. We will see later
that this guarantees that our presentation of $\mathsf{Com}^\f$
is also a Gr\"obner basis, but for the reverse order.

\subsection{Buchberger's Algorithm}

The Diamond Lemma gives us a recipe to (attempt to) fix any 
generating set to a Gr\"obner basis by uncovering hidden
leading terms through $S$-polynomials: borrowing terminology
from the original theory for commutative rings, we call it
the Buchberger Algorithm. We present it here in the form
of pseudo-code, and remark that the algorithm may not
terminate (as some ideals may admite infinite reduced 
Gr\"obner bases for certain choices of monomial orders).

 \begin{algorithm}
\caption{Buchberger's Algorithm}\label{algo:self-reduce}
\textsc{Input:} a set of generators $\mathcal V$ for an ideal $\II$
in a free shuffle operad.

\textsc{Output:} the reduced Gr\"obner basis of $\II$,
if it is finite.

\begin{adjustwidth}{0 cm}{2 cm}
\begin{algorithmic}[1]
\Procedure{BuchbergerAlg}{$\texttt{Polynomials}$}
	\State $\texttt{newSPolynomials} \gets \texttt{True}$
	\State $\texttt{ToComplete} \gets \texttt{Polynomials}$
	\While {$\texttt{newSPolynomials}$} 
		\State $\texttt{ToComplete} \gets 
					\textsc{SelfReduce}(\texttt{ToComplete})$
		\State $\texttt{ToComplete} 
					\gets \textsc{Sort}(\texttt{ToComplete},
								\texttt{grapathpermlex})$
		\State $\texttt{SPolynomials} \gets \varnothing$
		\State $\texttt{newSPolynomials} \gets \texttt{False}$
		\For {$g_1,g_2 \in \texttt{ToComplete} $}
			\State $\texttt{LocalOverlap}\gets
			 \textsc{Overlaps}(g_1,g_2)$
			 \State $\texttt{DoneOverlaps} \gets \varnothing$
			\While {$\texttt{LocalOverlaps}\neq \varnothing$}
			\State $\texttt{Overlap} \gets 	
				\texttt{LocalOverlaps}.\textsc{Pop}()$
			\State $\texttt{DoneOverlaps}.\textsc{Add}(\texttt{Overlap})$
			\State $\texttt{LocalSPoly} \gets 
				\textsc{SPoly}(g_1,g_2,\texttt{Overlap}) $
			\State $\texttt{RemainderSPoly} \gets \textsc{Reduce}(
			\texttt{LocalSPoly},\texttt{ToComplete})$
			\If {$\texttt{RemainderSPoly}\notin \texttt{SPolynomials}$}
				\State $\texttt{newSPolynomials} 
					\gets \texttt{True}$
				\State $\texttt{SPolynomials}.
					\textsc{Add}(\texttt{RemainderSPoly})$
				\EndIf
				\EndWhile
			\EndFor
			\State  
				$\texttt{ToComplete}.\textsc{Union}(\texttt{SPolynomials})$
\EndWhile
\State \Return $\texttt{ToComplete}$
\EndProcedure
\end{algorithmic}
\end{adjustwidth}
\end{algorithm}

\subsection{Exercises}

\begin{question}\label{ex:DiamondLemma}
Prove the implications $(4)\Longrightarrow (1) \Longrightarrow
(2)\Longrightarrow (3)$ in the Diamond Lemma.
\end{question}

\begin{question}\label{ex:fillin}
Fill in the missing details in the proof the implicatoin
$(3)\Longrightarrow (4)$ of the Diamond Lemma, which require
the use of Lemma~\ref{lemma:repl}
and Proposition~\ref{prop:repllead}. 
\end{question}

\begin{question}\label{ex:SpolyLie}
Complete the reduction of the $S$-polynomial coming from the
self-overlap of the Jacobi identity in the presentation of
the operad $\mathsf{Lie}^\f$ with respect to the 
\texttt{grapathpermlex} order of tree monomials.
\end{question}

\begin{question} 
Redo the previous exercise using the Haskell Operad Calculator.
\end{question}

\begin{question} The dendriform operad is a ns quadratic operad
generated by two operations $a: (x_1,x_2)\mapsto x_1\prec x_2$ and $
b: (x_1,x_2) \mapsto x_1\succ x_2$ subject to the following
quadratic relations:
\begin{align*}
(x_1\succ x_2)\prec x_3 &= x_1 \succ (x_2\prec x_3) \\
(x_1  \prec x_2) \prec x_3 &=x_1 \prec (x_2 \prec x_3 +x_2 \succ x_3),\\
x_1 \succ (x_2 \succ x_3) &= (x_1 \prec x_2 +x_1 \succ x_2) \succ x_3  .
\end{align*}
Consider the \texttt{grapathlex} order for the total order of
generators given $a < b$.
\begin{tenumerate}
\item Determine the leading terms of the three relations, and
check that this set of relations is self-reduced.
\item Determine all overlapping ambiguities between these
leading terms.
\item Choose one overlapping ambiguity and check that the
corresponding $S$-polynomial rewrites to zero.
\item Use the Haskell Operad Calculator to prove that the
remaining $S$-polynomials reduce to zero.
\item Conclude that
the quadratic relations above constitute a Gr\"obner basis for
$\mathsf{Dend}$ and this choice of order.
\item Now consider the order $b < a$. What happens?
\end{tenumerate}
\end{question}
\begin{figure}
\begin{verbatim}
# Operadic Buchberger Configuration File
# --------------------------------------

# Actions
actions: normalise 

# Time limit (seconds)
time limit: 

# Count limit
count limit: 

# Output options	
output:  new final  

#Field
field: 

# Operad type
operad type: asymmetric unsigned

# Measure 
measure: deglex perm

# Signature
signature:  a(2) b(2)

# Theory 
a( b(* *) * ) - b( * a( * * ) )
b( b( * * ) * ) - b( * b( * * ) ) + b( a ( * * ) * )
a( a( * * ) * ) - a( * a( * * ) ) - a( * b ( * * ) )
\end{verbatim}
\caption{A minimal configuration file to compute a Gr\"obner basis of
$\mathsf{Dend}$.}
\end{figure}

\begin{figure}
\begin{verbatim}
*Main> main

Configuration:

actions:        normalise 
count limit:    
arity limit:    6
time limit:     none
output:         final theory
field:          rationals
operad type:    unsigned asymmetric operad
measure:        degree-lexicographic permutation 
signature:      a(2) b(2)
theory:

  a(b(* *) *)  -  b(* a(* *))
  b(b(* *) *)  -  b(* b(* *))  +  b(a(* *) *)
  a(a(* *) *)  -  a(* a(* *))  -  a(* b(* *))

Arity: 4   Stable rewrite rules: 3   
Current critical pairs: 4   Queued critical pairs: 0

No new rewrite rules

Success! Complete theory: 

  a(b(* *) *)  ->  b(* a(* *))
  a(a(* *) *)  ->  a(* b(* *))  +  a(* a(* *))
  b(b(* *) *)  ->  - b(a(* *) *)  +  b(* b(* *))
\end{verbatim}
\caption{The result of running the previous code in
the Haskell Operad Calculator, showing that the
given relations (theory) are a Gr\"obner basis for
the \texttt{grapathlex} order with $a<b$.}
\end{figure}


\begin{figure}

\begin{verbatim}

Configuration:

actions:        normalise count 
count limit:    6
arity limit:    6
time limit:     none
output:         final theory
field:          rationals
operad type:    unsigned shuffle operad
signature:      x(2) y(2)
measure:        quantum(x,y) degree-lexicographic permutation 
theory:

  y(y(1 2) 3)  -  y(y(1 3) 2)  -  y(1 y(2 3))
  y(x(1 2) 3)  -  x(1 y(2 3))  -  x(y(1 3) 2)
  y(1 x(2 3))  -  x(y(1 2) 3)  -  x(y(1 3) 2)
  y(x(1 3) 2)  +  x(1 y(2 3))  -  x(y(1 2) 3)
  x(x(1 2) 3)  -  x(1 x(2 3))
  x(x(1 3) 2)  -  x(1 x(2 3))

Arity: 4   Stable rewrite rules: 6   
Current critical pairs: 24   
Queued critical pairs: 0

Success! Complete theory: 

  x(x(1 3) 2)  ->  x(x(1 2) 3)
  x(1 x(2 3))  ->  x(x(1 2) 3)
  y(x(1 3) 2)  ->  x(y(1 2) 3)  -  x(1 y(2 3))
  y(1 x(2 3))  ->  x(y(1 2) 3)  +  x(y(1 3) 2)
  y(x(1 2) 3)  ->  x(y(1 3) 2)  +  x(1 y(2 3))
  y(y(1 2) 3)  ->  y(y(1 3) 2)  +  y(1 y(2 3))

Counting normal forms:

  arity | normal forms
     3  |       6
     4  |      24
     5  |     120
     6  |     720
\end{verbatim}
\caption{Verification that the associative, Jacobi and 
Leibniz rule give a Gr\"obner basis for the
Poisson operad with respect to the $\mathsf{QM}$ order.}
\end{figure}