
\section{Technical homological results}

\subsection{Comparison theorem}

\subsection{Big Koszul complexes are acylic}

\begin{theorem}\label{thm:augacyclic}
Let $\PP$ be an operad and let $\CC$ be a conilpotent cooperad. All four
Koszul complexes associated to the canonical twisting cochains
$\pi$ and $\iota$ are all acyclic. 
\end{theorem}
\subsection{Fundamental theorem}

\begin{theorem}\label{thm:fundamental}
Let $\PP$ (resp. $\CC$) be a connected weight graded dg operad
(resp. conilpotent cooperad), and let $\tau : \CC\longrightarrow
\PP$ be a twisting morphism. Then the following four assertions are
equivalent:
\begin{tenumerate}
\item The right Koszul complex $\CC\circ_\tau \PP$ is acyclic.
\item The left Koszul complex $\PP\circ_\tau \CC$ is acyclic.
\item The map $f_\tau : \CC\longrightarrow \B{\PP}$ is a quasi-isomorphism.
\item The map $f^\tau : \Omega(\CC)\longrightarrow \PP$ is a quasi-isomorphism.
\end{tenumerate}
\end{theorem}

\section{Deformation theory}

\begin{proposition}
The assignment
$
d^r : \hom_\Sigma(\CC,\PP)
 	\longrightarrow 
 	\operatorname{Der}(\CC\circ_\Sigma \PP)
$ that sends 
a morphism $\tau : \CC\longrightarrow \PP$ of symmetric
sequences to the derivation $d_\tau^r : \CC\circ\PP \longrightarrow
\CC\circ \PP$ is a morphism of dg Lie algebras.
\end{proposition}

\hacer{Write the proof.}


\section{Further topics}

\subsection{Quillen homology of operads}

\hacer{Work of Dotsenko--Khoroshkin.}

\begin{lemma}
\hacer{Koszul iff quadratic model.}  \end{lemma}
  
  \begin{definition}
  \hacer{Inclusion exclusion operad.}
  \end{definition}
  
  \begin{definition}
    \hacer{Inclusion exclusion adapted to monomial relations.}
  \end{definition}
  
  \begin{proposition}
  \hacer{Homological perturbation. Another
  proof of Hoffbeck's theorem.}
  \end{proposition}
\subsection{Algebras over operads}
Operads are important not in and of themselves 
but through their representations, more commonly
called \emph{algebras over operads}. In fact,
one can usually `create' an operad by declaring
what kind of algebras it governs. If the
algebra has certain operations of certain
arities, these define the generators of the operad,
and the relations these operators must satisfy 
give us the relations presenting our operad. 

\begin{definition}
A $\PP$-algebra structure on a vector
space $V$ is the datum of a map of operads 
$\PP \longrightarrow \End_V$.
\end{definition}

Alternatively, one can consider the situation
when $\otimes$ is closed and has a right
adjoint $\hom$ (the internal hom) so that
what we want are maps
\[\label{eq:Palgebramap}
\gamma_{V,n} : 
\PP(n)\otimes_{S_n} V^{\otimes n} \longrightarrow V \]
declaring how each $\mu \in\PP(n)$ acts as an operation
$\mu : V^{\otimes n} \longrightarrow V$. 
It follows that a
 $\PP$-algebra structure on $V$ is the same as the datum
of maps as in \ref{eq:Palgebramap} that satisfy the following
conditions:

\begin{tenumerate}
\item \emph{Associativity:} let $\nu\in \PP(n)$ and
$\nu_i \in \PP(k_i)$ for $i\in [n]$, and pick
$w_i \in V^{\otimes k_i}$. Set $v_i = \gamma_{V,k_i}(\nu_i,w_i) \in V$
and $\mu = \gamma_\PP(\nu;\nu_1,\ldots,\nu_n)$. Then 
\[ \gamma_{V,k_1+\cdots+k_n}(\mu; w_1,\ldots,w_n) = 
	\gamma_{V,n}(\nu ;v_1,\ldots, v_n).\]
\item \emph{Equivariance:} for $\nu\in\PP(n)$, $v_1\otimes\cdots \otimes v_n \in V^{\otimes n}$
and $\sigma\in S_n$, we have that
\[ \gamma_{V,n}(\nu \sigma, v_{\sigma 1} \otimes \cdots \otimes v_{\sigma n}) = \gamma_{V,n}(\nu; v_1\otimes\cdots \otimes v_n). \]
\item \emph{Unitality:} if $1\in\PP(1)$ is the unit, then
$\gamma_{V,1}(1,v) = v$.  
\end{tenumerate}


\hacer{Example: recognition
principle. Gerstenhaber algebras and Hochschild
complex. }

\subsection{Homotopy algebras}

\pagebreak


\section{Classical theory}

%\subsection{Linear self-reduction}
%Let $M$ be a matrix with $m$ rows
%and $n$ columns. The following is the
%usual algorithm to put $M$ in row reduced
%echelon form, written as pseudo-code:
%\begin{algorithm}
%\caption{Linear self-reduction algorithm}\label{algo:gauss}
%\begin{adjustwidth}{2 cm}{2 cm}
%\begin{algorithmic}[1]
%\Procedure{LinearSelfReduce}{$\texttt{Matrix}$} 
%\State $\texttt{lead} \gets 0$
%\State $\texttt{A} \gets \texttt{Matrix}$
%\State $\texttt{rows} \gets \textsc{Width}(\texttt{A})$
%\State $\texttt{cols} \gets \textsc{Height}(\texttt{A})$
%	\For {$r\in [0,\texttt{ rows})$}
% 			\If {\texttt{cols} $\leqslant$\texttt{ lead}}
% 			\State \textbf{stop}
% 			\EndIf
% 	\State $i\gets r$
% 	\While {$\texttt{A}[i,\texttt{lead}]=0$}
% 			\State $i\gets i+1$
% 			 \If {\texttt{rows} $=i$}
%  					\State $i\gets r$
%  					\State $\texttt{lead} \gets \texttt{lead}+1$
%  					\If {\texttt{cols} $=$\texttt{ lead }}
%   						\State \textbf{stop}
%   					\EndIf
%   			\EndIf
%   	\EndWhile
%   \If {$i\neq r$}
%   		\State \textbf{swap} $\texttt{A}[r,:]$ and $\texttt{A}i,:]$
%   		\State $\texttt{A}[r,:]
%   			\gets \texttt{A}[r,:]/\texttt{A}[r,\texttt{lead}]$
%   		\For {$i \in [0,\texttt{rows})$}
%   			 \If {$i\neq r$}
%   			 \State $\texttt{A}[i,:] \gets 
%   			 	\texttt{A}[i,:] - 
%   			 		\texttt{A}[i, \texttt{lead}]\texttt{A}[r,:] $
%   			\EndIf
%   			\EndFor
%   \State $\texttt{lead} \gets \texttt{lead}+1$
%   \EndIf
%\EndFor
%\EndProcedure
%\end{algorithmic}
%\end{adjustwidth}
%\end{algorithm}

\subsection{Non-commutative Gr\"obner bases}
\hacer{Explain the analogous results for associative
algebras, and the case of the Steenrod algebra.}
