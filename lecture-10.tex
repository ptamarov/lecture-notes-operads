\section{Koszul complexes}\label{lecture:KD2}

\textbf{Goals.} Define twisting morphisms, unravel the
bar-cobar adjunction. Give the definition of the
Koszul complexes associated to a quadratic operad,
and give a second definition of Koszulness.

\subsection{Twisting morphisms}

We begin with some more operadic algebra, this time focusing
on certain variants of the circle product that will help us
state and prove some (perhaps technical) results about
operads, conilpotent cooperads, and their (co)bar
constructions. 

\medskip

\textbf{Infinitesimal composites.}
Since we will use them often, let us define two flavours of
``infinitesimal composites''  between two symmetric sequences $\XX$ and $\YY$.
First, for a third sequence $\YY'$, let us write $\XX\circ (\YY;\YY')$
for the subfunctor of $\XX\circ (\YY\oplus \YY')$ which is linear
in $\YY'$. In other words,  we consider corollas whose root 
vertex is decorated by
$\XX$, and all whose leaves are  decorated by an element 
of $\YY$ except for one
which is decorated by an element of~$\YY'$. This construction
enjoys some sort of ``mixed functoriality'' with the usual
composition product:

\begin{definition}
Let $f:\XX \longrightarrow \XX'$ and $g: \YY\longrightarrow \YY'$
be maps of symmetric sequences. We define the infinitesimal
composite $f\circ' g$ to be the map
\[ f\circ' g: \XX\circ \YY \longrightarrow \XX'\circ (\YY;\YY') \]
such that
$(f\circ' g)(x;y_1,\ldots,y_k) = 
\sum_{i=1}^k (f(x);y_1,\ldots,g(y_i),\ldots,y_k)$.
\end{definition}

As usual, Koszul signs will appear if our sequences are dg.
We write $\XX\circ_{(1)} \YY$ for $\XX\circ (\kk,\YY)$. In other
words, we consider corollas whose root vertex is decorated by
$\XX$, and all whose leaves are ``empty'', except for one
which is decorated by an element of $\YY$. If $f$ and $g$
are as above, there is a map
\[
f\circ_{(1)} g : \XX\circ_{(1)} \YY \longrightarrow \XX'\circ_{(1)} \YY'
\]
such that $(f\circ_{(1)} g )(x;1,\ldots,1,y,1,\ldots,1) = 
 (f(x);1,\ldots,1,g(y),1,\ldots,1)$.
As we suggest,
these two constructions are better behaved than
composite product when it comes to linearity:

\begin{lemma}
For any four symmetric sequences $\XX,\YY,\YY_1,\YY_2$,
we have a natural isomorphism
\[ \XX\circ (\YY;\YY_1\oplus \YY_2) \longrightarrow 
				\XX\circ (\YY;\YY_1) \oplus \XX\circ (\YY;\YY_2).\]
In particular, if $\YY=\kk$, we have a natural isomorphism				
\[ \XX\circ_{(1)} (\YY_1\oplus \YY_2) \longrightarrow 
				(\XX\circ_{(1)} \YY_1) \oplus (\XX\circ_{(1)} \YY_2),\]
so that the functor $-\circ_{(1)}-$ is bilinear.
\end{lemma}

It is useful to remark that although $f\circ' g$ and $f\circ_{(1)} g$
look similar, they have different (co)domains and they act in different
ways on summands: while the former outputs a sum of terms, the latter
outputs one term only.

\medskip

\textbf{Infinitesimal (de)compositions.}
Let $\PP$ be a dg operad and let $\CC$ be a conilpotent
dg cooperad, and let us assume (once and for all, throughout
this section) that they are both (co)augmented. In this
situation, we can consider morphisms of dg cooperads
$\CC \longrightarrow \B{\PP}$, and we claim these
can be described purely in terms of linear maps
$\tau :\CC \longrightarrow \PP$ satisfying some conditions.

Since $\PP$ is an operad, we can then consider its 
\emph{infinitesimal composition map} $\gamma_{(1)} :\PP \circ_{(1)} 
	\PP\longrightarrow \PP$ which encodes its partial
	compositions. Similarly, we can consider the 
	infinitesimal decomposition map $\Delta_{(1)}:
\CC\longrightarrow  \CC\circ_{(1)} \CC$ obtained
by projecting the decomposition map of $\CC$ onto 
$\CC\circ_{(1)} \CC$.
\begin{figure}[h]
\begin{tikzpicture}
	\begin{pgfonlayer}{main}
		\node [style=leaf] (0) at (-0.5, 1) {};
		\node [circle,draw=black, fill=pagecolor, inner sep=2 pt,minimum size=6 pt] (1) at (0, 0) { $c$};
		\node [style=leaf] (2) at (-1.5, 1) {};
		\node [style=leaf] (3) at (0.5, 1) {};
		\node [style=leaf] (4) at (1.5, 1) {};
		\node [style=leaf] (12) at (0, -1) {};
	\end{pgfonlayer}
	\begin{pgfonlayer}{bg}
		\draw (2.center) to (1.center);
		\draw (1.center) to (0.center);
		\draw (3.center) to (1.center);
		\draw (1.center) to (4.center);
		\draw (1.center) to (12.center);
	\end{pgfonlayer}
\end{tikzpicture}
\mathsymbol{1.5}{\leadsto}
\begin{tikzpicture}
	\begin{pgfonlayer}{main}
		\node [style=leaf] (0) at (0, -1) {};
		\node [style=inner] (1) at (0, 0) {\scriptsize $ f(c_{(1)})$};
		\node [style=leaf] (2) at (-1.25, 1) {};
		\node [style=leaf] (3) at (0, 1) {};
		\node [style=inner] (4) at (1.25, 1) {\scriptsize $ g(c_{(2)})$};
		\node [style=leaf] (5) at (0.75, 2) {};
		\node [style=leaf] (6) at (1.75, 2) {};
	\end{pgfonlayer}
	\begin{pgfonlayer}{bg}
		\draw (2.center) to (1.center);
		\draw (1.center) to (3.center);
		\draw (5.center) to (4.center);
		\draw (4.center) to (6.center);
		\draw (4.center) to (1.center);
		\draw (1.center) to (0.center);
	\end{pgfonlayer}
\end{tikzpicture}
	\mathsymbol{1.5}{\leadsto}
	\begin{tikzpicture}
\begin{pgfonlayer}{main}
		\node [style=leaf] (0) at (-0.5, 1) {};
		\node [circle,draw=black, fill=pagecolor, inner sep=2 pt,minimum size=6 pt] (1) at (0, 0) {$p$};
		\node [style=leaf] (2) at (-1.5, 1) {};
		\node [style=leaf] (3) at (0.5, 1) {};
		\node [style=leaf] (4) at (1.5, 1) {};
		\node [style=leaf] (12) at (0, -1) {};
	\end{pgfonlayer}
	\begin{pgfonlayer}{bg}
		\draw (2.center) to (1.center);
		\draw (1.center) to (0.center);
		\draw (3.center) to (1.center);
		\draw (1.center) to (4.center);
		\draw (1.center) to (12.center);
	\end{pgfonlayer}
	\end{tikzpicture}
\caption{A schematic version of the star product: we use ``Sweedler notation'' for
the infinitesimal decomposition of $\CC$ and write $p$ a generic
summand $f(c_{(1)})\circ_i g(c_{(2)})$ appearing in the final result.}
\label{fig:starprod}
\end{figure}
\begin{definition}
Let $f,g\in \hom_\Sigma(\CC,\PP)$ be maps of symmetric
sequences. We define their star product $f\star g$ 
as the composition
$\gamma_{(1)} \circ (f\circ_{(1)} g)\circ \Delta_{(1)}$,
as depicted in Figure~\ref{fig:starprod}. We give 
$\hom_\Sigma(\CC,\PP)$ the usual differential
\[
\partial(f) = d_\PP \circ f  - (-1)^{|f|} f\circ d_\CC.
\]
\end{definition}

One can show that $\hom_\Sigma(\CC,\PP)$ along with
the star product $-\star -$ becomes a pre-Lie algebra,
 which is dg if $\CC$ or $\PP$ are. By anti-symmetrizing
 $-\star -$, we thus obtain a dg Lie algebra structure
 on $\hom_\Sigma(\CC,\PP)$. As in any dg Lie algebra,
 the degree $-1$ elements satisfying a particular equation
 play an important role and, for historical reasons,
  we give them a name here 
 different from the usual ``Maurer--Cartan element'':
 
\begin{definition}
We write $\Tw(\CC,\PP)$ for the set of degree $-1$ maps
$\tau : \CC\longrightarrow \PP$
such that $\tau\eta =0$, $\varepsilon\tau = 0$
and 
\[
\partial(\tau) + \tau\star \tau = 0,
\]
and call it the set of \emph{twisting morphisms} from
$\CC$ to $\PP$. We call the equation in the display the
\emph{Maurer--Cartan equation} for $\tau$.
\end{definition}


\subsection{Adjunction}
The bar construction admits a dual construction for
coaugmented conilpotent cooperads $\CC$, which we call
the cobar construction and write $\Omega(\CC)$. These
two functors fit into a diagram
\[
 \Omega(-)  : \mathsf{Coop}^c   \adjoint
  \mathsf{Op} :\B{-}
\]
from the category of augmented operads (which we may allow to be
dg, see Exercise~\ref{ex:dgbar}) to the category of
\emph{conilpotent} augmented dg cooperads. The following
theorem can be considered as one of the incarnations of Koszul
duality between operads and conilpotent cooperads.

\begin{theorem}\label{thm:adjunction}
The bar and cobar functors form an adjoint pair
$ \Omega(-)  : \mathsf{Coop}^c   \adjoint
  \mathsf{Op} :\B{-}$.
More precisely, 
for every operad $\PP$ and every conilpotent cooperad
$\CC$, the following are in natural bijection:
\begin{tenumerate}
\item Maps of conilpotent dg cooperads $\CC \longrightarrow \B{\PP}$.
\item Maps of dg operads $\Omega(\CC) \longrightarrow\PP$.
\item Twisting morphisms $\CC \longrightarrow \PP$.
\end{tenumerate}
\end{theorem}

\begin{proof}
Let us consider a map $f : \CC \longrightarrow \B{\PP}$ of cooperads
with $\CC$ conilpotent. Forgetting for the moment that the codomain
has a differential, we notice that there exists a unique map of
symmetric collections $\CC \longrightarrow s\overline{\PP}$
that induces it. Equivalently, there is a degree $-1$ map
$\tau : \CC \longrightarrow \PP$ such that $\varepsilon\tau = 0$
and $\tau\eta = 0$ (this second equality follows since $\B{\PP}$ is
augmented and connected)
which induces $f$. The fact that $f$ commutes with the differential of
$\B{\PP}$ is equivalent to the Maurer--Cartan equation
\[
\partial \tau + \tau\star\tau = 0.
\]
The degree $-1$ map $\tau : \CC\longrightarrow \PP$ in turn can
be turned into a degree zero map $s^{-1}\overline{\CC}\longrightarrow
\PP$ which, by the universal property of the free operad construction
induces a map $\Omega(\CC) \longrightarrow \PP$, which in fact commutes with
the differential, as one can check this is also equivalent
to the equation above. We conclude that the two hom-sets in
the statement of this theorem are in bijection with the set
of symmetric sequence morphisms $\tau : \CC\longrightarrow \PP$
that satisfy the above Maurer--Cartan equation and
such that both $\tau\eta$ and $\varepsilon\tau$ vanish,
which are precisely the twisting morphisms we have defined.
\end{proof}

Since it will be useful for us, let us make the bijection
in the theorem above more explicit. To do this, let us consider
the map $\pi : \B{\PP} \longrightarrow \PP$ obtained
as the projection onto $s\overline{\PP}$ and the degree
$-1$ inclusion into $\PP$. Then the bijection
\[
\hom_{\mathsf{Coop}^c}(\CC,\B{\PP})
 	\longrightarrow
 	 \mathsf{Tw}(\CC,\PP)
\]
is given by post-composition with $\pi$, while the
inverse is given by extending the
resulting degree zero map 
$\CC\longrightarrow \PP$ to a unique coderivation
$\CC\longrightarrow \B{\PP}$. We call $\pi$
the universal twisting cochain (it is one, as per
Exercise~\ref{ex:universaltw}), and observe that it
corresponds to the identity map of $\B{\PP}$ under
the bijection above. 

Dually, there is a degree $-1$ inclusion $\iota:
\CC\longrightarrow \Omega(\CC)$ which is also a 
twisting cochain ---the universal twisting cochain
for the conilpotent cooperad $\CC$--- and the bijection
\[
\hom_{\mathsf{Op}}(\Omega(\CC),\PP)
 	\longrightarrow
 	 \mathsf{Tw}(\CC,\PP)
\]
is given by pre-composition with $\iota$. To summarize,
for any map $\CC\longrightarrow \PP$ we have a commutative
diagram
\[
\begin{tikzcd}
{} & \Omega(\CC) \arrow[dr] & {} \\
\CC \arrow[ur,"\iota"]
	\arrow[dr]
	\arrow[rr,"\tau"] & {} & \PP \\
{} & \B{\PP} \arrow[ur,"\pi",swap] & {} 
\end{tikzcd}
\]
where the diagonal maps are the universal twisting cochains, 
the anti-diagonal maps are maps of dg (co)operads and
the only horizontal arrow is the corresponding twisting cochain.
In other words, every twisting cochain $\CC\longrightarrow \PP$
factors uniquely as the composition of the universal twisting
cochain $\iota$ followed by a morphism of dg operads, and
as the composition of a morphism of dg conilpotent cooperads
followed by the universal twisting cochain $\pi$.

\subsection{Koszul complexes}

\textbf{Left and right modules over operads.} Since it will be useful
later, let us begin by noticing that operads admit representations other
than their algebras (which we introduce in the Appendix). We will
use this to define certain objects that are central to our lectures:
the Koszul complexes of twisting cochains.

\begin{definition}
Let $\PP$ be an operad. A right $\PP$-module is a symmetric
sequence $\MM$ along with a map $\rho :\MM\circ\PP\longrightarrow \MM$
that is unital and associative for the unit and the composition
map of $\PP$, that is, the following diagrams commute:
\[
\begin{tikzcd}
(\MM\circ \PP)\circ \PP 
	\arrow[r,"\rho\circ 1"]
	\arrow[dd,"\alpha",swap] 				& \MM\circ \PP \arrow[dr,"\rho"] \\
	{} 						& {} 			& \MM\\
\MM\circ (\PP \circ \PP) \arrow[r,"1\circ\gamma"] & 
	\MM \circ \PP \arrow[ur,"\rho",swap]
\end{tikzcd}
\qquad
\begin{tikzcd}
\MM\circ \kk \arrow[dd,"1\circ\eta",swap] \arrow[dr,"\cong"] & {} \\
 {}                               &  \MM \\
\MM\circ \PP \arrow[ur,"\rho",swap]
\end{tikzcd}
\]
A map of right $\PP$-modules is a morphism
of symmetric sequences $f:\MM\longrightarrow \MM'$ 
such that the following diagram commutes:
\[
\begin{tikzcd}[row sep = large, column sep = large]
\MM\circ \PP \arrow[r,"\rho"]
	\arrow[d,"f\circ 1",swap] & \MM \arrow[d,"f"]\\
\MM'\circ \PP \arrow[r,"\rho'",swap] & \MM'
\end{tikzcd}
\]
Finally, a free right $\PP$-module is any module that is isomorphic
to one of the form $\MM = \XX\circ\PP$ where the map $\rho : \MM\circ\PP
\longrightarrow \MM$ is given by the composition
\[
(\XX\circ \PP)\circ\PP 
\stackrel{\alpha}{\longrightarrow}
\XX \circ (\PP \circ \PP) 
	\stackrel{1\circ\gamma}{\longrightarrow} 
		\XX\circ \PP.
\]
\end{definition}

\emph{Note.} If we replace the word ``right'' above by ``left'' we obtain
the corresponding definitions for \emph{left $\PP$-modules}. A word of
caution is in order, however, as right modules and left modules in
case of operads behave completely differently. For example, right
actions are linear, in the sense that their signature is of the
form
\[
(m;p_1,\ldots,p_k) \longrightarrow \rho(m;p_1,\ldots,p_k)
\] 
and in particular there is only one argument in $\MM$. On the
other hand, left actions are not linear, in the sense that their
signature is of the form
\[
(p;m_1,\ldots,m_k) \longrightarrow \lambda(p;m_1,\ldots,m_k)
\] 
for multiple $m_1,\ldots,m_k\in \MM$: an element in $\PP$
of arity $k$ must act simultaneously on $k$ elements of $\MM$
at once and, since modules have no units, there is no way
to linearize this. We will have the opportunity to see
how this distinction will affect our development of the
theory in some cases.

\begin{definition}
Let $\MM$ be a left $\PP$-module. An endomorphism
$f: \MM \longrightarrow \MM$ of symmetric sequences is
called a derivation if for all $m_1,\ldots,m_k\in\MM$
and $p\in\PP(k)$ we have that $f \lambda = \lambda(1\circ' f)$
\end{definition}
 
In the case of right $\PP$-modules, there is no difference
between a derivation and a right linear map ---in both cases
the relevant requirement is that $f\rho = \rho(f\circ 1)$---, but we will use
both names for consistency. Note that the condition above of
being a derivation is linear, as $f$ appears only once in each
use of $\lambda$, so left linear maps and left derivations
are slightly different. However, they are in natural bijection
when we restrict ourselves to the class of free modules:

\begin{lemma}\label{lemma:derivations}
Let $\PP$ be an operad and let $\MM$ be a free
left (resp. right) $\PP$-module with basis $\XX$.
There is a natural bijection between maps of 
symmetric sequences $\XX\longrightarrow \MM$
and left (resp. right) derivations $\MM\longrightarrow \MM$.
More precisely:
\begin{tenumerate}
\item If $\MM$ is left free, then the unique derivation $f:\MM\longrightarrow\MM$
which extends $\varphi: \XX\longrightarrow \MM$ is given by
\[
f = d_\PP \circ 1 + (\gamma_{(1)}\circ 1)(1\circ'\varphi).
\]
\item  If $\MM$ is right free, then the unique derivation $f:\MM\longrightarrow\MM$ 
which extends $\varphi: \XX\longrightarrow \MM$ is given by
\[
f = 1\circ' d_\PP  + (1\circ\gamma)(\varphi\circ 1).
\]
\end{tenumerate}
\end{lemma}

\textbf{Koszul complexes.}
Let $\tau : \CC\longrightarrow \PP$ be a twisting cochain,
and let us explain how to produce a map that assigns $\tau$
two complexes $\CC\circ_\tau \PP$ and $\PP\circ_\tau\CC$, 
the \emph{right (resp. left) Koszul complex} associated to $\tau$.
In the first case, let us consider the free right $\PP$-module
$\CC\circ \PP$, and let us consider the map
$\CC \longrightarrow \CC\circ \PP$
obtained as the composition $(1\circ_{(1)} \tau)\Delta_{(1)}$.
By Lemma~\ref{lemma:derivations}, there is a unique derivation
of right $\PP$-modules
$d_\tau^r : \CC\circ \PP \longrightarrow \CC\circ \PP$
extending the map above. It is given by the following composition:
\[
\begin{tikzcd}[column sep = large]
{} & (	\CC\circ_{(1)} \CC )\circ\PP \arrow[r,"(1\circ_{(1)}\tau)\circ 1"] 
	& (	\CC\circ_{(1)} \PP )\circ
	\PP  \arrow[dr,"\cong"]\\ 
\CC\circ \PP \arrow[ur,"\Delta_{(1)}\circ 1"]\arrow[dr,"d_\tau^r",swap] & {} & {} & \CC\circ (\PP;\PP\circ\PP)\arrow[dl,"1\circ(1;\gamma)"]\\ 
{} &  \CC\circ \PP &  \CC\circ (\PP;\PP)\arrow[l,"\cong"] &{}
\end{tikzcd}
 \]
Dually, we write $d_\tau^l$ for the unique derivation of the free left 
$\PP$-module $\PP\circ\CC$ extending the map $\CC\longrightarrow \PP\circ \CC$
given by the composition $(\tau\circ 1)\Delta$. 


\begin{proposition}
The derivation $\partial_\tau = d_\CC\circ 1  + 1\circ' d_\PP + d_\tau^r$
of $\CC\circ \PP$
is such that $\partial_\tau^2 = d_{\partial(\tau) + \tau\star\tau}^r$. In 
particular, it squares to zero if and only if $\tau$ is a twisting
morphism. Similarly, the derivation  $\partial_\tau = d_\PP\circ 1  + 1\circ' 
d_\CC + d_\tau^l$ of $\PP\circ \CC$ is such that 
$\partial_\tau^2 = d_{\partial(\tau) + \tau\star\tau}^l$,
so that it squares to zero if and only if $\tau$ is a twisting
morphism.
\end{proposition}

\begin{proof}
The proof involves a computation, though the computation may get
rather cumbersome, so we provide the requisite
details
in the Appendix. In fact, we will
prove something a bit stronger: the assignment that takes 
a morphism $\tau : \CC\longrightarrow \PP$ of symmetric
sequences to the derivation $d_\tau^r : \CC\circ\PP \longrightarrow
\CC\circ \PP$ is a morphism of dg Lie algebras, where we put
on $\hom_\Sigma(\CC,\PP)$ the dg Lie algebra structure induced by
the pre-Lie star product $-\star-$ and we give the space of
derivations $\operatorname{Der}(\CC\circ\PP)$ the usual
structure of a Lie algebra through the commutator of
derivations (recall that the composition of two derivations
need not be a derivation, but the commutator of derivations
is a derivation).

To see how this claim implies the proposition, we notice
that $\partial_\tau\circ\partial_\tau = \frac{1}{2}[\partial_\tau,\partial_\tau]$
and that if we write $\partial_\tau = d_{\CC\circ\PP} + d_\tau^r$, then
\begin{tenumerate}
\item The original differential squares to zero, so that $[d_{\CC\circ\PP},d_{\CC\circ\PP}] = 0$,
\item The fact this is a morphism of complexes shows that $[d_{\CC\circ\PP},d_\tau^r] = d_{\partial(\tau)}^r$ and,
\item The claim above proves that $[d_\tau^r,d_\tau^r] = d_{[\tau,\tau]}^r$.
\end{tenumerate}
This completes the proof of the proposition.
\end{proof}

\begin{definition}
We write $\CC\circ_\tau \PP$ for the complex in right $\PP$-modules
$(\CC\circ \PP,\partial_\tau)$, and call it the \emph{right Koszul
complex associated to $\tau$}. Analogously, we write $\PP\circ_\tau \CC$
for the complex in free left $\PP$-modules $(\PP\circ \CC,\partial_\tau)$
and call it the \emph{left Koszul complex associated to $\tau$}.
\end{definition}

We are now ready to make concrete our claim that Koszul (co)operads are
those that admit economic resolutions: we did not say of \emph{what}
object, but we can do that now. The only observation we need to make
is the following:

\begin{definition}
Let $\PP$ be a quadratic operad. The map $\kappa : \PP^\antishriek
\longrightarrow \PP$ obtained as the composition
$\PP^\antishriek \hookrightarrow \B{\PP} \longrightarrow \PP$
is a twisting cochain.
\end{definition} 

Note that $\kappa$ can be described even in simpler terms as
the degree $-1$ map obtained as the compositions of the projection
$\PP^\antishriek \twoheadrightarrow s\XX$ onto weight $1$,
the desuspension map $s\XX\stackrel{s^{-1}}\longrightarrow \XX$
and the inclusion $\XX \hookrightarrow \PP$. 

\begin{proof}
This is actually a straightforward computation,
since in our situation $\PP$ has no differential
and we only need to show that $\kappa\star \kappa = 0$. 
But $\kappa\star\kappa : \PP^\antishriek \longrightarrow \PP$
is obtained as the composition
\[
\PP^\antishriek\stackrel{\Delta_{(1)}}\longrightarrow 
 	\PP^\antishriek \circ_{(1)} 	\PP^\antishriek 
 	 \longrightarrow
 	 	\PP \circ_{(1)} 	\PP 
 	 		\stackrel{\gamma_{(1)}}\longrightarrow 
 	 		\PP
\]
where the map $\kappa\circ_{(1)} \kappa$ in the middle is only non-zero
on elements of $s\XX\circ_{(1)} s\XX \subseteq \PP^\antishriek\circ_{(1)} \PP^\antishriek$, and in this case the decomposition of $\PP^\antishriek$
lands on $s^2\mathcal R$ by construction. This means that
the image after using the composition map of $\gamma$ lands in $\RR$,
which is zero in $\PP$, as we claimed.
\end{proof}

With this at hand, here is our second definition of Koszulness.
By Theorem~\ref{thm:fundamental} in the Appendix, the following
is equivalent to our original definition of Koszulness of a quadratic
operad, stating that the inclusion $\PP^\antishriek \longrightarrow 
\B{\PP}$ is a quasi-isomorphism of cooperads. 

\begin{definition}[Second definition]
A quadratic operad $\PP$ is Koszul if and only if the
right Koszul complex $\PP^\antishriek \circ_\kappa \PP$
is a resolution of the trivial module $\kk$ in right
$\PP$-modules. Equivalently, 
a quadratic operad $\PP$ is Koszul if and only if the
left Koszul complex $\PP \circ_\kappa \PP^\antishriek$
is a resolution of the trivial module $\kk$ in left
$\PP$-modules.
\end{definition}

\subsection{Exercises}

\begin{question}
Suppose that $\PP$ is locally finite dimensional and that $\PP(1) = \kk$. 
Show that $\B{\PP}$ and $\Omega(\PP^*)$ are dual to each other.
\end{question}

\begin{question}
Prove the claim in Theorem~\ref{thm:adjunction} that the Maurer--Cartan
equation for the map $\tau : \CC\longrightarrow \PP$ corresponding
to a morphism of dg cooperads $f : \CC\longrightarrow \B{\PP}$
and to a morphism of dg operads $g  :\Omega(\CC) \longrightarrow \PP$
is equivalent to the condition that $f$ and $g$ commute with the
differentials.
\end{question}


\begin{question}\label{ex:universaltw}
Show that the universal twisting cochain
$\pi:\B{\PP}\longrightarrow \PP$ is indeed
a twisting cochain, and that the Maurer--Cartan
equation is equivalent to the condition that 
$\PP$ is a dg operad. Do the same for the 
 universal twisting cochain $\iota : \CC 
\longrightarrow \Omega(\CC)$, and 
\end{question}


\begin{question}
Show that $\PP \longmapsto \pi_\PP$ is natural, in
the sense that for a morphism of operads $\PP\longrightarrow 
\PP'$ we have a commutative square
\[
\begin{tikzcd}[ampersand replacement=\&]
\B{\PP} \arrow[r] \arrow[d] \&  \PP \arrow[d] \\
\B{\PP'} \arrow[r] \& \PP'
\end{tikzcd}
\]
Do the same for $\iota$. Show that the bijection described in Theorem~\ref{thm:adjunction} 
is natural.  
\end{question}