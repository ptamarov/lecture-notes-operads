\section{Quadratic operads}\label{lecture:quadraticops}

 \textbf{Goal.} Introduce weight graded gadgets,
 define operads by generators and relations, and
 introduce quadratic operads. Give plenty of examples
 of `real life' quadratic operads to work on:
 Hilbert series, Koszul dual, bar construction. 
 
 \subsection{Weight gradings and presentations}
 The notion of a quadratic operad is based on the observation
 every free operad has a canonical `weight grading' by the
 number of internal vertices of a tree. Let us make this
 precise.
 
\begin{definition}
A symmetric sequence $\XX$ is weight graded if for
each finite set the component $\XX(I)$ admits a 
decomposition $\XX(I) = \bigoplus_{j\geqslant 0}
\XX^{(j)}(I)$. A symmetric operad $\PP$ is weight
graded if its underlying symmetric sequence
is weight graded and its composition maps
preserve the weight grading.
\end{definition}

Thus, a weight graded operad must have composition maps 
of the form
\[ \PP^{(a)}(k) \otimes 
	\PP^{(b_1)}(n_1) \otimes \cdots \otimes \PP^{(b_k)}(n_k)
	 	\longrightarrow \PP^{(b)}(n) \]
where $b=b_1+\cdots+b_k$ and $n = n_1+\cdots+n_k$. In the
case we consider partial composition maps, observe we have
instead maps of the form
\[\circ_i :  \PP^{(a)}(m)\otimes  \PP^{(b)}(n)
	\longrightarrow  \PP^{(a+b)}(m+n-1). \] 
The free operad $\FF_\XX$ is weight graded by the number
of internal vertices of a tree (that is, we put $\XX$ in
weight one, and extend the weight to trees by counting 
occurrences of elements of $\XX$. More generally, if
$\XX$ admits a weight grading, then $\FF_\XX$ inherits
this weight grading: the weight of a tree monomial is the
sum of the weight of the decorations of its vertices,
 and we write $\FF_\XX^{(n)}$ for the
homogeneous component of weight $n\in\NN_0$. If we do
not specify a weight grading on $\FF_\XX$, we will 
always assume we are taking the canonical weight grading above.


\begin{definition} An ideal in an operad $\PP$
is a subcollection $\mathcal{I}$ for which
both $\gamma(\mathcal{I}\circ \PP)$ and
$\gamma(\PP\circ_{(1)} \mathcal{I})$ are contained in
$\mathcal{I}$.
The quotient of $\PP/\mathcal{I}$ is again an
operad, called the quotient of $\PP$ 
by $\mathcal{I}$. Every subcollection $\RR$
of $\PP$ is contained in a smallest
ideal, called the \emph{ideal generated by $\RR$}.
\end{definition}

The notion of ideals and of free operads allow us
to define operads by generators and relations.

\begin{definition}
We write $\FF(\XX,\RR)$ for the quotient of 
$\FF_\XX$
by the ideal generated by a subcollection $\RR$
of $\FF_\XX$. 
 We say $\PP$ is presented by generators
$\XX$ and relations $\RR$ if there is an
isomorphism $\FF(\XX,\RR) \longrightarrow
\PP$.
\end{definition}

Note that if $\PP$ is symmetric, the definition
requires that $\mathcal{I}$ be stable under
the symmetric group actions, so we may 
sometimes specify $\RR$ by a generating set 
only, and understand
that $(\RR)$ is generated by the $\Sigma$-orbit
of $\RR$.

\bigskip

\textbf{Some examples.}
To illustrate the definitions above, let us
give three examples of algebraic operads whose
associated algebras are probably well known
to the reader: 
\begin{tenumerate}
\item The associative operad is generated by a 
binary operation $\mu$ generating the regular
representation of $S_2$ subject to the relation
$\mu\circ_1 \mu = \mu \circ_2 \mu$. 
\item The commutative operad is generated by a 
binary operation which instead generates the
trivial representation of $S_2$ and is
also associative. Both of this 
and the previous
example arise as the linearization of a set operad.
\item 
The Lie operad is generated by a single binary 
operation $\beta$ that generates the sign 
representation of $S_2$ subject to the only
relation
$(\beta \circ_1 \beta)(1+\tau+\tau^2) = 0$
where $\tau = (123)\in S_3$ is the $3$-cycle. 
\end{tenumerate}
We write these operads $\mathsf{As},\mathsf{Com}$
and $\mathsf{Lie}$ and, following J.-L. Loday,
call them the \emph{three graces}. We have that
\[ \As(n) = \kk S_n,\quad
 	\Com(n) = \kk, \quad
 	 \Lie(n) = \operatorname{Ind}_{\mathbb Z/n}^{S_n} \kk_\zeta \] 
 where $\kk_\zeta$ is a character of $\mathbb Z/n$
 for a primitive $n$th root of the unit. Concretely,
 the last equality is stating that if we fix a primitive
 $k$th root of unity $\zeta_k$, and if we let $\rho_k$ be
 the standard $k$-cycle of $S_k$, the free Lie
 algebra $L(V)\subseteq T(V)$ identifies  
 in each weight degree $k$ with those $v
 \in V^{\otimes k}$ such that $\rho_k v = \zeta_k v$; see~\cite{Klyachko1975,MO187545}.
 
\begin{note} It is not always advantageous
to define an operad by generators and relations:
the operad pre-Lie can be defined explicitly
in terms of labeled rooted trees and a grafting
operation, as done by Chapoton--Livernet, and
this `presentation' is very useful in practice,
for example, to show that the pre-Lie operad
is Koszul.
\end{note}

\subsection{Quadratic operads}
An operad $\PP$ is \emph{quadratic} if it admits a presentation
$\FF(\XX,\RR)$ where $\RR \subseteq \FF(\XX)^{(2)}$. 
That is, $\PP$ is generated by some collection of
operations $\XX$ and all its defining relations are of the form
\[ \sum \lambda_{\mu,\nu}^i \mu \circ_i \nu = 0  \] 
where $\ari(\mu)+\ari(\nu)$ is constant. An operad is
\emph{binary quadratic} if moreover $\XX = \XX(2)$ or,
what is the same, all the generating operations of $\PP$
are of arity two (binary). 
 A \emph{quadratic-linear presentation} of an operad $\PP$
is a presentation $\FF(\XX,\RR)$ of $\PP$ where $\RR 
\subseteq \XX \oplus \FF(\XX)^{(2)}$. That is, it is
a presentation of the form
\[ \sum \lambda_{\mu,\nu}^i \,\mu \circ_i \nu 
 + \sum \lambda_\rho \,\rho = 0   \] 
 where $\ari(\mu)+\ari(\nu) = \ari(\rho)+1$ is constant.
 Every operad admits a quadratic-linear
presentation, albeit with possibly with infinitely many generators, 
We point the reader to~\cite{Loday2012}*{Section 7.8} for a comprehensive treatment of operads
with quadratic linear relations.
 
 Let us define a quadratic datum to be a pair $(\XX,\RR)$
 where $\XX$ is a symmetric sequence and $\RR\subseteq
 \FF_\XX^{(2)}$. A map of quadratic data $(\XX_1,\RR_1) 
 \longrightarrow (\XX_2,\RR_2)$ is a map $\XX_1
 \to \XX_2$ of symmetric sequences for which the induced
 map on free operads sends $\RR_1$ to $\RR_2$. The 
 assignment $(\XX,\RR) \longrightarrow \FF(\XX,\RR)$
 defines a functor from the category of quadratic data
 to the category of quadratic operads.
 
 %%Exercise: find two different QD giving same operad.
 \bigskip
 
\textbf{More examples.} The presentations of the 
associative, commutative and Lie operad above are
quadratic. The following are also quadratic operads:

\emph{The Gerstenhaber operad}. The symmetric operad
$\mathsf{Ger}$ and its cousin,
the Poisson operad $\mathsf{Poiss}$ belong to the two parameter
family $\mathsf{Poiss}(a,b)$ of binary quadratic operads generated
by two operations $x_1x_2$ and $[x_1,x_2]$ of respective degrees
$a$ and $b$, so that
the first is commutative associative, the second is a Lie
bracket, and they satisfy the Leibniz rule. With this
at hand $\mathsf{Ger} =\mathsf{Poiss}(0,-1)$ while  
$\mathsf{Poiss} = \mathsf{Poiss}(0,0)$.

\smallskip

\emph{The pre-Lie operad}. The operad $\mathsf{PreLie}$ and its
quotient, the Novikov operad $\mathsf{Nov}$, are quadratic
binary operads generated by a single operation $x_1\circ x_2$
with no symmetries. The first one is subject to the right-symmetry
condition for the associator
\[ 
x_1\circ (x_2\circ x_3) - (x_1\circ x_2)\circ x_3	=
x_1\circ (x_3\circ x_2) - (x_1\circ x_3)\circ x_2.
\]
The second 
operad is obtained by further imposing the left-permutative
relation that 
\[ x_1\circ (x_2\circ x_3) = x_2\circ (x_1\circ x_3).\]
The permutative operad $\mathsf{Perm}$ is the binary
operad generated by a single operation with no symmetries
satisfying the last quadratic equation.

\smallskip

\emph{The operad of totally associative $k$-ary algebras}. 
$\mathsf{tAs}_k$ (and its commutative counterpart). It is generated
by a $k$-ary non-symmetric operation $\alpha$ subject to
the relations $\alpha \circ_i \alpha =\alpha\circ_k \alpha$
for all $i\in [k]$. One can consider $\alpha$ to be
totally symmetric, and obtain the operad of totally
associative commutative $k$-ary algebras.

\smallskip

\emph{The operad of partially associative $k$-ary algebras}.
$\mathsf{pAs}^k$ (and its Lie counterpart). It is generated
by a $k$-ary non-symmetric operation $\alpha$ of degree $k-2$
subject to the single relation
\[ 
	\sum_{i=1}^k (-1)^{(k-1)(i-1)} \alpha\circ_i \alpha = 0.\]
One can consider a $k$-ary totally antisymmetric operation
$\beta$ of degree $1$, and obtain the operad of Lie $k$-algebras, 
which is subject to the single equation
\[
 \sum_{\substack{A\sqcup B = [2k-3] \\
 |A|=k-1,|B|=k-2}}  (\beta\circ_1\beta)\sigma_{A,B} = 0.
 \]

\smallskip

\emph{The operad of anti-associative algebras.} $\mathsf{As}^-$ is
generated by a single operation of degree zero with no symmetries
satisfying the `anti-associative law'
\[ x_1(x_2x_3) + (x_1x_2)x_3 = 0. \]

\subsection{Exercises}


 \begin{question}
During Lecture~\ref{lecture:quadraticops} we introduced the associative and commutative operads
through binary quadratic presentations. Show  that 
for all $n\geqslant 1$ the space $\mathsf{Ass}(n)$ is the
regular representation of $S_n$, and that for 
all $n\geqslant 1$ the space $\mathsf{Com}(n)$ is the 
trivial representation.
\end{question}

\begin{question} Use the presentation of the Poisson operad given
during Lecture~\ref{lecture:quadraticops} to show that $\dim\mathsf{Poiss}(n)\leqslant n!$
for all $n\geqslant 1$\footnote{There are at least three different
ways to show that equality holds.}. 
\end{question}

\begin{question} Let $x_1x_2$ be the associative binary generator of
$\mathsf{Ass}$ and let us consider the operations (which are symmetric
and antisymmetric, respectively)
\[x_1\cdot x_2 = \frac{1}{2}(x_1x_2+x_2x_1), \quad 
	 [x_1,x_2] = \frac{1}{2}(x_1x_2-x_2x_1) 	
	 \]
obtained by `polarization'. Show that the second is a Lie bracket,
and that the first is a commutative (but not associative) product
that satisfies the Leibniz rule for $[x_1,x_2]$, and whose associator
is equal to $[x_2,[x_1,x_3]]$. This is called the \emph{Livernet--Loday
presentation} of the associative operad.
\end{question}

\begin{question} 
During Lecture~\ref{lecture:quadraticops}, we introduced to operad $\mathsf{tCom}_k$ of totally
associative commutative $k$-ary algebras. It is generated by a single
fully symmetric operation $\mu$ or arity $k$ subject to the relations
$\mu\circ_1\mu = \mu\circ_i\mu$
for each $i\in [k]$ (and all its symmetric translates).
Show that $\mathsf{tCom}_k(n)$ is either the one dimensional trivial
representation or zero depending on $n$. What values must
$n$ take so that it is non-zero? 
\end{question}

\begin{question}
The permutative operad $\mathsf{Perm}$ is generated by a single
binary operation $x_1x_2$ with no symmetries which is associative,
and such that
\[ x_1(x_2x_3) = x_2(x_1x_3). \]
Show that $\mathsf{Perm}(n)$ is of dimension $n$ and is
isomorphic as a representation to $\mathrm{Ind}_{S_{n-1}}^{S_n}\mathbb{C}$
where $\mathbb{C}$ is the trivial representation.
\end{question}



We have defined quadratic operads as precisely those
operads presented by (homogeneous) quadratic relations
on some set of generators. Let us explore how to create
maps between them.



\begin{question}
Suppose that $(\mathcal{X},\mathcal{R})$ and $(\mathcal{Y},\mathcal Q)$
are quadratic data. Show that a
map of sequences $f: \mathcal{X} \longrightarrow \mathcal{Y}$
induces a map on the corresponding quadratic operads if and only if
the induced map $F = \mathcal{F}_f$ sends $\mathcal{R}$ to
$\mathcal{Q}$.
\end{question}

\begin{question}
Show that:
\begin{tenumerate}
\item The augmentation map
$\mathbb C S_2\longrightarrow \mathbb C$ (that
sends $1$ and $(12)$ to $1$) induces a surjective map of
operads $\mathsf{Ass}
\longrightarrow \mathsf{Com}$.
\item The inclusion map
$\mathbb{C}^- \longrightarrow \mathbb{C}S_2$ that assigns $1$ to 
$1-(12)$ induces a map of operads $\mathsf{Lie}\longrightarrow 
\mathsf{Ass}$ and also a map of operads $\mathsf{Lie}\longrightarrow 
\mathsf{PreLie}$.
\item The projection $\mathsf{Ass}
\longrightarrow \mathsf{Com}$ actually factors
through $\mathsf{Perm}$. 
\end{tenumerate}
In each case, what is the interpretation at the level
of algebras?
\end{question}
