\documentclass[fleqn,a4paper, twoside]{article} 

%% for print
%\usepackage[
%	top = 1.15 in, 
%	bottom = 1.25 in,
%	left = 1.15 in, 
%	right = 1.15 in,
%	includehead]{geometry}
%% font sizes
%\usepackage{scrextend}
%\changefontsizes{12pt}
% for online reading
\usepackage[
	top = 1.15 cm, 
	bottom = 2.25 cm,
	left = 1.15 cm, 
	right =1.15 cm,
	includehead]{geometry}
% font sizes
\usepackage{scrextend}
\changefontsizes{14pt}

\usepackage[all]{nowidow}


\usepackage[utf8x]{inputenc}

\usepackage{amscd,amssymb,amsmath}
\usepackage{amsrefs}

%nimbus roman
\usepackage{mathptmx}
\usepackage[T1]{fontenc}


\usepackage{booktabs}
\newcommand{\ra}[1]{\renewcommand{\arraystretch}{#1}}

\usepackage[utf8x]{inputenc}
\usepackage{amsfonts,amssymb,amsmath,amsrefs}
\usepackage{graphicx}
\usepackage[british]{babel}
\usepackage{caption}
\usepackage{mathdots}
\usepackage{mathtools} 
\usepackage{amsmath}
\usepackage{amsfonts}
\usepackage{amssymb}
\usepackage{graphicx}
\usepackage{subfigure}
\usepackage{makeidx}
\usepackage{multicol}
\usepackage{array}
\usepackage{cancel}
\usepackage{polynom}
\usepackage[pdf,all]{xy}
\xyoption{line}
\CompileMatrices


%% Side captions %%
\usepackage{sidecap}

\usepackage{wrapfig} %%https://es.sharelatex.com/learn/latex/Wrapping_text_around_figures

\usepackage{lipsum}
%% SMALL HYPHEN %%
\mathchardef\hy="2D % Define a "math hyphen"

 %% TAMAÑO C\eLDAS %%     
\usepackage{pgf,tikz}
\usepackage{tikz-cd}
\usetikzlibrary{calc}
\usetikzlibrary{matrix,arrows}
\usepackage{stackrel}
\usepackage[shortlabels]{enumitem} %Para listas estilo Weibel
\usepackage{stmaryrd} %Para double brackets
\usepackage{setspace} %Para espaciado de renglones
\spacing{1.15}
%\onehalfspacing %Mejor lectura
\usepackage{etoolbox}
\usetikzlibrary{trees}


\patchcmd{\section}{\normalfont}{\normalfont\large}{}{}

\usepackage[bb=ams, cal=euler, scr=rsfso , frak=euler]{mathalpha}

\usepackage{fancyhdr}
\pagestyle{fancy}
\fancyhead[RE]{\small\it Algebraic operads}
\fancyhead[LO]{\small\it Matamzee 2021}
\fancyhead[RO,LE]{\small\textbf\thepage}
\renewcommand{\headrulewidth}{0 pt}
\cfoot{}
\newcommand{\0}{\langle 0\rangle}

\fancypagestyle{references}
{\fancyhead[RE]{\small\it References}
\fancyhead[LO]{\small\it References}
\fancyhead[RO,LE]{\small\bf\thepage}
\fancyfoot[L,R,C]{}
\renewcommand{\headrulewidth}{0 pt}}

 %%%%%%%%%%%%%%%%%%%%%%%

\newcommand{\XX}{\mathcal{X}}
\renewcommand{\AA}{\mathcal{A}}
\newcommand{\YY}{\mathcal{Y}}
\newcommand{\End}{\operatorname{End}}
\newcommand{\RR}{\mathcal{R}}
\newcommand{\II}{\mathcal{I}}
\newcommand{\QQ}{\mathcal{Q}}
\newcommand{\FF}{\mathcal{F}}

\newcommand{\ari}{\operatorname{ar}}
\newcommand{\cone}{\operatorname{cone}}
\newcommand{\hem}{\hspace{0.5 em}} 
%\input{treesp.tex} For trees, maybe add again later

 \newcommand{\stt}{\mathbin{\text{\tikz 
[x=1ex,y=1ex,line width=.1ex,line 
join=round] \draw (0,0) rectangle (1,1) 
(1,1) -- (0,0)  (1,0) -- (0,1);}}}

%%%%%% ENUMERATE STUFF %%%%%%
\usepackage{enumitem}
\listfiles
\setlist[enumerate]{label= (\arabic*)}


\newenvironment{tenumerate}{
 \begin{enumerate}
  \setlength{\itemsep}{0pt}
  \setlength{\parskip}{0pt}
}{\end{enumerate}}

\newenvironment{titemize}{
\begin{itemize}
  \setlength{\itemsep}{0pt}
  \setlength{\parskip}{0pt}
}{\end{itemize}}

\definecolor{newcol}{rgb}{0,0,0}
\definecolor{deepblue}{rgb}{0.0, 0.28, 0.67}
%{0.5, 0.0, 0.13}
\DeclareTextFontCommand{\new}{\color{black}\em}
%\DeclareTextFontCommand{\new}{\color{black}\em}

\usepackage[pdftex, colorlinks,bookmarks 
= true,bookmarksnumbered = true]{hyperref}

\hypersetup{colorlinks,linkcolor={deepblue},
	citecolor={deepblue},urlcolor={deepblue}}  
%
\usepackage{sectsty}
\chapterfont{\color{newcol}}  % sets colour of chapters
\sectionfont{\color{newcol}}  % sets colour of sections
\subsectionfont{\color{newcol}}  % sets colour of sections

% let \[ and \] be the same as \begin{equation} and \end{equation}
\makeatletter
\AtBeginDocument{%
  \let\[\@undefined
  
\DeclareRobustCommand{\[}{\begin{equation}}%
  \let\]\@undefined
  
\DeclareRobustCommand{\]}{\end{equation}}%
}
\makeatother 
% but only print equation numbers if needed
\mathtoolsset{showonlyrefs,showmanualtags}

\usepackage{amsthm}
\usepackage{thmtools}
\newtheoremstyle{mytheorem}
  {\topsep}   % ABOVESPACE
  {\topsep}   % BELOWSPACE
  {\itshape}  % BODYFONT
  {0pt}       % INDENT (empty value is the same as 0pt)
  {\bfseries\color{newcol}} % HEADFONT
  {\color{newcol}}         % HEADPUNCT
  {5pt plus 1pt minus 1pt} % HEADSPACE
  {}          % CUSTOM-HEAD-SPEC
  
\theoremstyle{mytheorem}
\newtheorem{theorem}{Theorem}[section]
\newtheorem{prop}[theorem]{Proposition}
\newtheorem{cor}[theorem]{Corollary}
\newtheorem{lemma}[theorem]{Lemma}
\newtheorem*{conj*}{Conjecture}
\newtheorem{question}[theorem]{Question}
\newtheorem{conj}{Conjecture}

 \newtheoremstyle{introthm}
  {\topsep}   % ABOVESPACE
  {\topsep}   % BELOWSPACE
  {\itshape}  % BODYFONT
  {0pt}       % INDENT (empty value is the same as 0pt)
  {\bfseries\color{newcol}} % HEADFONT
  {\color{newcol}{.}}         % HEADPUNCT
  {5pt plus 1pt minus 1pt} % HEADSPACE
  {}          % CUSTOM-HEAD-SPEC
 
  
\theoremstyle{introthm}
\newtheorem{introthm}{Theorem}
\newtheorem*{introcor}{Corollary}
\renewcommand{\theintrothm}{\Alph{introthm}}
 % Para teoremas en la introducción

%%%%%%%% TODO NOTES %%%


%% TODO NOTES! %%
\usepackage{xargs}                      
% Use more than one optional parameter in a new commands
%\usepackage[pdftex,dvipsnames]{xcolor}  
% Coloured text etc.
\usepackage[colorinlistoftodos,prependcaption,textsize=small]{todonotes}
\newcommandx{\unsure}[2][1=]{\todo[linecolor=blue,backgroundcolor=blue!25!white,bordercolor=blue,#1]{#2}}
\newcommandx{\change}[2][1=]{\todo[linecolor=blue,backgroundcolor=blue!25,bordercolor=blue,#1]{#2}}
\newcommandx{\info}[2][1=]{\todo[linecolor=OliveGreen,backgroundcolor=OliveGreen!25,bordercolor=OliveGreen,#1]{#2}}
\newcommandx{\improvement}[2][1=]{\todo[linecolor=Plum,backgroundcolor=Plum!25,bordercolor=Plum,#1]{#2}}
\newcommandx{\thiswillnotshow}[2][1=]{\todo[disable,#1]{#2}}

\definecolor{col1}{rgb}{0.8, 0.8, 1.0}
\definecolor{col2}{rgb}{0.9, 0.9, 0.98}
\definecolor{col3}{rgb}{0.71, 0.49, 0.86}

  
\newcommand{\ind}{\operatorname{Ind}_\PP}

\newcommand{\cof}{\rightarrowtail}
\newcommand{\foc}{\leftarrowtail}
\newcommand{\weq}{\overset{\sim}{\longrightarrow}}
\newcommand{\tcf}{\overset{\sim}{\cof}}

\newtheoremstyle{mydefinition}
  {\topsep}   % ABOVESPACE
  {\topsep}   % BELOWSPACE
  {}  % BODYFONT
  {0pt}       % INDENT (empty value is the same as 0pt)
  {\bfseries\color{newcol}} % HEADFONT
  {\color{newcol}}         % HEADPUNCT
  {5pt plus 1pt minus 1pt} % HEADSPACE
  {}          % CUSTOM-HEAD-SPEC
  
\theoremstyle{mydefinition}
\newtheorem{definition}[theorem]{Definition}
\newtheorem{obs}[theorem]{Observation}
\newtheorem{rmk}[theorem]{Remark}
\newtheorem{problem}[theorem]{Problem}
\newtheorem{example}[theorem]{Example}
\newtheorem{note}[theorem]{Note}
\newtheorem{variant}[theorem]{Variant}


\newtheoremstyle{mydefinition2}
  {\topsep}   % ABOVESPACE
  {\topsep}   % BELOWSPACE
  {}  % BODYFONT
  {0pt}       % INDENT (empty value is the same as 0pt)
  {\bfseries\color{newcol}} % HEADFONT
  {\color{newcol}{.}}         % HEADPUNCT
  {5pt plus 1pt minus 1pt} % HEADSPACE
  {}          % CUSTOM-HEAD-SPEC
  
\theoremstyle{mydefinition2}
\newtheorem*{definition*}{Definition}
\newtheorem*{remark*}{Remark}
\newtheorem*{obs*}{Observation}
\newtheorem*{example*}{Example}

% named theorem %

% for specifying a name
\theoremstyle{plain} % just in case the style had changed
\newcommand{\thistheoremname}{}
\newtheorem{genericthm}[theorem]{\thistheoremname}
\newenvironment{namedthm}[1]
  {\renewcommand{\thistheoremname}{#1}%
   \begin{genericthm}}
  {\end{genericthm}}

\newcommand{\imor}{\interleave\kern-.45em\longrightarrow}
\renewcommand{\qedsymbol}{$\blacktriangleleft$}
\newcommand\place{\mathord-}
\newcommand{\ps}{\mathbin\parallel}
\newcommand{\GSet}{\mathsf{Fin}^\times}
\newcommand{\FSet}{\mathsf{Fin}}
\newcommand{\Fin}{\mathsf{Fin}}
\newcommand{\Set}{\mathsf{Set}}
\newcommand{\DCSH}{\mathsf{DCSH}}
\newcommand{\kMod}{{}_\kk\mathsf{Mod}}
\newcommand{\kmod}{{}_\kk\mathsf{mod}}
\newcommand{\kCh}{{}_\kk\mathsf{Ch}}
\newcommand{\Sp}{\mathsf{Sp}}
\newcommand{\kSp}{{}_\kk\mathsf{Sp}}
\newcommand{\GVec}{\mathsf{Vec}_q^\times}
\newcommand{\Opr}{\mathsf{Opr}}
\newcommand{\NS}{\mathsf{NS}}
\DeclarePairedDelimiter\abs{\lvert}{\rvert}
\newcommand{\vv}{\vert}
\newcommand{\Der}{\operatorname{Der}}
\newcommand{\HH}{\mathrm{HH}}
\newcommand{\As}{\mathsf{As}}
\newcommand{\Com}{\mathsf{Com}}
\newcommand{\Lie}{\mathsf{Lie}}

\newcommand{\aut}{\operatorname{aut}}
\definecolor{sqsqsq}{rgb}{0.13,0.13,0.13}
\definecolor{aqaqaq}{rgb}{0.63,0.63,0.63}

\newcommand{\SC}{SC}
\newcommand\id{\mathrm{id}}
\newcommand\category[1]{\mathsf{#1}}
\newcommand{\Mod}{\mathsf{Mod}}
\newcommand{\Gmod}{\mathsf{GMod}}
\newcommand{\SMod}{{}_\Sigma\mathsf{dgMod}}
\newcommand{\gSMod}{{}_\Sigma\mathsf{gMod}}
\newcommand{\coker}{\operatorname{coker}}
\newcommand{\Ho}{\operatorname{Ho}}
\newcommand\spe[1]{\mathcal{#1}}
\renewcommand{\tt}{\otimes}
\newcommand{\E}{\mathcal{E}}
\newcommand{\CC}{\mathcal{C}}
\newcommand{\DC}{\mathsf{DC}}
\newcommand{\DA}{\mathsf{DA}}
\newcommand{\dd}{\partial}
\newcommand{\OO}{\mathcal O}
\newcommand{\?}{\,?\,}
\newcommand{\n}{[n]}
\newcommand\cls[1]{\llbracket#1\rrbracket}
\newcommand{\NN}{\mathbb N}
\renewcommand{\k}{[k]}
\newcommand{\bt}{\bullet}
\newcommand{\kk}{\Bbbk}
\newcommand{\Aut}{\operatorname{Aut}}
\newcommand{\Ext}{\operatorname{Ext}}
\newcommand{\Tor}{\operatorname{Tor}}
\newcommand{\PAlg}{\mathsf{Alg}_{\geqslant 0}}
\newcommand{\Cog}{\mathsf{Cog}}
\newcommand{\Alg}{\mathsf{Alg}}
\newcommand{\Cxs}{\mathsf{Ch}_\kk}
\newcommand\inter[1]{\llbracket#1\rrbracket}
\newcommand{\Cell}{\operatorname{Cell}}
\newcommand{\Sing}{\operatorname{Sing}}
\newcommand{\Sull}{A_{\mathrm{PL}}}

\newcommand\lab{\mathsf{lab}}
\newcommand{\Q}{{\mathbb{Q}}}
\newcommand{\Z}{{\mathbb{Z}}}
\newcommand{\hoq}{\!\sslash\!\!}
\newcommand{\PP}{{\mathcal{P}}}

\definecolor{newterm-color}{RGB}{0, 0, 0}
\newcommand\newterm[1]{%
  \textcolor{newterm-color}{\itshape #1}%
}

\theoremstyle{mytheorem}
\newtheorem*{theorem*}{Theorem}
\newtheorem*{question*}{Question}

% named theorem %

% for specifying a name
\theoremstyle{plain} % just in case the style had changed
% front matter stlye 
\renewenvironment{abstract}{%
\small\begin{center}
\begin{minipage}{.9\textwidth}
%\textbf{\textcolor{newcol}{Abstract.}}
}
{\par\noindent\end{minipage}\end{center}\vspace{3 em}}
%
\makeatletter
\renewcommand\@maketitle{%
\hfill
\begin{center}\begin{minipage}{0.9 	\textwidth}
\centering
\vskip 6em
\let\footnote\thanks 
{\LARGE \@title \par }
\vspace{1 em}
%\hrulefill
\vskip 1 em
{\large \@author \par}
\vspace{3.5 em}

\end{minipage}\end{center}
\par
}
\makeatother
%
%%%%%%%%%%%%%%%%%%%%%%%%%%%%%%%%%%%%%%%%%%%%%%%%%%%%%%


\DeclareMathOperator\sgn{sgn}
\DeclareMathOperator\fsch{\mathfrak{sch}}

%\definecolor{newterm-color}{RGB}{0, 51, 153}
%\newcommand\newterm[1]{%
%\textcolor{newterm-color}{\bfseries\itshape #1}%
%}

% textual claims in equations
\newcommand\claim[2][.8]{%
  \begin{minipage}{#1\displaywidth}%
  \itshape
  #2
  \end{minipage}%
}

\usepackage{float}

\tikzcdset{arrow style=tikz, diagrams={>=stealth}}

%%Addresses

\newcommand{\Addresses}{{% additional braces for segregating \footnotesize
  \bigskip
  \footnotesize
  \textsc{School of Mathematics, Trinity College Dublin}, College Green, Dublin 2, Ireland, D02 PN40\par\nopagebreak
  \textit{Addresses:} \texttt{morenofdezjm@gmail.com, pedro@maths.tcd.ie}
 }}
  

\usepackage{titletoc}

\titlecontents{chapter}
[0.2em] %
{\bigskip}
%{\contentslabel[\thecontentslabel.]{2em}\hspace{0.667em}}%\thecontentslabel
{\makebox[2em][r]{\thecontentslabel.}\hspace{0.333em}}%\thecontentslabel
{\hspace*{-2em}}
{\hfill\contentspage}[\smallskip]

\titlecontents{section}% <section>
[0.2em]% <left>
{\small}% <above-code>
{\thecontentslabel.\hspace{3pt}}%<numbered-entry-format>; you could also 
%use  {\thecontentslabel. } to show the numbers
{}% <numberless-entry-format>
{\enspace\titlerule*[0.5pc]{.}\contentspage}%<filler-page-format>
\titlecontents*{subsection}% <section>
[1em]% <left>
{\footnotesize}% <above-code>
{\thecontentslabel. \hspace{3pt}}% <numbered-entry-format>; you could also 
%use {\thecontentslabel. } to show the numbers
{}% <numberless-entry-format>
{}% <filler-page-format>
[ --- \ ]% <separator>
[]% <end>
\setcounter{tocdepth}{2}% Display up to \subsection in ToC

\setlength\parskip{3 pt}
\setlength\parindent{0 em}

\raggedbottom 
\makeindex

\title{\vspace{-5 em}\setstretch{0.85}{\textbf{Matamzee 2021}}}
\author{P. Tamaroff}
\date{August 2021}
%\address{}
%\email{ptamarov@gmail.com}
\begin{document}
\maketitle


%\tableofcontents
%\section{Introduction}\label{sec:intro}
%\addcontentsline{toc}{section}{\nameref{sec:intro}}
\thispagestyle{empty}

\tableofcontents

\section{August 21st: D. Lewanski}

\textbf{A natural basis for intersection theory.}
Consider some numbers with $\sum d_i = 3g-3+n$, of the form
$\langle \tau_{d_1},\ldots,\tau_{d_n}\rangle_g \in \mathbb Q$,
and a corresponding generating series in $x^d$ which we 
write $A_{g,n}(x)$. We assume that the numbers above are
symmetric in $n$ so that $A_{g,n}(x)$ is a symmetric
polynomial in the variables $x$.

\textbf{Questions:} (1) What polynomials do we obtain?
(2) Symmetric polynomials have different
bases, like the elementary symmetric polynomials. Do
the $\{A_{g,n}\}$ behave nicely for either of these 
bases? 

\subsection{Witten--Kontsevich}

Let $\overline{\mathcal{M}}_{g,n}$ be the 
Deligne--Mumford--Knudsen moduli space of 
stable singular genus $g$ curves with $n$
marked points. 

Each curve $C$ in this space admits 
a marked point labeled $i$ which is 
a smooth point, and has a cotangent bundle
$T_{p_i}^*(C)$. This gives a bundle 
$\mathcal L_i$ over $\overline{\mathcal{M}}_{g,n}$
and the number $\langle \tau_{d_1},\ldots,\tau_{d_n}\rangle_g$
is in fact 
\[ \int_{\overline{\mathcal{M}}_{g,n}}
 \psi_1^{d_1}\cdots \psi_n^{d_n} \]
 where  $\psi_i = c_1(\mathcal{L}_i)$. Here
 $c_1(E)\in H^2(X,\mathbb Q)$ is the Chern class
 of a bundle, defined [as follows].
 
 \textbf{Potential.} 
 Define the \emph{potential function} for $\overline{M}$ as
 follows:
 \[ F(t) = \sum_{3g-3+n\geqslant 0} \langle \tau_{d_1},\ldots,\tau_{d_n}\rangle_g t^{\vec{d}} / \vec{d}! \]
 
 \begin{theorem}[Kontsevich]
 If $U = \partial^2_0 F$ then the following Korteweg--De Vries
 equation holds:
 \[ 
 	\partial_1 U = 
 	 	U \partial_0 U + \frac{1}{12} \partial^3_0 U.\]
 	 	\end{theorem}
 	 	\subsection{Some closed formulas}
 	 In genus zero, we have 
 	 that	\[ \int_{\overline{\mathcal{M}}_{0,n}}
 \psi_1^{d_1}\cdots \psi_n^{d_n} = \binom{n-3}{d_1,\ldots,d_n} \]
 	 	so that
 	$A_{0,n} = e_1^{n-3}$. In genus one, we have that
 	 \[ \int_{\overline{\mathcal{M}}_{1,n}}
 \psi_1^{d_1}\cdots \psi_n^{d_n} = 
 	\frac{1}{24} \left( \binom{n}{d_1,\ldots,d_n} 
 	  - \sum_{b_1,\ldots,b_n \in \{0,1\}} \binom{n - b}{d_i - b_i}  (b - 2)!\right) 
 	  	\]
 	  	where $b = \sum b_i$ (and at least two $b_i$ are non
 	  	zero), so that
 	  	\[ A_{1,n} = e_1^n - \sum_{k=2}^n (k-2)! e_k e_1^{n-k}.\]
 	  	
 	  	\begin{theorem}
 	  	There exists an algorithm to explicitly compute
 	  	$A_{g,n}$ for a fixed $g\geqslant 0$. In particular,
 	  	there exists a way to compute them in the basis of
 	  	elementary symmetric polynomials, of the form
 	  	\[ 
 	  	 A_{g,n}(x) = \sum_{\lambda} C_{g,n}(\lambda)
 	  	  	e_\lambda e_1^{n-\ell(\lambda)} ,\]
 	  	  	which stabilizes for $n\geqslant 2g-2$.
 	  	\end{theorem}
 	  	
 	  	\begin{conj}[Eynard--Lewanski--Ooms]
 	  	If we write $A_{g,n}$ in the basis of elementary
 	  	symmetric polynomials, there should exist a
 	  	clear pattern where some coefficients are
 	  	zero:
 	  	\[ A_{g,n}(x) = 
 	  	 	\sum_{\substack{|\lambda|\leqslant 3g-3+n, \\
 	  	 	\lambda_i\geqslant 2,\ell(\lambda)\leqslant g}} C_{g,n}(\lambda) 
 	  	 	e_\lambda e_1^{3g-3+n-|\lambda|}\]  	
 	  	\end{conj}
 	  	The conjecture has been checked for $g\leqslant 7$
 	  	and every $n$, and for $n\leqslant 3$ for every $g\geqslant 0$ ($n=1$ by Witten, $n=2$ by Dijkgraaf, $n=3$ by Zagier). 
% 	  	graphs on surfaces
% 	  	counting surfaces
 	  	
 \section{August 22nd: G. Baverez}
 
 \textbf{Liouville Conformal Field Theories.} 
  Gaussian free field.
 Gaussian multiplicative chaos.
 
\subsection{Motivation}

Let $\Sigma$ be a Riemann surface with a Riemannian
metric $g$, and let us write
\[ [g] = \{ e^{2\omega} g : \omega \in C^\infty(\Sigma)\}. \]
 
 Solutions to the Liouville equation
 \[ K(e^{2\omega}g ) = e^{-2\omega}(K_g - 2\Delta_g \omega)\]
 are the minimizers of
 \[ S_L(\omega) = 
 	\frac{1}{2\pi} \int_\Sigma \left(
 		|d\omega|^2_g + K_g \omega
 	 	+ 2\pi \mu e^{2\sigma} \right) d\mathsf{vol}_g\]
 	 	
 	 	where $e^{-2\omega}(K_g- 2\Delta_g \omega) = -2\pi \mu$. One can modify this operator as follows:
 	  	 	\[ S_L(\omega,\gamma) = 
 	\frac{1}{2\pi} \int_\Sigma \left(
 		|d\omega|^2_g +Q  K_g \omega
 	 	+ 2\pi \mu e^{\gamma\sigma} \right) d\mathsf{vol}_g\]
 	 	where $Q =2/\gamma+\gamma/2$. 
 	 	
 	 	\textbf{Goal:} make sense of the path integral
 	 	\[ P = \int e^{-S_L(\omega)} D\omega ,\] 
 	 	which is not defined for various reasons: $S_L$ is
 	 	actually a distribution, and the space of all $\omega$s
 	 	is infinite dimensional.
 	 	
 	 	\subsection{White noise}
 	 	
 	 	Let $H$ be a Hilbert space, and let $\xi_v$ be
 	 	random variables $\xi_h : \Omega \longrightarrow 
 	 	\mathbb R$ for $h\in H$ defined as follows: pick
 	 	a basis $(e_n)$ of $H$, and let $\xi_{e_n}$
 	 	be normal Gaussian $N(0,1)$ and i.i.d., and extend this
 	 	linearly to $H$. We have that
 	 	$
 	 	 \mathbb E[\xi_v\xi_w] = \langle v,w\rangle.
 	 	 $
 	 	We call the datum $H \longrightarrow \mathcal F(\Omega)$
 	 	a \emph{white noise on $H$}.  
 	 	
 	 	If $F\in L^2(\Omega)$ is smooth and non-zero on
 	 	finitely many directions, then there exist $f$
 	 	smooth such that $F = f(\xi_1,\ldots,\xi_n)$, 
 	 	and we write
 	 	\[ dF = \sum_{i=1}^n \partial_k f e_k .\]
 	 	
 	 	Gaussian: 
 	 	\[ 
 	 	\mathbb E(e^{a\xi_u - \alpha^2/2 \mathbb E(\xi_n^2)} F)
 	 	 = \mathbb E(f(\xi+\alpha u))
 	 	\]
 	 	
 	 	$(\Sigma,g)$ with $-\Delta_g \geqslant 0$. Let
 	 	$H^1(\Sigma)^\circ = H^1(\Sigma)/\mathbb R$  
\end{document}
