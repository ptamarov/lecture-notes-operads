
\documentclass[fleqn, a4paper, twoside]{article} 

%% for print
\usepackage[
	top = 1.15 in, 
	bottom = 1.25 in,
	left = 1.15 in, 
	right = 1.15 in,
	includehead]{geometry}
% font sizes
\usepackage{scrextend}
\changefontsizes{12pt}
% for online reading
%\usepackage[
%	top = 1.15 cm, 
%	bottom = 2.25 cm,
%	left = 1.15 cm, 
%	right =1.15 cm,
%	includehead]{geometry}
% font sizes
%\usepackage{scrextend}
%\changefontsizes{14pt}

\usepackage[all]{nowidow}
\usepackage{microtype}
\usepackage{changepage}

\usepackage[utf8x]{inputenc}

\usepackage{amscd,amssymb,amsmath}
\usepackage{amsrefs}

%nimbus roman
\usepackage{mathptmx}
\usepackage[T1]{fontenc}


\usepackage{booktabs}
\newcommand{\ra}[1]{\renewcommand{\arraystretch}{#1}}

\newcommand{\lead}[1]{\operatorname{LT}(#1)}
\newcommand{\leadm}[1]{\operatorname{LM}(#1)}
\newcommand{\repl}[3]{\square_{#1}^{#2}(#3)}
\newcommand{\leadc}[1]{\operatorname{LC}(#1)}
\usepackage[utf8x]{inputenc}
\usepackage{amsfonts,amssymb,amsmath,amsrefs}
\usepackage{graphicx}
\usepackage[british]{babel}
\usepackage{caption}
\usepackage{mathdots}
\usepackage{mathtools} 
\usepackage{amsmath}
\usepackage{amsfonts}
\usepackage{amssymb}
\usepackage{graphicx}
\usepackage{subfigure}
\usepackage{makeidx}
\usepackage{multicol}
\usepackage{array}
\usepackage{cancel}
\usepackage{polynom}
\usepackage[pdf,all]{xy}
\xyoption{line}
\CompileMatrices

\usepackage{xcolor}
% Link colors and styles, change accordingly, these are the official colours of Trinity 
\definecolor{trinityblue}{rgb}{0.05, 0.45,0.75}
\definecolor{trinitygray}{rgb}{0.33, 0.34,0.35}

%% Tree macros %%

\newcommand{\fork}[7]{
\begin{tikzpicture}[scale = .75]
\tikzstyle{inner}=[circle,draw=black, fill=white, inner sep=1pt,minimum size=5pt]
\tikzstyle{leaf}=[circle, draw=white, fill=white, inner sep=3 pt,minimum size=5 pt]
	\begin{pgfonlayer}{main}
		\node [style=inner] (0) at (0, 0) {$#3$};
		\node [style=inner] (2) at (-1, 1) {$#1$};
		\node [style=inner] (9) at (1, 1) {$#2$};
		\node [style=leaf] (10) at (0, -1.25) {};
		\node [style=leaf] (11) at (-1.5, 2) {$#4$};
		\node [style=leaf] (12) at (-0.5, 2) {$#5$};
		\node [style=leaf] (13) at (0.5, 2) {$#6$};
		\node [style=leaf] (14) at (1.5, 2) {$#7$};
	\end{pgfonlayer}
	\begin{pgfonlayer}{bg}
		\draw (2.center) to (0.center);
		\draw (0.center) to (9.center);
		\draw (0.center) to (10.center);
		\draw (2.center) to (11.center);
		\draw (12.center) to (2.center);
		\draw (13.center) to (9.center);
		\draw (9.center) to (14.center);
	\end{pgfonlayer}
\end{tikzpicture}}

\newcommand{\leftc}[5]{
	\begin{tikzpicture}[scale = .75]
\tikzstyle{inner}=[circle,draw=black, fill=white, inner sep=1pt,minimum size=5pt]
\tikzstyle{leaf}=[circle, draw=white, fill=white, inner sep=3 pt,minimum size=5 pt]
\begin{pgfonlayer}{main}
		\node[style = inner] (0) at (-1, 1) {$#1$};
		\node[style = inner] (1) at (0, 0) {$#2$};
		\node[style = leaf] (2) at (1, 1) {$#5$};
		\node[style = leaf] (3) at (0, 2) {$#4$};
		\node[style = leaf] (4) at (-2, 2) {$#3$};
		\node (5) at (0, -1) {};
		\end{pgfonlayer}
	\begin{pgfonlayer}{bg}
		\draw (1) to (0);
		\draw (2.center) to (1);
		\draw (3.center) to (0);
		\draw (4.center) to (0);
		\draw (5.center) to (1);
		\end{pgfonlayer}	
\end{tikzpicture}
}

\newcommand{\Leftc}[7]{
	\begin{tikzpicture}[scale = .75]
\tikzstyle{inner}=[circle,draw=black, fill=white, inner sep=1pt,minimum size=5pt]
\tikzstyle{leaf}=[circle, draw=white, fill=white, inner sep=3 pt,minimum size=5 pt]
\begin{pgfonlayer}{main}
		\node[style = inner] (0) at (-1, 1) {$#2$};
		\node[style = inner] (1) at (0, 0) {$#1$};
		\node[style = leaf] (2) at (1, 1) {$#7$};
		\node[style = leaf] (3) at (0, 2) {$#6$};
		\node[style = inner] (4) at (-2, 2) {$#3$};
		\node[style = leaf] (5) at (-3, 3) {$#4$};
		\node[style = leaf] (6) at (-1, 3) {$#5$};
		\node (7) at (0, -1) {}; %theRoot
		\end{pgfonlayer}
	\begin{pgfonlayer}{bg}
		\draw (1) to (0);
		\draw (2.center) to (1);
		\draw (3.center) to (0);
		\draw (4.center) to (0);
		\draw (7.center) to (1);
		\draw (4.center) to (6);
		\draw (4.center) to (5);
		\end{pgfonlayer}	
\end{tikzpicture}
}

\newcommand{\Leftr}[7]{
	\begin{tikzpicture}[scale = .75]
\tikzstyle{inner}=[circle,draw=black, fill=white, inner sep=1pt,minimum size=5pt]
\tikzstyle{leaf}=[circle, draw=white, fill=white, inner sep=3 pt,minimum size=5 pt]
\begin{pgfonlayer}{main}
		\node[style = inner] (0) at (-1, 1) {$#2$};
		\node[style = inner] (1) at (0, 0) {$#1$};
		\node[style = leaf] (2) at (1, 1) {$#7$};
		\node[style = inner] (3) at (0, 2) {$#3$};
		\node[style = leaf] (4) at (-2, 2) {$#6$};
		\node[style = leaf] (5) at (-1, 3) {$#4$};
		\node[style = leaf] (6) at (1, 3) {$#5$};
		\node (7) at (0, -1) {}; %theRoot
		\end{pgfonlayer}
	\begin{pgfonlayer}{bg}
		\draw (1) to (0);
		\draw (2.center) to (1);
		\draw (3.center) to (0);
		\draw (4.center) to (0);
		\draw (7.center) to (1);
		\draw (3.center) to (6);
		\draw (3.center) to (5);
		\end{pgfonlayer}	
\end{tikzpicture}
}


\newcommand{\rightc}[5]{
\begin{tikzpicture}[scale = .75]
\tikzstyle{inner}=[circle,draw=black, fill=white, inner sep=1pt,minimum size=5pt]
\tikzstyle{leaf}=[circle, draw=white, fill=white, inner sep=3 pt,minimum size=5 pt]
\begin{pgfonlayer}{main}
		\node[style = leaf] (0) at (-1, 1) {$#3$};
		\node[style = inner] (1) at (0, 0) {$#1$};
		\node[style = inner] (2) at (1, 1) {$#2$};
		\node[style = leaf] (3) at (0, 2) {$#4$};
		\node[style = leaf] (4) at (2, 2) {$#5$};
		\node (5) at (0, -1) {};
		\end{pgfonlayer}
	\begin{pgfonlayer}{bg}
		\draw (1) to (0);
		\draw (2.center) to (1);
		\draw (3.center) to (2);
		\draw (4.center) to (2);
		\draw (5.center) to (1);
		\end{pgfonlayer}	
\end{tikzpicture}
}
\newcommand{\mathsymbol}[2]{
	\begin{tikzpicture}
        \node at(0,0){};
        \node at(0,#1){$#2$};
    \end{tikzpicture}
}
%% Side captions %%
\usepackage{sidecap}

%% Exercises %%
\usepackage{exsheets} %Para guias de ej.
\SetupExSheets{headings = runin}

\usepackage{wrapfig} %%https://es.sharelatex.com/learn/latex/Wrapping_text_around_figures

\usepackage{lipsum}
%% SMALL HYPHEN %%
\mathchardef\hy="2D % Define a "math hyphen"

 %% TAMAÑO C\eLDAS %%     
\usepackage{pgf,tikz}

\pgfdeclarelayer{bg}    % declare background layer
\pgfsetlayers{bg,main}  % set the order of the layers (main is the standard layer)


\usepackage{tikz-cd}
\usetikzlibrary{calc}
\usetikzlibrary{matrix,arrows}
\usepackage{stackrel}
\usepackage[shortlabels]{enumitem} %Para listas estilo Weibel
\usepackage{stmaryrd} %Para double brackets
\usepackage{setspace} %Para espaciado de renglones
\spacing{1.15}
%\onehalfspacing %Mejor lectura
\usepackage{etoolbox}
\usetikzlibrary{trees}


\patchcmd{\section}{\normalfont}{\normalfont\large}{}{}

\usepackage[bb=ams, cal=euler, scr=rsfso , frak=euler]{mathalpha}

%%%Sha
  \DeclareFontFamily{U}{wncy}{}
    \DeclareFontShape{U}{wncy}{m}{n}{<->wncyr10}{}
    \DeclareSymbolFont{mcy}{U}{wncy}{m}{n}
    \DeclareMathSymbol{\ShaL}{\mathord}{mcy}{"58} 
    
    \usepackage{scalerel}[2016/12/29]

\newcommand{\Sha}{{\scaleobj{0.75}{\ShaL}}}
 
 
\usepackage{fancyhdr}
\pagestyle{fancy}
\fancyhead[RE]{\small\it Algebraic operads}
\fancyhead[LO]{\small\it Winter Lecture Series 2021/2022}
\fancyhead[RO,LE]{\small\textbf\thepage}
\renewcommand{\headrulewidth}{0 pt}
\cfoot{}
\newcommand{\0}{\langle 0\rangle}

\fancypagestyle{references}
{\fancyhead[RE]{\small\it References}
\fancyhead[LO]{\small\it References}
\fancyhead[RO,LE]{\small\bf\thepage}
\fancyfoot[L,R,C]{}
\renewcommand{\headrulewidth}{0 pt}}

 %%%%%%%%%%%%TikZ Styles%%%%%%%%%%%

%%%%%%%%%%%%%%

\newcommand{\XX}{\mathcal{X}}
\renewcommand{\AA}{\mathcal{A}}
\newcommand{\YY}{\mathcal{Y}}
\newcommand{\End}{\operatorname{End}}
\newcommand{\RR}{\mathcal{R}}
\newcommand{\II}{\mathcal{I}}
\newcommand{\QQ}{\mathcal{Q}}
\newcommand{\FF}{\mathcal{F}}
\newcommand{\GG}{\mathcal G}

\newcommand{\f}{\mathsf{f}}
\newcommand{\ari}{\operatorname{ar}}
\newcommand{\cone}{\operatorname{cone}}
\newcommand{\hem}{\hspace{0.5 em}} 
%\input{treesp.tex} For trees, maybe add again later

 \newcommand{\stt}{\mathbin{\text{\tikz 
[x=1ex,y=1ex,line width=.1ex,line 
join=round] \draw (0,0) rectangle (1,1) 
(1,1) -- (0,0)  (1,0) -- (0,1);}}}

%%%%%% ENUMERATE STUFF %%%%%%
\usepackage{enumitem}
\listfiles
\setlist[enumerate]{label= (\arabic*)}


\newenvironment{tenumerate}{
 \begin{enumerate}
  \setlength{\itemsep}{0pt}
  \setlength{\parskip}{0pt}
}{\end{enumerate}}

\newenvironment{titemize}{
\begin{itemize}
  \setlength{\itemsep}{0pt}
  \setlength{\parskip}{0pt}
}{\end{itemize}}

\definecolor{newcol}{rgb}{0,0,0}
\definecolor{deepblue}{rgb}{0.0, 0.28, 0.67}
%{0.5, 0.0, 0.13}
\DeclareTextFontCommand{\new}{\color{black}\em}
%\DeclareTextFontCommand{\new}{\color{black}\em}

\usepackage[pdftex, colorlinks,bookmarks 
= true,bookmarksnumbered = true]{hyperref}

% url colors
\hypersetup{colorlinks,
	linkcolor={trinityblue},
		citecolor={trinityblue},
			urlcolor={trinityblue}} 
			
\usepackage{sectsty}
\chapterfont{\color{newcol}}  % sets colour of chapters
\sectionfont{\color{newcol}}  % sets colour of sections
\subsectionfont{\color{newcol}}  % sets colour of sections

% let \[ and \] be the same as \begin{equation} and \end{equation}
\makeatletter
\AtBeginDocument{%
  \let\[\@undefined
  
\DeclareRobustCommand{\[}{\begin{equation}}%
  \let\]\@undefined
  
\DeclareRobustCommand{\]}{\end{equation}}%
}
\makeatother 
% but only print equation numbers if needed
\mathtoolsset{showonlyrefs,showmanualtags}

\usepackage{amsthm}
\usepackage{thmtools}
\newtheoremstyle{mytheorem}
  {\topsep}   % ABOVESPACE
  {\topsep}   % BELOWSPACE
  {\itshape}  % BODYFONT
  {0pt}       % INDENT (empty value is the same as 0pt)
  {\bfseries\color{newcol}} % HEADFONT
  {\color{newcol}}         % HEADPUNCT
  {5pt plus 1pt minus 1pt} % HEADSPACE
  {}          % CUSTOM-HEAD-SPEC
  
\theoremstyle{mytheorem}
\newtheorem{theorem}{Theorem}[section]
\newtheorem{proposition}[theorem]{Proposition}
\newtheorem{corollary}[theorem]{Corollary}
\newtheorem{lemma}[theorem]{Lemma}
\newtheorem*{conj*}{Conjecture}
%\newtheorem{question}[theorem]{Question}

 \newtheoremstyle{introthm}
  {\topsep}   % ABOVESPACE
  {\topsep}   % BELOWSPACE
  {\itshape}  % BODYFONT
  {0pt}       % INDENT (empty value is the same as 0pt)
  {\bfseries\color{newcol}} % HEADFONT
  {\color{newcol}{.}}         % HEADPUNCT
  {5pt plus 1pt minus 1pt} % HEADSPACE
  {}          % CUSTOM-HEAD-SPEC
 
  
\theoremstyle{introthm}
\newtheorem{introthm}{Theorem}
\newtheorem*{introcor}{Corollary}
\renewcommand{\theintrothm}{\Alph{introthm}}
 % Para teoremas en la introducción

%%%%%%%% TODO NOTES %%%


%% TODO NOTES! %%
\usepackage{xargs}                      
% Use more than one optional parameter in a new commands
%\usepackage[pdftex,dvipsnames]{xcolor}  
% Coloured text etc.
\usepackage[colorinlistoftodos,prependcaption,textsize=small]{todonotes}

\newcommand{\hacer}[1]{\todo[inline,linecolor=blue,backgroundcolor=blue!25!white,bordercolor=blue]{#1}}
\newcommandx{\unsure}[2][1=]{\todo[linecolor=blue,backgroundcolor=blue!25!white,bordercolor=blue,#1]{#2}}
\newcommandx{\change}[2][1=]{\todo[linecolor=blue,backgroundcolor=blue!25,bordercolor=blue,#1]{#2}}


\definecolor{col1}{rgb}{0.8, 0.8, 1.0}
\definecolor{col2}{rgb}{0.9, 0.9, 0.98}
\definecolor{col3}{rgb}{0.71, 0.49, 0.86}

%%%% Write pseudo-code
\usepackage{algorithm}
\usepackage[noend]{algpseudocode}

\makeatletter
\def\BState{\State\hskip-\ALG@thistlm}
\makeatother


\newcommand{\ind}{\operatorname{Ind}_\PP}

\newcommand{\cof}{\rightarrowtail}
\newcommand{\foc}{\leftarrowtail}
\newcommand{\weq}{\overset{\sim}{\longrightarrow}}
\newcommand{\tcf}{\overset{\sim}{\cof}}

\newtheoremstyle{mydefinition}
  {\topsep}   % ABOVESPACE
  {\topsep}   % BELOWSPACE
  {}  % BODYFONT
  {0pt}       % INDENT (empty value is the same as 0pt)
  {\bfseries\color{newcol}} % HEADFONT
  {\color{newcol}}         % HEADPUNCT
  {5pt plus 1pt minus 1pt} % HEADSPACE
  {}          % CUSTOM-HEAD-SPEC
  
\theoremstyle{mydefinition}
\newtheorem{definition}[theorem]{Definition}
\newtheorem{obs}[theorem]{Observation}
\newtheorem{rmk}[theorem]{Remark}
\newtheorem{problem}[theorem]{Problem}
\newtheorem{example}[theorem]{Example}
\newtheorem{note}[theorem]{Note}
\newtheorem{variante}[theorem]{Variant}


\newtheoremstyle{mydefinition2}
  {\topsep}   % ABOVESPACE
  {\topsep}   % BELOWSPACE
  {}  % BODYFONT
  {0pt}       % INDENT (empty value is the same as 0pt)
  {\bfseries\color{newcol}} % HEADFONT
  {\color{newcol}{.}}         % HEADPUNCT
  {5pt plus 1pt minus 1pt} % HEADSPACE
  {}          % CUSTOM-HEAD-SPEC
  
\theoremstyle{mydefinition2}
\newtheorem*{definition*}{Definition}
\newtheorem*{remark*}{Remark}
\newtheorem*{obs*}{Observation}
\newtheorem*{example*}{Example}

% named theorem %

% for specifying a name
\theoremstyle{plain} % just in case the style had changed
\newcommand{\thistheoremname}{}
\newtheorem{genericthm}[theorem]{\thistheoremname}
\newenvironment{namedthm}[1]
  {\renewcommand{\thistheoremname}{#1}%
   \begin{genericthm}}
  {\end{genericthm}}

\newcommand{\imor}{\interleave\kern-.45em\longrightarrow}
%\renewcommand{\qedsymbol}{$\blacktriangleleft$}
\newcommand\place{\mathord-}
\newcommand{\ps}{\mathbin\parallel}
\newcommand{\GSet}{\mathsf{Fin}^\times}
\newcommand{\FSet}{\mathsf{Fin}}
\newcommand{\Fun}{\mathsf{Fun}}
\newcommand{\Set}{\mathsf{Set}}
\newcommand{\DCSH}{\mathsf{DCSH}}
\newcommand{\kMod}{{}_\kk\mathsf{Mod}}
\newcommand{\kmod}{{}_\kk\mathsf{mod}}
\newcommand{\kCh}{{}_\kk\mathsf{Ch}}
\newcommand{\Sp}{\mathsf{Sp}}
\newcommand{\kSp}{{}_\kk\mathsf{Sp}}
\newcommand{\GVec}{\mathsf{Vec}_q^\times}
\newcommand{\Opr}{\mathsf{Opr}}
\newcommand{\NS}{\mathsf{NS}}
\DeclarePairedDelimiter\abs{\lvert}{\rvert}
\newcommand{\vv}{\vert}
\newcommand{\Der}{\operatorname{Der}}
\newcommand{\HH}{\mathrm{HH}}
\newcommand{\As}{\mathsf{As}}
\newcommand{\Com}{\mathsf{Com}}
\newcommand{\Lie}{\mathsf{Lie}}

\newcommand{\aut}{\operatorname{aut}}
\definecolor{sqsqsq}{rgb}{0.13,0.13,0.13}
\definecolor{aqaqaq}{rgb}{0.63,0.63,0.63}

\newcommand{\SC}{SC}
\newcommand\id{\mathrm{id}}
\newcommand\category[1]{\mathsf{#1}}
\newcommand{\Mod}{\mathsf{Mod}}
\newcommand{\Gmod}{\mathsf{GMod}}
\newcommand{\nsMod}{{}_{\mathrm{ns}}\mathsf{Mod}}

\newcommand{\SMod}{{}_\Sigma\mathsf{Mod}}
\newcommand{\dgSMod}{{}_\Sigma\mathsf{dgMod}}
\newcommand{\gSMod}{{}_\Sigma\mathsf{gMod}}
\newcommand{\coker}{\operatorname{coker}}
\newcommand{\Ho}{\operatorname{Ho}}
\newcommand\spe[1]{\mathcal{#1}}
\renewcommand{\tt}{\otimes}
\newcommand{\E}{\mathcal{E}}
\newcommand{\CC}{\mathcal{C}}
\newcommand{\DC}{\mathsf{DC}}
\newcommand{\DA}{\mathsf{DA}}
\newcommand{\dd}{\partial}
\newcommand{\OO}{\mathcal O}
\newcommand{\?}{\,?\,}
\newcommand{\n}{[n]}
\newcommand\cls[1]{\llbracket#1\rrbracket}
\newcommand{\NN}{\mathbb N}
\renewcommand{\k}{[k]}
\newcommand{\bt}{\bullet}
\newcommand{\kk}{\Bbbk}
\newcommand{\Aut}{\operatorname{Aut}}
\newcommand{\Ext}{\operatorname{Ext}}
\newcommand{\Tor}{\operatorname{Tor}}
\newcommand{\PAlg}{\mathsf{Alg}_{\geqslant 0}}
\newcommand{\Cog}{\mathsf{Cog}}
\newcommand{\Alg}{\mathsf{Alg}}
\newcommand{\Cxs}{\mathsf{Ch}_\kk}
\newcommand\inter[1]{\llbracket#1\rrbracket}
\newcommand{\Cell}{\operatorname{Cell}}
\newcommand{\Sing}{\operatorname{Sing}}
\newcommand{\Sull}{A_{\mathrm{PL}}}

\newcommand\lab{\mathsf{lab}}
\newcommand{\Q}{{\mathbb{Q}}}
\newcommand{\Z}{{\mathbb{Z}}}
\newcommand{\hoq}{\!\sslash\!\!}
\newcommand{\PP}{{\mathcal{P}}}

\definecolor{newterm-color}{RGB}{0, 0, 0}
\newcommand\newterm[1]{%
  \textcolor{newterm-color}{\itshape #1}%
}

\theoremstyle{mytheorem}
\newtheorem*{theorem*}{Theorem}
\newtheorem*{question*}{Question}

% named theorem %

% for specifying a name
\theoremstyle{plain} % just in case the style had changed
% front matter stlye 
\renewenvironment{abstract}{%
\small\begin{center}
\begin{minipage}{.9\textwidth}
%\textbf{\textcolor{newcol}{Abstract.}}
}
{\par\noindent\end{minipage}\end{center}\vspace{3 em}}
%
\makeatletter
\renewcommand\@maketitle{%
\hfill
\begin{center}\begin{minipage}{0.9 	\textwidth}
\centering
\vskip 6em
\let\footnote\thanks 
{\LARGE \@title \par }
\vspace{1 em}
%\hrulefill
\vskip 1 em
{\large \@author \par}
\vspace{3.5 em}

\end{minipage}\end{center}
\par
}
\makeatother
%
%%%%%%%%%%%%%%%%%%%%%%%%%%%%%%%%%%%%%%%%%%%%%%%%%%%%%%


\DeclareMathOperator\sgn{sgn}
\DeclareMathOperator\fsch{\mathfrak{sch}}

%\definecolor{newterm-color}{RGB}{0, 51, 153}
%\newcommand\newterm[1]{%
%\textcolor{newterm-color}{\bfseries\itshape #1}%
%}

% textual claims in equations
\newcommand\claim[2][.8]{%
  \begin{minipage}{#1\displaywidth}%
  \itshape
  #2
  \end{minipage}%
}

\usepackage{afterpage}

\newcommand\blankpage{%
    \null
    \thispagestyle{empty}%
    %\addtocounter{page}{-1}%
    \newpage}


\usepackage{float}

\tikzcdset{arrow style=tikz, diagrams={>=stealth}}

%%Addresses

\newcommand{\Addresses}{{% additional braces for segregating \footnotesize
  \bigskip
  \footnotesize
  \textsc{Office F310, Non-linear Algebra Group}, MPIMiS, Leipzig, Germany \par\nopagebreak
  \textit{Addresses:} tamaroff@mis.mpg.de}
}
  

\usepackage{titletoc}

\titlecontents{chapter}
[0.2em] %
{\bigskip}
%{\contentslabel[\thecontentslabel.]{2em}\hspace{0.667em}}%\thecontentslabel
{\makebox[2em][r]{\thecontentslabel.}\hspace{0.333em}}%\thecontentslabel
{\hspace*{-2em}}
{\hfill\contentspage}[\smallskip]

\titlecontents{section}% <section>
[0.2em]% <left>
{\small}% <above-code>
{\thecontentslabel.\hspace{3pt}}%<numbered-entry-format>; you could also 
%use  {\thecontentslabel. } to show the numbers
{}% <numberless-entry-format>
{\enspace\titlerule*[0.5pc]{.}\contentspage}%<filler-page-format>
\titlecontents*{subsection}% <section>
[1em]% <left>
{\footnotesize}% <above-code>
{\thecontentslabel. \hspace{3pt}}% <numbered-entry-format>; you could also 
%use {\thecontentslabel. } to show the numbers
{}% <numberless-entry-format>
{}% <filler-page-format>
[ --- \ ]% <separator>
[]% <end>
\setcounter{tocdepth}{2}% Display up to \subsection in ToC

\setlength\parskip{3 pt}
\setlength\parindent{0 em}

\raggedbottom 
\makeindex

\title{\vspace{-5 em}\setstretch{0.85}{\textbf{Algebraic operads, Koszul duality and Gr\"obner bases: an introduction}}}
\author{P. Tamaroff}
\date{August 22 and 24}
%\address{}
%\email{ptamarov@gmail.com}
\begin{document}
\maketitle


%\tableofcontents
%\section{Introduction}\label{sec:intro}
%\addcontentsline{toc}{section}{\nameref{sec:intro}}
\thispagestyle{empty}

\begin{abstract}
This lecture series aim to offer a gentle introduction
to the theory of algebraic operads, starting with the
elements of the theory, and progressing slowly towards
more advanced themes, including (inhomogeneous)
Koszul duality theory, Gr\"obner bases and higher
structures. The course  will consists of approximately
twelve lectures, along with extra talks by
willing participants, with the goal of introducing extra
material to the course, and making them more
familiar with the theory.
%We will survey the Koszul duality theory introduced by Ginzburg--Kapranov and, more particularly, it's extension to inhomogenous 
%(quadratic-linear) operads, appearing originally in the work of 
%G\'alvez-Carrillo--Tonks--Vallette. We'll do this with a view 
%towards its application to the computation of the homotopy 
%quotient of the BV operad by the circle action
%(Khoroshkin--Markarian--Shadrin, etc) and the corresponding 
%geometrical statement (Drummond-Cole). If time permits, I'll 
%mention some other interesting (old and new) related results 
%related to non-commutative analogues, among others. 
\end{abstract}


\vfill


\hfill	Leipzig, \today


\afterpage{\blankpage}

\newpage

\vspace*{\fill} 
\emph{These notes were written during the Winter Lecture
Series for the academic term 2021-2022 at the \emph{Max
Planck Institut f\"ur Mathematike in den Naturwissenschaften}.
We acknowledge the excellent working
conditions during the time the lecture series took place,
which in particular allowed to produce these lecture notes. }
\vspace*{\fill} 
 \afterpage{\blankpage}

\newpage

\thispagestyle{empty}

\tableofcontents

\vfill

\hfill \texttt{Draft: \today} 

\afterpage{\blankpage}

\pagebreak

\thispagestyle{empty}


\section{Motivation and history}

\textbf{Goals.} The goals of this lecture is to
give a broad picture of the history and 
pre-history of operads, and some current trends,
and give a road-map for the course.  

\subsection{Introduction and motivation}

Operads (topological operads, more precisely)
originally appeared as tools in algebraic
topology and homotopy theory, 
specifically in the study of iterated loop 
spaces (May, 1972 and Boardman and Vogt before).
They also appeared as \emph{comp algebras} in Gerstenhaber's 
work on Hochschild cohomology and topologically as 
Stasheff's `associahedra' for his homotopy
characterization of loop spaces (both in 1963).
The theory of operads, in particular topological
and algebraic, saw itself very much influenced by
homological algebra, category theory, algebraic
geometry, rational homotopy theory and mathematical
phyisics. Here we list a few examples:

\begin{tenumerate}
\item (Stasheff, Sugawara) Study homotopy associative 
$H$-spaces, Stasheff implicitly discovers a topological
ns operad $K$ with $C_*(K) = \mathsf{As}_\infty$ and a
recognition principle for $A_\infty$-spaces.

\item (Boardmann--Vogt) Study infinite loop spaces,
build a PROP (a version of an $E_\infty$-operad) 
and obtain a recognition principle for infinite
loop spaces.

\item (Kontsevich) Uses $L_\infty$-algebras and
configuration spaces to prove his deformation
quantization theorem that every Poisson manifold
admits a deformation quantization.

\item (Kontsevich) The above is
implied by the formality theorem: the
Lie algebra of polyvector fields is 
$L_\infty$-quasi-isomorphic to the Hochschild
complex, and $f_1 = \mathsf{HKR}$. 

\item (Tamarkin) Approaches this result through 
the formality of the little disks operad $D_2$.
Proves that the Hochschild complex of a polynomial
algebra is \emph{intrinsically formal} as a 
Gerstenhaber algebra.

\item (Manifold calculus) Describes the
homotopy type of embedding spaces as certain 
derived operadic module maps and to
produces their explicit deloopings
using little disk operads, due to 
Goodwillie--Weiss, Boavida de Brito–-Weiss,
Turchin, Arone--Turchin, Dwyer–-Hess,
Ducoulombier–-Turchin. 

\item (Ginzburg--Kapranov, Fresse,
Vallette, Hinich) Koszul duality for algebraic 
operads and cousins allows to develop a robust 
homotopy theory of homotopy algebras, cohomology
theory, deformation theory, Quillen homology, etc. 

\item (Deligne conjecture and variants) The study
of natural operations on the Hochschild complex
of an associative algebra lead to a manifold of 
results beginning with the proof that there is
an action of the little disks operad $D_2$ on
it, and the ultimate version by Markl--Voronov,
who proved that the operad of natural operations
on it has the homotopy type of $C_*(D_2)$.  
\end{tenumerate}

Operads are modeled by trees (planar or
non-planar, rooted or not), and relaxing these
graphs allows us to produce other type of 
algebraic structures. The following table
gives the reader a ``taxonomy cheat sheet''
for operads and their kin; we will, for better
or worse, defer from diving into the curious
world that lies beyond operads, but encourage
the reader to do this for themselves (and find
out what ``wheeled structures'' are, and how
they fit in the table below).

\begin{center}
\begin{tabular}{@{}llcc@{}} \toprule
Type & Graph & Compositions & Due to \\ \midrule
PROPs & Any graph & Any  & Adams--MacLane \\
Modular & Any graph & $\xi_{i,j}$, $\circ_{i,j}$  & Getzler--Kapranov\\ 
Properads & Connected graphs & Any & B. Vallette \\
Dioperads & Trees & ${}_i\circ_j$ (no genus) & W. L. Gan \\
Half-PROPs & Trees & $\circ_j$ , ${}_i\circ$ & 
 Markl--Voronov \\ 
Cyclic operads & Trees &  $\circ_{i,j}$ & Getzler--Kapranov\\ 
Symmetric operads & Rooted trees & $\circ_i$  & J. P. May \\ 
\bottomrule
\end{tabular}
\end{center}

\subsection{Koszul duality}
Koszul duality was invented by Steward Priddy in
the seventies~\cite{Priddy1970}, with the objective of streamlining
computations of certain cohomology theories for
classes of algebras (notably, Lie and associative 
algebras). One of the reasons this was (and still is)
relevant is that such cohomology groups play a central
role in the computation of other more complicated invariants
of algebras and topological spaces: in particular,
the cohomology of the Steenrod algebra famously featured
in Adam's spectral sequence computing the stable homotopy
groups of spheres at each prime. In Priddy's own words:
\[
\claim{
The purpose of this paper is to construct resolutions for a 
large class of algebras which includes the Steenrod algebra
and the universal enveloping algebras. 
It is a basic problem of homological algebra to compute the 
cohomology algebras of various augmented algebras. Unfortunately, 
the canonical tool for attacking this problem ---the bar resolution--- is often intractable. In some instances,
however, one is able to find a simpler resolution.
}
\]
Priddy developed his theory for both ``inhomogeneous''
and ``homogeneous'' quadratic algebras ---those presented
in coordinates by quadratic equations in their variables---
and, while in the homogeneous case his formalism gave the
answer immediately, the inhomogenous case required an
additional step, which nonetheless simplified the existing
methods considerably.  

Although Koszul duality nowadays has a much broader meaning
and casts an immense net in modern day algebra, representation
theory, combinatorics, topology and geometry, 
among other areas of mathematics, in this lecture
series we will follow Priddy's motivation and see it as an
instance of a phenomenon in which certain algebraic objects
have very economical ---and thus computationally and theoretically
useful--- resolutions. An interested reader can 
consult~\cite{KellerKoszul2003,Positselski2011,
Sinha2010,MO329,holstein2021categorical} to obtain a broader
view of this phenomenon, and in particular find a wide
variety of answers to the question ``...but what \emph{exactly}
is Koszul duality?''.  
%
%With the risk of diverging 
%from our initial definition, a ``Koszul duality phenomenon'' 
%comes in the shape of a Quillen equivalence between 
%two model categories of algebraic objects of some kind,
%which allows to consider ``object-wise Koszul duality phenomenona''.
%The latter then induce equivalences between categories of
%``representations'' of such objects.

Naturally, one of the reasons why Koszul duality has cemented
itself in modern day mathematics is that it appears often:
algebraic structures of interest have an inclination to be
quadratic and, when in luck, Koszul. These can be anything
from Lie, commutative or associative algebras, to Feynmann
categories, dg categories, operads and their kin. In this
lecture series, we will focus on algebraic operads: our goal
is to introduce the reader to algebraic operads in general
and to quadratic operads in particular, define what
it means for such operads to be Koszul, and explore
the consequences this property has on the operads
and its representations.

Although, as we
mentioned, we will take a rather old fashioned point of 
view and think of Koszul operads as those operads 
having a ``nice resolution'', we aim to give the reader
a modern outlook on the current methods available to
prove that an operad is Koszul, and some relatively 
new developments in the area from the last two (or
maybe three) decades:
the inhomogenous Koszul duality for (pr)operads due
to Galvez-Carillo--Tonks--Vallette, which
followed the original theory of Ginzburg--Kapranov,
the use of filtered distributive laws of Dotsenko
which followed the methods of Markl, and the general
theory stating Koszul operads give rise to good
notions of algebras up to homotopy, due to Vallette.
Naturally, we will also focus on the classical developments,
and on the effective methods of Hoffbeck and 
Dotsenko--Khoroshkin, which we detail in the next
section.

\subsection{Gr\"obner bases}
\hacer{Write introduction to Groebner bases.}
\hacer{References for introduction.}
\newpage

\subsection{Exercises}

\textbf{A. Symmetric groups.} Operads are meant to encode
operations on objects \emph{along with their symmetries},
which is done through the representation theory of the
symmetric groups. The following exercises will remind
you of some basic facts about them.


\begin{question} Let $I = [n]$ so that $\Aut(I) = S_n$ is 
the symmetric group on $n$ letters. For each ordered 
partition $\pi = (\pi_1,\ldots,\pi_k)$, let $\lambda$
be the ordered partition of $n$ with $\lambda_i = \# \pi_i$
for $i\in [k]$. Show that the permutations of
$[n]$ that preserve $\pi$ determine a subgroup of $S_n$
isomorphic to $S_\lambda := S_{\lambda_1}\times
\cdots \times S_{\lambda_k}$. 
\end{question}


\begin{question} Consider the subgroup of $S_n$ corresponding
to the ordered partition of $[n]$ given by $([1,k],[k+1,n])$, along with the 
inclusion $S_k\times S_{n-k} \hookrightarrow S_n$. Show that
a set of representatives for the cosets of this inclusion
in $S_n$ is given by the \emph{$(k,n-k)$-shuffles}, those
permutations $\sigma\in S_n$ that preserve the linear order in
$[1,k]$ and $[k+1,n]$. Conclude that there are exactly
$\binom nk$ shuffles of type $(k,n-k)$ on $[n]$. Define
shuffles associated to other partitions of $n$.
\end{question}

\medskip
\textbf{B. Categories.} The language of categories and
functors permeates most of modern algebra and geometry,
and in particular is useful to work with operads and
other combinatorial structures defines by graphs. The
following will remind you of some important notions
we will use during the course.

\begin{question} A category $\mathcal C$ is the datum of
a set of objects $\operatorname{Ob}(\mathcal{C})$, and
for each $x,y\in \operatorname{Ob}(\mathcal{C})$ a
set $\mathcal C(x,y)$ of morphisms from $x$ to $y$.
Moreover, we require the existence of an associative
and unital composition law
\[-\circ -: \mathcal C(y,z) \times  \mathcal C(x,y) 
	\longrightarrow \mathcal C(x,z). \] 
	The latter means there are distinguished elements
	$1_x\in \mathcal{C}(x,x)$ for each object of $\mathcal{C}$
	that induce the identity $-\circ 1_x$ an $1_x\circ -$
	of any $\mathcal{C}(-,x)$ and $\mathcal{C}(x,-)$.
	Find examples of categories: sets, finite sets,
	rings, vector spaces, open subsets, posets, and 
	others.
\end{question}

\begin{question} A functor $F: \mathcal{C}_1\longrightarrow 
\mathcal{C}_2$ is a datum that assigns to each object
$x$ of the domain an object $F(x)$ of the codomain,
and to each morphism $f:x\to y$ a morphism $F(f)$ such
that $F(f\circ g) = F(f) \circ F(g)$ and $F(1_x) = 1_{Fx}$
for each pair of composable arrows $f$ and $g$ and each
object $x$ of $\mathcal C_1$. Find examples of functors
between the examples of categories you found above.

\end{question}
\begin{question} A monoidal category is a category $\mathcal C$ 
along with the datum of a bifunctor $\otimes :
\mathcal{C}\times \mathcal{C}\longrightarrow \mathcal{C}$
along with an associator and left and right units. 
A monoidal category is \emph{strict} if the associator and left
and right units are identities.
\begin{enumerate}
\item Expand on the details of these definitions. Define what
a braided monoidal category and what a symmetric monoidal category are.
\item Exhibit monoidal
structures the following categories: sets, vector spaces,
linear representations of a group $G$, topological spaces,
associative algebras, Lie algebras, and others.
\end{enumerate}
 \emph{Hint.} In the case of Lie algebras,
 consider the category of Lie groups with its
 canonical tensor product (the cartesian product) 
 and the functor $G\longmapsto T_e(G)$ to decide
 what the tensor product of two Lie algebras is.
\end{question}


\begin{question}
If $(\mathcal{V},\otimes)$ is a monoidal category, we say $\mathcal{C}$ 
is a $\mathcal V$-enriched category if each hom-set 
$\mathcal{C}(x,y)$ is an object of $\mathcal{V}$ and
there is a composition law 
\[-\circ -: \mathcal C(y,z) \otimes  \mathcal C(x,y) 
	\longrightarrow \mathcal C(x,z). \] 
which consists of morphisms in $\mathcal{V}$, and which is 
associative and unital. Note that an ordinary category is
just a category enriched over the category of sets.
A linear category is a category enriched over the
category of vector spaces, an additive category is a
category enriched over Abelian groups.
Expand on what this means. Find about Abelian
categories, and ponder over the difference: an additive
category is a category with structure, while an
Abelian category is a category with additional properties.
\end{question}

\begin{question}\label{ex:skeleton}
A category $\mathcal D$ is skeletal if no two distinct
objects in it are isomorphic. We say that $\mathcal{D}$ is
the skeleton of $\mathcal{C}$ if:
\begin{tenumerate}
\item It is a full subcategory of $\mathcal{C}$: for each
pair of objects $x,y\in\mathcal{D}$, we have that $\mathcal{D}(x,y) = \mathcal{C}(x,y)$.
\item The inclusion of $\mathcal{D}$ in $\mathcal{C}$ is
essentially surjective: every object of $\mathcal{C}$ is
isomorphic to an object of $\mathcal{D}$.
\item $\mathcal{D}$ is skeletal.
\end{tenumerate}
Show that every small category admits a skeleton, and
compute the skeleton of the following categories: sets,
finite sets, finite dimensional vector spaces over
a field.
\end{question}
\medskip

\begin{question}
Suppose that $x$ is an object in a symmetric monoidal
category $(\mathcal{C},\tau)$. For each $n\in\NN$ and 
each $i\in [n-1]$ define $\tau_i : x^{\otimes n} 
\longrightarrow x^{\otimes n}$ by
\[ \tau_i = 1^{i-1} \otimes \tau \otimes 1^{n-i-1}.\]
Show that the assignment
$(i,i+1)\in S_n\longmapsto \tau_i
	 \in\operatorname{Aut}( x^{\otimes n})$ 
	 is a group homomorphism. \emph{Note.} This
	 produces in particular a map $S_2\longrightarrow 
	 \operatorname{Aut}(x^{\otimes 2})$ that sends
	 the transposition $(12)\in S_2$ to the
	 flip $\tau_{x,x}:x\otimes x\longrightarrow 
	 x\otimes x$.  
\end{question}

\begin{question} A product and permutation category (abbreviated `PROP')
is a monoidal category $\mathcal{C}$ whose set of objects is $\mathbb N = 
\{0,1,2,\ldots\}$ and its tensor product is addition (in particular, it is strict and symmetric). Unravel the definitions:
\begin{tenumerate}
\item Use that $n = 1+\cdots + 1$ to show that $\mathcal{C}(m,n)$ is a right $S_n$-module.
\item Similarly, show that $\mathcal{C}(m,n)$ is also a left $S_m$-module.
\item Show these two actions are compatible (i.e. they commute).
\item Show that the product $+$ induces a \emph{horizontal} composition rule 
\[ \mathcal{C}(m,n) \times \mathcal{C}(m',n') 	\longrightarrow
 	\mathcal{C}(m+m',n+n'). \]
\item Interpret the usual categorical product as a \emph{vertical} composition rule 
\[ \mathcal{C}(n,k) \times \mathcal{C}(m,n) 	\longrightarrow
 	\mathcal{C}(m,k). \]
\end{tenumerate}
Consider the definition of a PROP enriched over a symmetric strict 
monoidal category, like $\mathsf{Vect}$ (these are called $\mathbb{k}$-linear
PROPs). Define the category of PROPs. 
\end{question}

\emph{Note.} For 
each $n\in\mathrm{Ob}(\mathcal C)$ the
object $\mathcal{C}(n,n)$ is a monoid under composition
 that receives a map $S_n\longrightarrow \mathcal{C}(n,n)$.
 Under the interpretation above, the image of $\sigma$
  is equal to both the left and the right action of $S_n$ 
  on the identity map $n\to n$. In particular, the twist
  $\tau$ of $\mathcal{C}$ is equal to $(12)\mathrm{id}_2$,
  and may (or may not) be trivial.
  
  \medskip
  
\textbf{C. Graded spaces and complexes.} When studying
algebraic structures like operads, it will be necessary
to use some tools from homological algebra: graded spaces,
chain complexes, differentials, their homology, among
others. The following exercises are intended to familiarize
you with the elements of homological algebra, but we will
look at them in more detail during the course.

\begin{question} A $\mathbb Z$-graded vector space (usually 
just called a graded vector space) is a vector
space $V$ with a direct sum decomposition
\[ V  = \bigoplus_{n\in\mathbb Z} V_n.\]
If $v\in V_n$ we say that $v$ is homogeneous of
degree $n$. Find out about the category of
graded vector spaces, specifically:
\begin{titemize}
\item What are its (degree zero) morphisms?
\item What are its (degree homogeneous) morphisms?
\item What is the tensor product of two graded spaces?
\item What is the natural isomorphism $V\otimes W\longrightarrow W\otimes V$?
\item How does the last item relate to the `Koszul sign
rule'?
\end{titemize}
\end{question}

\begin{question} A differential graded (dg) vector space,
usually called a complex, is a pair $(V,d)$ where $V$ is 
a graded vector space $V$ and $d: V \to V$ is a homogeneous
map of degree $-1$ such that $d^2=0$. Repeat the previous
exercise replacing $\mathsf{gVect}$ with $\mathsf{Ch}$,
the category of complexes of vector spaces.
\end{question}

\begin{question} If $(V,d)$ is a complex, then $Z(V) = \ker d$ 
is called its space of cycles, and $B(V) = \operatorname{im} d$ 
is called
its space of boundaries. The quotient $Z(V)/B(V)$ is called
the homology of $V$, and is written $H(V)$. Show that a map 
of complexes $f:V\to W$ induces a map $Z(V) \to Z(W)$ and 
in turn a map $H(f) : H(V) \to H(W)$. 
\end{question}
 
\begin{question} (Leisure) Find a book on homological algebra
and read about the \emph{snake lemma} and the \emph{five lemma}.
If you are very motivated, read about double complexes and spectral
sequences.
\end{question} 

\afterpage{\blankpage}

\newpage
\section{Symmetric modules and algebraic operads}

\textbf{Goals.}
We will define
some related gadgets (symmetric collections,
algebras, modules, endomorphism operads)
necessary to introduce operads. 
Then, we define what an operad is (topological,
algebraic, symmetric, non-symmetric). 
We will then give some
(not so) well known examples of topological
and algebraic operads.

\subsection{Basic definitions}
\emph{What is an operad?} A group is a model of
$\operatorname{Aut}(X)$ for $X$ a set, an algebra
is a model of $\End(V)$ for $V$
a vector space. Equivalently, groups are the
gadgets that act on objects by automorphisms,
and algebras are the gadgets that act
on objects by their (linear) endomorphisms. 
Operads are the gadgets that act on
objects through operations with many 
inputs (and one output), and at the same
time keep track of symmetries when
the inputs are permuted.

The underlying objects to operads are known as
\emph{symmetric sequences}: a symmetric sequence
(also known as an $\Sigma$-module or symmetric 
module) is a sequence of vector spaces
$\XX = (\XX(n))_{n\geqslant 0}$ such that for
each $n\in\NN_0$ there is a right action of
$S_n$ on $\XX(n)$. We usually consider \emph{reduced}
$\Sigma$-modules, those for which $\XX(0)=0$.

A map of $\Sigma$-modules is a collection of maps
$(f_n : \XX_1(n) \longrightarrow \XX_2(n)\}_{n\geqslant 0}$,
each equivariant for the corresponding group action. 
This defines the category $\SMod$ of symmetric
sequences, and whenever we think of symmetric sequences
using this definition, we will say we are considering a 
a biased or skeletal approach to them.

In parallel, it is convenient to consider the 
category $\GSet$ of finite sets and bijections.
An object in this category is a finite set $I$,
and a morphism $\sigma : I\longrightarrow J$ is a
bijection. Since every finite set $I$ with $n$
elements is (non-canonically) isomorphic to 
$[n] =\{1,\ldots,n\}$, the following holds:

\begin{lemma} The skeleton of $\GSet$ is
equal to the category with objects the finite sets
$[n]$ for $n\geqslant 0$ and with morphisms the
bijections $[n]\longrightarrow [n]$ (and no morphism
between $[n]$ and $[m]$ if $m\neq n$).
\end{lemma}

\begin{proof}
This is Exercise~\ref{ex:skeleton}.
\end{proof}

We set ${}_\Sigma\Mod  = 
\Fun(\GSet,\mathsf{Vect}^{\mathrm{op}})$,
so that a $\Sigma$-module is a pre-sheaf of vector
spaces $I\longmapsto \XX(I)$ assigning to each
isomorphism $\tau : I\longrightarrow J$ an isomorphism
$\XX(\tau): \XX(J)\longrightarrow \XX(I)$. When we
think of $\Sigma$-modules as pre-sheaves, we will say we 
are taking an unbiased approach, will if we specify only
its values on natural numbers, we will say we are taking the
biased or skeletal approach; we will come back to this later.

With this at hand, 
we can in turn define the \emph{Cauchy product}
of two $\Sigma$-modules $\XX$ and $\YY$
\[ (\XX\otimes_\Sigma \YY)(I) = 
 	\bigoplus_{S\sqcup T= I}
 		 \XX(S)\otimes \YY(T)\] 
where the right-hand is the usual tensor product of
vector spaces
and the sum runs through partitions of $I$ into
two disjoint sets. The symmetric product is then
defined by 
\[ (\XX\circ_\Sigma \YY)(I) 
 	= \bigoplus_{\pi \vdash I} \XX(\pi) 
 		\otimes \YY^{\otimes k}(\pi)\] 
as the sum runs through (ordered) partitions of $I$.
These two products will be central in what follows.

\begin{lemma}
The category ${}_\Sigma\Mod$ with $\circ_\Sigma$ is
monoidal with unit the species taking the value $\kk e_x$ at 
the singleton sets $\{x\}$ and zero everywhere else. The same
category is monoidal for $\otimes_\Sigma$ with unit
the species taking the value $\kk$ at $\varnothing$
and zero everywhere else.
\end{lemma}

We will use the notation $\kk$ for the base field but
also for the unit for the composition product $\circ_\Sigma$,
hoping it will not cause much confusion. It will be useful 
later to think of $\kk$ as a twig or ``stick''.

Observe that the associator for $\circ_\Sigma$ is
not too simple and involves reordering certain
factors of tensor products in $\mathsf{Vect}$. In
particular, replacing vector spaces by graded vector
spaces or complexes will create signs in the
associator.

We are now ready to define the prototypical symmetric
 sequence that carries the structure of an algebraic 
 operad. 
 
\begin{definition}
The \emph{endomorphism operad} of a space $V$ is the symmetric sequence $\End_V$
where for each $n\in\NN$ we set $\End_V(n) = \End(V^\otimes, V)$. 
The symmetric group $S_n$ acts on the right
on $\End_V(n)$ 
so that $(f\sigma)(v) = f(\sigma v)$ for
$v\in V^{\otimes n}$, where $S_n$ acts on
the left on $V^{\otimes n}$ by $(\sigma v)_i
= v_{\sigma i}$. The composition maps are defined
by $\gamma(f;g_1,\ldots,g_n) = f\circ (g_1\otimes\cdots \otimes g_n)$. 
\end{definition}

The following two operations on permutations 
will streamline our definition of (algebraic)
operads.

\medskip

\textbf{Two useful maps.} For each $k\geqslant 1$
and each tuple $\lambda = (n_1,\ldots,n_k)$ 
with sum $n$
there is a map
\[ S_k \longrightarrow S_{n_1+\cdots+n_k} \]
that sends $\sigma\in S_k$ to the permutation
$\lambda(\sigma)$ of $[n]$ that permutes the blocks 
$\pi_i = \{n_1+\cdots+n_{i-1}+1,\ldots
			n_1+\cdots+n_{i-1}+n_i\}$
			according to $\sigma$.
There is also a map
\[S_{n_1}\times \cdots \times  S_{n_k} 
	\longrightarrow S_{n_1+\cdots+n_k}  \] 
	that sends a tuple of permutations 
	$(\sigma_1,\ldots,\sigma_k)$ to the
	permutation $\sigma_1\#\cdots \# \sigma_k$
	that acts like $\sigma_i$ on the block $\pi_i$
	as above. These operations are illustrated
	in Figure~\ref{fig:1}. With these at hand, 
one can check that these composition maps
satisfy the following axioms:

\begin{figure}
$$\lambda = (2,1,2), \quad \sigma = 312
	\quad \leadsto \quad \lambda(\sigma) =  34512 \in S_5
	$$
	$$
	(213,213,132)\in S_3\times S_3\times S_3 \quad \leadsto \quad 213546798\in S_9 $$
\caption{The useful operations}
\label{fig:1}
\end{figure}

\begin{tenumerate}
\item \emph{Associativity}: let $f\in \End_V(n)$,
and consider $g_1,\ldots,g_n \in \End_V$ and
for each $i\in [n]$ a tuple $h_i= (h_{i1},\ldots,h_{i n_i})$
were $n_i= \mathrm{ar}(g_i)$. Then for
$f_i = \gamma(g_i; h_{i1},\ldots,h_{in_i})$ 
and $g= \gamma(f; g_1,\ldots,g_n)$ we have
that
\[ \gamma(f;f_1,\ldots,f_n) = 
	\gamma(g; h_1,\ldots,h_n).\]
\item \emph{Intrinsic equivariance}: for
each $\sigma\in S_k$ and $\lambda = (\ari(g_1),\ldots,\ari(g_k))$ we have that
\[ \gamma(f\sigma; g_1,\ldots,g_k) = 	
	\gamma(f; g_{\sigma 1} ,\ldots, 
		g_{\sigma k})\lambda(\sigma),\]
	
\item \emph{Extrinsic equivariance}: for each
tuple of permutations $(\sigma_1,\ldots,\sigma_k) \in S_{n_1} \times
\cdots \times S_{n_k}$, if $\sigma = \sigma_1\#\cdots\#\sigma_k$, we have that
\[\gamma(f,g_1\sigma_1,\ldots,g_k\sigma_k) = 
	\gamma(f; g_1,\ldots,g_k)\sigma.\]
\item \emph{Unitality:} the identity $1\in\End_V(1)$
satisfies $\gamma(1;g) = g$ and $\gamma(g;1,\ldots,1) = g$ for every $g\in\End_V$.
\end{tenumerate}

\begin{definition} 
A symmetric operad (in vector spaces) is an
$\Sigma$-module $\PP$ along with a composition
map $\gamma : \PP\circ \PP \longrightarrow \PP$
of signature
\[\gamma : \PP(k)\otimes 
	\PP(n_1) \otimes \cdots \otimes \PP(n_k)
	 	\longrightarrow \PP(n_1+\cdots+n_k)\]
along with a unit $1\in \PP(1)$, that satisfy
the axioms above. 
\end{definition} 

\begin{variante} A non-symmetric operad is
an operad whose underlying object is a collection
(with no symmetric group actions). Operads in
topological spaces or chain complexes require
the composition maps to be morphisms (that is,
continuous maps or maps of chain complexes,
respectively) and, more generally, operads 
defined on a symmetric monoidal category
require, naturally, that the composition
maps be morphisms in that category. 
\end{variante}

\textbf{Pseudo-operads.}
One can define operads through \emph{partial 
composition maps}, modeling the honest partial
composition map
\[ f\circ_i g = f(1,\ldots,1,g,1,\ldots,1)\] 
in $\End_V$. These composition maps satisfy the
following properties:

\begin{tenumerate}
\item \emph{Associativity}: for
each $f,g,h\in\End_V$, and $\delta = i-j+1$,
we have
\[ 
(f \circ_j g) \circ_i h  = 
 	\begin{cases} 
 		 (f \circ_i h) \circ_{\ari(f)+j-1} g
 		  	& \delta \leqslant 0  \\
 		  	f\circ_j (g \circ_\delta h) &
 		  	\delta\in [1,\ari(g)] \\
 		  	(f \circ_\delta h) \circ_j g & \delta > \ari(g)
 		   \end{cases}
 		 \]
\item \emph{Intrinsic equivariance}: for
each $\sigma\in S_k$, we have that
\[  (f\sigma) \circ_i g  = (f\circ_{\sigma i} g)\sigma'\]
where $\sigma'$ is the same permutation as $\sigma$
that treats the block $\{i,i+1,\ldots,i+\ari(g)-1\}$
as a single element. 
\item \emph{Extrinsic equivariance}: 
 for each $\sigma\in S_k$, we have that
\[  f \circ_i (g\sigma)  = (f\circ_i g)\sigma''\]
where $\sigma''$ acts by only permuting the
block $\{i,\ldots,i+\ari(g)-1\}$ according
to $\sigma$.
\item \emph{Unitality:} the identity $1\in\End_V(1)$
satisfies $1 \circ_1 g = g$ and $g\circ_i 1 = g$ for every $g\in\End_V$ and $1\leqslant i\leqslant \ari(g)$.
\end{tenumerate}

\begin{definition}
A symmetric operad (in vector spaces) is an
$\Sigma$-module $\PP$ along with partial composition
map of signature
\[ -\circ_i -  : \PP(m)\otimes \PP(n) 
	\longrightarrow \PP(m+n-1) \]
and a unit $1\in\PP(1)$ satisfying the axioms above.
\end{definition}

It is not hard to see (but must be checked at least once)
that an operad with $\PP(n) = 0$ for $n\neq 1$ is
exactly the same as an associative algebra. 

\textbf{Warning!} If one does not require
the existence of a unit, the notion of a
\emph{pseudo-operad} by Markl (defined by partial
compositions) does not coincide with the
notion of an operad as defined by May.


\subsection{Constructing operads by hand}

One can define operads in various ways. For example,
one can define the underlying collection explicitly,
and give the composition maps directly:
\begin{tenumerate}
\item \emph{Commutative operad.} The reduced symmetric topological (or set)
operad with $\mathsf{Com}(n)$ a single point for each
$n\in \NN$, and composition maps the unique
map from a point to a point.
\item \emph{Associative operad.} 
The reduced set operad with
$\mathsf{As}(n) = S_n$
the regular representation and composition maps
\[ S_k \times S_{n_1} \times
\cdots \times S_{n_k} \longrightarrow S_{n_1+\cdots n_k} \]
the unique equivariant map that sends the tuple
of identities to the identity.
\item  \emph{Stasheff operad.}
Let $K_{n+2}$ be the subset of $I^n$ (the
product of $n$ copies of $I=[0,1]$) 
consisting of tuples $(t_1,\ldots,t_{n+2})$
such that $t_1\cdots t_k\leqslant 2^{-k}$
for $j\in [n+2]$. The boundary of 
$K_{n+2}$ consists of those points such
that for some $j\in [n+2]$ we have
either $t_j$ or $t_1\cdots t_j = 2^{-j}$.
It is tedious (but otherwise doable)
to show that for each pair $(r,s)$ of
natural numbers and each $i\in [r]$
there exists an inclusion
\[ \circ_i : K_{r+1} \times K_{s+1} \longrightarrow
 	K_{r+s+1} \] 
that defines on the sequence of
spaces $\{K_{n+2}\}_{n\geqslant 0}$
the structure of a non-symmetric operad.
We will see in the exercise a realization
of $K_n$ as the convex hull of points
with positive integer coordinates
(due to J.-L. Loday)  using planar
binary rooted trees, which will make the
operad structure more transparent.

\item If $M$ is a monoid, there is an
operad $\mathbb W_M$ with $\mathbb{W}_M(n) =
M^n$ such that 
\[(m_1,\ldots,m_s) \circ_i (m_1',\ldots,m_t') = 
 	(m_1,\ldots,m_{i-1}, m_im_1',\ldots,m_im_t',m_{i+1},\ldots, m_s).\] We call it the
 \emph{word operad of $M$}. Its underlying
 symmetric collection is $\mathsf{As}\circ M$. 
 
 \item Write $\operatorname{Aff}(\mathbb C) = \mathbb{C}\times \mathbb{C}^\times$ for the group of affine transformations of
$\mathbb C$ with group law $(z,\lambda)(w,\mu) = (z+\lambda w,\lambda\mu)$. In turn, define for each finite set $I$ the topological space
\[ \mathcal{C}(I) = \{ (z_i,\lambda_i)\in \operatorname{Aff}(\mathbb C)^I  : |z_i-z_j|>|\lambda_i|+|\lambda_j| \}.\] 
The group law of $\operatorname{Aff}(\mathbb C)$ allows us to
define an operad structure on $\mathcal{C}(I)$ using the
exact same definition as in the word operad of a monoid. 
The subspaces $\mathcal{D}_2^{\mathrm{fr}}(I)
	\subseteq \mathcal{C}(I)$
where $|z_i|+|\lambda_i|\leqslant 1$ for all $i\in I$, and 
where the inequality is strict unless $z_i=0$ is called
the \emph{framed little disks operad}. The little disks operad
is the suboperad where $\lambda_i = 1$ for all $i\in I$, and 
we write it $\mathcal{D}_2(I)$.

 \item The operad of rooted trees $\mathsf{RT}$ has
 $\mathsf{RT}(n)$ the collection of rooted threes with $n$
 vertices labeled by $[n]$, and the composition $T \circ_j T'$
  is obtained by inserting $T'$ at the $j$th vertex of $T$
  and reattaching the children of that vertex to $T'$ in
  all possible ways. For example, if
%  \[ T=\vcenter{\xymatrix{*++[o][F-]{1} \ar@{-}[d] & 
%*++[o][F-]{3} \ar@{-}[dl]\\
%*++[o][F-]{2}}}
%\ {\hbox{\rm and}}\ \  S=\vcenter{
%\xymatrix{*++[o][F-]{2} \ar@{-}[d] \\
%*++[o][F-]{1}}}\]
then we have that  
%  \[ T\circ_2 S=\quad
% \vcenter{\SelectTips{cm}{}\xymatrix@-1pc{
%*++[o][F-]{1}\ar@{-}[d]&*++[o][F-]{4}\ar@{-}[dl]\\
%*++[o][F-]{3}\ar@{-}[d]\\
%*++[o][F-]{2}}} +\quad
%\vcenter{\SelectTips{cm}{}\xymatrix@-1pc{
%*++[o][F-]{1}\ar@{-}[d]\\
%*++[o][F-]{3}\ar@{-}[d]&*++[o][F-]{4}\ar@{-}[dl]\\
%*++[o][F-]{2}}}\quad +\quad
%\vcenter{\SelectTips{cm}{}\xymatrix@-1pc{
%*++[o][F-]{4}\ar@{-}[d]\\
%*++[o][F-]{3}\ar@{-}[d]&*++[o][F-]{1}\ar@{-}[dl]\\
%*++[o][F-]{2}}} \quad + \quad
%\vcenter{\SelectTips{cm}{}\xymatrix@-1pc{
%*++[o][F-]{1}\ar@{-}[dr] & *++[o][F-]{3}\ar@{-}[d]&*++[o][F-]{4}\ar@{-}[dl] \\
%&*++[o][F-]{2}}}. \]
\end{tenumerate}
\subsection{Exercises}

 \begin{question}
 Follow the lecture notes and read about the
 partial definition of an operad (and what
 a Markl operad is). Show that a unital
 pseudo-operad is the same as a unital 
 May operad. 
 \end{question}
 
 \begin{question}
Define the category of collections
in $\mathsf{Vect}$
using the biased approach and the 
unbiased approach (this requires considering
\emph{totally ordered} sets instead of sets,
and their order preserving bijections. We will
write them with calligraphic letters but
use subscripts, so $\mathcal X$ has ns
components $\{\mathcal X_n\}_{n\geqslant 1}$.

\begin{enumerate}
\item Show
that it supports a non-symmetric Cauchy
product given by 
\[ (\mathcal{X}\otimes\mathcal{Y})_n =
  \bigoplus_{i+j=n} \mathcal{X}_i\otimes
   	\mathcal{Y}_j.\]
   	
   	\item Use this and the unbiased approach to 
   	argue that the ns counterpart of a
   	`subset of $I$' is an interval:
   	a totally ordered subset of $I$ of the
   	form $[i,j] = \{ x\in I : i\leqslant x \leqslant j\}$.
 \item Use the previous item to define the
 non-symmetric composition of ns collections.
 Define the generating function associated
 to a collection, and show it behaves
 well with respect to the products above. 
\end{enumerate}
\end{question}

\begin{question}
Since every finite totally ordered set
is, in particular, a finite set (and
every order preserving function is a
fortiori a function) there is a 
map of categories 
$ \mathsf{FinOrd}^\times \longrightarrow
 	\mathsf{FinSet}^\times$
which induces a map that `forgets the
symmetries' ${}_\Sigma\mathsf{Mod}	\longrightarrow\mathsf{Coll}$. 
Show that there is a functor that assigns
a ns sequence $\mathcal{X}$ to the 
sequence $\mathcal{X}_\Sigma(n) =
\kk S_n\otimes \mathcal{X}_n$ which is
left adjoint and monoidal. 
\end{question}
 
\begin{question}
Describe the associator for $\circ_\Sigma$ in the
category of differential graded collections. 
In particular,
write down the signs explicitly. Explain
how this is related to the signs
in the parallel composition axiom
for \emph{graded operads} that read as follows:
for elements $f,g$ and $h$
in an operad (of homogeneous arities)
and $\delta = i-j+1$, we have that
\[ 
(f \circ_j g) \circ_i h  = 
 	\begin{cases} 
 		(-1)^{|g||h|}
 		 (f \circ_i h) \circ_{\ari(f)+j-1} g
 		  	& \delta \leqslant 0  \\
 		 \phantom{(-1)^{|g||h|}(} 	f\circ_j (g \circ_\delta h) &
 		  	\delta\in [1,\ari(g)] \\
 		(-1)^{|g||h|}
   	(f \circ_\delta h) \circ_j g
   		 & \delta > \ari(g).
 		   \end{cases}
 		 \]
\end{question}

\begin{question}
A (unital associative) monoid $x$ in a monoidal category 
$(\mathcal C,\otimes,\alpha,\rho,\lambda,1)$ is an object
along with maps $\mu: x\otimes x\to x$ and $\eta : 1
 \longrightarrow x$ such that $\mu$ is associative, 
 that is $\mu (\mu\otimes 1) = \mu(1\otimes \mu)
 \alpha_{x,x,x}$, and unital for
$\eta$, that is $\mu(\eta\otimes 1)=\rho_x$
and $\mu(1\otimes \eta) = \lambda_x$.
Show that a $\Sigma$-operad is exactly the
same as a monoid in $({}_\Sigma\mathsf{Mod},\circ_\Sigma)$.
\end{question}

\begin{question}
We write $\mathsf{End}$ for
 category of endofunctors of $\mathsf{Vect}$. Show
 that there is a \emph{monoidal}
 functor $S:{}_\Sigma\mathsf{Mod}
 \longrightarrow \mathsf{End}$ that assigns
 $\mathcal{X}$ to $V\longmapsto \bigoplus_{n\geqslant 0} \mathcal{X}(n)\otimes_{\Sigma_n} V^{\otimes n}$.
 It is called the \emph{Schur functor} associated
 to $\mathcal{X}$. The endofunctors in the essential
 image of $S$ are called \emph{analytic}.
\end{question}

\begin{question}
	 If $\mathcal{X}$
	 is a symmetric sequence, describe the $\Sigma_n$
	 action on $\mathcal{X}^{\otimes n}$ where
	 $\otimes$ is the Cauchy product. Observe that
	 it commutes with the $\Aut(I)$ action on 
	 $\mathcal{X}^{\otimes n}(I)$.
\end{question}
	 
\begin{question}
Define ${}_\Sigma\mathsf{Mod}(\CC)$
for any symmetric monoidal category $(\CC,\otimes,1)$
(such as the category of sets, or topological spaces,
or chain complexes, among others) along with
its \emph{symmetric composition product}
 $-\circ_\Sigma - $.
 \end{question}
 
\begin{question} Prove that non-unital Markl operads
and non-unital May operads differ.
To do this, consider the non-unital ns operad
$\PP$ such that $\PP(2)$ and $\PP(4)$
are its only non-zero components, and are
both one dimensional, and define
\[ \gamma : \PP(2)\otimes \PP(2)\otimes \PP(2)
 	\longrightarrow \PP(4) \]
to be an isomorphism, and all other maps zero. 
Check that $\PP$ is a May operad, and
show that $\PP$ is not a Markl
operad by exploring the consequences
of the equality
\[ \mu(\mu,\mu) = (\mu \circ_2\mu)\circ_1 \mu \]
in any Markl operad.
\end{question}

\begin{question}
 Check that examples (1), (2), (4), (5) in page 10
 are indeed all operads.
\end{question}
%
%\begin{question}
%Follow these steps to construct the
%Stasheff operad as a sequence of 
%convex polytopes $K_2',K_3',\ldots$
%for which the boundary of $K_{n+1}'$
%is a union of products $K_{r+1}'\times
%K_{s+1}'$ with $r+s=n$ indexes by
%planar rooted trees with two internal
%vertices.
%\end{question}
%\begin{enumerate}
%\item Let us write $T_n$ for the collection
%of planar rooted \emph{binary} 
%trees with $n+1$ leaves, which we 
%order from left to right. Explain how
%this gives a total order on the vertices,
%which we will thus call $1,\ldots,n$.
%\item  For
%each $t\in T_n$ and each vertex $i$ of
%$t$, let $L(i)$ denote the number of paths
%from $i$ to a leaf of $t$ going through
%its left child, and let $R(i)$ denote
%the the number of paths
%from $i$ to a leaf of $t$ going through
%its right child. We define
%\[ x(t) = (L(1)R(1),\ldots,L(n)R(n))
%	\in \mathbb N^n. \]
%Show that $x(t)$ always lies in the
%hyperplane $x_1+\cdots x_n = \binom{n}{2}$.	
%We write $K_{n+2}'$ for the convex hull of 
%the points $\{ x(t) : t\in T_n \}$.
%\emph{Hint.} Any planar binary rooted tree
%$t$ decomposes into a left tree $L_t$
%and a right tree $R_t$ by looking at the
%children of the unique child of the root.
%Express $W(t) = \sum_{i=1}^n x(t)$ in terms of $W(R_t)$ and $W(L_t)$.
%
%\item 
%Show that the polytope $K_{n+2}'$ 
%is of dimension $n$, 
%and its $k$-cells for $k\in [n]$
%are in bijection with planar rooted
%trees with $n-k+1$ internal vertices
%and $n+2$ leaves. Conclude, in particular,
%that its codimension one faces are
%in bijection with planar rooted trees
%with $2$ internal vertices and $n+2$
%leaves.
%
%\item Suppose that $t$ has $r+1$
%leaves and that $t'$ has $s+1$ leaves,
%and consider the grafting $t\circ_i t'$.
%We define $x(t)\circ_i x(t')$ by
%$x(t\circ_i t')$. Show that this
%defines a map
%\[\circ_i :  K_{r+1}'\times K_{s+1}' 
%	\longrightarrow K_{r+s+1}'.\]
%\item Show the maps above give the
%collection $\{K_{n+2}'\}_{n\geqslant 0}$
%the structure of a ns operad.
%\end{enumerate}

\begin{question}
Suppose that $T\in\mathsf{RT}(n)$ and
that $T'\in \mathsf{RT}(m)$, where $\mathsf{RT}$
is the symmetric collection
 of rooted trees of Lecture 1,
and let $\mathrm{In}(T,i)$ denote the
set of incoming edges of $T$ at the
vertex labeled $i$. For each function
$f: \mathrm{In}(T,i)\longrightarrow [m]$,
define the tree $T\circ_i^f T'$ by
replacing vertex $i$ of $T$ by $T'$ and
attaching the loose incoming edges of 
vertex $i$ to the vertices of $T'$
according to the map $f$: the edge
$e\in \mathrm{In}(T,i)$ is attached
to vertex $f(e)\in T'$. Finally,
define $T\circ_i T'$ by taking the
sum through all possible functions
$f$. Show that this gives $\mathsf{RT}$
the structure of a unital pseudo-operad,
and thus of a usual operad, with unit
the tree with no edges and one vertex.
\end{question}


\begin{question}
 Describe the operation $T\star T' = S(T,T')$ where
$S$ is the rooted tree above in terms of 
insertions of $T'$ in $T$ and regrafting of incoming 
edges. Show that it satisfies the following \emph{pre-Lie
identity}:
\[
  (T\star T')\star T'' -  T \star (T' \star T'' ) =
    (T\star T'')\star T' -  T \star (T'' \star T' ) 
 	\]
 	by explicitly interpreting the left hand side in
 	terms of certain insertions of $T'$ and $T''$ in $T$,
 	and showing the resulting sum of trees is symmetric
 	in $T'$ and $T''$.

\end{question}

\begin{question}
Suppose that $\mathcal P$ is an operad
and that $\mathcal X\subseteq \mathcal P$
is a symmetric subsequence. We say
$\mathcal X$ generates $\mathcal P$
if every element of $\mathcal P$ is
an iterated composition of elements of
$\mathcal X$. 
Show that the rooted trees operad 
$\mathrm{RT}$ 
is generated by the symmetric subsequence
given by the two labeled rooted trees with
two vertices:
\[ 
S=\vcenter{
\xymatrix{*++[o][F-]{2} \ar@{-}[d] \\
*++[o][F-]{1}}},\qquad
S(12)=\vcenter{
\xymatrix{*++[o][F-]{1} \ar@{-}[d] \\
*++[o][F-]{2}}}
\]
spanning the regular representation of
$S_2$. Follow these steps:
\end{question}
\begin{enumerate}
\item Suppose that $T$ is an $n$-rooted tree
and let $J$ be a subset of $[n]$ corresponding
to leaves of $T$ that are the children of a
vertex $i\in T$. Let $T'$ be the
tree obtained by erasing all these leaves
and replacing the vertex label by a new
symbol $\ast$, and let $T''$ be the rooted
tree with root $i$ and children labeled
by $J$. Show that $T'\circ_\ast T'' = T$.
\item Use the above and induction on the
number of vertices to show it suffices to prove
the claim for the corollas, that is, trees with
one internal root vertex.
\item Let us write $T_n$
for the operation in $\mathsf{RT}(n)$
corresponding to a corolla with root $1$,
so in particular $T_2 = S$.
Show that 
\[ 
T_n = 
 T_2\circ_1 T_{n-1} - 
  	\sum_{i=1}^{n-1} (T_{n-1}\circ T_i)\sigma_i
\]
where $\sigma_i = (i+1,i+2,\ldots,n)\in S_n$
is a cycle, and use this to conclude.
\end{enumerate}

\emph{Note.} The operation $T_n$ is usually
denoted $\{x_1; x_2,\ldots,x_n\}$ and is called
a \emph{symmetric brace}, and the equation
above is usually written in the form
\[ 
\{x_1; x_2,\ldots,x_n\} = 
 \{\{x_1; x_2,\ldots,x_{n-1}\}; x_n\}
 - \sum_{i=1}^{n-1} \{ x_1; x_2,
 \ldots, x_{i-1}, \{x_i ; x_n \},
 x_{i+1},\ldots,x_{n-1}\}.
 \]
 
 \begin{question}
 Let $\mathcal X$ be a symmetric sequence,
 and define the derivative $\partial\mathcal X$ 
 of $\mathcal X$ to be symmetric sequence
 with $(\partial\mathcal X)(I) = \mathcal X(I^*)$
 where $I^* = I \sqcup \{ I \}$. Note that
 $S_I$ acts on $I^*$ fixing the element $I$.
 Show that $(\partial \mathcal X)(n)$ 
 is isomorphic to the restriction of $\mathcal{X}(n+1)$ to $S_n = \mathrm{Fix}(n+1)$, and
 conclude that 
 \[ \partial_z f_{\mathcal X}(z) = 
  			f_{\partial\mathcal X}(z). \]	
  Let $s$ be the sequence of singletons and
  define the pointing of operation by
  $\mathcal X^\bullet = s\otimes_\Sigma \partial   \mathcal{X}$. Determine the representation
  $\mathcal{X}^\bullet(n)$ in terms of
  $\mathcal{X}(n)$. 
 \end{question}

\afterpage{\blankpage}
\newpage
\section{Free operads and presentations}

\textbf{Goals.} We will define algebraic operads
by generators and relations, and with this at
hand define quadratic and quadratic-linear
presentations of operads. 

\subsection{Trees}

Operads and their kin are gadgets modeled after
combinatorial graph-like objects. Operads themselves are modeled after
rooted trees, so it is a good idea to have a concrete definition of 
what a rooted tree is. We will also consider planar rooted trees,
and trees with certain decorations, so it is a good idea to digest
the definitions carefully to later embellish them.

A rooted tree $\tau$ is the datum of a finite set $V(\tau)$
of vertices along with a partition $V(\tau) = 
\mathrm{Int}(\tau)\sqcup L(\tau) \cup R(\tau)$,
where the first are the \emph{interior} vertices, $L$ are the leaves, and $R(\tau)$ is
a singleton, called the root of $\tau$. We also require there is 
a function $p :V(\tau)\smallsetminus R(\tau) \longrightarrow V(\tau)$,
describing the edges of $\tau$, 
with the following properties: call a vertex $v\in V(\tau)$ a child 
of $w\in V(\tau)$ if $v\in p^{-1}(w)$. Then:
\begin{tenumerate}
\item The root $r\in R(\tau)$ has exactly one child.
\item The leaves of $\tau$ have no children.
\item For each non-root vertex $v$ there exist a unique sequence
$(v_0,v_1,\ldots,v_k)$ such that $p(v_{i-1}) = v_{i}$ for $i\in [k]$
with $v_0 = v$ and $v_k = r$.
\end{tenumerate}
We will call a non-leaf vertex that has no children a \emph{stump}
(or an endpoint, or a cherry-top).
A tree is reduced if has no stumps and all of its non-root and
non-leaf vertices have at least two children. We will also call
the root the (unique) output vertex $\tau$, and the leaves
the input vertices of $\tau$. 

A planar rooted tree is a rooted tree $\tau$
along with a linear order in each of the fibers
of the parent function $p$ of $\tau$. In short,
the children of each vertex are linearly ordered,
so we are effectively considering a drawing of
$\tau$ in the plane, where the clockwise
orientation gives us the order at each
vertex. 

Two rooted trees $\tau$ and $\tau'$ are isomorphic
if there exists a bijection $f : V(\tau) \longrightarrow 
V(\tau')$ that preserves the root, the input vertices and the
interior vertices, so that $p'\circ f = p$ where we
also write $f$ for the induced bijection $f : V(\tau)\smallsetminus r \longrightarrow 
V(\tau')\smallsetminus r'$. Two planar rooted trees
are isomorphic if in addition $f$ respects the linear order
at each vertex.

For example, consider the rooted tree $\tau$ with
$V = \{1,2,3\}\cup \{4,5\}\cup \{0\}$, that is,
three leaves, two interior vertices and the root.
Then the choice of $p : [5] \to \llbracket 5 \rrbracket$ 
with $p(\{1,2\}) = 4$, $p(\{3,4\}) = 5$, $p(5) = 0$
gives a tree isomorphic to the one with 
with $p(\{1,2\}) = 3$, $p(\{3,4\}) = 5$, $p(5) = 0$.
On the other hand, if we consider the vertices linearly
ordered by their natural order, these two planar rooted trees 
are no longer isomorphic. 

\begin{definition}
For a finite set $I$, an $I$-labeled tree $T$
is a pair $(\tau,f)$ where $\tau$ is a 
reduced rooted tree, along with
a bijection $f : I \longrightarrow L(\tau)$.
Two $I$-labeled trees $T$ an $T'$ are isomorphic
if there exists a pair $(g,\sigma)$ where
$g$ is an isomorphism between $\tau$ and $\tau'$
and $\sigma$ is an automorphism of $I$ such that
$g\mid_{L(\tau)}\circ f = \sigma\circ f'$. 
\end{definition}

Suppose that $(\tau,f)$ is an $I$-tree and that
$(\tau',f')$ is a $J$-tree, and that $i\in I$. We define
$K=I\cup_i J = I\sqcup J \smallsetminus i$ and the
$K$-tree $\tau\circ_i \tau'$ as follows:
\begin{tenumerate}
\item Its leaves are $L(\tau\circ_i \tau') = L(\tau)\sqcup L(\tau')\smallsetminus f^{-1}(i)$.
\item Its internal vertices are $V(\tau)\sqcup V(\tau')$, with
root $r$. 
\item The parent function $q$ is defined by declaring that:
	\begin{titemize} 
	\item $q$ coincides with $p$ on $V(\tau)$, 
	\item $q(w) = p(f^{-1}(i))$ if
$w$ is the unique children of the root of $\tau'$, 
	\item  $q$
coincides with $p'$ on $V(\tau')\smallsetminus \{r',w\}$.
\end{titemize}
\item The leaf labeling is the unique bijection $L(\tau\circ_i \tau') \longrightarrow I\circ_i J$ extending $f$ and $f'$.
\end{tenumerate} 

\subsection{Tree monomials}
Let us now consider an 
(unbiased) reduced symmetric sequence $\XX$ which
we will think of as an \emph{alphabet}. A tree monomial in the
alphabet $\XX$ 
is a pair $(\tau,x)$ where $\tau$ is a reduced rooted
tree and $x : \mathrm{Int}(\tau) \longrightarrow \XX$ is a map
with the property that $x(v) \in \XX(p^{-1}(v))$. Observe that
reduced sequences and reduced trees correspond to each other, in the
sense that with this definition we can only decorate a stump
of $\tau$ with an element of $\XX(\varnothing)$. 

An $I$-labeled
$\XX$-tree $T$ is a triple $(\tau,x,f)$ where $(\tau,f)$ is $I$-labeled
and $(\tau,x)$ is an $\XX$-tree. We will say that $(\tau,x,f)$
is a (symmetric) tree monomial if $\XX$ is symmetric. If it
is just a collection, we will say that $(\tau,x,f)$ is a
ns tree monomial. In particular, if $T$ is an $I$-labeled
tree, and if $\sigma \in \Aut(I)$, there is another 
$I$-labeled tree $\sigma(T)=(\tau,f\sigma^{-1})$. 


Suppose that $T = (\tau,x,f)$ is a tree monomial on an
alphabet $\XX$, and let us pick a vertex $v$ of $\tau$
and a permutation $\sigma$ of the set $C = p^{-1}(v)$ of
children of $v$. We define the tree $\tau^\sigma$ as
follows: the datum defining $\tau$ remains unchanged
except $p$ is modified to $p^\sigma$ so that 
\[ p^\sigma(w) = 
\begin{cases}
 p(w) & \text{if $p^2(w)\neq v$} \\
 p(\sigma^{-1}(w')) & \text{if $p(w)=w'\in C$}.
\end{cases}
\]
Briefly, we just relabel the vertices of $\tau$
using $\sigma$. With this at hand, we define
$T^\sigma$ to be the tree monomial with underlying 
tree $\tau^\sigma$ and with $x$ modified to $x^\sigma$ so that
\[ x^\sigma(w) = 
\begin{cases}
 \sigma x(v) & \text{if $v=w$,} \\
 x(\sigma^{-1}(w')) & \text{if $p(w)=w'\in C$}.
\end{cases}
\]
Note that it is possible some children of $v$ are
leaves, in which case the definitions make sense if
we think of leaves as decorated by the unit of
$\kk$. 

\begin{example}
Let us consider the alphabet 
$\XX = \XX(2) = \{\ast \}$ where
the unique operation is antisymmetric.
Then we have the following equalities
of symmetric tree monomials:
\[ 
	\leftc{}{}{2}{1}{3}  \mathsymbol{1.5}{=}
\rightc{}{}{3}{1}{2}  
   \mathsymbol{1.5}{=-}
 {\leftc{}{}{1}{2}{3}}
  \mathsymbol{1.5}{.}\]
 
\end{example}


Let us now define for each $n\geqslant 1$ the
space $\FF_\XX(I)$ as the span of all tree monomials
$T = (\tau,f,x)$ on $\XX$ with leaves labeled by $I$,
modulo the subspace generated by all elements of the form
\[ R(T,v,\sigma) = T - T^\sigma \]
where $\sigma$ ranges through $\Aut(p^{-1}(v))$ as
$v$ ranges through the vertices of $\tau$. In case
all children of $v$ are leaves, this is saying that
the tree where $x_v$ is replace by $\sigma(x_v)$
is equal to the tree where the leaves of $T$ that are
children of $v$ are relabeled according to $\sigma$.
We also require that tree decorations behave like
tensors, so that $T = T_1+T_2$ if the decoration
of $T$ at a vertex $v$ is of the form $x_1 + x_2$
and for $i\in [2]$ the tree $T_i$ coincides with
$T$ except that it is decorated by $x_i$ at $v$.



\subsection{The free operad}
An algebraically inclined way to construct
(algebraic) operads is through generators and
relations. There is a forgetful functor
from the category of operads to the category
of collections. In general, it admits a left
adjoint, which is the free operad functor.

\begin{definition}
The \emph{free symmetric operad} on $\XX$ is the
symmetric sequence $\FF_\XX$ along with the composition
law obtained by grafting of trees. More precisely,
suppose that $T\in \FF_\XX(I)$ and that $T'\in\FF_\XX(J)$,
and that $i\in I$. We define $T'' =T\circ_i T' \in 
\FF_\XX(I\cup_I J)$ by taking its underlying labeled
tree to be $\tau\circ_i \tau'$, and by decorating it
in the unique way which extends the decorations of 
$T$ and $T'$.
\end{definition}

The following lemma shows that this indeed defines an operad.

\begin{lemma}
Tree grafting respects both $I$-tree 
isomorphisms and the relations $T\sim T^\sigma$
above, and hence is well defined on $\FF_\XX$.
\end{lemma}

\begin{proof}
This is Exercise~\ref{ex:grafting}.
\end{proof}

 We will later
interpret $\XX\longmapsto \FF_\XX$ as a \emph{monad},
thus giving another definition of operads. The
advantage of this `monadic approach' is its 
flexibility, which allow us to define other
operad like structures, like the ones 
mentioned in the introduction.
In this direction, a curious reader 
can consider the following 
equivalent definition:

\begin{definition}
The free operad generated by a symmetric
collection $X$ is defined inductively by
letting  $\FF_{0,X}=\kk$ be spanned
by the `twig' (tree with no vertices and one edge)
in arity zero and
\[ \FF_{n+1,\XX} = \kk\oplus (\XX\circ  \FF_{n,\XX} ), \]
and finally by setting
$\FF_\XX = \varinjlim_n \FF_{n+1,\XX}$.
The composition maps are defined by induction,
and the axioms are also checked by induction.
\end{definition}

Intuitively, the previous definition says that an element of
$\FF_\XX$ is either the twig, or corolla
with $n$ vertices decorated by $\XX$, whose leaves
have on them an element of $\FF_\XX$. The
final shape of $\FF_\XX$ will however
depend on the symmetric structure of $ \XX$. 
 \subsection{Exercises}


 
 \begin{question}
Let $\XX$ be a collection such that $\underline{\XX} = 
\XX(2)$. Compute a basis of tree monomials 
for the free operad over
$\XX$ in case $\XX(2)$ is:
\begin{tenumerate}
\item The regular representation of $S_2$.
\item The sign representation of $S_2$.
\item The trivial representation of $S_2$. 
\end{tenumerate} 
In all cases, decompose the $S_3$-module $\mathcal{F}_\XX(3)$
into irreducible representations.
\end{question}


\begin{question}\label{ex:grafting}
Show that tree grafting respects both $I$-tree
isomorphism and the relation $T\sim T^\sigma$,
and hence descends to $\FF_\XX$.
\end{question}

\begin{question}
Suppose that $\XX$ is an alphabet (in sets) that is
finite in each arity and such that $\XX(n) = \varnothing$
for $n=0,1$. Show that $\mathcal{F}_\XX$ is finite
in each arity. 
\end{question}
 
\begin{question}
Define non-symmetric tree monomials over a ns alphabet
$\XX$ and thus define the free \emph{non-symmetric}
operad over a collection $\XX$.
\end{question}

\begin{question}
Read the statement and proof of \emph{Theorem 5.4.2} in
`Algebraic Operads' that the colimit construction briefly
described in the lecture notes does give the free
operad on a symmetric collection.
\end{question}

\begin{question}
Consider the map from ns collections to symmetric sequences
that assigns $\XX$ to $\Sigma\otimes \XX$ such that
$(\Sigma\times \XX)(n) = \Sigma_n\times \XX(n)$ with its
corresponding symmetric group action. What is the relation
between the free ns operad on $\XX$ and the free symmetric
operad on $\Sigma\times \XX$?
\end{question}

\begin{question}
Let $V$ be an $S_2$-module, and let $\XX$ be the
symmetric collection with $\XX(2) = V$ and zero
everywhere else. Show that $\mathcal{F}_\XX(3)$
consists of three copies of $V^{\otimes 2}$ and
describe explicitly the action of $S_3$ on it.
\end{question}

\begin{question}
Show that the construction of the free operad we carried
out during \textbf{Lecture~2} indeed defines the free
operad on $\XX$ where $i:\XX \longrightarrow \mathcal F_\XX$
sends an element $x\in \XX(I)$ to the corolla whose unique
internal vertex is labeled by $x$ (and whose
leaves are labeled by $I$). 
\end{question}

 \begin{question}
 Follow the lecture notes and read about 
 weight gradings and the canonical weight
 grading on a free operad. 
 \end{question}

\afterpage{\blankpage}
\newpage

\section{Quadratic operads}

 \textbf{Goal.} Introduce weight graded gadgets,
 define operads by generators and relations, and
 introduce quadratic operads. Give plenty of examples
 of `real life' quadratic operads to work on:
 Hilbert series, Koszul dual, bar construction. 
 
 \subsection{Weight gradings and presentations}
 The notion of a quadratic operad is based on the observation
 every free operad has a canonical `weight grading' by the
 number of internal vertices of a tree. Let us make this
 precise.
 
\begin{definition}
A symmetric sequence $\XX$ is weight graded if for
each finite set the component $\XX(I)$ admits a 
decomposition $\XX(I) = \bigoplus_{j\geqslant 0}
\XX^{(j)}(I)$. A symmetric operad $\PP$ is weight
graded if its underlying symmetric sequence
is weight graded and its composition maps
preserve the weight grading.
\end{definition}

Thus, a weight graded operad must have composition maps 
of the form
\[ \PP^{(a)}(k) \otimes 
	\PP^{(b_1)}(n_1) \otimes \cdots \otimes \PP^{(b_k)}(n_k)
	 	\longrightarrow \PP^{(b)}(n) \]
where $b=b_1+\cdots+b_k$ and $n = n_1+\cdots+n_k$. In the
case we consider partial composition maps, observe we have
instead maps of the form
\[\circ_i :  \PP^{(a)}(m)\otimes  \PP^{(b)}(n)
	\longrightarrow  \PP^{(a+b)}(m+n-1). \] 
The free operad $\FF_\XX$ is weight graded by the number
of internal vertices of a tree (that is, we put $\XX$ in
weight one, and extend the weight to trees by counting 
occurrences of elements of $\XX$. More generally, if
$\XX$ admits a weight grading, then $\FF_\XX$ inherits
this weight grading: the weight of a tree monomial is the
sum of the weight of the decorations of its vertices,
 and we write $\FF_\XX^{(n)}$ for the
homogeneous component of weight $n\in\NN_0$. If we do
not specify a weight grading on $\FF_\XX$, we will 
always assume we are taking the canonical weight grading above.


\begin{definition} An ideal in an operad $\PP$
is a subcollection $\mathcal{I}$ for which
both $\gamma(\mathcal{I}\circ \PP)$ and
$\gamma(\PP\circ_{(1)} \mathcal{I})$ are contained in
$\mathcal{I}$.
The quotient of $\PP/\mathcal{I}$ is again an
operad, called the quotient of $\PP$ 
by $\mathcal{I}$. Every subcollection $\RR$
of $\PP$ is contained in a smallest
ideal, called the \emph{ideal generated by $\RR$}.
\end{definition}

The notion of ideals and of free operads allow us
to define operads by generators and relations.

\begin{definition}
We write $\FF(\XX,\RR)$ for the quotient of 
$\FF_\XX$
by the ideal generated by a subcollection $\RR$
of $\FF_\XX$. 
 We say $\PP$ is presented by generators
$\XX$ and relations $\RR$ if there is an
isomorphism $\FF(\XX,\RR) \longrightarrow
\PP$.
\end{definition}

Note that if $\PP$ is symmetric, the definition
requires that $\mathcal{I}$ be stable under
the symmetric group actions, so we may 
sometimes specify $\RR$ by a generating set 
only, and understand
that $(\RR)$ is generated by the $\Sigma$-orbit
of $\RR$.

\bigskip

\textbf{Some examples.}
To illustrate the definitions above, let us
give three examples of algebraic operads whose
associated algebras are probably well known
to the reader: 
\begin{tenumerate}
\item The associative operad is generated by a 
binary operation $\mu$ generating the regular
representation of $S_2$ subject to the relation
$\mu\circ_1 \mu = \mu \circ_2 \mu$. 
\item The commutative operad is generated by a 
binary operation which instead generates the
trivial representation of $S_2$ and is
also associative. Both of this 
and the previous
example arise as the linearization of a set operad.
\item 
The Lie operad is generated by a single binary 
operation $\beta$ that generates the sign 
representation of $S_2$ subject to the only
relation
$(\beta \circ_1 \beta)(1+\tau+\tau^2) = 0$
where $\tau = (123)\in S_3$ is the $3$-cycle. 
\end{tenumerate}
We write these operads $\mathsf{As},\mathsf{Com}$
and $\mathsf{Lie}$ and, following J.-L. Loday,
call them the \emph{three graces}. We have that
\[ \As(n) = \kk S_n,\quad
 	\Com(n) = \kk, \quad
 	 \Lie(n) = \operatorname{Ind}_{\mathbb Z/n}^{S_n} \kk_\zeta \] 
 where $\kk_\zeta$ is a character of $\mathbb Z/n$
 for a primitive $n$th root of the unit. Concretely,
 the last equality is stating that if we fix a primitive
 $k$th root of unity $\zeta_k$, and if we let $\rho_k$ be
 the standard $k$-cycle of $S_k$, the free Lie
 algebra $L(V)\subseteq T(V)$ identifies  
 in each weight degree $k$ with those $v
 \in V^{\otimes k}$ such that $\rho_k v = \zeta_k v$. 
 
\begin{note} It is not always advantageous
to define an operad by generators and relations:
the operad pre-Lie can be defined explicitly
in terms of labeled rooted trees and a grafting
operation, as done by Chapoton--Livernet, and
this `presentation' is very useful in practice,
for example, to show that the pre-Lie operad
is Koszul.
\end{note}

\subsection{Quadratic operads}
An operad $\PP$ is \emph{quadratic} if it admits a presentation
$\FF(\XX,\RR)$ where $\RR \subseteq \FF(\XX)^{(2)}$. 
That is, $\PP$ is generated by some collection of
operations $\XX$ and all its defining relations are of the form
\[ \sum \lambda_{\mu,\nu}^i \mu \circ_i \nu = 0  \] 
where $\ari(\mu)+\ari(\nu)$ is constant. An operad is
\emph{binary quadratic} if moreover $\XX = \XX(2)$ or,
what is the same, all the generating operations of $\PP$
are of arity two (binary). 
 A \emph{quadratic-linear presentation} of an operad $\PP$
is a presentation $\FF(\XX,\RR)$ of $\PP$ where $\RR 
\subseteq \XX \oplus \FF(\XX)^{(2)}$. That is, it is
a presentation of the form
\[ \sum \lambda_{\mu,\nu}^i \mu \circ_i \nu 
 + \sum \lambda_\rho \rho = 0   \] 
 where $\ari(\mu)+\ari(\nu) = \ari(\rho)+1$ is constant.
 Every operad admits a quadratic-linear
presentation, albeit with possibly with infinitely many generators, 
We will postpone the discussion of such presentations
to a later lecture.
 
 Let us define a quadratic datum to be a pair $(\XX,\RR)$
 where $\XX$ is a symmetric sequence and $\RR\subseteq
 \FF_\XX^{(2)}$. A map of quadratic data $(\XX_1,\RR_1) 
 \longrightarrow (\XX_2,\RR_2)$ is a map $\XX_1
 \to \XX_2$ of symmetric sequences for which the induced
 map on free operads sends $\RR_1$ to $\RR_2$. The 
 assignment $(\XX,\RR) \longrightarrow \FF(\XX,\RR)$
 defines a functor from the category of quadratic data
 to the category of quadratic operads.
 
 %%Exercise: find two different QD giving same operad.
 \bigskip
 
\textbf{More examples.} The presentations of the 
associative, commutative and Lie operad above are
quadratic. The following are also quadratic operads:

\emph{The Gerstenhaber operad}. The symmetric operad
$\mathsf{Ger}$ and its cousin,
the Poisson operad $\mathsf{Poiss}$ belong to the two parameter
family $\mathsf{Poiss}(a,b)$ of binary quadratic operads generated
by two operations $x_1x_2$ and $[x_1,x_2]$ of respective degrees
$a$ and $b$, so that
the first is commutative associative, the second is a Lie
bracket, and they satisfy the Leibniz rule. With this
at hand $\mathsf{Ger} =\mathsf{Poiss}(0,-1)$ while  
$\mathsf{Poiss} = \mathsf{Poiss}(0,0)$.

\smallskip

\emph{The pre-Lie operad}. The operad $\mathsf{PreLie}$ and its
quotient, the Novikov operad $\mathsf{Nov}$, are quadratic
binary operads generated by a single operation $x_1\circ x_2$
with no symmetries. The first one is subject to the right-symmetry
condition for the associator
\[ 
x_1\circ (x_2\circ x_3) - (x_1\circ x_2)\circ x_3	=
x_1\circ (x_3\circ x_2) - (x_1\circ x_3)\circ x_2.
\]
The second 
operad is obtained by further imposing the left-permutative
relation that 
\[ x_1\circ (x_2\circ x_3) = x_2\circ (x_1\circ x_3).\]
The permutative operad $\mathsf{Perm}$ is the binary
operad generated by a single operation with no symmetries
satisfying the last quadratic equation.

\smallskip

\emph{The operad of totally associative $k$-ary algebras}. 
$\mathsf{tAs}_k$ (and its commutative counterpart). It is generated
by a $k$-ary non-symmetric operation $\alpha$ subject to
the relations $\alpha \circ_i \alpha =\alpha\circ_k \alpha$
for all $i\in [k]$. One can consider $\alpha$ to be
totally symmetric, and obtain the operad of totally
associative commutative $k$-ary algebras.

\smallskip

\emph{The operad of partially associative $k$-ary algebras}.
$\mathsf{pAs}^k$ (and its Lie counterpart). It is generated
by a $k$-ary non-symmetric operation $\alpha$ of degree $k-2$
subject to the single relation
\[ 
	\sum_{i=1}^k (-1)^{(k-1)(i-1)} \alpha\circ_i \alpha = 0.\]
One can consider a $k$-ary totally antisymmetric operation
$\beta$ of degree $1$, and obtain the operad of Lie $k$-algebras, 
which is subject to the single equation
\[
 \sum_{\substack{A\sqcup B = [2k-3] \\
 |A|=k-1,|B|=k-2}}  (\beta\circ_1\beta)\sigma_{A,B} = 0.
 \]

\smallskip

\emph{The operad of anti-associative algebras.} $\mathsf{As}^-$ is
generated by a single operation of degree zero with no symmetries
satisfying the `anti-associative law'
\[ x_1(x_2x_3) + (x_1x_2)x_3 = 0. \]

\subsection{Exercises}


 \begin{question}
During Lecture 3 we introduced the associative and commutative operads
through binary quadratic presentations. Show  that 
for all $n\geqslant 1$ the space $\mathsf{Ass}(n)$ is the
regular representation of $S_n$, and that for 
all $n\geqslant 1$ the space $\mathsf{Com}(n)$ is the 
trivial representation.
\end{question}

\begin{question} Use the presentation of the Poisson operad given
during Lecture 3 to show that $\dim\mathsf{Poiss}(n)\leqslant n!$
for all $n\geqslant 1$\footnote{There are at least three different
ways to show that equality holds.}. 
\end{question}

\begin{question} Let $x_1x_2$ be the associative binary generator of
$\mathsf{Ass}$ and let us consider the operations (which are symmetric
and antisymmetric, respectively)
\[x_1\cdot x_2 = \frac{1}{2}(x_1x_2+x_2x_1), \quad 
	 [x_1,x_2] = \frac{1}{2}(x_1x_2-x_2x_1) 	
	 \]
obtained by `polarization'. Show that the second is a Lie bracket,
and that the first is a commutative (but not associative) product
that satisfies the Leibniz rule for $[x_1,x_2]$, and whose associator
is equal to $[x_2,[x_1,x_3]]$. This is called the \emph{Livernet--Loday
presentation} of the associative operad.
\end{question}

\begin{question} 
During Lecture 3, we introduced to operad $\mathsf{tCom}_k$ of totally
associative commutative $k$-ary algebras. It is generated by a single
fully symmetric operation $\mu$ or arity $k$ subject to the relations
$\mu\circ_1\mu = \mu\circ_i\mu$
for each $i\in [k]$ (and all its symmetric translates).
Show that $\mathsf{tCom}_k(n)$ is either the one dimensional trivial
representation or zero depending on $n$. What values must
$n$ take so that it is non-zero? 
\end{question}

\begin{question}
The permutative operad $\mathsf{Perm}$ is generated by a single
binary operation $x_1x_2$ with no symmetries which is associative,
and such that
\[ x_1(x_2x_3) = x_2(x_1x_3). \]
Show that $\mathsf{Perm}(n)$ is of dimension $n$ and is
isomorphic as a representation to $\mathrm{Ind}_{S_{n-1}}^{S_n}\mathbb{C}$
where $\mathbb{C}$ is the trivial representation.
\end{question}



We have defined quadratic operads as precisely those
operads presented by (homogeneous) quadratic relations
on some set of generators. Let us explore how to create
maps between them.



\begin{question}
Suppose that $(\mathcal{X},\mathcal{R})$ and $(\mathcal{Y},\mathcal Q)$
are quadratic data. Show that a
map of sequences $f: \mathcal{X} \longrightarrow \mathcal{Y}$
induces a map on the corresponding quadratic operads if and only if
the induced map $F = \mathcal{F}_f$ sends $\mathcal{R}$ to
$\mathcal{Q}$.
\end{question}

\begin{question}
Show that:
\begin{tenumerate}
\item The augmentation map
$\mathbb C S_2\longrightarrow \mathbb C$ (that
sends $1$ and $(12)$ to $1$) induces a surjective map of
operads $\mathsf{Ass}
\longrightarrow \mathsf{Com}$.
\item The inclusion map
$\mathbb{C}^- \longrightarrow \mathbb{C}S_2$ that assigns $1$ to 
$1-(12)$ induces a map of operads $\mathsf{Lie}\longrightarrow 
\mathsf{Ass}$ and also a map of operads $\mathsf{Lie}\longrightarrow 
\mathsf{PreLie}$.
\item The projection $\mathsf{Ass}
\longrightarrow \mathsf{Com}$ actually factors
through $\mathsf{Perm}$. 
\end{tenumerate}
In each case, what is the interpretation at the level
of algebras?
\end{question}

\afterpage{\blankpage}
\newpage



 \section{Koszul duality I}
\textbf{Goals.}
Give the definition of the Koszul dual
operad of a quadratic operad, and then compute
some Koszul duals. Give the definition of the
Koszul complexes associated to a quadratic operad,
and define Koszul operads.

\subsection{Differential graded sequences}


\textbf{Homologically graded $\Sigma$-modules.}
A (homologically) graded vector space is a
vector space $V$ along with a direct sum
decomposition $V = \bigoplus_{n\in\mathbb Z} V_n$.
We call the components of this sum the \emph{graded (or
homogeneous)
components of $V$}, and say that an element in 
on of these summands is \emph{homogeneous}. If
$v\in V_n$, we say that $v$ is \emph{homogeneous of
degree $n$} and write $|v|=n$. 

A map $f : V\longrightarrow W$ of graded vector spaces
is \emph{homogeneous of degree $n$} if $f(V_j)\subseteq W_{j+n}$ for
all $j\geqslant 1$. We write $\hom(V,W)$ for the
space of all homogeneous maps, which is itself a graded
vector space with $\hom(V,W)_n$ the space of all
graded maps of degree $n$ for each $n\in\mathbb Z$. 
In this way, we obtain the category $\mathsf{Vect}_\mathbb{Z}$
of graded vector spaces and graded maps. 

A \emph{differential graded (dg) vector space} is a pair 
$(V,d)$ where $V$ is a graded vector space and 
$d : V\longrightarrow V$ is a homogeneous map of degree 
$-1$ such that $d^2=0$. We usually will call $(V,d)$
a \emph{chain complex}. The collection of homogeneous
maps $V\longrightarrow W$ is again a chain
complex, with differential
\[ d\varphi 
	= d_V\varphi - (-1)^{|\varphi|} \varphi d_W. \]
A homogeneous map of degree zero such that $d(\varphi)=0$
is called a \emph{chain map}.
It is
convenient to also consider \emph{cohomologically graded}
vector spaces, by formally inverting the order of $\mathbb{Z}$
and letting $V^n = V_{-n}$ for all $n\in\mathbb Z$. 

\bigskip

\textbf{Monoidal structure.} If $V$ and $W$ are
graded vector spaces, we define their tensor product
by setting
\[ (V\otimes W)_n = \bigoplus_{i+j = n} V_i\otimes W_j \]
for all $n\in\mathbb Z$, and setting the symmetry map
\[\tau : V\otimes W \longrightarrow W\otimes V\]
to be $\tau(v\otimes w) = (-1)^{|v||w|}w\otimes v$
on homogeneous elements, and extending it linearly on all of
$V\otimes W$. This makes $\mathsf{Vect}_\mathbb{Z}$ into a
symmetric monoidal category with unit the graded vector
space with $V_0 = \kk$ and $V_n = 0$ for $n\neq 0$.
The tensor product of maps $f: V\longrightarrow V'$
and $g : W\longrightarrow W'$ acts
in such a way that $f\otimes g : V\otimes V'
\longrightarrow W\otimes W'$ is the map
\[ (f\otimes g)(v\otimes w)  = (-1)^{|g||v|} f(v)\otimes g(w).\]
In case $V$ and $W$ are in fact dg, their tensor product is
also dg with $d_{V\otimes W} = d_V\otimes 1+ 1\otimes d_W$. 

\begin{definition}
A (homologically) graded $\Sigma$-module $\XX$
is a $\Sigma$-module taking values in the category
of graded vector spaces. Similarly, a dg $\Sigma$-module
is one taking values in dg vector spaces.
\end{definition}

\textbf{The endomorphism operad functor on dg modules.}
Let us consider the most natural way to create dg modules
from dg vector spaces, as we did in the case of usual
vector spaces. Namely, we may as before consider the
\emph{endomorphism operad} of a dg vector space $V$
by setting, for each $n\geqslant 0$,
\[ \End_V(n) = \hom(V^{\otimes n},V) \]
where these consists of homogeneous maps of dg vector
spaces. In particular, each of these arity components is 
itself a dg vector space, and the (total or partial)
composition maps
of the resulting operad are maps of dg vector spaces.

Of particular importance to us will be the \emph{suspension}
operation on dg vector spaces. Let us write $s$ for the
unique dg vector space with $s_1 = \mathbb C$ and zero
elsewhere, and similarly let us write $s^{-1}$ for the
unique dg vector space with $s^{-1} = \mathbb{C}$
and zero elsewhere. The \emph{suspension} of the dg vector
space $V$ is the tensor product $s\otimes V$, which
we write more simply $sV$, and whose basis elements we
write $sv$ for $v\in V$. Thus $|sv| = |v|+1$ for all 
homogeneous $v\in V$. Similarly, we define the
\emph{desuspension} $s^{-1}V$.

\begin{note}
The differential of $sV$ is given by $d(sv) = -s dv$. Can you explain why this is so using
the Koszul sign rule?
\end{note}
 
The following lemma shows that $V\mapsto \End_V$ is 
monoidal for the \emph{Hadamard product} of operads on the
target (and the usual tensor product on the domain): 
 \begin{lemma}\label{lemma:hadamard}
The map $\Phi : \End_V\otimes \End_W\longrightarrow \End_{V\otimes W}$
that assigns $\varphi \otimes \psi \in \End_V(n)\otimes \End_W(n)$
to the map
\[ \Phi(\varphi,\psi)(v,w) = (-1)^\varepsilon \varphi(v)\otimes\psi(w)\]
where $\varepsilon = \sum_{i=1}^n (|w_1| +\cdots + |w_{i-1}|+|\psi|)|v_i|$
is an isomorphism of operads provided $V$ and $W$
are locally finite.
 \end{lemma}
 
 \begin{proof}
 This is Exercise~\ref{ex:suspensions}
 \end{proof}

In particular, we see that $\End_{sV}$ is canonically isomorphic
with $\End_s\otimes \End_V$, and hence that algebra structures on $sV$
are related to algebra structures on $V$ through the operad $\End_s$.
Let us give it a name. 

\subsection{The Koszul dual}

\newcommand{\sus}{\mathscr{S}}
\textbf{Suspensions.} 
We call $\End_{s}$ the suspension operad
and write it $\sus$. Note that $\End_{s}(n)$ is
the sign representation of $\Sigma_n$ put in degree $1-n$.

\begin{proposition} For each $n\geqslant 1$ let us
we write $\nu_n$ for the unique map in $\End_s(n)$ 
that sends $s^n$ to $s$. Then for every $m\geqslant 1$
we have that
\[ \nu_n \circ_i \nu_m = (-1)^{(i-1)(m-1)} \nu_{m+n-1}. \]
In particular, the binary operation $\nu := \nu_2$ of degree
$-1$ generates $\End_s$,
and presents it as a quadratic operad subject to the 
anti-associativity relation
\[ \nu \circ_1\nu + \nu\circ_2 \nu = 0.\]
\end{proposition}
\begin{proof}
 This is Exercise~\ref{ex:suspensionoperad}.
\end{proof}

If $\PP$ is an operad, then the arity-wise tensor product
$\sus\otimes \PP$ is called the suspension of $\PP$
and we write it $\sus\PP$ or $\PP\{1\}$. Dually, we
write $\sus^{-1}$ for the desuspension operad
defined by $\End_{s^{-1}\kk}$. 

\begin{note} As we just observed,
the operad  $\sus\PP$ has the property that
$\sus\PP(sV) = s\PP(V)$, so that algebras over $\sus\PP$
are exactly those vector spaces $V$ such that $s^{-1}V$ is a
$\PP$-algebra. Equivalently, $sV$ is a $\sus\PP$-algebra
if and only if $V$ is a $\PP$-algebra. 
\end{note}

\textbf{Pairings.} We define a pairing between $\FF_\XX$ and
$\FF_{s^{-1}\sus^{-1}\XX^*}$ as follows (the appearance of
the suspensions will be evident later):
\[ \langle \Sigma\nu^* \circ_j \Sigma\mu*, 
	\rho \circ_i \tau  \rangle
   = \delta_{ij} (-1)^{\varepsilon}
   	\nu^*(\rho)\mu^*(\tau). \]
where $\varepsilon_1 = (\ari(\nu)-1)(|\mu|+i-1)+|\nu||\mu|$ and $\varepsilon_2$ counts the total
number of inversions in the shuffle permutations
appearing in the two tree monomials.
If $\XX = \XX(2)$ is binary and has no homological degrees, 
this simplifies to
\[ \langle \Sigma\nu^* \circ_i \Sigma\mu*, 
	\rho \circ_i \tau  \rangle
   =  \begin{cases}
    	(-1)^\varepsilon \nu^*(\rho)\mu^*(\tau) & i=1 \\
    	-	\nu^*(\rho)\mu^*(\tau) & i = 2.
    	\end{cases} \] 
where $\varepsilon$ depends on the decoration
of the leaves (it is $1$ if both decorations
are equal, and is $-1$ if exactly one is the
shuffle $132$. 

\begin{definition}
The Koszul dual operad of a quadratic operad $\PP$ 
generated by $\XX$ subject to relations $\RR$, is
the operad $\PP^!$ generated by $s^{-1}\sus^{-1}\XX^*$ 
and subject to the orthogonal space of relations
$\RR^\perp$ according to the pairing above.
\end{definition} 

\begin{note}
Let $\PP$ be an operad. Then $\PP$ is quadratic if and
only if $\sus\PP$ is quadratic, and it is Koszul if and
only if $\sus\PP$ is Koszul. 
\end{note}

\textbf{Some examples.} Let us compute the Koszul duals of 
some of the quadratic operads we introduced in \textbf{Lecture 3}.
For simplicity, we will consider only those with binary 
generators of degree zero, though one can in the same way
carry out computations with generators of higher arities and
varying homological degrees.

\bigskip

\emph{The associative operad}. We saw previously that for
$\underline{\XX}$ consisting of a single operation
$x_1x_2$ with no symmetries, the
space $\FF_\XX(3)$ is twelve dimensional, spanned by
the $S_3$-orbits of $\alpha = x_1(x_2x_3)$ and $\beta =(x_1x_2)x_3$,
each of size six. We also noted that $\alpha-\beta$
spans a six dimensional submodule, complemented by the
orbit of $\alpha+\beta$. 

Using the pairing above, we see that
\[\langle \alpha,\alpha\rangle = 1,
	\quad \langle\beta,\beta\rangle = -1,
	\quad \langle \alpha,\beta\rangle = 0, \] 
from where it follows that the dual space to the associativity
relation is the corresponding associativity relation
$\alpha^* - \beta^*$ in $\XX^*$. In other words,
the associative operad is Koszul self-dual:
\[\mathsf{Ass}^! = \mathsf{Ass}.\]

It is important to note how the minus sign in our
definition of the pairing or, more generally, the
Koszul sign we have introduced, guaranteeing that
this pairing in equivariant, introduces the minus sign
in the dual of $\alpha+\beta$.

\bigskip

\emph{The commutative and Lie operads.}
We have computed that if $\XX(2)$ is the trivial representation
of $S_2$ spanned by some commutative operation $x_1x_2$,
then $\FF_\XX(3)$ is three dimensional, spanned by
$x_1(x_2x_3)$, $(x_1x_2)x_3$ and $(x_1x_3)x_2$.
Moreover, we verified that if we put
\[ \alpha = x_1(x_2x_3) -(x_1x_2)x_3, 	\quad 
     \beta =  x_1(x_2x_3) -(x_1x_3)x_2 \]
     then these two element span an $S_3$-submodule
     that is complemented by the $S_3$-submodule generated by
     \[ \gamma = x_1(x_2x_3) +(x_1x_2)x_3+  (x_1x_3)x_2.\]
 This is in fact an orthogonal complement as a direct computation
 shows, so we see that the orthogonal set of relations
 to the commutative associative relation is the dual of
 $\gamma$ for the dual antisymmetric operation $[x_1,x_2]$:
 this is exactly the Jacobi relation
 \[ 
 \gamma^* = -[x_1,[x_2,x_3]] +[[x_1,x_2],x_3]+[[x_1,x_3],x_2].
 \]
 It follows that the Koszul dual of the commutative operad is the
 Lie operad, and conversely:
 \[ \mathsf{Com}^! = \mathsf{Lie}, \quad
  \mathsf{Lie}^! = \mathsf{Com}.
  	\]
With this at hand, one can compute that the Poisson operad is self-dual:
one only needs to address the Leibniz relation.  

\bigskip

\emph{The pre-Lie and permutative operads. The Novikov operad.}
Recall the pre-Lie operad is generated by a single operation $x_1x_2$
with no symmetries, subject to the pre-Lie relation
\[(x_1x_2)x_3 - x_1(x_2x_3) - (x_1x_3)x_2 + x_1(x_3x_2). \]
One can check that the $S_3$-orbit $V$ of this element is three dimensional,
so let us write $\alpha_1,\alpha_2$ and $\alpha_3$ for the translates of
this relation in $\FF_\XX(3)$. 

This orbit is complemented by the orbit $W$ of the associativity relation
$(x_1x_2)x_3 - x_1(x_2x_3)$ and the orbit $U$ of the permutative relation
$(x_1x_2)x_3 - (x_1x_3)x_2$. The first is six dimensional, as we already
computed, while the second is three dimensional. It is a direct computation
to check that $V^\perp$ identifies with the nine dimensional subspace
$U^*\oplus W^*$. 

Thus, we see that the operad of pre-Lie algebra is Koszul dual to that
of permutative algebras:
\[ \mathsf{PreLie}^! = \mathsf{Perm},\quad
 	\mathsf{Perm}^! = \mathsf{PreLie}.\]
One can use this to show that the operad controlling Novikov algebras,
those pre-Lie algebras whose product is \emph{left} permutative
\[ x_1(x_2x_3) = x_2(x_1x_3) \]
is almost Koszul self-dual: we have that $\mathsf{Nov}^! = 
\mathsf{Nov}^{\mathrm{op}}$, by which we mean the resulting
operad controls pre-Lie algebras with associator symmetric
in the \emph{first two} variables (left-symmetric) and 
whose pre-Lie operation is \emph{right} permutative.
\subsection{Exercises}

\begin{question}\label{ex:suspensions}
Show the map $\Phi_{V,W}$ of Lemma~\ref{lemma:hadamard}
is an isomorphism for $V$ and $W$ locally finite dimensional
dg symmetric sequences.
\end{question}

\begin{question}\label{ex:suspensionoperad}
Show that the suspension operad is binary quadratic
generated by a single operation $\nu$ of degree $-1$
that is ``anti-associative'', in the sense that
$\nu\circ_1\nu + \nu\circ_2\nu=0$. 
\end{question}
\begin{question} Show that:
\begin{tenumerate}
\item $\mathsf{Ass}$ is Koszul
self dual.
\item $\mathsf{Com}$ and $\mathsf{Lie}$
are Koszul dual to each other.
\item $\mathsf{PreLie}$
and $\mathsf{Perm}$ are Koszul dual to each
other.
\item the Poisson operad is Koszul self-dual.
\end{tenumerate}
\end{question}

\begin{question} 
The operad $\mathsf{Nov}$ of Novikov algebras
is the quotient of the (right) pre-Lie operad by the 
left permutative relation
$x_1(x_2x_3) = x_2(x_1x_3)$.
Show that $\mathsf{Nov}$ is Koszul dual to its
``opposite'' operad $\mathsf{Nov}^\mathrm{op}$
controlling left pre-Lie algebras satisfying the
right permutative relation. 
\end{question}

\begin{question}
Show that:
\begin{tenumerate}
\item 
The Koszul dual of the operad controlling
totally associative $k$-ary algebras is the
operad controlling partially associative
$k$-ary algebras.
\item The Koszul dual of the operad controlling
commutative totally associative $k$-ary algebras
is the operad controlling $k$-ary Lie algebras. 
\end{tenumerate}
\end{question}

\begin{question}
Let $x_1x_2$ be a binary operation and consider
the two relations:
\[
  R = (x_1x_2)x_3 -
   	\sum_{\sigma \in S_3} 
   		\lambda_\sigma \sigma(x_1(x_2x_3)),
   		\qquad
   		S = x_1(x_2x_3) - 
   	\sum_{\sigma \in S_3} 
   		\lambda_\sigma \sigma^{-1}((x_1x_2)x_3).
 	\]
Show that the resulting quadratic operads $\FF(x_1x_2)/(R)$
and $\FF(x_1x_2)/(S)$ are Koszul dual to each other.
\end{question}

\begin{question}
Show that in the case of binary operads,
the bilinear form we constructed 
during the lectures is $S_3$-invariant.
\end{question}

%\afterpage{\blankpage}
\newpage

\section{Shuffle operads}

\textbf{Goal.} Introduce shuffle operads
and prove that the free symmetric operad
on a reduced symmetric collection is
isomorphic, as a shuffle operad,
to the free shuffle operad on the
corresponding shuffle collection. 

\subsection{Shuffle operads}

Recall that the category of ns collections
on some category $\mathsf{C}$ 
consists of those pre-sheaves on
the category of finite ordered sets
and order preserving bijections with
values in $\mathsf{C}$: a ns collection
on $\mathsf{C}$ is simply a list of
objects of $\mathsf{C}$ indexed by
the non-negative integers (considered
as totally ordered sets of finite 
cardinality). 

\begin{definition}
An ordered partition $\pi$ of length $n$
of a finite totally order set
set is called \emph{shuffling} if
$\min \pi_i < \min \pi_{i+1}$ for
each $i\in [n-1]$. Equivalently, 
a surjection $f:I\longrightarrow [n]$
with $I$ a totally ordered set
is called \emph{shuffling} if
$\min f^{-1}(i) < \min f^{-1}(i+1)$
for each $i\in [n-1]$.
\end{definition}

Although totally ordered
sets along with bijections form a rather
dull category, this category
admits a composition product, which we call
the \emph{shuffle composition product},
defined as follows, and which will turn
out to be crucial for our purposes.

\begin{definition}
For each pair of ns collections $\XX$
and $\YY$, we define the ns collection
$\XX\circ_{\Sha} \YY$ so that on each
totally order finite set we have that 
\[
(\XX\circ_{\Sha} \YY)(I)
	=
	 \bigoplus_{\substack{r\geqslant 1
	 	\\ f: I \longrightarrow [r]}}
	 \XX([r])\otimes 
	 \YY(f^{-1}(1))
	 	\otimes
	 		\cdots
	 			\otimes
	 				\YY(f^{-1}(r))
\]
where the sum runs through all $r\geqslant 1$
and all possible 
shuffling surjections 
$f : I \longrightarrow [r]$.
\end{definition}

One can prove that this product is
associative, in the same way that one
proves $\circ_\Sigma$ and $\circ_\mathrm{ns}$
are. In some way, the shuffle composition
product interpolates between the symmetric
composition product, which contains ``too
many'' summands, and the ns composition
product, which contains too few. We
leave the following proposition as an exercise.

\begin{proposition}
The category of ns collections along with
the shuffle composition product is 
monoidal with the same unit as that of the
ns composition product. \qed
\end{proposition}

Note that we can also define a shuffle
Cauchy product, by looking at shuffling
partitions of a finite order set that
have length two. Although we will not study
the resulting monoidal category here,
we remark it gives rise to interesting
monoids, usually known as shuffle algebras.

\begin{definition}
A shuffle operad is a monoid in the category of 
ns collections with the shuffle composition 
product. 
\end{definition}

Thus, a shuffle operad consists of the
datum of a ns sequence $\PP$ along with
shuffle composition maps, one for each
shuffle partition $\pi$ of a finite ordered
set $I$ of the form
\[
\gamma_\pi : \PP(r)\otimes
		\PP(\pi_1)\otimes\cdots\otimes\PP(\pi_r)
		 	\longrightarrow \PP(I)
\]
that satisfy suitable associativity and
unitality axioms. Precisely, let us pick
a finite totally ordered set $I$,
a shuffling partition $\pi$ of $I$,
and let us assume that we pick a shuffling
partition $\pi^{(i)}$ of each block of $\pi$.
There is a unique way to order the collection
of blocks of these to obtain a shuffling
partition $\pi'$ of $I$.
For each part $\pi_i$ of $\pi$
and each $(g_i;\vec{h}_i) \in \PP(\pi_i)\otimes
\PP[\pi^{(i)}]$, let us write
$f_i = \gamma_{\pi^{(i)}}(g_i;\vec{h}_i)$,
and let $\vec{h}$ be obtained for the
tuple $(\vec{h}_1,\ldots,\vec{h}_r)$
by reordering the entries according to $\pi'$.
Then
\[ 
\gamma_\pi(f ; f_1,\ldots,f_r) =
 \gamma_{\pi'}(\gamma_\pi(f;g_1,
 \ldots,g_r); \vec{h} ).
	\]
Moreover, for each finite set $I$,
if $\{I\}$ and $I$ denote the corresponding
partitions into one block and into singletons,
we a fixed $1\in \PP(1)$ such that for
every $\nu\in\PP(I)$ we have
\[ \gamma_{\{I\}}(1;\nu) = \nu , 
\quad  \gamma_I(\nu ; 1,\ldots,1 ) = 
\nu.\]
Naturally, one can consider partial compositions
on a shuffle operad, but carefully noting that
for each $i$, there exist many different
shuffling partitions $\pi$ of the form
\[
 (1,\ldots,i-1,A,j_1,\ldots,j_s)
 	\]
 where $\min(A) = i$. Namely, for each
 $[n]$ we need simply choose a subset 
 $A$ of $[n]\smallsetminus [i-1]$ that contains
 $i$, and this can be done by choosing a subset
 of $[n]\smallsetminus [i]$ and appending 
 $i$. 
\begin{definition}
An ideal of a shuffle operad $\PP$ is a
ns subcollection $\mathcal{I}$ such that
\[ \gamma_\pi(\nu_0;\nu_1,\ldots,\nu_r)\in 
\mathcal{I}\] if at least one of $\nu_i$
is in $\mathcal{I}$ for some $i\in [0,r]$.
\end{definition}

As we will see later, ideals of shuffle operads
are slightly more refined than those in
symmetric operads. For example, the ideal 
generated by the left comb $(x_1x_2)x_3$
in a symmetric operad
automatically contains its two translates,
while in a shuffle operad, the three ideals
corresponding to these three possible shuffle
tree monomials are different. 

\subsection{Free shuffle operad}

Let us now give an explicit description
of the free shuffle operad on a ns
collection. Since we have already defined
the free symmetric and non-symmetric operad on
a collection (of the appropriate kind), we
already have almost all the language necessary to 
define it.

\begin{definition}
Let $\tau$ be a planar tree, which
we draw on the plane with the counter-clockwise
orientation. Begin
at the left side of root edge, and transverse the 
``boundary'' of the tree in the counter-clockwise
direction. This path will meet the vertices
of $\tau$ in some order, and we call this
total order the \emph{canonical planar order}
of its vertices.
\end{definition}

Observe that this
also orders the edges of $\tau$, and
the leaves (which are given the usual
left-to-right planar order). 

Now let $\XX$ be a ns collection and let
$T$ be a planar tree monomial with variables
in $\XX$, and let us pick a bijective labelling
$\mathsf{n} : L(\tau) 
\longrightarrow [n]$ of the leaves of
$\tau$. This induces a
labelling of the vertices of $\tau$
inductively by inductively labelling
$v$ with the minimum label appearing 
among its set of children. 

\begin{definition}
We say a leaf labelling of a planar tree 
monomial $T$
is shuffling if the induced order on the
children of each of its vertices coincides with
the canonical planar order. A pair
$(T,\mathsf{n})$ where $\mathsf{n}$
is a shuffling leaf labelling is called
a shuffle tree monomial.
\end{definition}

We now define the ns collection 
$\mathrm{Tree}^\Sha_\XX$ so that for each
finite totally ordered set $I$ the set
$\mathrm{Tree}^\Sha_\XX(I)$ consists of those
shuffle tree monomials on $\XX$ with shuffling
labellings by $I$. We write $\FF^\Sha_\XX$
for the corresponding linear ns collection.

\begin{figure}[h]
\[ 
	\leftc{}{}{1}{2}{3}
 \quad
		\leftc{}{}{1}{3}{2} 
		\qquad
		\rightc{}{}{1}{2}{3}
		\]
		\caption{The three
		shuffle trees with three
		leaves on a binary generator.}\end{figure}

Suppose that $T$ and $T'$ are shuffle tree
monomials on $[n]$ and $[m]$, that $i\in [n]$
and that we pick a shuffling partition $\pi$ of
$[m+n-1]$ whose only non-singleton part
is of the form 
\[ \{i=j_1,j_2,\dots,j_m\}.\]
We define the tree monomial $T\circ_\pi T'$
by grating the tree $T'$ at the leaf of $T$
labelled by $i$, with its leaf labels
renumbered through the unique order
preserving bijection $j_i \longmapsto
i$, and we renumber the leaf labels
of $T$ distinct from $1,\ldots,i-1$ using
the remaining blocks of $\pi$. This
defines the ``partial shuffle composition''
of shuffle tree monomials.

\begin{figure}
\begin{tikzpicture}
\tikzstyle{inner}=[circle,draw=black, fill=white, inner sep=1pt,minimum size=7.5 pt]
\tikzstyle{leaf}=[circle, draw=white, fill=white, inner sep=3 pt,minimum size=5 pt]

	\begin{pgfonlayer}{main}
		\node [style=inner] (0) at (0, 0) {};
		\node [style=leaf] (1) at (-4, 2) {$1$};
		\node 				(2a) at (-3, 2) {$\cdots$};
		\node [style=leaf] (2) at (-2, 2) {$i-1$};
		\node [style=inner] (3) at (-0.5, 2) {};
		\node [style=leaf] (4) at (-2.25, 4) {$i$};
		\node [style=leaf] (4a) at (-1.25, 4) {$\cdots$};
		\node [style=leaf] (5) at (1, 4) {$j_m$};
		\node [style=leaf] (6) at (0, 4) {$j_{m-1}$};
		\node [style=leaf] (7) at (1, 2) {$k_1$};
		\node [style=leaf] (8) at (2, 2) {$k_2$};
		\node 				(8a) at (3, 2) {$\cdots$};
		\node [style=leaf] (9) at (4, 2) {$k_n$};
		\node [style=leaf] (10) at (0, -2) {};
	\end{pgfonlayer}
	\begin{pgfonlayer}{bg}
		\draw (0.center) to (1.center);
		\draw (2.center) to (0.center);
		\draw (0.center) to (3.center);
		\draw (3.center) to (4.center);
		\draw (3.center) to (6.center);
		\draw (3.center) to (5.center);
		\draw (7.center) to (0.center);
		\draw (0.center) to (8.center);
		\draw (0.center) to (9.center);
		\draw (0.center) to (10.center);
	\end{pgfonlayer}
\end{tikzpicture}

\caption{The two-level trees corresponding
 to partial compositions of shuffle operads}
\end{figure}
We may as well define the ``total shuffle
composition'' of a tree $T_0$ with trees
$T_1,\ldots,T_n$ along a shuffling partition
$\pi = (\pi_1,\ldots,\pi_n)$ with $T_i$ having
as many leafs as $\pi_i$ for each $i\in [n]$
Concretely, we consider for each such $i$
the unique order preserving bijection
between $\pi_i$ and the labels of $T_i$,
and graft $T_i$ at the input of $T_0$
labelled by $\min \pi_i$. 

\begin{proposition}
The shuffle composition of shuffle tree
monomials is again a shuffle tree
monomial.
\end{proposition} 

\begin{proof}
This is Exercise~\ref{ex:shufflecomp}. The
idea is to note that the local increasing
condition is not broken, and this is clear
on each $T_i$ since we simply relabelled their
leafs with an isomorphic totally order
set, 
while it is not broken on
$T_0$ since we grafted the $T_i$s using
a shuffling partition.
\end{proof}

With this at hand, we can state and prove the
main result in this section. 

\begin{proposition}
The ns collection
$\FF_\XX^\Sha$ with its corresponding
shuffle composition is the \emph{free shuffle
operad} generated by $\XX$, where the
inclusion $\XX\longrightarrow \FF_\XX^\Sha$
sends an element in $\XX$ to the corresponding
corolla with its unique shuffling leaf
labelling. \qed
\end{proposition}

\subsection{Forgetful functor}

Since every finite totally order 
set $I$ is in particular a finite set
$I^{\f}$
after forgetting the order,
we have a functor
$\XX \longmapsto \XX^{\f}$ that assigns
a symmetric collection $\XX$ to the ns
collection $\XX^{\f}$ such that
\[ \XX^{\f}(I) = \XX(I^{\f}) \]
for each finite order set $I$. 
We call this the \emph{forgetful functor}
from symmetric to ns collections. 
The following will be central in what
follows.

\begin{proposition}
The forgetful functor $\SMod \longrightarrow 
\nsMod$ is strong monoidal for the corresponding 
symmetric and shuffle composition products
when restricted
to \emph{reduced} collections, in the sense
that for each pair $\XX$ and $\YY$ with
$\YY$ reduced there is a natural isomorphism
\[
(\XX\circ_\Sigma \YY)^{\f} \longrightarrow
 \XX^{\f}\circ_\Sha \YY^{\f}.
\]
\end{proposition}

\begin{proof}
Let us begin by proving that if $\YY$ is a
reduced symmetric sequence then $\YY^{\otimes n}$
is a free $S_n$-module for every $n\geqslant 1$.
This is of course true for $n=1$. For $n>1$,
it suffices to exhibit an $S_n$-basis. 
For each finite totally ordered set $I$, let
us consider the components of $\YY^{\otimes n}(I^{\f})$,
and note that since $\YY$ is reduced they are of the form
\[
\YY(\pi_1)\otimes \YY(\pi_n)
\]
where $\pi$ is a partition of $I$ into $n$ blocks with
at least one element. For each such partition $\pi$
of $I$, there exists a unique permutation $\sigma\in S_n$
such that $(\sigma\pi)_i = \pi_{\sigma^{-1}(i)}$ is
shuffling, and this proves that $\YY^{\otimes n}(I^{\f})$ is
isomorphic to the free $S_n$-module generated
by $(\YY^{\f})^{\otimes_\Sha n}(I)$.
It follows that for each $n\geqslant 1$ we have a
natural isomorphism
\[ 
\XX(n)\otimes_{S_n}\YY^{\otimes n}(I^{\f})
 \longrightarrow 
  \XX^{\f}(n)\otimes (\YY^{\f})^{\otimes_\Sha n}(I)
\] 
which gives us the desired isomorphism
$
(\XX\circ_\Sigma \YY)^{\f} \longrightarrow
 \XX^{\f}\circ_\Sha \YY^{\f}.
$
\end{proof}

\begin{corollary}
For each reduced symmetric collection $\XX$, 
there is a natural isomorphism of shuffle operads
\[
(\FF_\XX^\Sigma)^{\f}
 \longrightarrow \FF_{\XX^{\f}}^\Sha.
\]
Moreover, if $I$ is an ideal in $\FF_\XX^\Sigma$
then $I^\f$ is an ideal in $\FF_{\XX^\f}^\Sha$ and the
resulting quotient shuffle operads are naturally
isomorphic via the induced map
\[
(\FF_\XX^\Sigma/ I )^{\f}
 \longrightarrow \FF_{\XX^{\f}}^\Sha/ I^\f.
\]
\end{corollary}

In particular, shuffle tree monomials on $\XX^{\f}$, when
considered with their non-planar tree structure,
give us a basis of the free symmetric operad on 
$\XX$, and we can study any presentation of a symmetric
operad through the resulting presentation of the
corresponding shuffle operad. 

\subsection{Exercises}

\begin{question}\label{ex:shufflecomp}
 Show that the shuffle composition of shuffle tree
monomials is again a shuffle tree
monomial.
\end{question}
 
\begin{question}
Use the definition of shuffle trees
to compute a basis
of $\mathrm{Tree}_\XX^\Sha(4)$ in case $\XX$ consists
of a single symmetric or antisymmetric
operation. What happens if the operation
is not symmetric?
\end{question}

\begin{question} Explain how 
$\XX^f \circ_\Sha \YY^f$ fails
 to identify with  $(\XX\circ_\Sigma \YY)^f$ 
 in case $\YY$ is
 not reduced. 
\end{question}

\begin{question}
Go through the definition of the shuffle
compositions $\gamma_\pi$ for shuffle
tree monomials, and show that it maps
shuffle tree monomials to shuffle tree
monomials.
\end{question}

\begin{question} 
Give an example of a shuffle operad that
is not obtained from a symmetric operad
through the forgetful functor. \emph{Suggestion:}
ideals coming from symmetric operads must be
stable under the (now non-existent group action).
Can you find a shuffle ideal that is ``not very
symmetric''?
\end{question}

\begin{question}
Write down a presentation of the following as
shuffle operads: the  
commutative operad, the
Lie operad, the
associative operad, and the
operad of $3$-ary totally commutative
associative algebras.
\end{question}

\begin{question}
Repeat the theme of the last four exercises
with any other (quadratic) operad
of your choice.
\end{question}

\newpage
\section{Monomial orders}


\subsection{Some reminders}

In the following, we will anchor ourselves in the 
rewriting theory that exists for associative monoids
in sets and the corresponding theory for
associative algebras. Since we are not assuming
the reader is familiar with this, let us give
a brief recollection of the basics.

\begin{definition}
An associative monoid is a set $M$ along with
an associative multiplication $\mu : M\times M
\longrightarrow M$. Given a set $X$, we write
$\langle X\rangle$ for the free monoid on 
$X$, which is given by the set
$
 \bigsqcup_{n\geqslant 1} X^{n}
 $
of all \emph{words the alphabet $X$} with
product the isomorphism
$ X^n\times X^m \cong X^{m+n}$
for each $m,n\geqslant 1$.
\end{definition}

We are interested in finding bases of 
free objects by ideals and, to do this,
we will resort to ordering our free objects.
This will allow us to give a (terminating)
algorithm whose input will be a set of 
relations and an ordering, and whose
output (among other things) will 
be a basis of our quotient object.

\begin{definition}

An ordered monoid is a pair $(M,\prec)$ where $M$ is a monoid
and $\prec$ is a total order on $M$ that satisfies the
following three conditions:
\begin{tenumerate}
\item It is a well-order: every non-empty subset of $M$
has a minimum. 
\item The product map of $M$ is increasing in both of its
arguments for $\prec$. 
\end{tenumerate}
A \emph{monomial order} on the alphabet $X$ is, by definition,
the structure of an ordered monoid on the free monoid $\langle X\rangle$
generated by $X$.
\end{definition}

Explicitly, the last condition requires that if $m_1,m_2,m_3\in M$
and if $m_1\prec m_2$ then it follows that $m_3m_1\prec m_3m_2$ 
and $m_1m_3\prec m_2m_3$. If the alphabet $X$ is given a total order, 
then we can produce a monomial order on it as follows:

\begin{definition}
Let $\prec$ be a total order on $X$. 
The graded lexicographic order on $\langle X\rangle$
induced by $\prec$, which we write $\prec_\ell$,
is such that $w\prec_\ell w'$ if and only if
\begin{tenumerate}
\item The word $w$ is shorter than $w'$, or else
\item We have
$w = w_1 xw_2$ and $w' = w_1 y w_2'$ with
$x\prec y$ in $X$.
\end{tenumerate}
\end{definition}

It is important to note that the lexicographic order defined only
by the second condition is \emph{not} a well-order, and it is 
not increasing for the concatenation product: for example,
if $x\prec y$ then $x \prec x^2$ but $x^2y \prec xy$. 

\begin{lemma}
The graded lexicographic order is a monomial order on $X$
for any choice total order $\prec$.
\end{lemma}

\begin{proof}
It is clear that the resulting order is total, for either
two words are of distinct length, or they are of the same length
and differ at and entry, or else they are equal. To see the
order behaves well with respect to the concatenation product,
we observe that the function $w\longmapsto \mathrm{Length}(m)$
is additive for the concatenation product, so if $w$ is longer
than $w'$, then $ww''$ will be longer than $w'w''$ and,
similarly, $w''w$ will be longer than $w''w'$. If $w$ and 
$w'$ have the same length, then it is clear that
$w''w'\prec w''w$ if and only if $w'w''\prec ww''$ if
and only if $w' \prec w$. To see that the order is a well
order, let us consider a collection $W$ of words. Then,
in particular, there exists a least natural number $n$
such that $W$ contains words of length $n$ but not of $n-1$. 
In this case, it follows that the minimum of $W$, if it
exists, must be contained in the set $X^n$, and this
set is well ordered by the lexicographical order if
$X$ it itself well ordered: we can find the minimum 
by induction on $n$.
\end{proof}

We now recall from \textbf{Lecture 1} the definition
of the \emph{word operad of a monoid $M$}.

\begin{definition}
Let $M$ be an associative monoid. The symmetric
operad $\mathbb W_M$ is defined by $\mathbb{W}_M(n) =
M^n$ for each $n\geqslant 1$, and its partial composition
product is defined for each $s,t\geqslant 1$ and each
$i\in [s]$ by the rule
\[(m_1,\ldots,m_s) \circ_i (m_1',\ldots,m_t') = 
 	(m_1,\ldots,m_{i-1}, m_im_1',\ldots,m_im_t',m_{i+1},\ldots, m_s).\] 
\end{definition}

The reader should verify that $\mathbb W_M$ is isomorphic,
as a symmetric sequence, to the composition product $\mathrm{Ass}\circ M$,
where we consider $M$ a symmetric sequence concentrated in arity $1$. 

\subsection{Two statistics}

Of particular interest to us is the case $\XX$ is
a reduced symmetric sequence in sets, and we let
$\underline{\XX} = \bigsqcup_{n\geqslant 1} \XX(n)$
be the underlying alphabet of $\XX$. We will use
the notation $\XX^*$ for the free monoid $\langle 
\underline{\XX}\rangle$. By definition, there
exists a unique map of shuffle operads
$
\pi : \FF_\XX^\Sha \longrightarrow
						\mathbb W_{\XX^*} 				
$ 
extending the map $\XX \longrightarrow \mathbb W_{\XX^*}$
that assigns $x\in \XX(n)$ to the element
$(x,\ldots,x)\in \mathbb W_{\XX^*}(n)$.

\begin{definition}
For each shuffle tree monomial $T$, we call 
$\pi(T)$ the \emph{path sequence of $T$.}
\end{definition}

The path sequence of a shuffle tree monomial $T$ can
be computed in a straight-forward way, as the following
lemma shows. The useful observation that the previous
definition allows us to make is that the path
sequence statistic is compatible with shuffle 
compositions of tree monomials, in the sense the
path sequence of a composition of tree 
monomials equals the compositions
of the corresponding path sequences of these
tree monomials.
 
\begin{figure}[h]
\[ 
	\leftc{x}{y}{1}{2}{3}
 \quad
		\leftc{x}{y}{1}{3}{2} 
		\qquad
		\rightc{y}{x}{1}{2}{3}
		\]
		\[ \hspace{0.4 in}(yx,yx,y)
			\hspace{.8 in} (yx,y,yx)
				\hspace{0.8 in}(y,yx,yx)\]
		\caption{An example of the computation of path sequences.}
		\label{fig:paths}
		\end{figure}
		
\begin{lemma}
Let $\XX$ be reduced. 
The path sequence of $T$ is the tuple in $\mathbb{W}_{\XX^*}(n)$
where $n$ is the number of leaves of $T$, obtained by recording at the
$i$th entry the word in $\XX$ read by travelling from the root of
$T$ to the leaf labelled by $i$.
\end{lemma}

More generally, in case $\XX$ has $0$-ary variables, we must look at 
all \emph{endpoints} of a tree monomial. Since we will not be
interested in non-reduced alphabets, we let the curious reader
explore this modification on their own. It is useful to remark
in situations like this that $\underline{\XX}$ is obtained 
through a disjoint union of the components
of $\XX$: the path sequence of $x\circ_1 y$ for $x$ and $y$ unary
is $(xy)$, and the path sequence of $x\circ_1 y'$ for $x$ unary and
$y'$ nullary is `also' $(xy')$, but these are \emph{distinct}
in the free monoid $\XX^*$. 

\begin{proof}
This is Exercise~\ref{ex:pathseq}.
\end{proof}

Let us now consider the unique  map of shuffle operads
$\sigma : \FF_\XX^\Sha \longrightarrow
						\mathsf{Ass}$ 
extending the map $\XX \longrightarrow \mathsf{Ass}$
that assigns $x\in \XX(n)$ to the identity
$1\in \mathsf{Ass}(n) = S_n$. 

\begin{definition}
If $T$ is a shuffle tree monomial. 
we call $\sigma(T)$ the (leaf) permutation sequence of $T$.
We call the pair $(\pi(T),\sigma(T))$ the
path-permutation data of $T$. 
\end{definition}

As before, this statistic of $T$ has a simpler description,
that can be read off directly from $T$, and the previous
definition tells us that the leaf permutation
sequence of a tree monomial behaves well with
respect to shuffle compositions.

\begin{figure}
\[
\fork{x}{y}{x}{1}{3}{2}{4}
\qquad
\fork{x}{y}{x}{1}{4}{2}{3}
\qquad
\fork{x}{y}{x}{1}{2}{3}{4}
\]
\caption{Three different shuffle three monomials with the same path
sequence, but different permutation sequence.}
\end{figure}

\begin{lemma}
The permutation sequence of $T$
 is obtained by reading the leaf labelling of $T$ from
left to right and recording it as a permutation in
``two line notation''.
\end{lemma}

The main result of this section tells us that it suffices
for us to order sequences of words in the alphabet $\XX$
and permutations in order to order shuffle tree monomials.

\begin{theorem}
The map $\FF_\XX^\Sha \longrightarrow \mathbb{W}_{\XX^*}\times
\mathsf{Ass}$ of the free shuffle operad on $\XX$ into the 
Hadamard product of $\mathbb{W}_{\XX^*}$ and $\mathsf{Ass}$
induced by $\pi$ and $\sigma$ is injective.
In other words, the path-permutation datum of a shuffle
tree monomial determines it uniquely.
\end{theorem}

Let us call the map in the statement of the theorem the
\emph{path-permutation inclusion}.

\begin{proof}
We will sketch a proof, and ask the reader to fill in the
details as an exercise; we proceed by induction on the
total length of the path sequence of a tree monomial so that,
for example, the path sequences appearing in Figure~\ref{fig:paths}
have all length five. First, let us show that the path sequence
determines the planar structure of our tree monomial uniquely:

If the length is zero, then the path sequence $\pi$
is empty, and we are simply considering the trivial tree monomial.
Let us consider now some positive length $\ell$ and search,
among all words $w$ appearing in $\pi$, that which has
the largest possible length and smallest possible coordinate,
let us say this coordinate is $i$.

If $w$ ends in a $0$-ary variable of $\XX$, this means the
$i$th leaf of $T$ ends at a stump, and we can remove it, 
and continue by induction. If not, then $w$ ends with
some variable $x\in \XX(k)$, and the way we have chosen
it implies that the $i$th leaf (in the planar order)
is the first child of $x$, and that all other children
of $x$ are also leaves. It follows that $w$
and the next $k-1$ words in $\pi$ all end with $x$,
and that $\pi$ is obtained as a non-symmetric composition
with $(x,\ldots,x)$. By pruning $x$ from $\pi$, we
can proceed by induction. 

Now that we know the path sequence recovers the planar structure
of $T$ uniquely, let us pick some path-permutation datum
$(\pi,\sigma)$. Then, reorder the entries of $\pi$ using
$\sigma^{-1}$ to recover the planar structure of $T$,
and then label its leafs according to $\sigma$, to 
recover the whole shuffle structure.
\end{proof}


\subsection{Ordered shuffle operads}

We can now proceed to define ordered shuffle operads. 

\begin{definition}
A set shuffle operad $\PP$ is order if for each $n\geqslant 0$
the component $\PP(n)$ is well-ordered and if shuffle compositions
are increasing in each of its arguments: for each $n\geqslant 1$,
all elements $(T_0;T_1,\ldots,T_n) \in\PP(k)\times \PP(n_1)
\times \cdots\times \PP(n_k)$ and all shuffling partitions
of $[n_1+\cdots +n_k]$, we have that
\[ 
\gamma_\pi(T_0;T_1,\ldots,T_i, \ldots, T_n)  \prec
\gamma_\pi(T_0;T_1,\ldots,T_i',\ldots,T_n)
\]
whenever $T_i\prec T_i'$  for some $i\in [0,n]$ as elements
of $\PP(n_1+\cdots+ n_k)$. 
\end{definition}

In particular, we can apply this definition in the case $\PP$
is the free set shuffle operad on some alphabet $\XX$. As
promised, let use 
the injection $(\pi,\sigma)$ to endow tree monomials with
well-orders. 

\begin{proposition}
Let $(M,\prec)$ be an ordered monoid. The
word operad on $\mathbb W_M$ is an ordered operad
through the lexicographical order of words.
\end{proposition}

\begin{proof}
This is Exercise~\ref{ex:orderedM}.
\end{proof}

In particular, we can consider the case in which $M = \XX^*$
is endowed with the graded lexicographical order induced by
a total order on $\underline{\XX}$, which implies the following
corollary.

\begin{corollary}
Suppose that $\underline{\XX}$ is given a total order, and that
we give the free monoid $\XX^*$ the induced graded 
lexicographical order. Then
the word operad $\mathbb{W}_{\XX^*}$ is an ordered
shuffle operad with the lexicographical order.
\end{corollary}

We leave it as an exercise to the reader to show that the
associative operad is an ordered shuffle operad if we
use on it the lexicographic order on permutations
(seen as strings of numbers, in one line notation).
All our work is now, done:

\begin{definition}
Let $\XX$ be an alphabet and suppose that we
give the monoid $\XX^*$ a monomial order $\prec$.
The \emph{path-permutation extension} of $\prec$ is the
unique order on $\FF_{\XX}^\Sha$ induced by
the path-permutation inclusion, where we use the
the induced lexicographic
order on $\mathbb{W}_{\XX^*}$ first, and the
lexicographic order on $\mathsf{Ass}$ second.
\end{definition}

Naturally, one can switch the roles of the two
factors of the path-permutation inclusion to get
the \emph{permutation-path extension} of a monomial
order on $\XX^*$. We will explore other variations
in the exercises.

\begin{definition}
Let us fix a total order $\prec$ on $\underline{X}$, and let us
consider the induced graded lexicographic order on 
$\XX^*$, where we first compare the lenght of a 
word, and then use the lexicographic order induced
by the total order. The path-permutation extension 
on $\FF_\XX^\Sha$ is called the  \emph{graded path-permutation 
lexicographic order} induced by $\prec$.
\end{definition}

\hacer{Add examples of orderings.}

\newpage

\subsection{Exercises}
 
\begin{question}\label{ex:pathseq}
Show that the path sequence of a tree monomial, as defined
using the universal property of the free shuffle operad,
coincides with its combinatorial definition obtained
by reading the entries of the tree from the
root to the leaves.
\end{question}
 
\begin{question}
Let $X$ be a finite set and let us give 
$\langle X\rangle$ the graded lexicographical order with
respect to a fixed total order on $X$. Show this is a
monomial order.
\end{question}

\begin{question}\label{ex:orderedM}
Suppose $(M,\prec)$ is an ordered monoid and
we let us give the shuffle operad
$\mathbb{W}_M$ the induced lexicographical 
order. Show that the resulting order is a monomial
order.
\end{question}

\begin{question}
Consider the ns collection $\XX$ with $\underline{\XX} = \XX(2)$
a singleton. Show that we can always find a monomial order that
singles out one of the three shuffle tree monomial basis
elements of $\FF_\XX^\Sha(3)$ as the largest. 
\end{question}

\begin{question}
Consider the ns collection $\XX$ with $\underline{\XX} = \XX(2)
= \{x,y\}$, and the ``mixed'' shuffle tree monomials in 
$\FF_\XX^\Sha(3)$ that have $x$ and $y$ (one at the top,
the other at the bottom). Explore what leading terms
you can obtain by choosing different induced orders
on $\FF_\XX^\Sha$.
\end{question}

 \pagebreak
 
\section{Gr\"obner bases}

\textbf{Goal.} Define the long division algorithm for shuffle
tree polynomials. Prove Gr\"obner bases for shuffle operads exist
and reduced Gr\"obner bases are unique.

\subsection{Tree insertion}

\begin{definition}
Let $T'$ and $T$ be tree monomials over some fixed alphabet $\XX$.\
We say that $T'$ divides $T$ if the underlying tree $\tau$
of $T$ contains a subtree $\tau_0$ isomorphic to the underlying
tree $\tau'$ of $T'$, whose induced shuffling labelling
 and decorations coincide with that of $T'$.
 \end{definition}
\begin{figure}[h]
\[
\fork{x}{y}{x}{1}{3}{2}{4}
\mathsymbol{1.5}{\text{is divisible by}}
\begin{tikzpicture}[scale = .75]
\tikzstyle{inner}=[circle,draw=black, fill=white, inner sep=1pt,minimum size=5pt]
\tikzstyle{leaf}=[circle, draw=white, fill=white, inner sep=3 pt,minimum size=5 pt]
	\begin{pgfonlayer}{main}
		\node [style=inner] (0) at (0, 0) {$x$};
		\node [style=inner] (2) at (-1, 1) {$x$};
		\node [style=leaf] (9) at (1, 1) {$2$};
		\node [style=leaf] (10) at (0, -1.25) {};
		\node [style=leaf] (11) at (-1.5, 2) {$1$};
		\node [style=leaf] (12) at (-0.5, 2) {$3$};
	\end{pgfonlayer}
	\begin{pgfonlayer}{bg}
		\draw (2.center) to (0.center);
		\draw (0.center) to (9.center);
		\draw (0.center) to (10.center);
		\draw (2.center) to (11.center);
		\draw (12.center) to (2.center);
	\end{pgfonlayer}
\end{tikzpicture}
\mathsymbol{1.5}{\text{and by}}
\begin{tikzpicture}[scale = .75]
\tikzstyle{inner}=[circle,draw=black, fill=white, inner sep=1pt,minimum size=5pt]
\tikzstyle{leaf}=[circle, draw=white, fill=white, inner sep=3 pt,minimum size=5 pt]
	\begin{pgfonlayer}{main}
		\node [style=inner] (0) at (0, 0) {$x$};
		\node [style=leaf] (2) at (-1, 1) {$1$};
		\node [style=inner] (9) at (1, 1) {$y$};
		\node [style=leaf] (10) at (0, -1.25) {};
		\node [style=leaf] (13) at (0.5, 2) {$2$};
		\node [style=leaf] (14) at (1.5, 2) {$3$};
	\end{pgfonlayer}
	\begin{pgfonlayer}{bg}
		\draw (2.center) to (0.center);
		\draw (0.center) to (9.center);
		\draw (0.center) to (10.center);
		\draw (13.center) to (9.center);
		\draw (9.center) to (14.center);
	\end{pgfonlayer}
\end{tikzpicture}
\]
\caption{A ``fork'' and its divisors of weight two. Notice the induced labelling of the right comb, coming from the
shuffling labelling $1<2<4$.}
\end{figure}

The following lemma asserts that the combinatorial notion
of divisibility coincides with the algebraically inclined
notion of divisibility, that of belonging to the ideal
generated by the divisor. 

\begin{lemma}
A tree monomial $T$ is divisible by another
tree monomial $T'$ if and only if it can be obtained
from $T'$ by iterated shuffle compositions with
other tree monomials.
\end{lemma}

\begin{proof}
It is clear that if $T$ is obtained from $T'$ by
iterated shuffle compositions with tree monomials,
then $T$ is divisible by $T'$. Conversely, suppose
$T'$ divides $T$. If the root of $T'$ is not that of
$T$, then we can write $T$ as a composition of several
tree monomials, one which is divisible by $T'$ and
which shares the root with it, so we may assume this is
the case. One this is done, we see that $T$ is in fact
obtained by grafting tree monomials at the leaves of $T'$,
and completes the proof.
\end{proof}

\begin{figure}[t]

\[
\Leftc{x}{y}{x}{1}{3}{2}{4}
\mathsymbol{1.5}{\text{is divisible by}}
\leftc{x}{y}{1}{3}{2}
\mathsymbol{1.5}{\text{and by}}
\leftc{y}{x}{1}{2}{3}
\]
\caption{A right comb with two divisors that have
the same underlying planar structure but different
induced labelling.}
\end{figure}
\begin{definition}
Suppose that $T'$ is a divisor of $T$, and let us
assume that $T'$ has $\ell'$ leaves and $T$
has $\ell$ leaves. We define the insertion operation
\[
\repl{T'}{T}{-}: \FF_\XX^\Sha(\ell') \longrightarrow 	
 \FF_\XX^\Sha(\ell) 
\]
that replaces the divisor $T'$ of $T$ by any other
shuffle tree monomial with $\ell'$ leaves
in $T$, and extend it linearly, making sure
that leaf labels are respected.
\end{definition}
\begin{figure}[b]
\[
\Leftc{x}{x}{x}{1}{2}{3}{4}
\mathsymbol{1.5}{\text{ is divisible ``twice'' by}}
\leftc{x}{x}{1}{2}{3}
\]
\caption{A right comb with two divisors that have the
same induced shuffle tree structure, but happen
at different places of the tree.}
\end{figure}
\begin{lemma}
Let $\mathcal{V}$ be a subset of the free shuffle operad on $\XX$.
Then the ideal generated by $\mathcal{V}$ is explicitly obtained
as the linear span of all insertions $\repl{T'}{T}{f}$
as $T,T'$ range through pairs $(T',T)$ with $T'$ a divisor
of $T$ and $f\in \mathcal V(\ell')$.
\end{lemma}

\begin{figure}[h]
\[
\begin{tikzcd}
{} &\phantom{\hspace{-3.5 em}}\Leftc{x}{x}{x}{}{}{}{} 
	\arrow[dr] 
	\arrow[dl] \\
\fork{x}{x}{x}{}{}{}{}&  {} & \Leftr{x}{x}{x}{}{}{}{}
\end{tikzcd}
\]
\caption{Two possible results of substituting
a left comb by a right comb in the leftmost tree.}
\end{figure}

\begin{proof}
By construction, $(\mathcal V)$ is the linear span of all possible
shuffle compositions where at least one summand is contained in
$\mathcal V$. Since shuffle compositions are multilinear, we can
assume that all terms appearing in such shuffle compositions (except,
possibly, for that in~$\mathcal V$) are tree monomials, in which case
the resulting shuffle composition coincides with an insertion
operation.
\end{proof}

\subsection{Long division}

Suppose that we fix a tree monomial order on the free
shuffle operad $\FF_\XX^\Sha$ and $f \in \FF_\XX^\Sha(n)$
is a tree \emph{polynomial}. The support of $f$ is the (finite)
set of tree monomials that appear in $f$ with non-zero coefficient.
We say the tree monomial $T$ is
the \emph{leading monomial of $f$} if $T$ is the largest monomial
that appears in the support of $f$, and we write the corresponding
summand in $f$ by $\leadm{f}$. This summand is accompanied by a
coefficient, which we call the leading coefficient of $f$ and
write $\leadc{f}$. Thus, any $f$ can be written in the form
\[ f = \leadc{f} \leadm{f} + f_0 \]
where all monomials appearing in $f_0$ are smaller than $\leadm{f}$.
We call $\leadc{f} \leadm{f}$ the leading term of $f$ and write
it $\lead{f}$. We begin with a preparatory result.

\begin{figure}[h]
\[ 
	\mathsymbol{1.5}{f_1 =}	\underline{\leftc{x}{y}{1}{2}{3}}
\mathsymbol{1.5}{+}		\leftc{x}{y}{1}{3}{2} 
		\mathsymbol{1.5}{+}	
		\rightc{y}{x}{1}{2}{3}
		\]
		\[ 
	\mathsymbol{1.5}{f_2 =}	\leftc{x}{x}{1}{2}{3}
\mathsymbol{1.5}{-}	\underline{\leftc{x}{y}{1}{3}{2} }
		\]
		\[ 
	\mathsymbol{1.5}{f_3 =}	 \underline{\leftc{x}{y}{1}{2}{3}}
		\mathsymbol{1.5}{-}	
	\rightc{x}{x}{1}{2}{3} \mathsymbol{1.5}{+}	 \rightc{y}{y}{1}{2}{3}
		\]
		\caption{Leading terms underlined in some tree polynomials, for
		the \texttt{grpathpermlex} order induced by $y>x$.}
		\label{fig:orders}
		\end{figure}
		
		
\begin{proposition}\label{prop:repllead}
Suppose that $T'$ is a divisor of $T$ and let $f\in \FF_\XX^\Sha(\ell')$.
Then the leading term of the insertion $\repl{T'}{T}{f} = \square_{T'}^T(f)$
is equal to $\repl{T'}{T}{\lead{f}}$.
\end{proposition}

\begin{proof}
For tree polynomials $f_0,f_1,\ldots,
f_n$ and any shuffling partition $\pi$, we have that
\[ \lead{\gamma_\pi(f_0;f_1,\ldots,f_n) }= 
 		\gamma_\pi(\lead{f_0};\lead{f_1},\ldots,\lead{f_n}).
 		\]
This is clear, since the shuffling compositions are by definition
(strictly) increasing for $\prec$. Since we know that $\repl{T'}{T}{f}$
is obtained from $f$ be iterating shuffle compositions, this proves
the statement of the proposition. 
\end{proof}

Let $\mathcal{V}$ be a subset of the free shuffle operad on $\XX$. A
tree monomial $T$ is reduced with respect to $\mathcal{V}$ if it is
not divisible by any of the leading terms of polynomials appearing in it.
A polynomial is reduced with respect to $\mathcal V$ if it is a linear
combination of tree monomials that are reduced. We say $\mathcal V$
is self-reduced if each $v\in \mathcal V$ is reduced with respect to
$\mathcal V\smallsetminus v$, and that is is linearly self-reduced
if no leading term of one element divides the leading term of 
another. 

\begin{definition}[Reduction]\label{def:reduce}
Suppose that $f$ and $g$ are polynomials of the same arity, and
that $f$ is divisible by the leading term of $g$, in other words,
suppose that $\lead{f} = \repl{T'}{T}{\lead{g}}$ for some
tree monomial $T$ and a divisor $T'$. In this case,
the reduction of $f$ with respect to $g$, which we write
$r_g(f)$, is defined by
\[
r_g(f) = f - \frac{\leadc{f}}{\leadc{g}} \repl{T'}{T}{g}.
\]
\end{definition}

The following lemma tells us that the reduced term $r_g(f)$ 
behaves like a ``remainder by division'', in the sense it is
either zero of ``smaller than $f$''.

\begin{lemma}\label{lemma:smaller}
For all $f$ and $g$ such that $r_g(f)$ is defined, either
$r_g(f)= 0$, or else we have that $\lead{r_g(f)} \prec \lead{f}$.
\end{lemma}

\begin{proof}
If $r_g(f)$ is non-zero, we have that its leading coefficient is 
equal to the leading coefficient of $f- \frac{\leadc{f}}{\leadc{g}}\repl{T'}{T}{g}$. But the leading term of the second term is, by
Proposition~\ref{prop:repllead}, equal to 
\[ \frac{\leadc{f}}{\leadc{g}}\repl{T'}{T}{\lead{g}} =  \leadc{f} \leadm{f} 
	= \lead{f}.\]
It follows that the terms appearing in $r_g(f)$ are all strictly smaller
than the leading term of $f$, as we wanted. 
\end{proof}

\textbf{Long division algorithm.}
Let us now define the long division algorithm for shuffle operads.
Its input is a polynomial $f$ and a finite set $\mathcal V$, both
in $\FF_\XX$, and its
output is a reduced element $\overline{f}$ with respect to $\mathcal
V$ such that $f-\overline{f} \in (\mathcal V)$ and $\lead{\overline{f}}
\preceq \lead{f}$. We can verbosely describe the algorithm as follows: 
\begin{tenumerate}
\item If our input polynomial
$f$ is zero, just return zero. If not, ensure $\mathcal V$
is linearly self-reduced, using the linear self-reduction
Algorithm~\ref{algo:gauss}.
\item If there is an element $v$ in $\mathcal V$ 
whose leading term divides the
leading term of $f$, pick that with the largest leading term.\footnote{This
 can be done since we already ensured $\mathcal V$ is linearly
self reduced.}
\item Let $f'$ be the remainder of division of $f$
by $v$ using the reduction procedure of Definition~\ref{def:reduce}.
Recursively call the algorithm to compute the result of long
division of $f'$ by $V$.
\item If not, then $\lead{f}$ is $\mathcal V$-reduced, 
so let $f'$ be the result of long division of $f-\lead{f}$ by $\mathcal V$, and return $\lead{f} + f'$.
\end{tenumerate}
The following is what this algorithm looks like in pseudo-code:
\begin{algorithm}
\caption{Long division algorithm}\label{euclid}
\begin{adjustwidth}{1 cm}{1 cm}
\begin{algorithmic}[1]
\Procedure{LongDivision}{\texttt{TreePolynomial},\texttt{TreePolynomials}}
\If {$\texttt{TreePolynomial}=0$} \Return $0$
	 \Else
	 \State $\texttt{Dividend} \gets \texttt{TreePolynomial}$
	 \State $\texttt{Divisors} \gets 
	 	\textsc{LinearSelfReduce}(\texttt{TreePolynomials})$
	 \State $\texttt{Factors} \gets \{ v\in \texttt{Divisors} :\leadm{v} \text{ divides } \lead{g}\}$
	\If {$\texttt{Factors} \neq \varnothing$}
	\State $\texttt{LargestFactor}\gets \textsc{Largest}(\texttt{Factors})$   
	\State $\texttt{Dividend}\gets \textsc{Reduce}(\texttt{Dividend},\texttt{LargestFactor})$ 
	\State $\texttt{Dividend}\gets \textsc{LongDivision}(\texttt{Dividend},\texttt{Divisors})$  
		\EndIf
	\State $\texttt{LeadDividend} \gets \lead{\texttt{Dividend}}$
	\State $\texttt{Dividend} \gets \texttt{Dividend}-\texttt{LeadDividend}$
	\State $\texttt{Dividend} \gets \textsc{LongDivison}(\texttt{Dividend},\texttt{Divisors})$
	\EndIf
	\State \Return $\texttt{LeadDividend} + \texttt{Dividend}$
\EndProcedure
\end{algorithmic}
\end{adjustwidth}
\end{algorithm}


\begin{lemma}
The long division algorithm terminates, and its
output is a reduced element $\overline{f}$ with respect to $\mathcal
V$ such that $f-\overline{f} \in (\mathcal V)$ and $\lead{\overline{f}}
\preceq \lead{f}$.
\end{lemma}


\begin{proof}
At each step, the leading monomial of $f$ is decreased Lemma~\ref{lemma:smaller},
so the fact that $\prec$ is a well order guarantees our algorithm
terminates. It also guarantees that the output will satisfy
$\lead{\overline{f}} \preceq \lead{f}$. Let us suppose, for the sake
of a contradiction, that the output of the algorithm is not
always reduced. Among such problematic polynomials $f$,
let us pick one $f$ with the smallest leading term, which is
possible since $\prec$ is a well order. If $\lead{f}$ is not
reduced, then the first step of our algorithm applies long
division to $r_g(f) = f'$ for some $g\in\mathcal V$, and
by Lemma~\ref{lemma:smaller}, this is either zero or has a smaller
leading term than $f'$, so it must be reduced. If $\lead{f}$
is reduced, then the second step of the algorithm applies long
division to $f-\lead{f}$, which has smaller leading term than
$f$, so the output is again reduced. Finally, note that
at each step of the algorithm we subtract an element of
$\mathcal V$, so the coset of $f$ is not modified,
which concludes our proof.
\end{proof}

\begin{proposition}
Suppose that $\mathcal I$ is an ideal in the free shuffle
operad generated by $\XX$. Then those shuffle monomials that
are reduced with respect to $\mathcal I$ form a basis
for the quotient operad $\FF_\XX / \mathcal I$.  
\end{proposition}

\begin{proof}
Let us first show that these monomials span the quotient operad.
This follows, for the long division algorithm guarantees
we can always replace a non-reduced polynomial $f$ by a reduced
one without affecting its coset. To see they are linearly
independent, suppose that $f$ is a polynomial reduced with
respect to $I$, and that $f\in \mathcal I$. Then we see that
$\lead{f} \in \lead{\mathcal I}$. But this can only happen if
$f=0$ (or else $f$ would not even be linearly reduced with respect
to $\mathcal I$). 
\end{proof}

In practice, we
have little control over the multitude of leading terms
that may appear in the elements of $\mathcal I$,
and Gr\"obner bases are designed to regain this control. 
 
 \bigskip
 
 \textbf{Self-reduction algorithm.} Suppose that
 $\mathcal V$ is a finite subset of $\FF_\XX^\Sha$. 
 The following
 algorithm takes as input this generating set,
 and outputs a self-reduced subset $\mathcal V'$
 that generates the same ideal as $\mathcal V$. 
 This is what this algorithm looks like in pseudo-code, 
 though we are being slightly imprecise: $\mathcal V$ is
 not a matrix, so we cannot feed it to our linear self
 reduction algorithm as is: we pick a total order on
 $\mathcal V$, and then use the corresponding matrix
 written in the shuffle tree monomial basis. 
 
 \begin{algorithm}
\caption{Self-reduction algorithm}\label{algo:self-reduce}
\begin{adjustwidth}{2 cm}{2 cm}
\begin{algorithmic}[1]
\Procedure{SelfReduce}{$\texttt{Polynomials}$}
	\State $\texttt{ToReduce} \gets \textsc{LinearSelfReduce}(\texttt{Polynomials})$
\If {$\texttt{ToReduce}$ is self reduced} \Return $\texttt{ToReduce}$
\Else 
\State $\texttt{Largest} 
	\gets \textsc{Largest}(\texttt{ToReduce})$
\State $\texttt{ToReduce} \gets \textsc{SelfReduce}(\texttt{ToReduce}\smallsetminus \texttt{Largest} )$
\State $\texttt{NewElement} \gets \textsc{LongDivision}(\texttt{Largest} ,\texttt{ToReduce})$
\State $\texttt{ToReduce}\gets \texttt{ToReduce}\cup \texttt{NewElemet}$
\EndIf
\State \Return $\textsc{SelfReduce}(\texttt{ToReduce})$
\EndProcedure
\end{algorithmic}
\end{adjustwidth}

\end{algorithm}

 \begin{proposition}\label{lemma:self-reduction}
 The self-reduction algorithm terminates for each
 finite set $\mathcal V$ and returns a self-reduced
 set $\mathcal V'$ with $(\mathcal V) = (\mathcal V')$.
\end{proposition}  

\begin{proof}
This is Exercise~\ref{ex:self-reduction}.
\end{proof}
\subsection{Existence and uniqueness}

\begin{lemma}
Let $\mathcal I$ be an ideal of $\FF_\XX$. The subspace 
\[
\lead{\mathcal I} = \langle T \in \FF_\XX : \text{ $T = 
	\lead{f}$ for some  $f\in\mathcal I$} \rangle.
\]
spanned by leading terms of elements of $\mathcal I$ is
again an ideal of $\FF_\XX$.
\end{lemma}

\begin{proof}
By construction $\lead{\mathcal I}$ is a subspace of $\FF_\XX$, so 
it suffices we prove it is an ideal. By multilinearity, it suffices
to show it is stable under compositions with respect to tree
monomials if at least one term is already in $\lead{\mathcal I}$.
To do this, note that if
$T$ is the leading term of some $f\in\mathcal I$, then by
Proposition~\ref{prop:repllead}, for any tree monomials
$T_0,\ldots,T_{i-1},T_{i+1},\ldots,T_n$, the leading term
of the composition 
\[ \gamma_\pi(T_0;T_1,\ldots,T_{i-1},f,T_{i+1},\ldots,T_n)\in \mathcal I\]
is precisely $\gamma_\pi(T_0;T_1,\ldots,T_{i-1},T,T_{i+1},\ldots,T_n)$,
which proves this belongs to $\lead{\mathcal I}$.
\end{proof}

\begin{definition}
Let $\mathcal I$ be an ideal of $\FF_\XX$. We say that a subset
$\mathcal G$ of $\mathcal I$ is a \emph{Gr\"obner basis of $\mathcal I$}
(with respect to our fixed monomial order) if the set of leading monomials
of $\mathcal G$ generate the ideal of leading terms of $\mathcal I$.
A Gr\"obner basis which is self-reduced is called reduced.
\end{definition}

\begin{lemma}
Let $\mathcal I$ be an ideal and let $\mathcal G$ be a 
Gr\"obner basis of $\mathcal I$. Then $\mathcal G$ 
generates $\mathcal I$.  
\end{lemma}

\begin{proof}
Suppose that there is some $f\in \mathcal I$ that is not
generated by $\mathcal G$, and let us pick one with the least 
possible leading term. Since $\mathcal G$ generates the
ideal of leading terms of $\mathcal I$, we can reduce
the leading term of $f$ with respect to $\mathcal G$
without modifying its coset in $\mathcal I$,
and obtain an element that is generated by $\mathcal G$.
But then $f$ itself is generated by $\mathcal G$,
which is a contradiction.
\end{proof}


\begin{proposition}
A set $\mathcal G$ is a Gr\"obner basis if the cosets of monomials
reduced with respect to it form a basis of the quotient operad. 
In this case, the result of long division of a polynomial by
$\mathcal G$ is independent of the choices or the order in
 which we perform the reductions. 
\end{proposition}

\begin{proof}
To begin, observe that 
the cosets of monomials that are reduced with respect
to $\mathcal G$ form a basis of the quotient operad precisely
when every coset of $\mathcal I$ contains a unique element
that is reduced with respect to $\mathcal G$. 

By the long division algorithm, it follows that every coset
contains at least one element that is reduced with respect to $\mathcal G$,
so it suffices we prove that this element is unique if and only if
$\mathcal G$ is a Gr\"obner basis.

Thus, first suppose that $\mathcal G$ is a Gr\"obner basis, but that
there exist two $\mathcal G$-reduced monomials that have the same
coset modulo $\mathcal I$. This means there exists a $\mathcal G$-reduced 
polynomial in $\mathcal I$, which means that its leading term is
$\mathcal G$-reduced, which is impossible.

Conversely, Suppose that $\mathcal G$ is not a Gröbner basis. It follows
that there is an element $f\in\mathcal I$ which is reduced with respect to 
$\mathcal G$. If we let $\overline{f}$ be the result of the long division of
$f$ by $\mathcal G$, we see we obtain a non-trivial linear combination of reduced 
monomials belonging to $\mathcal I$, so that there is not a unique
reduced representative for the zero coset.

Finally, suppose that for some $f$, two different choices of order of reductions yield two different outputs. Then, then there exist a coset $f + I$ contains two different elements that are reduced with respect to $\GG$, hence reduced monomials are linearly dependent, a contradiction.
\end{proof}

\begin{theorem}
Every ideal admits a unique reduced Gr\"obner basis.
\end{theorem}

\begin{proof}
We begin by proving uniqueness, which will in fact tell us
how to prove these exist. Thus, suppose $\GG$ is a 
Gr\"obner basis of an idea $\II$, so that
$\leadm{\GG}$ generates $\leadm{\II}$. If $\GG$ is
also reduced, then $\leadm{\GG}$ coincides with the set
\[ 
 \mathcal M = \{ T\in \leadm{\II} : \text{$T$ is not divisible by any
 other element of  $\leadm{\II}$}.
 	\}
	\]
of all minimal elements of $\leadm{\II}$ partially ordered
with respect to divisibility. To see this, not that if
$T\in \mathcal M$ then this must be divisible by at least
one element $g$ of $\GG$, and this can only happen 
if $\leadm{g} = T$. Conversely, if $T$ is a leading monomial
in $\leadm{\GG}$ then it is certainly a leading monomial
of $\leadm{\II}$. If $T'$ is any other
leading monomial of $\leadm{\II}$ that divides $T$, then there is
$T''\in \leadm{\GG}$ that divides $T'$, and hence $T$. But
since $\GG$ is reduced, this happens only if $T''=T$, and
hence no other leading term divides $T$.

In addition, the fact that $\GG$ is reduced guarantees
that for each $T\in\leadm{\GG}$ there exists a unique element
$g\in\GG$ such that $g= T-h$ and $h$ is reduced with
respect to $\II$. It follows that $h$ must be equal to the
unique element in the coset $T+\II$ that is reduced
with respect to $\II$.

To prove existence, we consider the set $\mathcal{M}$ above,
and let $\GG$ consist of those elements of the form $T-h$
where $T\in \mathcal{M}$ and $h$ is the unique element in
the coset $T+\II$ that is reduced with respect to $\II$. 
By our definition of $\mathcal M$, the set $\GG$ is self-reduced, 
so it suffices we show that it is a Gr\"obner basis. 
To do this, notice that every element of $\leadm{\II}$
is divisible by an element of $\mathcal M$: if not,
the smallest element which is not divisible by some
element of $\mathcal M$ is either not divisible by
any other element of $\leadm{\II}$, which makes it an element
of $\mathcal M$, or otherwise is divisible by some
smaller element of $\leadm{\II}$, and hence actually
does have a divisor from $\mathcal M$. Thus,
$\leadm{\GG}$ generates $\leadm{\II}$, as we wanted.
\end{proof}

It is important to point out that our proof above is highly
non-constructive: we are considering the poset of
leading terms of $\II$ under divisibility, 
which we admits a (possibly infinite) set of 
minimal elements, and arguing these constitute the
reduced Gr\"obner basis of~$\II$. In the next
lecture, we will learn how to begin with any generating
set of $\II$, and complete it to a (possibly infinite)
reduced Gr\"obnber basis.
 
\subsection{Exercises}

\begin{question}
Prove the claim made in the last paragraph above:
the ideal of leading terms of $\II$, partially ordered by
divisibility, admits a possibly infinite set of
minimal elements. \emph{Hint:} divisibility
refines our choice of total order $\prec$: if 
$T$ divides $T'$, then $T\prec T'$. 
\end{question}

\begin{question}\label{ex:self-reduction}
Translate the self-reduction algorithm into prose.
and prove Proposition~\ref{lemma:self-reduction}.
\end{question}

\vfill

\hacer{More exercises for Section 8} 

\afterpage{\blankpage}
\newpage

\section{Computing Gr\"obner bases}

\textbf{Goal.} 
Define overlapping ambiguities and $S$-polynomials of overlapping
ambiguities. State and prove Bergman's Diamond Lemma. 
Give Buchberger's algorithm for computing Gr\"obner bases.

\subsection{$S$-polynomials}

As a motivating example, consider the operad
with a single binary operation $x_1x_2$ and no
symmetries, which is \emph{anti-associative}, that is,
\[
\underline{(x_1x_2)x_3}+ x_1(x_2x_3)  = 0.
\] 
At the same time, let us choose the usual \texttt{grapathpermlex}
order so that the leading term of the relation above
is the underlined one. The following computation shows that,
among the leading terms of elements in the ideal
generated by this relation, all trees of weight three appear,
which might be at first unexpected:
\begin{align*}
\underline{((x_1x_2)x_3))x_4} + (x_1(x_2x_3))x_4 -
\underline{((x_1x_2)x_3)x_4}- ((x_1 x_2)(x_3x_4)) &=  \\
 (x_1(x_2x_3))x_4 - ((x_1 x_2)(x_3x_4)) &= \\
 (x_1(x_2x_3))x_4 - ((x_1 x_2)x_3)x_4 &=  2(x_1(x_2x_3))x_4.
\end{align*}
If the characteristic is not two, then this term is non-zero
and, up to a sign, equal to the other four tree monomials
of weight three. Thus, it follows that this quadratic operad
is in fact three dimensional! Let us introduce the device 
that captures this behaviour.

\begin{definition}
Let $g_1,g_2$ be two shuffle polynomials over an alphabet $\XX$.
We say that the monomials $\leadm{g_1}$ and $\leadm{g_2}$ for
an overlap ambiguity if they have a \emph{small common multiple},
that is, there exists tree monomial $T$ properly divisible by  
$\leadm{g_1}$ and $\leadm{g_2}$, such that $T$ is the result
of merging these along an overlap. The element
\[
S_T(g_1,g_2) = 
\repl{T_1}{T}{g_1} - \repl{T_2}{T}{g_2}	
\]
is called the $S$-polynomial of this overlapping ambiguity.
\end{definition}

\hacer{Example of associative operad}

\begin{definition} Let $\II$ be an ideal generated by some
set $\GG$, and for an element $f\in \II$, let us consider
a representation as a linear combination of insertions of
elements of $\GG$, of the form
\[
f = \sum_{i\in I} \lambda_i \repl{S_i}{T_i}{g_i}
\]
where $T_i = \leadm{g_i}$ for all $i\in I$. We call the element
$S = \max \{ S_i : i \in I\}$ the parameter of this representation.
\end{definition}

Note that any $S$-polynomial $S_T(g_1,g_2)$ admits a representation
of parameter $T$ (although this term will cancel!). We say a 
representation of $S_T(g_1,g_2)$ is non-trivial if its parameter
is smaller than $T$.

\subsection{Diamond Lemma}
The following result is one of the most useful ways one can
verify a subset of an ideal is a Gr\"obner basis.

\begin{theorem}[Diamond Lemma]
For a self-reduced set of generators $\GG$ of an ideal $\II$,
the following statements are equivalent:
\begin{tenumerate}
\item The set $\GG$ is a Gr\"obner basis of $\II$.
\item Every $S$-polynomial reduces to zero modulo $\GG$.
\item Every $S$-polynomial admits a non-trivial representation.
\item Every $f\in \II$ admits a representation with
parameter $\leadm{f}$.
\end{tenumerate}
\end{theorem}

\begin{proof}
The implications $(4)\Longrightarrow (1)\Longrightarrow (2)
\Longrightarrow (3)$ are straightforward, and we leave them
as a guided exercise. The hardest part of the proof is
showing that (3) implies (4), which we go through in
detail here. To prove it, let us assume that (3) holds, but
that (4) does not. Then, there exists some $f\in\II$
so that every representation of $f$ has parameter 
larger than $\leadm{f}$, and let us pick a representation
giving a counterexample with the following properties:
\begin{tenumerate}
\item The parameter $T$ of the representation is minimum.
\item Among representations with parameter $T$, we choose one
with $k=\{ i\in I : S_i=T\}$ minimum.
\end{tenumerate}
We now consider those divisors $T_1,T_2,\ldots,T_k$
of $T$ appearing in the representation and focus on the last two
divisors $T_{k-1}$
and $T_k$: notice that $k>1$, for else there is not room for
cancellations to occur so that the leading monomial of $f$ is smaller than $T$. 

We claim that we can always arrange it so that
we obtain a new representation of $f$ that breaks either
the first or the second condition. To do this, we will consider
the relative position of the divisors $T_{k-1}$ and $T_k$
in $T$.

\emph{Case 1:} one divisor is contained in the other. In this
situation, this means that $g_k = g_{k-1}$ since $\GG$ is
reduced, which implies we can merge these two terms into one.
If their coefficients sum to zero, then either we get a 
representation with smaller parameter if $k=2$, or with
smaller $k$ in general, which cannot be.

\emph{Case 2:} the divisors are disjoint. 
\emph{Case 3:} the divisors are disjoint.
\hacer{Write (3) implies (4).}
\end{proof}
\subsection{Buchberger Algorithm}

\subsection{Exercises}
\newpage
\appendix

\section{Algebras over operads}

\hacer{Write an appendix about algebras over operads.}

\subsection{Algebras over operads}
Operads are important not in and of themselves 
but through their representations, more commonly
called \emph{algebras over operads}. In fact,
one can usually `create' an operad by declaring
what kind of algebras it governs. If the
algebra has certain operations of certain
arities, these define the generators of the operad,
and the relations these operators must satisfy 
give us the relations presenting our operad. 

\begin{definition}
A $\PP$-algebra structure on a vector
space $V$ is the datum of a map of operads 
$\PP \longrightarrow \End_V$.
\end{definition}

Alternatively, one can consider the situation
when $\otimes$ is closed and has a right
adjoint $\hom$ (the internal hom) so that
what we want are maps
\[\label{eq:Palgebramap}
\gamma_{V,n} : 
\PP(n)\otimes_{S_n} V^{\otimes n} \longrightarrow V \]
declaring how each $\mu \in\PP(n)$ acts as an operation
$\mu : V^{\otimes n} \longrightarrow V$. 
It follows that a
 $\PP$-algebra structure on $V$ is the same as the datum
of maps as in \ref{eq:Palgebramap} that satisfy the following
conditions:

\begin{tenumerate}
\item \emph{Associativity:} let $\nu\in \PP(n)$ and
$\nu_i \in \PP(k_i)$ for $i\in [n]$, and pick
$w_i \in V^{\otimes k_i}$. Set $v_i = \gamma_{V,k_i}(\nu_i,w_i) \in V$
and $\mu = \gamma_\PP(\nu;\nu_1,\ldots,\nu_n)$. Then 
\[ \gamma_{V,k_1+\cdots+k_n}(\mu; w_1,\ldots,w_n) = 
	\gamma_{V,n}(\nu ;v_1,\ldots, v_n).\]
\item \emph{Equivariance:} for $\nu\in\PP(n)$, $v_1\otimes\cdots \otimes v_n \in V^{\otimes n}$
and $\sigma\in S_n$, we have that
\[ \gamma_{V,n}(\nu \sigma, v_{\sigma 1} \otimes \cdots \otimes v_{\sigma n}) = \gamma_{V,n}(\nu; v_1\otimes\cdots \otimes v_n). \]
\item \emph{Unitality:} if $1\in\PP(1)$ is the unit, then
$\gamma_{V,1}(1,v) = v$.  
\end{tenumerate}



\hacer{Example: recognition
principle. Gerstenhaber algebras and Hochschild
complex. }

\pagebreak


\section{Classical theory}

\subsection{Linear self-reduction}
Let $M$ be a matrix with $m$ rows
and $n$ columns. The following is the
usual algorithm to put $M$ in row reduced
echelon form, written as pseudo-code:
\begin{algorithm}
\caption{Linear self-reduction algorithm}\label{algo:gauss}
\begin{adjustwidth}{2 cm}{2 cm}
\begin{algorithmic}[1]
\Procedure{LinearSelfReduce}{$\texttt{Matrix}$} 
\State $\texttt{lead} \gets 0$
\State $\texttt{A} \gets \texttt{Matrix}$
\State $\texttt{rows} \gets \textsc{Width}(\texttt{A})$
\State $\texttt{cols} \gets \textsc{Height}(\texttt{A})$
	\For {$r\in [0,\texttt{ rows})$}
 			\If {\texttt{cols} $\leqslant$\texttt{ lead}}
 			\State \textbf{stop}
 			\EndIf
 	\State $i\gets r$
 	\While {$\texttt{A}[i,\texttt{lead}]=0$}
 			\State $i\gets i+1$
 			 \If {\texttt{rows} $=i$}
  					\State $i\gets r$
  					\State $\texttt{lead} \gets \texttt{lead}+1$
  					\If {\texttt{cols} $=$\texttt{ lead }}
   						\State \textbf{stop}
   					\EndIf
   			\EndIf
   	\EndWhile
   \If {$i\neq r$}
   		\State \textbf{swap} $\texttt{A}[r,:]$ and $\texttt{A}i,:]$
   		\State $\texttt{A}[r,:]
   			\gets \texttt{A}[r,:]/\texttt{A}[r,\texttt{lead}]$
   		\For {$i \in [0,\texttt{rows})$}
   			 \If {$i\neq r$}
   			 \State $\texttt{A}[i,:] \gets 
   			 	\texttt{A}[i,:] - 
   			 		\texttt{A}[i, \texttt{lead}]\texttt{A}[r,:] $
   			\EndIf
   			\EndFor
   \State $\texttt{lead} \gets \texttt{lead}+1$
   \EndIf
\EndFor
\EndProcedure
\end{algorithmic}
\end{adjustwidth}
\end{algorithm}

\subsection{Non-commutative Gr\"obner bases}
\hacer{Explain the analogous results for associative
algebras, and the case of the Steenrod algebra.}

\afterpage{\blankpage}

\newpage

\listoftodos


\bibliographystyle{alpha}
\bibliography{bibliography}

\end{document}


\section{Koszul duality II}

\subsection{Bar construction}

\subsection{Twisting cochains}

\subsection{The Koszul complexes}

\subsection{Exercises}
\newcommand{\antishriek}{\text{\raisebox{\depth}{\textexclamdown}}}
\begin{definition}
A quadratic operad is Koszul if one (and hence all)
of the following equivalent conditions are
satisfied:
\begin{tenumerate}
\item The cohomology group $H^s(\mathsf{B}(\PP))$ is zero for $s>0$.
\item We have $\mathrm{rate}(\PP)=1$.
\item The inclusion
$H^0(\mathsf{B}(\PP)) \longrightarrow \mathsf{B}^*(\PP)$
is a quasi-isomorphism. 
\end{tenumerate}
We call $H^0(\mathsf{B}(\PP))$ the
\emph{Koszul dual cooperad to $\PP$} and 
write it $\PP^\antishriek$.
\end{definition}

\textcolor{trinityblue}{Add equivalence with simpler
definition using twisting cochains.}
%\item The Koszul complex $\PP\circ_\tau \PP^\antishriek$ is acyclic.
%\item The Koszul complex $\PP^\antishriek\circ_\tau \PP$ is acyclic.
%\item The canonical map $\Omega \PP^\antishriek \longrightarrow \PP$
%is a quasi-isomorphism.


\section{Methods to prove Koszulness I}

\subsection{Distributive law methods}

\subsection{Koszul (co)homology}

\subsection{Exercises}

\section{Pre-Lie algebras and algebraic operads}

\subsection{Pre-Lie algebras associated to operads}

\subsection{The rooted trees operad}

\subsection{Exercises}




\section{Gr\"obner bases for algebraic operads II}

\subsection{Tree monomials}
\subsection{Admissible orders}
\subsection{Exercises}

\section{Methods to prove Koszulness II}

\subsection{Gr\"obner basis arguments}
\subsection{Filtration arguments}
\subsection{Exercises}

\section{Operads in algebraic topology 1}

\subsection{The little disks operad}
\subsection{Stasheff's operad}
\subsection{Recognition principles}


\section{Koszul duality II}

\subsection{Inhomogeneous duality}\label{sec:inhom}

Suppose that $\PP$ admits a quadratic linear presentation
defined by $\XX$ and $\RR\subseteq \XX\oplus \FF_\XX^{(2)}$.
There is a projection $q : \FF_\XX \longrightarrow \FF_\XX^{(2)}$
and we define $q\PP = \FF_{\XX}/(q\RR)$, and call it 
the \emph{quadratic operad associated to $\PP$}. We say
a quadratic-linear presentation is admissible if it
satisfies the following conditions:
\begin{tenumerate}
\item There are no superfluous generators in $\XX$,
that is, have that $\RR\cap \XX = 0$.
\item No new quadratic relations can be deduced from the
quadratic-linear relations, that is
\[ (\RR\circ_{(1)} \XX + \XX\circ_{(1)} \RR) \cap \FF_\XX^{(2)}
\subseteq \RR\cap \FF_\XX^{(2)}.\]
\end{tenumerate}
When condition (1) is satisfied, there is a map 
$f: q\RR \longrightarrow \XX$ such that $\RR = \{
r - f(r) : r\in q\RR \}$ is the graph of $f$. The operad
$\PP$ is filtered by weight, and we write $\mathrm{gr}(\PP)$
for the resulting operad. There is a surjection
$ q\PP \longrightarrow \mathrm{gr}(\PP)$
that is an isomorphism in weights $0$ and $1$, but
not necessarily in weight $2$.

The map $f: q\RR \longrightarrow \XX$ induces a map
$d_f : q\PP^\antishriek \longrightarrow \FF_{s\XX}^c$
which is the unique coderivation that correstricts on
$s\XX$ to the composition
\[ \PP^\antishriek  \xrightarrow{\pi} 
 s^2 q\RR \xrightarrow{s^{-1}f} s\XX.\] 
 Moreover, the following holds:
 \begin{tenumerate}
 \item The coderivation $d_f$ maps into $q\PP^\antishriek$
 if and only if 
 $(\RR\circ_{(1)} \XX + \XX\circ_{(1)} \RR) \cap \FF_\XX^{(2)} \subseteq q\RR$.
 \item If condition (2) above is satisfied, then the previous
 condition holds, as $\RR\cap \FF_\XX^{(2)}\subseteq q\RR$,
 and $d_f^2=0$.
 \end{tenumerate}
 
\begin{definition}
A quadratic-linear presentation of $\PP$ is inhomogeneous
Koszul if and only if it satisfies conditions (1) and (2)
and if the quadratic operad $q\PP$ is Koszul. In this case,
we call $(q\PP^\antishriek,d_f)$ the Koszul dual conilpotent
dg-cooperad
of $\PP$ and write it $\PP^\antishriek$.
\end{definition}

 \textbf{Warning!} Although not every operad 
is quadratic, every operad admits
a inhomogeneous Koszul presentation (Exercise 3.8.10
in \cite{LodVal}).  The problem is finding `economical
and useful' one.
Concretely, if we choose $\XX = \#\PP$ (the symmetric 
sequence underlying $\#\PP$) and 
quadratic-linear relations $\# \mu \circ_i \#\nu = \#(\mu\circ_i \nu)$ for every $\mu,\nu\in\PP$ and $i\in [1,\ari(\mu)]$,
then $\PP^\antishriek = \mathsf{B}(\PP)$ is the
bar construction of $\PP$ with its usual differential.

The main
theorem about inhomogeneous Koszul operads is the
following:

\begin{theorem}
If $\PP$ admits an inhomogenous Koszul presentation then
the canonical morphism
 $\Omega \PP^\antishriek \longrightarrow \PP$
determined by $s^{-1}\PP^\antishriek \twoheadrightarrow
\XX \hookrightarrow \PP$ is a quasi-isomorphism of
operads.
\end{theorem}

In general, one can show that the map $\Omega\mathsf{B}(\PP)
\longrightarrow \PP$ is a quasi-isomorphism: this is the
original approach of Ginzburg--Kapranov) who instead
define a duality functor $\mathsf{D}(\PP)$ along with a
quasi-isomorphism
$\mathsf{D}\mathsf{D}(\PP) \longrightarrow \PP$,
and define the Koszul dual operad to $\PP$ as
$H_\Delta(\mathsf{D}(\PP))$. The
theorem above gives us a more economical resolution of
$\PP$ in case it is quadratic-linear Koszul. Note that
the bar-cobar construction is, more or less, obtained
as the `least economical' resolution arising from the
corresponding `least economical' quadratic-linear 
presentation of an operad $\PP$. 

\section{Prototype of inhomogeneous duality}
\textbf{Goals.}
We will define the operad governing BV algebras 
and show it admits a small inhomogeneous Koszul 
presentation. We will compute the homology of the 
corresponding dg-cooperad, and
explain how it gives rise to the Gravity operad
of E. Getzler. At the same time, we will explain
how the homotopy quotient of $\mathsf{BV}$ by
the circle action is the hypercommutative
operad of Yu. I. Manin.
\subsection{Definition and computations}
\newcommand{\BV}{\mathsf{BV}}
The BV operad is an algebraic symmetric operad, which we
write $\BV$, generated by a binary commutative associative
operation $\mu$ that we will write $x_1x_2$ 
of degree zero and a unary square-zero operation
$\Delta$ of degree $-1$ that satisfy the following 
homogeneous $7$-term
relation:
\[ 
 \Delta(x_1x_2x_3) = x_1\Delta(x_2x_3)+
 x_2\Delta(x_1x_3) + x_3\Delta(x_1x_2)
  - x_1x_2\Delta(x_3) - x_1x_3\Delta(x_2)
   	- x_2x_3\Delta(x_1).\]
Batalin--Vilkovisky algebras appear in several
areas of mathematics:
\begin{tenumerate}
\item (Algebra) Vertex operator algebras, cohomology of Lie algebras,
bar construction of $A_\infty$-algebras.
\item (Algebraic geometry) Gromov--Witten invariants and moduli spaces of curves (quantum cohomology, Frobenius manifolds), chiral algebras (geometric Langlands program),
\item (Differential geometry) The sheaf of 
polyvector fields of an orientable (resp. Poisson or
Calabi--Yau) manifold, the differential 
forms of a manifold (Hodge decomposition 
in the Riemannian case), Lie algebroids,
Lagrangian (resp. coisotropic) intersections.
\item (Noncommutative geometry) 
The Hoschchild cohomology of a symmetric 
algebra  and the cyclic Deligne conjecture,
non-commutative differential operators.
\item (Algebraic topology)
2-fold loop spaces on topological spaces 
carrying an action of the circle,
topological conformal field theories, 
Riemann surfaces, string topology. 
\item  (Mathematical physics)
BV quantization (gauge theory), BRST
cohomology, string theory, topological field theory, Renormalization theory.

\end{tenumerate}

 One can express the seven term relation by saying
 that $[\Delta,\mu] = \beta$ is a derivation
 for $\mu$, where $[f,g]$
 is the operadic commutator (\`a la Gerstenhaber)
 defined by
 \[ [f,g] = \sum_{i=1}^{\ari(f)} f\circ_i g 
 	  -(-1)^{|f||g|} \sum_{j=1}^{\ari(g)} g\circ_j f. \]
 This suggests defining $\beta = [\Delta,m]$, and presenting
 the BV-operad by quadratic-linear relations:
 \[ [\mu,\mu] = 0 ,\quad \Delta^2 =0 , \quad
  	[\Delta,\mu] = \beta, \quad
  	\beta\circ_1 \mu = \mu \circ_2 \beta +
  	 	\mu\circ_1 \beta (23).\]
  This presentation satisfies condition (1), but it
  \emph{does not} satisfy condition (2): one can
  deduce that $\beta$ is a Lie bracket of degree $-1$
 and that $\Delta$ is a derivation for $\beta$
  from the first three equations. In other words, one
  can deduce that $(\Delta,\beta)$ defines the
  datum of a dg Lie algebra purely from the
  fact that $\Delta^2=0$ and that $\mu$ is associative. 
  
  \begin{lemma}
  The BV-operad admits a quadratic-linear presentation
  satisfying conditions (1) and (2) 
  given by generators $\mu,\beta,\Delta$ of arities
  $2$, $2$ and $1$ and degrees $0$, $-1$ and $-1$,
  respectively. The operation $\mu$ is associative 
  commutative, $\beta$ is a Lie bracket, $\Delta$
  squares to zero, and 
 \[	\mathsf{Leib}(\Delta,\mu) = \beta, \quad
  	\mathsf{Leib}(\beta,\mu)=0, \quad
  	\mathsf{Leib}(\Delta,\beta) = 0.\]
  	  \end{lemma}
  	  
  	  \begin{theorem}
  	  The quadratic operad $q\BV$ is Koszul.
  	  \end{theorem}
  	  
  	  \begin{proof}
  	  We will use the distributive law criterion of
  	  Markl, adapted to the case the operads in his
  	  result have unary operators. One can first
  	  show that $\BV(n)$ and its
  	  quadratic counterpart both have dimension $2^n n!$
  	  by a Gr\"obner basis argument,
  	  and then that $q\BV$ is obtained from
  	  a distributive law between the quadratic 
  	  operads $\mathsf{Ger}$ and 
  	  $\mathsf{D} = \kk[\Delta]/(\Delta^2)$, in light of the
  	  relations
  	  \[\Delta(x_1x_2) = x_1\Delta(x_2) + \Delta(x_1)x_2, 
  	  \quad 
  	   \Delta[x_1,x_2] = [x_1,\Delta x_2] + [\Delta x_1,x_2]. \]
  	  One can prove this again by a dimension counting argument
  	  using the result above. With this at hand, we
  	  observe that $\mathsf{Ger}$ is Koszul, as it is
  	  in turn obtained from a distributive law between
  	  $\mathsf{Com}$ and $\sus\mathsf{Lie}$, which are
  	  both Koszul, and that $\mathsf{D}$ is
  	  Koszul (as any algebra with trivial multiplications
  	  is). We conclude that $q\BV$ is Koszul, and that we
  	  have isomorphisms of symmetric sequences
  	  \[ q\BV \cong \Com \circ \sus\Lie\circ \mathsf{D},
  	    	  \quad q\BV^\antishriek \cong
  	    	  T^c(\delta) \circ \Com^c \circ \sus^{-1}\Lie^c,\]
  	  which will be useful to describe the dg-cooperad
  	  $\BV^{\antishriek}$.
  	  \end{proof}

\subsection{The differential}  	  
  	 Let us note that a generic element
  	 of $\Com^c \circ \sus^{-1}\Lie^c$ consists of a
  	 corolla decorated by Lie words on a Lie bracket
  	 of degree $1$. Any Lie word can always be
  	 written uniquely as a linear combination of
  	 words in the form
  	 $\ell = [x_1, x_{\sigma 2} , \ldots , x_{\sigma n} ]
     $
     where $\sigma\in S_n$ fixes $1$, and we adopt the
     right bracketing convention:
     \[ [y_1,y_2,\ldots,y_n ] = [y_1,[y_2,\ldots,y_n]].\]
     We call $x_1$ the `head' of $\ell$. 
     We will then write an element of $\Com^c \circ \sus^{-1}\Lie^c$ generically by
  	 \[ \ell_1\odot \cdots  \odot \ell_n \]
	 where $\ell_i$ is a Lie word supported on $\pi_i$,
	 with `head' $x_j$ with $j_i = \min\pi_i$ and such that
	 $\min \pi_1 < \cdots < \min \pi_n$. 
	 	 
	 \begin{theorem}
	 A generic element of $q\BV^\antishriek$ is of the form
	     \[x =\delta^k\otimes \ell_1\odot \cdots  \odot \ell_n \]
	  and the differential $d$ of $q\BV^\antishriek$ is
	  \[ dx = \sum_{i=1}^n(-1)^{\varepsilon_i} \delta^{k-1} \otimes \ell_1\odot
	   \cdots \odot \ell_i^{(1)} \odot \ell_i^{(2)}
	    \odot \cdots \odot \ell_n,
	  	 \]	  	 
	  	 where $\ell \longmapsto 
	  	 	\sum \ell^{(1)}\otimes \ell^{(2)}$
	  	 is the binary component of the decomposition map in
	  	 $\sus^{-1}\Lie^c$.	  	 
	  	 \end{theorem}
  	  
%The operad Gerstenhaber, Gravity, Hypercomm, PreLie,
%Perm, Zinbiel, Dias, Leib, NAP (Livernet) and the 
%Connes--Kreimer Hopf algebra of
%renormalization. 
\subsection{Exercises}

%\section{Further contents}

%\subsection{Iterated integrals}
%$C^*_{\mathrm{dR}}(\Omega X)$, Chen iterated integrals,
%work of Getzler--Jones for $\Omega^2 X$. 

\bibliographystyle{alpha}
\bibliography{biblio}


\Addresses

\end{document}

	  	 
	  	
\subsection{The gravity operad and the hypercommutative operad}
  	  
  	 There is a unique square zero derivation $\Delta$ 
	 of $\mathsf{Ger}$
	 of degree $-1$ that sends the commutative
	 product $x_1x_2$ to the Lie bracket $[x_1,x_2]$,
	 which arises from the action of $S^1$ on $D_2$
	 by turning the whole configuration.
	 This derivation is acyclic, and $\mathsf{Grav}$
	 is, by definition, $\ker\Delta   =
	 \operatorname{im}\Delta  $.	 
	 	 
	 \begin{lemma}
	 Let $M$ be an $S^1$-space with a free action, and
	 let $\Delta : H_*(M) \longrightarrow H_{*+1}(M)$
	 be the square zero operator arising from the
	 fundamental class of $S^1$. Then
	 \[ \ker\Delta \cong \Sigma H_*(M/S^1) . \]
	 \end{lemma} 

	 The action of $S^1$ on little disks is free
	 (except in arity one) 
	 with quotient homotopy equivalent to the moduli
	 space of marked curves of genus zero, so 
	 	 	  \[ \ker\Delta =  
	 	 	  \Sigma H_*(D_2/S^1) = \Sigma H_*(\mathcal{M}_{0,\bullet+1}). \]
	 
	 \begin{theorem}
	 One can realize $\mathsf{Grav}$ as the suboperad
	 of $\mathsf{Ger}$ generated by
	 \[ \{x_1,\ldots,x_n\} =  
	 \sum_{i<j} \{x_i,x_j\}
	  x_1\cdots \widehat{x_i}\cdots \widehat{x_j} \cdots x_n, 
	  	\quad n\geqslant 2.\]
	 Similarly, one can realize it as the suboperad of	 
	 $\mathsf{BV}$ generated by 
	 \[ \{x_1,\ldots,x_n\} = 
	  \Delta(x_1\cdots x_n) - \sum_{i=1}^n x_1\cdots \Delta x_i
	   	\cdots x_n, \quad n\geqslant 2.\]
	   	We have that $\Sigma H_*(\mathcal{M}_{0,n+1}) \cong
	   	\mathsf{Grav}(n)$ for $n\geqslant 2$ where the
	   	operation $\{x_1,\ldots,x_n\}$ corresponds to the
	   	suspension of the generator of 
	   	$H_0(\mathcal{M}_{0,n+1})$.
	  \end{theorem}
	  
	 \begin{theorem}
	 The homology of the dg-cooperad 
	 $\mathsf{BV}^\antishriek$ (and hence
	 the bar homology of $\mathsf{BV}$) is
	 isomorphic to
	 \[ T(\delta) \oplus \sus^{-1}\mathsf{Grav}^*,\]
	 and the  
	 homotopy quotient of $\mathsf{BV}$ by
	 $\Delta$ is quasi-isomorphic to $\mathsf{Hycomm}$,
	 the Koszul dual of $\mathsf{Grav}$. These
	 two last operads are Koszul.
	 \end{theorem}



\subsection{The homotopy quotient}

\begin{definition} 
Let $f:\PP\to \mathcal Q$ be a morphism of operads, 
and factor $f$ into a cofibration
$i: \PP\longrightarrow \FF$ followed
by a trivial fibration $j:\FF\longrightarrow \QQ$.
We define the homotopy cofibre $C_f$ of $f$
as the quotient of $\FF$ by the ideal generated
by $\PP$. We write $\PP\sslash \QQ$ for
the homology of $C_f$.
\end{definition}

Note that $\PP\sslash \mathcal Q$ is independent 
of the choice of factorization of $f$.
Usually we can take $\FF$ of the form 
$(\PP \star \FF_\XX,d)$ where $\star$ is the
coproduct in the category of operads, 
so $C_f$ is
isomorphic to $(\FF_\XX,\bar{d})$. 
If $\alpha\in\PP$, we define the homotopy
quotient of $\PP$ as the homotopy cofibre of
the inclusion
$i : \PP_\alpha \longrightarrow \PP$
where $\PP_\alpha$ is the suboperad of $\PP$ 
generated by $\alpha$, and write it more
simply by $\PP\sslash \alpha$.  
In this case, to compute $\PP\sslash\alpha$, it suffices we compute a quasi-free
model $\FF = (\FF_\XX,d)$ of $\PP$ ---which in
particular we can assume contains an 
isomorphic copy of $\FF_\alpha$, and compute
the homology of the quotient $\FF/ (\FF_\alpha).$

This algebraic homotopy quotient is related
to the more geometrical homotopy quotient
with respect to the action of a topological
group $G$. Concretely, the exact functor 
$ t:\mathsf{Top} \longrightarrow \mathsf{Top}_G$
from the category of topological spaces to 
the category of topological spaces with a 
$G$-action, that sends a space $X$ to the
same space $tX$ with the trivial action, admits
a (non-Quillen-exact) left adjoint $F$ and
the homotopy quotient is obtained
as the left derived functor $\mathbb LF$ of
$F$, which we usually write $X\longmapsto X_{hG}$. At the level of homotopy categories, it
follows that we have an adjunction isomorphism
$[X_{hG},Y] \longrightarrow [X,tY]$.

Naturally, we may pass from the geometric to
the algebraic setting through the cohomology
functor, by considering the sphere $S^1$ whose
cohomology algebra coincides with 
$\Bbbk[\Delta]$. In this way, we can consider
the category $\mathsf{Op}_\Delta$ of dg-operads
under $\Bbbk[\Delta]$, and the exact functor
$t:\mathsf{Op} \longrightarrow \mathsf{Op}_\Delta$
that assigns an operad $\PP$ to the trivial
map $\Bbbk[\Delta]\to \PP$ sending $\Delta$ to $0$. The homotopy quotient functor $\PP\longmapsto \PP\sslash \Delta$ is the left adjoint, at the level of homotopy categories,
to $t$: we have an adjunction isomorphism
$[ \PP\sslash \Delta, \mathcal Q]
 	\longrightarrow [ \PP,t\mathcal Q]$. 
 	  