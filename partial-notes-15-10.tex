
\documentclass[fleqn,a4paper, twoside]{article} 

%% for print
\usepackage[
	top = 1.15 in, 
	bottom = 1.25 in,
	left = 1.15 in, 
	right = 1.15 in,
	includehead]{geometry}
% font sizes
\usepackage{scrextend}
\changefontsizes{12pt}
% for online reading
%\usepackage[
%	top = 1.15 cm, 
%	bottom = 2.25 cm,
%	left = 1.15 cm, 
%	right =1.15 cm,
%	includehead]{geometry}
% font sizes
%\usepackage{scrextend}
%\changefontsizes{14pt}

\usepackage[all]{nowidow}

\usepackage{changepage}

\usepackage[utf8x]{inputenc}

\usepackage{amscd,amssymb,amsmath}
\usepackage{amsrefs}

%nimbus roman
\usepackage{mathptmx}
\usepackage[T1]{fontenc}


\usepackage{booktabs}
\newcommand{\ra}[1]{\renewcommand{\arraystretch}{#1}}

\usepackage[utf8x]{inputenc}
\usepackage{amsfonts,amssymb,amsmath,amsrefs}
\usepackage{graphicx}
\usepackage[british]{babel}
\usepackage{caption}
\usepackage{mathdots}
\usepackage{mathtools} 
\usepackage{amsmath}
\usepackage{amsfonts}
\usepackage{amssymb}
\usepackage{graphicx}
\usepackage{subfigure}
\usepackage{makeidx}
\usepackage{multicol}
\usepackage{array}
\usepackage{cancel}
\usepackage{polynom}
\usepackage[pdf,all]{xy}
\xyoption{line}
\CompileMatrices

\usepackage{xcolor}
% Link colors and styles, change accordingly, these are the official colours of Trinity 
\definecolor{trinityblue}{rgb}{0.05, 0.45,0.75}
\definecolor{trinitygray}{rgb}{0.33, 0.34,0.35}

%% Tree macros %%

\newcommand{\leftc}[5]{
	\begin{tikzpicture}
\tikzstyle{inner}=[circle,draw=black, fill=white, inner sep=1pt,minimum size=5pt]
\tikzstyle{leaf}=[circle, draw=white, fill=white, inner sep=3 pt,minimum size=5 pt]
\begin{pgfonlayer}{main}
		\node[style = inner] (0) at (-1, 1) {$#1$};
		\node[style = inner] (1) at (0, 0) {$#2$};
		\node[style = leaf] (2) at (1, 1) {$#5$};
		\node[style = leaf] (3) at (0, 2) {$#4$};
		\node[style = leaf] (4) at (-2, 2) {$#3$};
		\node (5) at (0, -1) {};
		\end{pgfonlayer}
	\begin{pgfonlayer}{bg}
		\draw (1) to (0);
		\draw (2.center) to (1);
		\draw (3.center) to (0);
		\draw (4.center) to (0);
		\draw (5.center) to (1);
		\end{pgfonlayer}	
\end{tikzpicture}
}

\newcommand{\rightc}[5]{
\begin{tikzpicture}
\tikzstyle{inner}=[circle,draw=black, fill=white, inner sep=1pt,minimum size=5pt]
\tikzstyle{leaf}=[circle, draw=white, fill=white, inner sep=3 pt,minimum size=5 pt]
\begin{pgfonlayer}{main}
		\node[style = leaf] (0) at (-1, 1) {$#3$};
		\node[style = inner] (1) at (0, 0) {$#1$};
		\node[style = inner] (2) at (1, 1) {$#2$};
		\node[style = leaf] (3) at (0, 2) {$#4$};
		\node[style = leaf] (4) at (2, 2) {$#5$};
		\node (5) at (0, -1) {};
		\end{pgfonlayer}
	\begin{pgfonlayer}{bg}
		\draw (1) to (0);
		\draw (2.center) to (1);
		\draw (3.center) to (2);
		\draw (4.center) to (2);
		\draw (5.center) to (1);
		\end{pgfonlayer}	
\end{tikzpicture}
}
\newcommand{\mathsymbol}[2]{
	\begin{tikzpicture}
        \node at(0,0){};
        \node at(0,#1){$#2$};
    \end{tikzpicture}
}
%% Side captions %%
\usepackage{sidecap}

%% Exercises %%
\usepackage{exsheets} %Para guias de ej.
\SetupExSheets{headings = runin}

\usepackage{wrapfig} %%https://es.sharelatex.com/learn/latex/Wrapping_text_around_figures

\usepackage{lipsum}
%% SMALL HYPHEN %%
\mathchardef\hy="2D % Define a "math hyphen"

 %% TAMAÑO C\eLDAS %%     
\usepackage{pgf,tikz}

\pgfdeclarelayer{bg}    % declare background layer
\pgfsetlayers{bg,main}  % set the order of the layers (main is the standard layer)


\usepackage{tikz-cd}
\usetikzlibrary{calc}
\usetikzlibrary{matrix,arrows}
\usepackage{stackrel}
\usepackage[shortlabels]{enumitem} %Para listas estilo Weibel
\usepackage{stmaryrd} %Para double brackets
\usepackage{setspace} %Para espaciado de renglones
\spacing{1.15}
%\onehalfspacing %Mejor lectura
\usepackage{etoolbox}
\usetikzlibrary{trees}


\patchcmd{\section}{\normalfont}{\normalfont\large}{}{}

\usepackage[bb=ams, cal=euler, scr=rsfso , frak=euler]{mathalpha}

\usepackage{fancyhdr}
\pagestyle{fancy}
\fancyhead[RE]{\small\it Algebraic operads}
\fancyhead[LO]{\small\it Winter Lecture Series 2021/2022}
\fancyhead[RO,LE]{\small\textbf\thepage}
\renewcommand{\headrulewidth}{0 pt}
\cfoot{}
\newcommand{\0}{\langle 0\rangle}

\fancypagestyle{references}
{\fancyhead[RE]{\small\it References}
\fancyhead[LO]{\small\it References}
\fancyhead[RO,LE]{\small\bf\thepage}
\fancyfoot[L,R,C]{}
\renewcommand{\headrulewidth}{0 pt}}

 %%%%%%%%%%%%TikZ Styles%%%%%%%%%%%

%%%%%%%%%%%%%%
\newcommand{\XX}{\mathcal{X}}
\renewcommand{\AA}{\mathcal{A}}
\newcommand{\YY}{\mathcal{Y}}
\newcommand{\End}{\operatorname{End}}
\newcommand{\RR}{\mathcal{R}}
\newcommand{\II}{\mathcal{I}}
\newcommand{\QQ}{\mathcal{Q}}
\newcommand{\FF}{\mathcal{F}}

\newcommand{\ari}{\operatorname{ar}}
\newcommand{\cone}{\operatorname{cone}}
\newcommand{\hem}{\hspace{0.5 em}} 
%\input{treesp.tex} For trees, maybe add again later

 \newcommand{\stt}{\mathbin{\text{\tikz 
[x=1ex,y=1ex,line width=.1ex,line 
join=round] \draw (0,0) rectangle (1,1) 
(1,1) -- (0,0)  (1,0) -- (0,1);}}}

%%%%%% ENUMERATE STUFF %%%%%%
\usepackage{enumitem}
\listfiles
\setlist[enumerate]{label= (\arabic*)}


\newenvironment{tenumerate}{
 \begin{enumerate}
  \setlength{\itemsep}{0pt}
  \setlength{\parskip}{0pt}
}{\end{enumerate}}

\newenvironment{titemize}{
\begin{itemize}
  \setlength{\itemsep}{0pt}
  \setlength{\parskip}{0pt}
}{\end{itemize}}

\definecolor{newcol}{rgb}{0,0,0}
\definecolor{deepblue}{rgb}{0.0, 0.28, 0.67}
%{0.5, 0.0, 0.13}
\DeclareTextFontCommand{\new}{\color{black}\em}
%\DeclareTextFontCommand{\new}{\color{black}\em}

\usepackage[pdftex, colorlinks,bookmarks 
= true,bookmarksnumbered = true]{hyperref}

% url colors
\hypersetup{colorlinks,
	linkcolor={trinityblue},
		citecolor={trinityblue},
			urlcolor={trinityblue}} 
			
\usepackage{sectsty}
\chapterfont{\color{newcol}}  % sets colour of chapters
\sectionfont{\color{newcol}}  % sets colour of sections
\subsectionfont{\color{newcol}}  % sets colour of sections

% let \[ and \] be the same as \begin{equation} and \end{equation}
\makeatletter
\AtBeginDocument{%
  \let\[\@undefined
  
\DeclareRobustCommand{\[}{\begin{equation}}%
  \let\]\@undefined
  
\DeclareRobustCommand{\]}{\end{equation}}%
}
\makeatother 
% but only print equation numbers if needed
\mathtoolsset{showonlyrefs,showmanualtags}

\usepackage{amsthm}
\usepackage{thmtools}
\newtheoremstyle{mytheorem}
  {\topsep}   % ABOVESPACE
  {\topsep}   % BELOWSPACE
  {\itshape}  % BODYFONT
  {0pt}       % INDENT (empty value is the same as 0pt)
  {\bfseries\color{newcol}} % HEADFONT
  {\color{newcol}}         % HEADPUNCT
  {5pt plus 1pt minus 1pt} % HEADSPACE
  {}          % CUSTOM-HEAD-SPEC
  
\theoremstyle{mytheorem}
\newtheorem{theorem}{Theorem}[section]
\newtheorem{prop}[theorem]{Proposition}
\newtheorem{cor}[theorem]{Corollary}
\newtheorem{lemma}[theorem]{Lemma}
\newtheorem*{conj*}{Conjecture}
%\newtheorem{question}[theorem]{Question}

 \newtheoremstyle{introthm}
  {\topsep}   % ABOVESPACE
  {\topsep}   % BELOWSPACE
  {\itshape}  % BODYFONT
  {0pt}       % INDENT (empty value is the same as 0pt)
  {\bfseries\color{newcol}} % HEADFONT
  {\color{newcol}{.}}         % HEADPUNCT
  {5pt plus 1pt minus 1pt} % HEADSPACE
  {}          % CUSTOM-HEAD-SPEC
 
  
\theoremstyle{introthm}
\newtheorem{introthm}{Theorem}
\newtheorem*{introcor}{Corollary}
\renewcommand{\theintrothm}{\Alph{introthm}}
 % Para teoremas en la introducción

%%%%%%%% TODO NOTES %%%


%% TODO NOTES! %%
\usepackage{xargs}                      
% Use more than one optional parameter in a new commands
%\usepackage[pdftex,dvipsnames]{xcolor}  
% Coloured text etc.
\usepackage[colorinlistoftodos,prependcaption,textsize=small]{todonotes}
\newcommandx{\unsure}[2][1=]{\todo[linecolor=blue,backgroundcolor=blue!25!white,bordercolor=blue,#1]{#2}}
\newcommandx{\change}[2][1=]{\todo[linecolor=blue,backgroundcolor=blue!25,bordercolor=blue,#1]{#2}}
\newcommandx{\info}[2][1=]{\todo[linecolor=OliveGreen,backgroundcolor=OliveGreen!25,bordercolor=OliveGreen,#1]{#2}}
\newcommandx{\improvement}[2][1=]{\todo[linecolor=Plum,backgroundcolor=Plum!25,bordercolor=Plum,#1]{#2}}
\newcommandx{\thiswillnotshow}[2][1=]{\todo[disable,#1]{#2}}

\definecolor{col1}{rgb}{0.8, 0.8, 1.0}
\definecolor{col2}{rgb}{0.9, 0.9, 0.98}
\definecolor{col3}{rgb}{0.71, 0.49, 0.86}

  
\newcommand{\ind}{\operatorname{Ind}_\PP}

\newcommand{\cof}{\rightarrowtail}
\newcommand{\foc}{\leftarrowtail}
\newcommand{\weq}{\overset{\sim}{\longrightarrow}}
\newcommand{\tcf}{\overset{\sim}{\cof}}

\newtheoremstyle{mydefinition}
  {\topsep}   % ABOVESPACE
  {\topsep}   % BELOWSPACE
  {}  % BODYFONT
  {0pt}       % INDENT (empty value is the same as 0pt)
  {\bfseries\color{newcol}} % HEADFONT
  {\color{newcol}}         % HEADPUNCT
  {5pt plus 1pt minus 1pt} % HEADSPACE
  {}          % CUSTOM-HEAD-SPEC
  
\theoremstyle{mydefinition}
\newtheorem{definition}[theorem]{Definition}
\newtheorem{obs}[theorem]{Observation}
\newtheorem{rmk}[theorem]{Remark}
\newtheorem{problem}[theorem]{Problem}
\newtheorem{example}[theorem]{Example}
\newtheorem{note}[theorem]{Note}
\newtheorem{variante}[theorem]{Variant}


\newtheoremstyle{mydefinition2}
  {\topsep}   % ABOVESPACE
  {\topsep}   % BELOWSPACE
  {}  % BODYFONT
  {0pt}       % INDENT (empty value is the same as 0pt)
  {\bfseries\color{newcol}} % HEADFONT
  {\color{newcol}{.}}         % HEADPUNCT
  {5pt plus 1pt minus 1pt} % HEADSPACE
  {}          % CUSTOM-HEAD-SPEC
  
\theoremstyle{mydefinition2}
\newtheorem*{definition*}{Definition}
\newtheorem*{remark*}{Remark}
\newtheorem*{obs*}{Observation}
\newtheorem*{example*}{Example}

% named theorem %

% for specifying a name
\theoremstyle{plain} % just in case the style had changed
\newcommand{\thistheoremname}{}
\newtheorem{genericthm}[theorem]{\thistheoremname}
\newenvironment{namedthm}[1]
  {\renewcommand{\thistheoremname}{#1}%
   \begin{genericthm}}
  {\end{genericthm}}

\newcommand{\imor}{\interleave\kern-.45em\longrightarrow}
\renewcommand{\qedsymbol}{$\blacktriangleleft$}
\newcommand\place{\mathord-}
\newcommand{\ps}{\mathbin\parallel}
\newcommand{\GSet}{\mathsf{Fin}^\times}
\newcommand{\FSet}{\mathsf{Fin}}
\newcommand{\Fun}{\mathsf{Fun}}
\newcommand{\Set}{\mathsf{Set}}
\newcommand{\DCSH}{\mathsf{DCSH}}
\newcommand{\kMod}{{}_\kk\mathsf{Mod}}
\newcommand{\kmod}{{}_\kk\mathsf{mod}}
\newcommand{\kCh}{{}_\kk\mathsf{Ch}}
\newcommand{\Sp}{\mathsf{Sp}}
\newcommand{\kSp}{{}_\kk\mathsf{Sp}}
\newcommand{\GVec}{\mathsf{Vec}_q^\times}
\newcommand{\Opr}{\mathsf{Opr}}
\newcommand{\NS}{\mathsf{NS}}
\DeclarePairedDelimiter\abs{\lvert}{\rvert}
\newcommand{\vv}{\vert}
\newcommand{\Der}{\operatorname{Der}}
\newcommand{\HH}{\mathrm{HH}}
\newcommand{\As}{\mathsf{As}}
\newcommand{\Com}{\mathsf{Com}}
\newcommand{\Lie}{\mathsf{Lie}}

\newcommand{\aut}{\operatorname{aut}}
\definecolor{sqsqsq}{rgb}{0.13,0.13,0.13}
\definecolor{aqaqaq}{rgb}{0.63,0.63,0.63}

\newcommand{\SC}{SC}
\newcommand\id{\mathrm{id}}
\newcommand\category[1]{\mathsf{#1}}
\newcommand{\Mod}{\mathsf{Mod}}
\newcommand{\Gmod}{\mathsf{GMod}}
\newcommand{\SMod}{{}_\Sigma\mathsf{dgMod}}
\newcommand{\gSMod}{{}_\Sigma\mathsf{gMod}}
\newcommand{\coker}{\operatorname{coker}}
\newcommand{\Ho}{\operatorname{Ho}}
\newcommand\spe[1]{\mathcal{#1}}
\renewcommand{\tt}{\otimes}
\newcommand{\E}{\mathcal{E}}
\newcommand{\CC}{\mathcal{C}}
\newcommand{\DC}{\mathsf{DC}}
\newcommand{\DA}{\mathsf{DA}}
\newcommand{\dd}{\partial}
\newcommand{\OO}{\mathcal O}
\newcommand{\?}{\,?\,}
\newcommand{\n}{[n]}
\newcommand\cls[1]{\llbracket#1\rrbracket}
\newcommand{\NN}{\mathbb N}
\renewcommand{\k}{[k]}
\newcommand{\bt}{\bullet}
\newcommand{\kk}{\Bbbk}
\newcommand{\Aut}{\operatorname{Aut}}
\newcommand{\Ext}{\operatorname{Ext}}
\newcommand{\Tor}{\operatorname{Tor}}
\newcommand{\PAlg}{\mathsf{Alg}_{\geqslant 0}}
\newcommand{\Cog}{\mathsf{Cog}}
\newcommand{\Alg}{\mathsf{Alg}}
\newcommand{\Cxs}{\mathsf{Ch}_\kk}
\newcommand\inter[1]{\llbracket#1\rrbracket}
\newcommand{\Cell}{\operatorname{Cell}}
\newcommand{\Sing}{\operatorname{Sing}}
\newcommand{\Sull}{A_{\mathrm{PL}}}

\newcommand\lab{\mathsf{lab}}
\newcommand{\Q}{{\mathbb{Q}}}
\newcommand{\Z}{{\mathbb{Z}}}
\newcommand{\hoq}{\!\sslash\!\!}
\newcommand{\PP}{{\mathcal{P}}}

\definecolor{newterm-color}{RGB}{0, 0, 0}
\newcommand\newterm[1]{%
  \textcolor{newterm-color}{\itshape #1}%
}

\theoremstyle{mytheorem}
\newtheorem*{theorem*}{Theorem}
\newtheorem*{question*}{Question}

% named theorem %

% for specifying a name
\theoremstyle{plain} % just in case the style had changed
% front matter stlye 
\renewenvironment{abstract}{%
\small\begin{center}
\begin{minipage}{.9\textwidth}
%\textbf{\textcolor{newcol}{Abstract.}}
}
{\par\noindent\end{minipage}\end{center}\vspace{3 em}}
%
\makeatletter
\renewcommand\@maketitle{%
\hfill
\begin{center}\begin{minipage}{0.9 	\textwidth}
\centering
\vskip 6em
\let\footnote\thanks 
{\LARGE \@title \par }
\vspace{1 em}
%\hrulefill
\vskip 1 em
{\large \@author \par}
\vspace{3.5 em}

\end{minipage}\end{center}
\par
}
\makeatother
%
%%%%%%%%%%%%%%%%%%%%%%%%%%%%%%%%%%%%%%%%%%%%%%%%%%%%%%


\DeclareMathOperator\sgn{sgn}
\DeclareMathOperator\fsch{\mathfrak{sch}}

%\definecolor{newterm-color}{RGB}{0, 51, 153}
%\newcommand\newterm[1]{%
%\textcolor{newterm-color}{\bfseries\itshape #1}%
%}

% textual claims in equations
\newcommand\claim[2][.8]{%
  \begin{minipage}{#1\displaywidth}%
  \itshape
  #2
  \end{minipage}%
}

\usepackage{float}

\tikzcdset{arrow style=tikz, diagrams={>=stealth}}

%%Addresses

\newcommand{\Addresses}{{% additional braces for segregating \footnotesize
  \bigskip
  \footnotesize
  \textsc{Office F310, Non-linear Algebra Group}, MPIMiS, Leipzig, Germany \par\nopagebreak
  \textit{Addresses:} tamaroff@mis.mpg.de}
}
  

\usepackage{titletoc}

\titlecontents{chapter}
[0.2em] %
{\bigskip}
%{\contentslabel[\thecontentslabel.]{2em}\hspace{0.667em}}%\thecontentslabel
{\makebox[2em][r]{\thecontentslabel.}\hspace{0.333em}}%\thecontentslabel
{\hspace*{-2em}}
{\hfill\contentspage}[\smallskip]

\titlecontents{section}% <section>
[0.2em]% <left>
{\small}% <above-code>
{\thecontentslabel.\hspace{3pt}}%<numbered-entry-format>; you could also 
%use  {\thecontentslabel. } to show the numbers
{}% <numberless-entry-format>
{\enspace\titlerule*[0.5pc]{.}\contentspage}%<filler-page-format>
\titlecontents*{subsection}% <section>
[1em]% <left>
{\footnotesize}% <above-code>
{\thecontentslabel. \hspace{3pt}}% <numbered-entry-format>; you could also 
%use {\thecontentslabel. } to show the numbers
{}% <numberless-entry-format>
{}% <filler-page-format>
[ --- \ ]% <separator>
[]% <end>
\setcounter{tocdepth}{2}% Display up to \subsection in ToC

\setlength\parskip{3 pt}
\setlength\parindent{0 em}

\raggedbottom 
\makeindex

\title{\vspace{-5 em}\setstretch{0.85}{\textbf{Algebraic operads, Koszul duality and Gr\"obner bases: an introduction}}}
\author{P. Tamaroff}
\date{August 22 and 24}
%\address{}
%\email{ptamarov@gmail.com}
\begin{document}
\maketitle


%\tableofcontents
%\section{Introduction}\label{sec:intro}
%\addcontentsline{toc}{section}{\nameref{sec:intro}}
\thispagestyle{empty}

\begin{abstract}
This lecture series aim to offer a gentle introduction
to the theory of algebraic operads, starting with the
elements of the theory, and progressing slowly towards
more advanced themes, including (inhomogeneous)
Koszul duality theory, Gr\"obner bases and higher
structures. The course  will consists of approximately
twelve lectures, along with extra talks by
willing participants, with the goal of introducing extra
material to the course, and making them more
familiar with the theory.
%We will survey the Koszul duality theory introduced by Ginzburg--Kapranov and, more particularly, it's extension to inhomogenous 
%(quadratic-linear) operads, appearing originally in the work of 
%G\'alvez-Carrillo--Tonks--Vallette. We'll do this with a view 
%towards its application to the computation of the homotopy 
%quotient of the BV operad by the circle action
%(Khoroshkin--Markarian--Shadrin, etc) and the corresponding 
%geometrical statement (Drummond-Cole). If time permits, I'll 
%mention some other interesting (old and new) related results 
%related to non-commutative analogues, among others. 
\end{abstract}

\tableofcontents

\vfill
\hfill \emph{Draft, \today}

\pagebreak
\section{Motivation and history}

\textbf{Goals.} The goals of this lecture is to
give a broad picture of the history and 
pre-history of operads, and some current trends,
and give a road-map for the course.  

\subsection{Introduction and motivation}

Operads originally appeared as tools in homotopy theory, 
specifically in the study of iterated loop 
spaces (May, 1972 and Boardman and Vogt before).
They appeared as \emph{comp algebras} in Gerstenhaber's 
work on Hochschild cohomology and topologically as 
Stasheff's `associahedra' for his homotopy
characterization of loop spaces (both in 1963).
The theory of operads 
received new inspiration from homological algebra,
category theory, algebraic geometry and mathematical
physics:

\begin{tenumerate}
\item (Stasheff, Sugawara) Study homotopy associative 
$H$-spaces, Stasheff implicitly discovers a topological
ns operad $K$ with $C_*(K) = \mathsf{As}_\infty$ and a
recognition principle for $A_\infty$-spaces.

\item (Boardmann--Vogt) Study infinite loop spaces,
build a PROP (a version of an $E_\infty$-operad) 
and obtain a recognition principle for infinite
loop spaces.

\item (Kontsevich) Uses $L_\infty$-algebras and
configuration spaces to prove his deformation
quantization theorem that every Poisson manifold
admits a deformation quantization.

\item (Kontsevich) The above is
implied by the formality theorem: the
Lie algebra of polyvector fields is 
$L_\infty$-quasi-isomorphic to the Hochschild
complex, and $f_1 = \mathsf{HKR}$. 

\item (Tamarkin) Approaches this result through 
the formality of the little disks operad $D_2$.
Proves that the Hochschild complex of a polynomial
algebra is \emph{intrinsically formal} as a 
Gerstenhaber algebra.

\item (Manifold calculus) Describes the
homotopy type of embedding spaces as certain 
derived operadic module maps and to
produces their explicit deloopings
using little disk operads, due to 
Goodwillie--Weiss, Boavida de Brito–-Weiss,
Turchin, Arone--Turchin, Dwyer–-Hess,
Ducoulombier–-Turchin. 

\item (Ginzburg--Kapranov, Fresse,
Vallette, Hinich) Koszul duality for algebraic 
operads and cousins allows to develop a robust 
homotopy theory of homotopy algebras, cohomology
theory, deformation theory, Quillen homology, etc. 

\item (Deligne conjecture and variants) The study
of natural operations on the Hochschild complex
of an associative algebra lead to a manifold of 
results beginning with the proof that there is
an action of the little disks operad $D_2$ on
it, and the ultimate version by Markl--Voronov,
who proved that the operad of natural operations
on it has the homotopy type of $C_*(D_2)$.  
\end{tenumerate}

Algebraic operads are modeled by trees (planar or
non-planar, rooted or not), and relaxing these
graphs allows us to produce other type of 
algebraic structures:
\begin{center}
\begin{tabular}{@{}llcc@{}} \toprule
Type & Graph & Compositions & Due to \\ \midrule
PROPs & Any graph & Any  & Adams--MacLane \\
Modular & Any graph & $\xi_{i,j}$, $\circ_{i,j}$  & Getzler--Kapranov\\ 
Properads & Connected graphs & Any & B. Vallette \\
Dioperads & Trees & ${}_i\circ_j$ (no genus) & W. L. Gan \\
Half-PROPs & Trees & $\circ_j$ , ${}_i\circ$ & 
 Markl--Voronov \\ 
Cyclic operads & Trees &  $\circ_{i,j}$ & Getzler--Kapranov\\ 
Symmetric operads & Rooted trees & $\circ_i$  & J. P. May \\ 
\bottomrule
\end{tabular}
\end{center}

\subsection{Exercises}

The corresponding exercises to this lecture appear in \textbf{Exercise Sheet \#0}.

\pagebreak

\section{Symmetric sequences, composition products, unbiased approach}

\textbf{Goals.}
We will define
some related gadgets (symmetric collections,
algebras, modules, endomorphism operads)
necessary to introduce operads. 
Then, we define what an operad is (topological,
algebraic, symmetric, non-symmetric). 
We will then give some
(not so) well known examples of topological
and algebraic operads.

\subsection{Basic definitions}
\emph{What is an operad?} A group is a model of
$\operatorname{Aut}(X)$ for $X$ a set, an algebra
is a model of $\End(V)$ for $V$
a vector space. Equivalently, groups are the
gadgets that act on objects by automorphisms,
and algebras are the gadgets that act
on objects by their (linear) endomorphisms. 
Operads are the gadgets that act on
objects through operations with many 
inputs (and one output), and at the same
time keep track of symmetries when
the inputs are permuted.

The underlying objects to operads are known as
\emph{symmetric sequences}: a symmetric sequence
(also known as an $\Sigma$-module or symmetric 
module) is a sequence of vector spaces
$\XX = (\XX(n))_{n\geqslant 0}$ such that for
each $n\in\NN_0$ there is a right action of
$S_n$ on $\XX(n)$. We usually consider \emph{reduced}
$\Sigma$-modules, those for which $\XX(0)=0$.

A map of $\Sigma$-modules is a collection of maps
$(f_n : \XX_1(n) \longrightarrow \XX_2(n)\}_{n\geqslant 0}$,
each equivariant for the corresponding group action. 
This defines the category $\SMod$ of symmetric
sequences, and whenever we think of symmetric sequences
using this definition, we will say we are considering a 
a biased or skeletal approach to them.

In parallel, it is convenient to consider the 
category $\GSet$ of finite sets and bijections.
An object in this category is a finite set $I$,
and a morphism $\sigma : I\longrightarrow J$ is a
bijection. Since every finite set $I$ with $n$
elements is (non-canonically) isomorphic to 
$[n] =\{1,\ldots,n\}$, the following holds:

\begin{lemma} The skeleton of $\GSet$ is
equal to the category with objects the finite sets
$[n]$ for $n\geqslant 0$ and with morphisms the
bijections $[n]\longrightarrow [n]$ (and no morphism
between $[n]$ and $[m]$ if $m\neq n$).
\end{lemma}

\begin{proof}
This is in Exercise Sheet \#0. 
\end{proof}

We set ${}_\Sigma\Mod  = 
\Fun(\GSet,\mathsf{Vect}^{\mathrm{op}})$,
so that a $\Sigma$-module is a pre-sheaf of vector
spaces $I\longmapsto \XX(I)$ assigning to each
isomorphism $\tau : I\longrightarrow J$ an isomorphism
$\XX(\tau): \XX(J)\longrightarrow \XX(I)$. When we
think of $\Sigma$-modules as pre-sheaves, we will say we 
are taking an unbiased approach, will if we specify only
its values on natural numbers, we will say we are taking the
biased or skeletal approach; we will come back to this later.

With this at hand, 
we can in turn define the \emph{Cauchy product}
of two $\Sigma$-modules $\XX$ and $\YY$
\[ (\XX\otimes_\Sigma \YY)(I) = 
 	\bigoplus_{S\sqcup T= I}
 		 \XX(S)\otimes \YY(T)\] 
where the right-hand is the usual tensor product of
vector spaces
and the sum runs through partitions of $I$ into
two disjoint sets. The symmetric product is then
defined by 
\[ (\XX\circ_\Sigma \YY)(I) 
 	= \bigoplus_{\pi \vdash I} \XX(\pi) 
 		\otimes \YY^{\otimes k}(\pi)\] 
as the sum runs through (ordered) partitions of $I$.
These two products will be central in what follows.

\begin{lemma}
The category ${}_\Sigma\Mod$ with $\circ_\Sigma$ is
monoidal with unit the species taking the value $\kk e_x$ at 
the singleton sets $\{x\}$ and zero everywhere else. The same
category is monoidal for $\otimes_\Sigma$ with unit
the species taking the value $\kk$ at $\varnothing$
and zero everywhere else.
\end{lemma}

We will use the notation $\kk$ for the base field but
also for the unit for the composition product $\circ_\Sigma$,
hoping it will not cause much confusion. It will be useful 
later to think of $\kk$ as a twig or ``stick''.

Observe that the associator for $\circ_\Sigma$ is
not too simple and involves reordering certain
factors of tensor products in $\mathsf{Vect}$. In
particular, replacing vector spaces by graded vector
spaces or complexes will create signs in the
associator.

We are now ready to define the prototypical symmetric
 sequence that carries the structure of an algebraic 
 operad. 
 
\begin{definition}
The \emph{endomorphism operad} of a space $V$ is the symmetric sequence $\End_V$
where for each $n\in\NN$ we set $\End_V(n) = \End(V^\otimes, V)$. 
The symmetric group $S_n$ acts on the right
on $\End_V(n)$ 
so that $(f\sigma)(v) = f(\sigma v)$ for
$v\in V^{\otimes n}$, where $S_n$ acts on
the left on $V^{\otimes n}$ by $(\sigma v)_i
= v_{\sigma i}$. The composition maps are defined
by $\gamma(f;g_1,\ldots,g_n) = f\circ (g_1\otimes\cdots \otimes g_n)$. 
\end{definition}

The following two operations on permutations 
will streamline our definition of (algebraic)
operads.

\medskip

\textbf{Two useful maps.} For each $k\geqslant 1$
and each tuple $\lambda = (n_1,\ldots,n_k)$ 
with sum $n$
there is a map
\[ S_k \longrightarrow S_{n_1+\cdots+n_k} \]
that sends $\sigma\in S_k$ to the permutation
$\lambda(\sigma)$ of $[n]$ that permutes the blocks 
$\pi_i = \{n_1+\cdots+n_{i-1}+1,\ldots
			n_1+\cdots+n_{i-1}+n_i\}$
			according to $\sigma$.
There is also a map
\[S_{n_1}\times \cdots \times  S_{n_k} 
	\longrightarrow S_{n_1+\cdots+n_k}  \] 
	that sends a tuple of permutations 
	$(\sigma_1,\ldots,\sigma_k)$ to the
	permutation $\sigma_1\#\cdots \# \sigma_k$
	that acts like $\sigma_i$ on the block $\pi_i$
	as above. These operations are illustrated
	in Figure~\ref{fig:1}. With these at hand, 
one can check that these composition maps
satisfy the following axioms:

\begin{figure}
$$\lambda = (2,1,2), \quad \sigma = 312
	\quad \leadsto \quad \lambda(\sigma) =  34512 \in S_5
	$$
	$$
	(213,213,132)\in S_3\times S_3\times S_3 \quad \leadsto \quad 213546798\in S_9 $$
\caption{The useful operations}
\label{fig:1}
\end{figure}

\begin{tenumerate}
\item \emph{Associativity}: let $f\in \End_V(n)$,
and consider $g_1,\ldots,g_n \in \End_V$ and
for each $i\in [n]$ a tuple $h_i= (h_{i1},\ldots,h_{i n_i})$
were $n_i= \mathrm{ar}(g_i)$. Then for
$f_i = \gamma(g_i; h_{i1},\ldots,h_{in_i})$ 
and $g= \gamma(f; g_1,\ldots,g_n)$ we have
that
\[ \gamma(f;f_1,\ldots,f_n) = 
	\gamma(g; h_1,\ldots,h_n).\]
\item \emph{Intrinsic equivariance}: for
each $\sigma\in S_k$ and $\lambda = (\ari(g_1),\ldots,\ari(g_k))$ we have that
\[ \gamma(f\sigma; g_1,\ldots,g_k) = 	
	\gamma(f; g_{\sigma 1} ,\ldots, 
		g_{\sigma k})\lambda(\sigma),\]
	
\item \emph{Extrinsic equivariance}: for each
tuple of permutations $(\sigma_1,\ldots,\sigma_k) \in S_{n_1} \times
\cdots \times S_{n_k}$, if $\sigma = \sigma_1\#\cdots\#\sigma_k$, we have that
\[\gamma(f,g_1\sigma_1,\ldots,g_k\sigma_k) = 
	\gamma(f; g_1,\ldots,g_k)\sigma.\]
\item \emph{Unitality:} the identity $1\in\End_V(1)$
satisfies $\gamma(1;g) = g$ and $\gamma(g;1,\ldots,1) = g$ for every $g\in\End_V$.
\end{tenumerate}

\begin{definition} 
A symmetric operad (in vector spaces) is an
$\Sigma$-module $\PP$ along with a composition
map $\gamma : \PP\circ \PP \longrightarrow \PP$
of signature
\[\gamma : \PP(k)\otimes 
	\PP(n_1) \otimes \cdots \otimes \PP(n_k)
	 	\longrightarrow \PP(n_1+\cdots+n_k)\]
along with a unit $1\in \PP(1)$, that satisfy
the axioms above. 
\end{definition} 

\begin{variante} A non-symmetric operad is
an operad whose underlying object is a collection
(with no symmetric group actions). Operads in
topological spaces or chain complexes require
the composition maps to be morphisms (that is,
continuous maps or maps of chain complexes,
respectively) and, more generally, operads 
defined on a symmetric monoidal category
require, naturally, that the composition
maps be morphisms in that category. 
\end{variante}

\textbf{Pseudo-operads.}
One can define operads through \emph{partial 
composition maps}, modeling the honest partial
composition map
\[ f\circ_i g = f(1,\ldots,1,g,1,\ldots,1)\] 
in $\End_V$. These composition maps satisfy the
following properties:

\begin{tenumerate}
\item \emph{Associativity}: for
each $f,g,h\in\End_V$, and $\delta = i-j+1$,
we have
\[ 
(f \circ_j g) \circ_i h  = 
 	\begin{cases} 
 		 (f \circ_i h) \circ_{\ari(f)+j-1} g
 		  	& \delta \leqslant 0  \\
 		  	f\circ_j (g \circ_\delta h) &
 		  	\delta\in [1,\ari(g)] \\
 		  	(f \circ_\delta h) \circ_j g & \delta > \ari(g)
 		   \end{cases}
 		 \]
\item \emph{Intrinsic equivariance}: for
each $\sigma\in S_k$, we have that
\[  (f\sigma) \circ_i g  = (f\circ_{\sigma i} g)\sigma'\]
where $\sigma'$ is the same permutation as $\sigma$
that treats the block $\{i,i+1,\ldots,i+\ari(g)-1\}$
as a single element. 
\item \emph{Extrinsic equivariance}: 
 for each $\sigma\in S_k$, we have that
\[  f \circ_i (g\sigma)  = (f\circ_i g)\sigma''\]
where $\sigma''$ acts by only permuting the
block $\{i,\ldots,i+\ari(g)-1\}$ according
to $\sigma$.
\item \emph{Unitality:} the identity $1\in\End_V(1)$
satisfies $1 \circ_1 g = g$ and $g\circ_i 1 = g$ for every $g\in\End_V$ and $1\leqslant i\leqslant \ari(g)$.
\end{tenumerate}

\begin{definition}
A symmetric operad (in vector spaces) is an
$\Sigma$-module $\PP$ along with partial composition
map of signature
\[ -\circ_i -  : \PP(m)\otimes \PP(n) 
	\longrightarrow \PP(m+n-1) \]
and a unit $1\in\PP(1)$ satisfying the axioms above.
\end{definition}

It is not hard to see (but must be checked at least once)
that an operad with $\PP(n) = 0$ for $n\neq 1$ is
exactly the same as an associative algebra. 

\textbf{Warning!} If one does not require
the existence of a unit, the notion of a
\emph{pseudo-operad} by Markl (defined by partial
compositions) does not coincide with the
notion of an operad as defined by May.


\subsection{Constructing operads by hand}

One can define operads in various ways. For example,
one can define the underlying collection explicitly,
and give the composition maps directly:
\begin{tenumerate}
\item \emph{Commutative operad.} The reduced symmetric topological (or set)
operad with $\mathsf{Com}(n)$ a single point for each
$n\in \NN$, and composition maps the unique
map from a point to a point.
\item \emph{Associative operad.} 
The reduced set operad with
$\mathsf{As}(n) = S_n$
the regular representation and composition maps
\[ S_k \times S_{n_1} \times
\cdots \times S_{n_k} \longrightarrow S_{n_1+\cdots n_k} \]
the unique equivariant map that sends the tuple
of identities to the identity.
\item  \emph{Stasheff operad.}
Let $K_{n+2}$ be the subset of $I^n$ (the
product of $n$ copies of $I=[0,1]$) 
consisting of tuples $(t_1,\ldots,t_{n+2})$
such that $t_1\cdots t_k\leqslant 2^{-k}$
for $j\in [n+2]$. The boundary of 
$K_{n+2}$ consists of those points such
that for some $j\in [n+2]$ we have
either $t_j$ or $t_1\cdots t_j = 2^{-j}$.
It is tedious (but otherwise doable)
to show that for each pair $(r,s)$ of
natural numbers and each $i\in [r]$
there exists an inclusion
\[ \circ_i : K_{r+1} \times K_{s+1} \longrightarrow
 	K_{r+s+1} \] 
that defines on the sequence of
spaces $\{K_{n+2}\}_{n\geqslant 0}$
the structure of a non-symmetric operad.
We will see in the exercise a realization
of $K_n$ as the convex hull of points
with positive integer coordinates
(due to J.-L. Loday)  using planar
binary rooted trees, which will make the
operad structure more transparent.

\item If $M$ is a monoid, there is an
operad $\mathbb W_M$ with $\mathbb{W}_M(n) =
M^n$ such that 
\[(m_1,\ldots,m_s) \circ_i (m_1',\ldots,m_t') = 
 	(m_1,\ldots,m_{i-1}, m_im_1',\ldots,m_im_t',m_{i+1},\ldots, m_s).\] We call it the
 \emph{word operad of $M$}. Its underlying
 symmetric collection is $\mathsf{As}\circ M$. 
 
 \item Write $\operatorname{Aff}(\mathbb C) = \mathbb{C}\times \mathbb{C}^\times$ for the group of affine transformations of
$\mathbb C$ with group law $(z,\lambda)(w,\mu) = (z+\lambda w,\lambda\mu)$. In turn, define for each finite set $I$ the topological space
\[ \mathcal{C}(I) = \{ (z_i,\lambda_i)\in \operatorname{Aff}(\mathbb C)^I  : |z_i-z_j|>|\lambda_i|+|\lambda_j| \}.\] 
The group law of $\operatorname{Aff}(\mathbb C)$ allows us to
define an operad structure on $\mathcal{C}(I)$ using the
exact same definition as in the word operad of a monoid. 
The subspaces $\mathcal{D}_2^{\mathrm{fr}}(I)
	\subseteq \mathcal{C}(I)$
where $|z_i|+|\lambda_i|\leqslant 1$ for all $i\in I$, and 
where the inequality is strict unless $z_i=0$ is called
the \emph{framed little disks operad}. The little disks operad
is the suboperad where $\lambda_i = 1$ for all $i\in I$, and 
we write it $\mathcal{D}_2(I)$.

 \item The operad of rooted trees $\mathsf{RT}$ has
 $\mathsf{RT}(n)$ the collection of rooted threes with $n$
 vertices labeled by $[n]$, and the composition $T \circ_j T'$
  is obtained by inserting $T'$ at the $j$th vertex of $T$
  and reattaching the children of that vertex to $T'$ in
  all possible ways. For example, if
%  \[ T=\vcenter{\xymatrix{*++[o][F-]{1} \ar@{-}[d] & 
%*++[o][F-]{3} \ar@{-}[dl]\\
%*++[o][F-]{2}}}
%\ {\hbox{\rm and}}\ \  S=\vcenter{
%\xymatrix{*++[o][F-]{2} \ar@{-}[d] \\
%*++[o][F-]{1}}}\]
then we have that  
%  \[ T\circ_2 S=\quad
% \vcenter{\SelectTips{cm}{}\xymatrix@-1pc{
%*++[o][F-]{1}\ar@{-}[d]&*++[o][F-]{4}\ar@{-}[dl]\\
%*++[o][F-]{3}\ar@{-}[d]\\
%*++[o][F-]{2}}} +\quad
%\vcenter{\SelectTips{cm}{}\xymatrix@-1pc{
%*++[o][F-]{1}\ar@{-}[d]\\
%*++[o][F-]{3}\ar@{-}[d]&*++[o][F-]{4}\ar@{-}[dl]\\
%*++[o][F-]{2}}}\quad +\quad
%\vcenter{\SelectTips{cm}{}\xymatrix@-1pc{
%*++[o][F-]{4}\ar@{-}[d]\\
%*++[o][F-]{3}\ar@{-}[d]&*++[o][F-]{1}\ar@{-}[dl]\\
%*++[o][F-]{2}}} \quad + \quad
%\vcenter{\SelectTips{cm}{}\xymatrix@-1pc{
%*++[o][F-]{1}\ar@{-}[dr] & *++[o][F-]{3}\ar@{-}[d]&*++[o][F-]{4}\ar@{-}[dl] \\
%&*++[o][F-]{2}}}. \]
\end{tenumerate}
\subsection{Exercises}

The corresponding exercises to this lecture appear in \textbf{Exercise Sheet \#1}. During the exercise sessions, we will review pseudo-operads
and their partial insertions, and the category of operads. 
 
\pagebreak

\section{Free operads and presentations}

\textbf{Goals.} We will define algebraic operads
by generators and relations, and with this at
hand define quadratic and quadratic-linear
presentations of operads. 

\subsection{Planar and non-planar trees}

Operads and their kin are gadgets modeled after
combinatorial graph-like objects. Operads themselves are modeled after
rooted trees, so it is a good idea to have a concrete definition of 
what a rooted tree is. We will also consider planar rooted trees,
and trees with certain decorations, so it is a good idea to digest
the definitions carefully to later embellish them.

A rooted tree $\tau$ is the datum of a finite set $V(\tau)$
of vertices along with a partition $V(\tau) = 
\mathrm{Int}(\tau)\sqcup L(\tau) \cup R(\tau)$,
where the first are the \emph{interior} vertices, $L$ are the leaves, and $R(\tau)$ is
a singleton, called the root of $\tau$. We also require there is 
a function $p :V(\tau)\smallsetminus R(\tau) \longrightarrow V(\tau)$,
describing the edges of $\tau$, 
with the following properties: call a vertex $v\in V(\tau)$ a child 
of $w\in V(\tau)$ if $v\in p^{-1}(w)$. Then:
\begin{tenumerate}
\item The root $r\in R(\tau)$ has exactly one child.
\item The leaves of $\tau$ have no children.
\item For each non-root vertex $v$ there exist a unique sequence
$(v_0,v_1,\ldots,v_k)$ such that $p(v_{i-1}) = v_{i}$ for $i\in [k]$
with $v_0 = v$ and $v_k = r$.
\end{tenumerate}
We will call a non-leaf vertex that has no children a \emph{stump}
(or an endpoint, or a cherry-top).
A tree is reduced if has no stumps and all of its non-root and
non-leaf vertices have at least two children. We will also call
the root the (unique) output vertex $\tau$, and the leaves
the input vertices of $\tau$. 

A planar rooted tree is a rooted tree $\tau$
along with a linear order in each of the fibers
of the parent function $p$ of $\tau$. In short,
the children of each vertex are linearly ordered,
so we are effectively considering a drawing of
$\tau$ in the plane, where the clockwise
orientation gives us the order at each
vertex. 

Two rooted trees $\tau$ and $\tau'$ are isomorphic
if there exists a bijection $f : V(\tau) \longrightarrow 
V(\tau')$ that preserves the root, the input vertices and the
interior vertices, so that $p'\circ f = p$ where we
also write $f$ for the induced bijection $f : V(\tau)\smallsetminus r \longrightarrow 
V(\tau')\smallsetminus r'$. Two planar rooted trees
are isomorphic if in addition $f$ respects the linear order
at each vertex.

For example, consider the rooted tree $\tau$ with
$V = \{1,2,3\}\cup \{4,5\}\cup \{0\}$, that is,
three leaves, two interior vertices and the root.
Then the choice of $p : [5] \to \llbracket 5 \rrbracket$ 
with $p(\{1,2\}) = 4$, $p(\{3,4\}) = 5$, $p(5) = 0$
gives a tree isomorphic to the one with 
with $p(\{1,2\}) = 3$, $p(\{3,4\}) = 5$, $p(5) = 0$.
On the other hand, if we consider the vertices linearly
ordered by their natural order, these two planar rooted trees 
are no longer isomorphic. 

\begin{definition}
For a finite set $I$, an $I$-labeled tree $T$
is a pair $(\tau,f)$ where $\tau$ is a 
reduced rooted tree, along with
a bijection $f : I \longrightarrow L(\tau)$.
Two $I$-labeled trees $T$ an $T'$ are isomorphic
if there exists a pair $(g,\sigma)$ where
$g$ is an isomorphism between $\tau$ and $\tau'$
and $\sigma$ is an automorphism of $I$ such that
$g\mid_{L(\tau)}\circ f = \sigma\circ f'$. 
\end{definition}

Suppose that $(\tau,f)$ is an $I$-tree and that
$(\tau',f')$ is a $J$-tree, and that $i\in I$. We define
$K=I\cup_i J = I\sqcup J \smallsetminus i$ and the
$K$-tree $\tau\circ_i \tau'$ as follows:
\begin{tenumerate}
\item Its leaves are $L(\tau\circ_i \tau') = L(\tau)\sqcup L(\tau')\smallsetminus f^{-1}(i)$.
\item Its internal vertices are $V(\tau)\sqcup V(\tau')$, with
root $r$. 
\item The parent function $q$ is defined by declaring that:
	\begin{titemize} 
	\item $q$ coincides with $p$ on $V(\tau)$, 
	\item $q(w) = p(f^{-1}(i))$ if
$w$ is the unique children of the root of $\tau'$, 
	\item  $q$
coincides with $p'$ on $V(\tau')\smallsetminus \{r',w\}$.
\end{titemize}
\item The leaf labeling is the unique bijection $L(\tau\circ_i \tau') \longrightarrow I\circ_i J$ extending $f$ and $f'$.
\end{tenumerate} 
Let us now consider an 
(unbiased) reduced symmetric sequence $\XX$ which
we will think of as an \emph{alphabet}. An tree monomial in the
alphabet $\XX$ 
is a pair $(\tau,x)$ where $\tau$ is a reduced rooted
tree and $x : \mathrm{Int}(\tau) \longrightarrow \XX$ is a map
with the property that $x(v) \in \XX(p^{-1}(v))$. Observe that
reduced sequences and reduced trees correspond to each other, in the
sense that with this definition we can only decorate a stump
of $\tau$ with an element of $\XX(\varnothing)$. 

An $I$-labeled
$\XX$-tree $T$ is a triple $(\tau,x,f)$ where $(\tau,f)$ is $I$-labeled
and $(\tau,x)$ is an $\XX$-tree. We will say that $(\tau,x,f)$
is a (symmetric) tree monomial if $\XX$ is symmetric. If it
is just a collection, we will say that $(\tau,x,f)$ is a
ns tree monomial. In particular, if $T$ is an $I$-labeled
tree, and if $\sigma \in \Aut(I)$, there is another 
$I$-labeled tree $\sigma(T)=(\tau,f\sigma^{-1})$. 


Suppose that $T = (\tau,x,f)$ is a tree monomial on an
alphabet $\XX$, and let us pick a vertex $v$ of $\tau$
and a permutation $\sigma$ of the set $C = p^{-1}(v)$ of
children of $v$. We define the tree $\tau^\sigma$ as
follows: the datum defining $\tau$ remains unchanged
except $p$ is modified to $p^\sigma$ so that 

\[ p^\sigma(w) = 
\begin{cases}
 p(w) & \text{if $p^2(w)\neq v$} \\
 p(\sigma^{-1}(w')) & \text{if $p(w)=w'\in C$}.
\end{cases}
\]
Briefly, we just relabel the vertices of $\tau$
using $\sigma$. With this at hand, we define
$T^\sigma$ to be the tree monomial with underlying 
tree $\tau^\sigma$ and with $x$ modified to $x^\sigma$ so that
\[ x^\sigma(w) = 
\begin{cases}
 \sigma x(v) & \text{if $v=w$,} \\
 x(\sigma^{-1}(w')) & \text{if $p(w)=w'\in C$}.
\end{cases}
\]
Note that it is possible some children of $v$ are
leaves, in which case the definitions make sense if
we think of leaves as decorated by the unit of
$\kk$. 

\begin{example}
Let us consider the alphabet 
$\XX = \XX(2) = \{\beta \}$ where
the unique operation is antisymmetric.
Then we have the following equalities
of symmetric tree monomials:
\[ 
	\leftc{\beta}{\beta}{2}{1}{3}  \mathsymbol{1.5}{=}
\rightc{\beta}{\beta}{3}{1}{2}  
   \mathsymbol{1.5}{=-}
 {\leftc{\beta}{\beta}{1}{2}{3}}
  \mathsymbol{1.5}{.}\]
 
\end{example}


Let us now define for each $n\geqslant 1$ the
space $\FF_\XX(I)$ as the span of all tree monomials
$T = (\tau,f,x)$ on $\XX$ with leaves labeled by $I$,
modulo the subspace generated by all elements of the form
\[ R(T,v,\sigma) = T - T^\sigma \]
where $\sigma$ ranges through $\Aut(p^{-1}(v))$ as
$v$ ranges through the vertices of $\tau$. In case
all children of $v$ are leaves, this is saying that
the tree where $x_v$ is replace by $\sigma(x_v)$
is equal to the tree where the leaves of $T$ that are
children of $v$ are relabeled according to $\sigma$.
We also require that tree decorations behave like
tensors, so that $T = T_1+T_2$ if the decoration
of $T$ at a vertex $v$ is of the form $x_1 + x_2$
and for $i\in [2]$ the tree $T_i$ coincides with
$T$ except that it is decorated by $x_i$ at $v$.



\subsection{The free operad}
An algebraically inclined way to construct
(algebraic) operads is through generators and
relations. There is a forgetful functor
from the category of operads to the category
of collections. In general, it admits a left
adjoint, which is the free operad functor.

\begin{definition}
The \emph{free symmetric operad} on $\XX$ is the
symmetric sequence $\FF_\XX$ along with the composition
law obtained by grafting of trees. More precisely,
suppose that $T\in \FF_\XX(I)$ and that $T'\in\FF_\XX(J)$,
and that $i\in I$. We define $T'' =T\circ_i T' \in 
\FF_\XX(I\cup_I J)$ by taking its underlying labeled
tree to be $\tau\circ_i \tau'$, and by decorating it
in the unique way which extends the decorations of 
$T$ and $T'$.
\end{definition}

The following lemma shows that this indeed defines an operad.

\begin{lemma}
Tree grafting respects both $I$-tree 
isomorphisms and the relations $T\sim T^\sigma$
above, and hence is well defined on $\FF_\XX$.
\end{lemma}

\begin{proof}
This is in \textbf{Exercise Sheet \#2}. 
\end{proof}

 We will later
interpret $\XX\longmapsto \FF_\XX$ as a \emph{monad},
thus giving another definition of operads. The
advantage of this `monadic approach' is its 
flexibility, which allow us to define other
operad like structures, like the ones 
mentioned in the introduction.
In this direction, a curious reader 
can consider the following 
equivalent definition:

\begin{definition}
The free operad generated by a symmetric
collection $X$ is defined inductively by
letting  $\FF_{0,X}=\kk$ be spanned
by the `twig' (tree with no vertices and one edge)
in arity zero and
\[ \FF_{n+1,\XX} = \kk\oplus (\XX\circ  \FF_{n,\XX} ), \]
and finally by setting
$\FF_\XX = \varinjlim_n \FF_{n+1,\XX}$.
The composition maps are defined by induction,
and the axioms are also checked by induction.
\end{definition}

Intuitively, the previous definition says that an element of
$\FF_\XX$ is either the twig, or corolla
with $n$ vertices decorated by $\XX$, whose leaves
have on them an element of $\FF_\XX$. The
final shape of $\FF_\XX$ will however
depend on the symmetric structure of $ \XX$. 
 \subsection{Exercises}

The corresponding exercises to this lecture appear in \textbf{Exercise Sheet \#2}.

\pagebreak

%\begin{note}
%The free conilpotent cooperad $\FF^c_\XX$ can be defined
%in a completely analogous fashion: it has the same 
%underlying symmetric sequence as $\FF_\XX$ (much like
%$TV$ and $T^cV$ have the same underlying vector space)
%and it has decomposition maps obtained by `degrafting'
%trees.
%\end{note}

\section{Quadratic operads}

 \textbf{Goal.} Introduce weight graded gadgets,
 define operads by generators and relations, and
 introduce quadratic operads. Give plenty of examples
 of `real life' quadratic operads to work on:
 Hilbert series, Koszul dual, bar construction. 
 
 \subsection{Weight gradings and presentations}
 The notion of a quadratic operad is based on the observation
 every free operad has a canonical `weight grading' by the
 number of internal vertices of a tree. Let us make this
 precise.
 
\begin{definition}
A symmetric sequence $\XX$ is weight graded if for
each finite set the component $\XX(I)$ admits a 
decomposition $\XX(I) = \bigoplus_{j\geqslant 0}
\XX^{(j)}(I)$. A symmetric operad $\PP$ is weight
graded if its underlying symmetric sequence
is weight graded and its composition maps
preserve the weight grading.
\end{definition}

Thus, a weight graded operad must have composition maps 
of the form
\[ \PP^{(a)}(k) \otimes 
	\PP^{(b_1)}(n_1) \otimes \cdots \otimes \PP^{(b_k)}(n_k)
	 	\longrightarrow \PP^{(b)}(n) \]
where $b=b_1+\cdots+b_k$ and $n = n_1+\cdots+n_k$. In the
case we consider partial composition maps, observe we have
instead maps of the form
\[\circ_i :  \PP^{(a)}(m)\otimes  \PP^{(b)}(n)
	\longrightarrow  \PP^{(a+b)}(m+n-1). \] 
The free operad $\FF_\XX$ is weight graded by the number
of internal vertices of a tree (that is, we put $\XX$ in
weight one, and extend the weight to trees by counting 
occurrences of elements of $\XX$. More generally, if
$\XX$ admits a weight grading, then $\FF_\XX$ inherits
this weight grading: the weight of a tree monomial is the
sum of the weight of the decorations of its vertices,
 and we write $\FF_\XX^{(n)}$ for the
homogeneous component of weight $n\in\NN_0$. If we do
not specify a weight grading on $\FF_\XX$, we will 
always assume we are taking the canonical weight grading above.


\begin{definition} An ideal in an operad $\PP$
is a subcollection $\mathcal{I}$ for which
both $\gamma(\mathcal{I}\circ \PP)$ and
$\gamma(\PP\circ_{(1)} \mathcal{I})$ are contained in
$\mathcal{I}$.
The quotient of $\PP/\mathcal{I}$ is again an
operad, called the quotient of $\PP$ 
by $\mathcal{I}$. Every subcollection $\RR$
of $\PP$ is contained in a smallest
ideal, called the \emph{ideal generated by $\RR$}.
\end{definition}

The notion of ideals and of free operads allow us
to define operads by generators and relations.

\begin{definition}
We write $\FF(\XX,\RR)$ for the quotient of 
$\FF_\XX$
by the ideal generated by a subcollection $\RR$
of $\FF_\XX$. 
 We say $\PP$ is presented by generators
$\XX$ and relations $\RR$ if there is an
isomorphism $\FF(\XX,\RR) \longrightarrow
\PP$.
\end{definition}

Note that if $\PP$ is symmetric, the definition
requires that $\mathcal{I}$ be stable under
the symmetric group actions, so we may 
sometimes specify $\RR$ by a generating set 
only, and understand
that $(\RR)$ is generated by the $\Sigma$-orbit
of $\RR$.

\medskip

\textbf{The classics.}
The associative operad is generated by a 
binary operation $\mu$ generating the regular
representation of $S_2$ subject to the only relation
\[ \mu\circ_1 \mu = \mu \circ_2 \mu. \] 
Similarly, the commutative operad is generated by a 
binary operation which instead generates the
trivial representation of $S_2$. Both of these
examples arise as the linearization of a set operad.

The Lie operad is generated by a single binary 
operation $\beta$ that generates the sign 
representation of $S_2$ subject to the only
relation
\[ (\beta \circ_1 \beta)(1+\tau+\tau^2) = 0 \] 
where $\tau = (123)\in S_3$ is the $3$-cycle. 
We write these operads $\mathsf{As},\mathsf{Com}$
and $\mathsf{Lie}$ and, following J.-L. Loday,
call them the \emph{three graces}. We have that
\[ \As(n) = \kk S_n,\quad
 	\Com(n) = \kk, \quad
 	 \Lie(n) = \operatorname{Ind}_{\mathbb Z/n}^{S_n} \kk_\zeta \] 
 where $\kk_\zeta$ is a character of $\mathbb Z/n$
 for a primitive $n$th root of the unit. 
 
 
\begin{note} It is not always advantageous
to define an operad by generators and relations:
the operad pre-Lie can be defined explicitly
in terms of labeled rooted trees and a grafting
operation, as done by Chapoton--Livernet, and
this `presentation' is very useful in practice,
for example, to show that the pre-Lie operad
is Koszul.
\end{note}

\subsection{Quadratic operads}
An operad $\PP$ is \emph{quadratic} if it admits a presentation
$\FF(\XX,\RR)$ where $\RR \subseteq \FF(\XX)^{(2)}$. 
That is, $\PP$ is generated by some collection of
operations $\XX$ and all its defining relations are of the form
\[ \sum \lambda_{\mu,\nu}^i \mu \circ_i \nu = 0  \] 
where $\ari(\mu)+\ari(\nu)$ is constant. An operad is
\emph{binary quadratic} if moreover $\XX = \XX(2)$ or,
what is the same, all the generating operations of $\PP$
are of arity two (binary). 
 A \emph{quadratic-linear presentation} of an operad $\PP$
is a presentation $\FF(\XX,\RR)$ of $\PP$ where $\RR 
\subseteq \XX \oplus \FF(\XX)^{(2)}$. That is, it is
a presentation of the form
\[ \sum \lambda_{\mu,\nu}^i \mu \circ_i \nu 
 + \sum \lambda_\rho \rho = 0   \] 
 where $\ari(\mu)+\ari(\nu) = \ari(\rho)+1$ is constant.
 Every operad admits a quadratic-linear
presentation, albeit with possibly with infinitely many generators, 
We will postpone the discussion of such presentations
to a later lecture.
 
 Let us define a quadratic datum to be a pair $(\XX,\RR)$
 where $\XX$ is a symmetric sequence and $\RR\subseteq
 \FF_\XX^{(2)}$. A map of quadratic data $(\XX_1,\RR_1) 
 \longrightarrow (\XX_2,\RR_2)$ is a map $\XX_1
 \to \XX_2$ of symmetric sequences for which the induced
 map on free operads sends $\RR_1$ to $\RR_2$. The 
 assignment $(\XX,\RR) \longrightarrow \FF(\XX,\RR)$
 defines a functor from the category of quadratic data
 to the category of quadratic operads.
 
 %%Exercise: find two different QD giving same operad.
 \bigskip
 
\textbf{More examples.} The presentations of the 
associative, commutative and Lie operad above are
quadratic. The following are also quadratic operads:

\emph{The Gerstenhaber operad}. The symmetric operad
$\mathsf{Ger}$ and its cousin,
the Poisson operad $\mathsf{Poiss}$ belong to the two parameter
family $\mathsf{Poiss}(a,b)$ of binary quadratic operads generated
by two operations $x_1x_2$ and $[x_1,x_2]$ of respective degrees
$a$ and $b$, so that
the first is commutative associative, the second is a Lie
bracket, and they satisfy the Leibniz rule. With this
at hand $\mathsf{Ger} =\mathsf{Poiss}(0,-1)$ while  
$\mathsf{Poiss} = \mathsf{Poiss}(0,0)$.

\smallskip

\emph{The pre-Lie operad}. The operad $\mathsf{PreLie}$ and its
quotient, the Novikov operad $\mathsf{Nov}$, are quadratic
binary operads generated by a single operation $x_1\circ x_2$
with no symmetries. The first one is subject to the right-symmetry
condition for the associator
\[ 
x_1\circ (x_2\circ x_3) - (x_1\circ x_2)\circ x_3	=
x_1\circ (x_3\circ x_2) - (x_1\circ x_3)\circ x_2.
\]
The second 
operad is obtained by further imposing the left-permutative
relation that 
\[ x_1\circ (x_2\circ x_3) = x_2\circ (x_1\circ x_3).\]
The permutative operad $\mathsf{Perm}$ is the binary
operad generated by a single operation with no symmetries
satisfying the last quadratic equation.

\smallskip

\emph{The operad of totally associative $k$-ary algebras}. 
$\mathsf{tAs}_k$ (and its commutative counterpart). It is generated
by a $k$-ary non-symmetric operation $\alpha$ subject to
the relations $\alpha \circ_i \alpha =\alpha\circ_k \alpha$
for all $i\in [k]$. One can consider $\alpha$ to be
totally symmetric, and obtain the operad of totally
associative commutative $k$-ary algebras.

\smallskip

\emph{The operad of partially associative $k$-ary algebras}.
$\mathsf{pAs}^k$ (and its Lie counterpart). It is generated
by a $k$-ary non-symmetric operation $\alpha$ of degree $k-2$
subject to the single relation
\[ 
	\sum_{i=1}^k (-1)^{(k-1)(i-1)} \alpha\circ_i \alpha = 0.\]
One can consider a $k$-ary totally antisymmetric operation
$\beta$ of degree $1$, and obtain the operad of Lie $k$-algebras, 
which is subject to the single equation
\[
 \sum_{\substack{A\sqcup B = [2k-3] \\
 |A|=k-1,|B|=k-2}}  (\beta\circ_1\beta)\sigma_{A,B} = 0.
 \]

\smallskip

\emph{The operad of anti-associative algebras.} $\mathsf{As}^-$ is
generated by a single operation of degree zero with no symmetries
satisfying the `anti-associative law'
\[ x_1(x_2x_3) + (x_1x_2)x_3 = 0. \]

\subsection{Exercises}

The corresponding exercises to this lecture appear in \textbf{Exercise Sheet \#3}.

\pagebreak

% 
%\subsection{Algebras over operads}
%Operads are important not in and of themselves 
%but through their representations, more commonly
%called \emph{algebras over operads}. In fact,
%one can usually `create' an operad by declaring
%what kind of algebras it governs. If the
%algebra has certain operations of certain
%arities, these define the generators of the operad,
%and the relations these operators must satisfy 
%give us the relations presenting our operad. 
%
%\begin{definition}
%A $\PP$-algebra structure on a vector
%space $V$ is the datum of a map of operads 
%$\PP \longrightarrow \End_V$.
%\end{definition}
%
%Alternatively, one can consider the situation
%when $\otimes$ is closed and has a right
%adjoint $\hom$ (the internal hom) so that
%what we want are maps
%\[\label{eq:Palgebramap}
%\gamma_{V,n} : 
%\PP(n)\otimes_{S_n} V^{\otimes n} \longrightarrow V \]
%declaring how each $\mu \in\PP(n)$ acts as an operation
%$\mu : V^{\otimes n} \longrightarrow V$. 
%It follows that a
% $\PP$-algebra structure on $V$ is the same as the datum
%of maps as in \ref{eq:Palgebramap} that satisfy the following
%conditions:
%
%\begin{tenumerate}
%\item \emph{Associativity:} let $\nu\in \PP(n)$ and
%$\nu_i \in \PP(k_i)$ for $i\in [n]$, and pick
%$w_i \in V^{\otimes k_i}$. Set $v_i = \gamma_{V,k_i}(\nu_i,w_i) \in V$
%and $\mu = \gamma_\PP(\nu;\nu_1,\ldots,\nu_n)$. Then 
%\[ \gamma_{V,k_1+\cdots+k_n}(\mu; w_1,\ldots,w_n) = 
%	\gamma_{V,n}(\nu ;v_1,\ldots, v_n).\]
%\item \emph{Equivariance:} for $\nu\in\PP(n)$, $v_1\otimes\cdots \otimes v_n \in V^{\otimes n}$
%and $\sigma\in S_n$, we have that
%\[ \gamma_{V,n}(\nu \sigma, v_{\sigma 1} \otimes \cdots \otimes v_{\sigma n}) = \gamma_{V,n}(\nu; v_1\otimes\cdots \otimes v_n). \]
%\item \emph{Unitality:} if $1\in\PP(1)$ is the unit, then
%$\gamma_{V,1}(1,v) = v$.  
%\end{tenumerate}
%
%
%
%\textcolor{trinityblue}{Complete. Recognition
%principle. Gerstenhaber algebras and Hochschild
%complex.}
%
%

 \section{Koszul duality I}
\textbf{Goals.}
Give the definition of the Koszul dual
operad of a quadratic operad, and then compute
some Koszul duals. Give the definition of the
Koszul complexes associated to a quadratic operad,
and define Koszul operads.

\subsection{Differential graded sequences}


\textbf{Homologically graded $\Sigma$-modules.}
A (homologically) graded vector space is a
vector space $V$ along with a direct sum
decomposition $V = \bigoplus_{n\in\mathbb Z} V_n$.
We call the components of this sum the \emph{graded (or
homogeneous)
components of $V$}, and say that an element in 
on of these summands is \emph{homogeneous}. If
$v\in V_n$, we say that $v$ is \emph{homogeneous of
degree $n$} and write $|v|=n$. 

A map $f : V\longrightarrow W$ of graded vector spaces
is \emph{homogeneous of degree $n$} if $f(V_j)\subseteq W_{j+n}$ for
all $j\geqslant 1$. We write $\hom(V,W)$ for the
space of all homogeneous maps, which is itself a graded
vector space with $\hom(V,W)_n$ the space of all
graded maps of degree $n$ for each $n\in\mathbb Z$. 
In this way, we obtain the category $\mathsf{Vect}_\mathbb{Z}$
of graded vector spaces and graded maps. 

A \emph{differential graded (dg) vector space} is a pair 
$(V,d)$ where $V$ is a graded vector space and 
$d : V\longrightarrow V$ is a homogeneous map of degree 
$-1$ such that $d^2=0$. We usually will call $(V,d)$
a \emph{chain complex}. The collection of homogeneous
maps $V\longrightarrow W$ is again a chain
complex, with differential
\[ d\varphi 
	= d_V\varphi - (-1)^{|\varphi|} \varphi d_W. \]
A homogeneous map of degree zero such that $d(\varphi)=0$
is called a \emph{chain map}.
It is
convenient to also consider \emph{cohomologically graded}
vector spaces, by formally inverting the order of $\mathbb{Z}$
and letting $V^n = V_{-n}$ for all $n\in\mathbb Z$. 

\bigskip

\textbf{Monoidal structure.} If $V$ and $W$ are
graded vector spaces, we define their tensor product
by setting
\[ (V\otimes W)_n = \bigoplus_{i+j = n} V_i\otimes W_j \]
for all $n\in\mathbb Z$, and setting the symmetry map
\[\tau : V\otimes W \longrightarrow W\otimes V\]
to be $\tau(v\otimes w) = (-1)^{|v||w|}w\otimes v$
on homogeneous elements, and extending it linearly on all of
$V\otimes W$. This makes $\mathsf{Vect}_\mathbb{Z}$ into a
symmetric monoidal category with unit the graded vector
space with $V_0 = \kk$ and $V_n = 0$ for $n\neq 0$.
The tensor product of maps $f: V\longrightarrow V'$
and $g : W\longrightarrow W'$ acts
in such a way that $f\otimes g : V\otimes V'
\longrightarrow W\otimes W'$ is the map
\[ (f\otimes g)(v\otimes w)  = (-1)^{|g||v|} f(v)\otimes g(w).\]
In case $V$ and $W$ are in fact dg, their tensor product is
also dg with $d_{V\otimes W} = d_V\otimes 1+ 1\otimes d_W$. 

\begin{definition}
A (homologically) graded $\Sigma$-module $\XX$
is a $\Sigma$-module taking values in the category
of graded vector spaces. Similarly, a dg $\Sigma$-module
is one taking values in dg vector spaces.
\end{definition}

\textbf{The endomorphism operad functor on dg modules.}
Let us consider the most natural way to create dg modules
from dg vector spaces, as we did in the case of usual
vector spaces. Namely, we may as before consider the
\emph{endomorphism operad} of a dg vector space $V$
by setting, for each $n\geqslant 0$,
\[ \End_V(n) = \hom(V^{\otimes n},V) \]
where these consists of homogeneous maps of dg vector
spaces. In particular, each of these arity components is 
itself a dg vector space, and the (total or partial)
composition maps
of the resulting operad are maps of dg vector spaces.

Of particular importance to us will be the \emph{suspension}
operation on dg vector spaces. Let us write $s$ for the
unique dg vector space with $s_1 = \mathbb C$ and zero
elsewhere, and similarly let us write $s^{-1}$ for the
unique dg vector space with $s^{-1} = \mathbb{C}$
and zero elsewhere. The \emph{suspension} of the dg vector
space $V$ is the tensor product $s\otimes V$, which
we write more simply $sV$, and whose basis elements we
write $sv$ for $v\in V$. Thus $|sv| = |v|+1$ for all 
homogeneous $v\in V$. Similarly, we define the
\emph{desuspension} $s^{-1}V$.

\begin{note}
The differential of $sV$ is given by $d(sv) = -s dv$.
\end{note}
 
The following lemma shows that $V\mapsto \End_V$ is 
monoidal for the \emph{Hadamard product} of operads on the
target (and the usual tensor product on the domain): 
 \begin{lemma}
The map $\Phi : \End_V\otimes \End_W\longrightarrow \End_{V\otimes W}$
that assigns $\varphi \otimes \psi \in \End_V(n)\otimes \End_W(n)$
to the map
\[ \Phi(\varphi,\psi)(v,w) = (-1)^\varepsilon \varphi(v)\otimes\psi(w)\]
where $\varepsilon = \sum_{i=1}^n (|w_1| +\cdots + |w_{i-1}|+|\psi|)|v_i|$
is an isomorphism of operads.
 \end{lemma}
 
 \begin{proof}
 This is in Exercise Sheet \#4.
 \end{proof}

In particular, we see that $\End_{sV}$ is canonically isomorphic
with $\End_s\otimes \End_V$, and hence that algebra structures on $sV$
are related to algebra structures on $V$ through the operad $\End_s$.
Let us give it a name. 

\subsection{The Koszul dual}

\newcommand{\sus}{\mathscr{S}}
\textbf{Suspensions.} 
We call $\End_{s}$ the suspension operad
and write it $\sus$. Note that $\End_{s}(n)$ is
the sign representation of $\Sigma_n$ put in degree $1-n$.

\begin{prop} For each $n\geqslant 1$ let us
we write $\nu_n$ for the unique map in $\End_s(n)$ 
that sends $s^n$ to $s$. Then for every $m\geqslant 1$
we have that
\[ \nu_n \circ_i \nu_m = (-1)^{(i-1)(m-1)} \nu_{m+n-1}. \]
In particular, the binary operation $\nu := nu_2$ of degree
$-1$ generates $\End_s$,
and presents it as a quadratic operad subject to the 
anti-associativity relation
\[ \nu \circ_1\nu + \nu\circ_2 \nu = 0.\]
\end{prop}
\begin{proof}
 This is in Exercise Sheet \#4.
\end{proof}

If $\PP$ is an operad, then the arity-wise tensor product
$\sus\otimes \PP$ is called the suspension of $\PP$
and we write it $\sus\PP$ or $\PP\{1\}$. Dually, we
write $\sus^{-1}$ for the desuspension operad
defined by $\End_{s^{-1}\kk}$. 

\begin{note} As we just observed,
the operad  $\sus\PP$ has the property that
$\sus\PP(sV) = s\PP(V)$, so that algebras over $\sus\PP$
are exactly those vector spaces $V$ such that $s^{-1}V$ is a
$\PP$-algebra. Equivalently, $sV$ is a $\sus\PP$-algebra
if and only if $V$ is a $\PP$-algebra. 
\end{note}

\textbf{Pairings.} We define a pairing between $\FF_\XX$ and
$\FF_{s^{-1}\sus^{-1}\XX^*}$ as follows (the appearance of
the suspensions will be evident later):
\[ \langle \Sigma\nu^* \circ_j \Sigma\mu*, 
	\rho \circ_i \tau  \rangle
   = \delta_{ij} (-1)^{(\ari(\nu)-1)(|\mu|+i-1)+|\nu||\mu|}
   	\nu^*(\rho)\mu^*(\tau). \] 
If $\XX = \XX(2)$ is binary and has no homological degrees, 
this simplifies to
\[ \langle \Sigma\nu^* \circ_i \Sigma\mu*, 
	\rho \circ_i \tau  \rangle
   =  \begin{cases}
    	\phantom{-}\nu^*(\rho)\mu^*(\tau) & i=1 \\
    	-	\nu^*(\rho)\mu^*(\tau) & i = 2.
    	\end{cases} \] 

\begin{definition}
The Koszul dual operad of a quadratic operad $\PP$ 
generated by $\XX$ subject to relations $\RR$, is
the operad $\PP^!$ generated by $s^{-1}\sus^{-1}\XX^*$ 
and subject to the orthogonal space of relations
$\RR^\perp$ according to the pairing above.
\end{definition} 

\begin{note}
Let $\PP$ be an operad. Then $\PP$ is quadratic if and
only if $\sus\PP$ is quadratic, and it is Koszul if and
only if $\sus\PP$ is Koszul. 
\end{note}

\textbf{Some examples.} Let us compute the Koszul duals of 
some of the quadratic operads we introduced in \textbf{Lecture 3}.
For simplicity, we will consider only those with binary 
generators of degree zero, though one can in the same way
carry out computations with generators of higher arities and
varying homological degrees.

\bigskip

\emph{The associative operad}. We saw previously that for
$\underline{\XX}$ consisting of a single operation
$x_1x_2$ with no symmetries, the
space $\FF_\XX(3)$ is twelve dimensional, spanned by
the $S_3$-orbits of $\alpha = x_1(x_2x_3)$ and $\beta =(x_1x_2)x_3$,
each of size six. We also noted that $\alpha-\beta$
spans a six dimensional submodule, complemented by the
orbit of $\alpha+\beta$. 

Using the pairing above, we see that
\[\langle \alpha,\alpha\rangle = 1,
	\quad \langle\beta,\beta\rangle = -1,
	\quad \langle \alpha,\beta\rangle = 0, \] 
from where it follows that the dual space to the associativity
relation is the corresponding associativity relation
$\alpha^* - \beta^*$ in $\XX^*$. In other words,
the associative operad is Koszul self-dual:
\[\mathsf{Ass}^! = \mathsf{Ass}.\]

It is important to note how the minus sign in our
definition of the pairing or, more generally, the
Koszul sign we have introduced, guaranteeing that
this pairing in equivariant, introduces the minus sign
in the dual of $\alpha+\beta$.

\bigskip

\emph{The commutative and Lie operads.}
We have computed that if $\XX(2)$ is the trivial representation
of $S_2$ spanned by some commutative operation $x_1x_2$,
then $\FF_\XX(3)$ is three dimensional, spanned by
$x_1(x_2x_3)$, $(x_1x_2)x_3$ and $(x_1x_3)x_2$.
Moreover, we verified that if we put
\[ \alpha = x_1(x_2x_3) -(x_1x_2)x_3, 	\quad 
     \beta =  x_1(x_2x_3) -(x_1x_3)x_2 \]
     then these two element span an $S_3$-submodule
     that is complemented by the $S_3$-submodule generated by
     \[ \gamma = x_1(x_2x_3) +(x_1x_2)x_3+  (x_1x_3)x_2.\]
 This is in fact an orthogonal complement as a direct computation
 shows, so we see that the orthogonal set of relations
 to the commutative associative relation is the dual of
 $\gamma$ for the dual antisymmetric operation $[x_1,x_2]$:
 this is exactly the Jacobi relation
 \[ 
 \gamma^* = -[x_1,[x_2,x_3]] +[[x_1,x_2],x_3]+[[x_1,x_3],x_2].
 \]
 It follows that the Koszul dual of the commutative operad is the
 Lie operad, and conversely:
 \[ \mathsf{Com}^! = \mathsf{Lie}, \quad
  \mathsf{Lie}^! = \mathsf{Com}.
  	\]
With this at hand, one can compute that the Poisson operad is self-dual:
one only needs to address the Leibniz relation.  

\bigskip

\emph{The pre-Lie and permutative operads. The Novikov operad.}
Recall the pre-Lie operad is generated by a single operation $x_1x_2$
with no symmetries, subject to the pre-Lie relation
\[(x_1x_2)x_3 - x_1(x_2x_3) - (x_1x_3)x_2 + x_1(x_3x_2). \]
One can check that the $S_3$-orbit $V$ of this element is three dimensional,
so let us write $\alpha_1,\alpha_2$ and $\alpha_3$ for the translates of
this relation in $\FF_\XX(3)$. 

This orbit is complemented by the orbit $W$ of the associativity relation
$(x_1x_2)x_3 - x_1(x_2x_3)$ and the orbit $U$ of the permutative relation
$(x_1x_2)x_3 - (x_1x_3)x_2$. The first is six dimensional, as we already
computed, while the second is three dimensional. It is a direct computation
to check that $V^\perp$ identifies with the nine dimensional subspace
$U^*\oplus W^*$. 

Thus, we see that the operad of pre-Lie algebra is Koszul dual to that
of permutative algebras:
\[ \mathsf{PreLie}^! = \mathsf{Perm},\quad
 	\mathsf{Perm}^! = \mathsf{PreLie}.\]
One can use this to show that the operad controlling Novikov algebras,
those pre-Lie algebras whose product is \emph{left} permutative
\[ x_1(x_2x_3) = x_2(x_1x_3) \]
is almost Koszul self-dual: we have that $\mathsf{Nov}^! = 
\mathsf{Nov}^{\mathrm{op}}$, by which we mean the resulting
operad controls pre-Lie algebras with associator symmetric
in the \emph{first two} variables (left-symmetric) and 
whose pre-Lie operation is \emph{right} permutative.
\subsection{Exercises}
The corresponding exercises to this lecture appear in \textbf{Exercise Sheet \#4}.
\end{document}

\pagebreak

\section{Koszul duality II}
\subsection{Operadic homological algebra}

\subsection{Twisting cochains}

\subsection{The Koszul complexes}

\subsection{Exercises}

\subsection{Koszul duality II}
\newcommand{\antishriek}{\text{\raisebox{\depth}{\textexclamdown}}}
\begin{definition}
A quadratic operad is Koszul if one (and hence all)
of the following equivalent conditions are
satisfied:
\begin{tenumerate}
\item The cohomology group $H^s(\mathsf{B}(\PP))$ is zero for $s>0$.
\item We have $\mathrm{rate}(\PP)=1$.
\item The inclusion
$H^0(\mathsf{B}(\PP)) \longrightarrow \mathsf{B}^*(\PP)$
is a quasi-isomorphism. 
\end{tenumerate}
We call $H^0(\mathsf{B}(\PP))$ the
\emph{Koszul dual cooperad to $\PP$} and 
write it $\PP^\antishriek$.
\end{definition}

\textcolor{trinityblue}{Add equivalence with simpler
definition using twisting cochains.}
%\item The Koszul complex $\PP\circ_\tau \PP^\antishriek$ is acyclic.
%\item The Koszul complex $\PP^\antishriek\circ_\tau \PP$ is acyclic.
%\item The canonical map $\Omega \PP^\antishriek \longrightarrow \PP$
%is a quasi-isomorphism.
\subsection{The associative operad}



\subsection{Pathological examples}

Let $\mathsf{As}^-$ be the ns-quadratic operad generated
by an anti-associative binary operation $\mu$, that is
\[ \mu\circ_1\mu + \mu \circ_2 \mu = 0.\]
Let $\mathcal{O}_3$ be the ns-quadratic operad generated
by an binary operation $\mu$ that is nilpotent of 
order three, in the sense that any binary tree with
three (or more) occurrences of $\mu$ is zero:
\[ (\mu\circ_i \mu)\circ_j \mu = 0 \]
for all possible choices of $i$ and $j$. 

\begin{lemma}
The operad $\mathsf{As}^-$ is three dimensional 
with basis $\{1,\mu,\mu\circ_1\mu\}$, and $\mathcal{O}_3$
is four dimensional with basis $\{1,\mu,\mu\circ_1\mu,\mu \circ_2\mu\}.$
\end{lemma}

\begin{proof}
The second claim is immediate. For the first, one has
to show that anti-associativity implies any binary tree
of weight three vanishes. This is a simple computation:
\[ ((x_1x_2)x_3)x_4 = 
 -(x_1(x_2x_3))x_4 =
  x_1((x_2x_3)x_4) =
   -x_1(x_2(x_3x_4)) \] 
but also $((x_1x_2)x_3)x_4 = 
-(x_1x_2)(x_3x_4) =
  x_1(x_2(x_3x_4))$, 
  so that  $  x_1(x_2(x_3x_4))=- x_1(x_2(x_3x_4))$.
  Since all weight three trees in $\mathsf{As}^-$ are
  equal up to a sign, they must all be zero. 
\end{proof}

It follows that the bar construction 
$\mathsf{B}(\mathsf{As}^{-})$
is the free operad on a binary operation $\mu$ and a ternary
operation $\mu_3$, 
and that the bar construction $\mathsf{B}(\mathcal{O}_3)$
is the free operad on a binary operation $\mu$
and two ternary operations $\nu_3$ and $\tau_3$.
The differential of the first bar construction
is non-zero only on trees that contain a subtree
of the form $\mu\circ_1\mu$ or $\mu \circ_2 \mu$,
and 
 \[ d(\mu\circ_1\mu) = \mu_3, \quad
  d(\mu \circ_2 \mu) = -\mu_3, \]
  while the same is true for the second bar construction
  with the exception that
   \[ d(\mu\circ_1\mu) = \nu_3, \quad
  d(\mu \circ_2 \mu) = \tau_3.\]
  The computation of $H^*(\mathsf{As}^-)$ can
  be done, and one can show that
  \[ H_d(B(\mathsf{As}^-))_{(w)} \neq 0\] 
  only for $(d,w) \in \{ (2k+n,3k+n) : k,n\in\NN\}$. 
   It  follows that if we define 
  \[ \mathrm{rate}(\PP) = \sup \left\{ \frac{w-1}{d-1} : 
  	H_d(\mathsf{B}(\PP))_{(w)} \neq 0\right\},  \]
  	then $\mathrm{rate}(\mathsf{As}^-) = 3/2$.
Thus, $\mathsf{As}^-$ is not Koszul (as per the definition
below, which requires this `slope' to be $1$) but is quite
close to having slope $1$. In this sense, the rate of 
$\PP$ is a measure of how `dispersed' the bar homology
is with respect to the syzygy grading. 
  \begin{problem}
  Compute the bar cohomology $H^*(\mathcal{O}_3)$. 
  More generally, compute the bar cohomology of
  any monomial operad.
  \end{problem}

\section{Methods to prove Koszulness I}

\subsection{Distributive law methods}

\subsection{Koszul (co)homology}

\subsection{Exercises}

\section{Pre-Lie algebras and algebraic operads}

\subsection{Pre-Lie algebras associated to operads}

\subsection{The rooted trees operad}

\subsection{Exercises}


\section{Gr\"obner bases for algebraic operads I}

\subsection{Shuffle sequences}
\subsection{Shuffle operads}
\subsection{Exercises}

\section{Gr\"obner bases for algebraic operads II}

\subsection{Tree monomials}
\subsection{Admissible orders}
\subsection{Exercises}

\section{Methods to prove Koszulness II}

\subsection{Gr\"obner basis arguments}
\subsection{Filtration arguments}
\subsection{Exercises}

\section{Operads in algebraic topology 1}

\subsection{The little disks operad}
\subsection{Stasheff's operad}
\subsection{Recognition principles}


\section{Koszul duality II}

\subsection{Inhomogeneous duality}\label{sec:inhom}

Suppose that $\PP$ admits a quadratic linear presentation
defined by $\XX$ and $\RR\subseteq \XX\oplus \FF_\XX^{(2)}$.
There is a projection $q : \FF_\XX \longrightarrow \FF_\XX^{(2)}$
and we define $q\PP = \FF_{\XX}/(q\RR)$, and call it 
the \emph{quadratic operad associated to $\PP$}. We say
a quadratic-linear presentation is admissible if it
satisfies the following conditions:
\begin{tenumerate}
\item There are no superfluous generators in $\XX$,
that is, have that $\RR\cap \XX = 0$.
\item No new quadratic relations can be deduced from the
quadratic-linear relations, that is
\[ (\RR\circ_{(1)} \XX + \XX\circ_{(1)} \RR) \cap \FF_\XX^{(2)}
\subseteq \RR\cap \FF_\XX^{(2)}.\]
\end{tenumerate}
When condition (1) is satisfied, there is a map 
$f: q\RR \longrightarrow \XX$ such that $\RR = \{
r - f(r) : r\in q\RR \}$ is the graph of $f$. The operad
$\PP$ is filtered by weight, and we write $\mathrm{gr}(\PP)$
for the resulting operad. There is a surjection
$ q\PP \longrightarrow \mathrm{gr}(\PP)$
that is an isomorphism in weights $0$ and $1$, but
not necessarily in weight $2$.

The map $f: q\RR \longrightarrow \XX$ induces a map
$d_f : q\PP^\antishriek \longrightarrow \FF_{s\XX}^c$
which is the unique coderivation that correstricts on
$s\XX$ to the composition
\[ \PP^\antishriek  \xrightarrow{\pi} 
 s^2 q\RR \xrightarrow{s^{-1}f} s\XX.\] 
 Moreover, the following holds:
 \begin{tenumerate}
 \item The coderivation $d_f$ maps into $q\PP^\antishriek$
 if and only if 
 $(\RR\circ_{(1)} \XX + \XX\circ_{(1)} \RR) \cap \FF_\XX^{(2)} \subseteq q\RR$.
 \item If condition (2) above is satisfied, then the previous
 condition holds, as $\RR\cap \FF_\XX^{(2)}\subseteq q\RR$,
 and $d_f^2=0$.
 \end{tenumerate}
 
\begin{definition}
A quadratic-linear presentation of $\PP$ is inhomogeneous
Koszul if and only if it satisfies conditions (1) and (2)
and if the quadratic operad $q\PP$ is Koszul. In this case,
we call $(q\PP^\antishriek,d_f)$ the Koszul dual conilpotent
dg-cooperad
of $\PP$ and write it $\PP^\antishriek$.
\end{definition}

 \textbf{Warning!} Although not every operad 
is quadratic, every operad admits
a inhomogeneous Koszul presentation (Exercise 3.8.10
in \cite{LodVal}).  The problem is finding `economical
and useful' one.
Concretely, if we choose $\XX = \#\PP$ (the symmetric 
sequence underlying $\#\PP$) and 
quadratic-linear relations $\# \mu \circ_i \#\nu = \#(\mu\circ_i \nu)$ for every $\mu,\nu\in\PP$ and $i\in [1,\ari(\mu)]$,
then $\PP^\antishriek = \mathsf{B}(\PP)$ is the
bar construction of $\PP$ with its usual differential.

The main
theorem about inhomogeneous Koszul operads is the
following:

\begin{theorem}
If $\PP$ admits an inhomogenous Koszul presentation then
the canonical morphism
 $\Omega \PP^\antishriek \longrightarrow \PP$
determined by $s^{-1}\PP^\antishriek \twoheadrightarrow
\XX \hookrightarrow \PP$ is a quasi-isomorphism of
operads.
\end{theorem}

In general, one can show that the map $\Omega\mathsf{B}(\PP)
\longrightarrow \PP$ is a quasi-isomorphism: this is the
original approach of Ginzburg--Kapranov) who instead
define a duality functor $\mathsf{D}(\PP)$ along with a
quasi-isomorphism
$\mathsf{D}\mathsf{D}(\PP) \longrightarrow \PP$,
and define the Koszul dual operad to $\PP$ as
$H_\Delta(\mathsf{D}(\PP))$. The
theorem above gives us a more economical resolution of
$\PP$ in case it is quadratic-linear Koszul. Note that
the bar-cobar construction is, more or less, obtained
as the `least economical' resolution arising from the
corresponding `least economical' quadratic-linear 
presentation of an operad $\PP$. 

\section{Prototype of inhomogeneous duality}
\textbf{Goals.}
We will define the operad governing BV algebras 
and show it admits a small inhomogeneous Koszul 
presentation. We will compute the homology of the 
corresponding dg-cooperad, and
explain how it gives rise to the Gravity operad
of E. Getzler. At the same time, we will explain
how the homotopy quotient of $\mathsf{BV}$ by
the circle action is the hypercommutative
operad of Yu. I. Manin.
\subsection{Definition and computations}
\newcommand{\BV}{\mathsf{BV}}
The BV operad is an algebraic symmetric operad, which we
write $\BV$, generated by a binary commutative associative
operation $\mu$ that we will write $x_1x_2$ 
of degree zero and a unary square-zero operation
$\Delta$ of degree $-1$ that satisfy the following 
homogeneous $7$-term
relation:
\[ 
 \Delta(x_1x_2x_3) = x_1\Delta(x_2x_3)+
 x_2\Delta(x_1x_3) + x_3\Delta(x_1x_2)
  - x_1x_2\Delta(x_3) - x_1x_3\Delta(x_2)
   	- x_2x_3\Delta(x_1).\]
Batalin--Vilkovisky algebras appear in several
areas of mathematics:
\begin{tenumerate}
\item (Algebra) Vertex operator algebras, cohomology of Lie algebras,
bar construction of $A_\infty$-algebras.
\item (Algebraic geometry) Gromov--Witten invariants and moduli spaces of curves (quantum cohomology, Frobenius manifolds), chiral algebras (geometric Langlands program),
\item (Differential geometry) The sheaf of 
polyvector fields of an orientable (resp. Poisson or
Calabi--Yau) manifold, the differential 
forms of a manifold (Hodge decomposition 
in the Riemannian case), Lie algebroids,
Lagrangian (resp. coisotropic) intersections.
\item (Noncommutative geometry) 
The Hoschchild cohomology of a symmetric 
algebra  and the cyclic Deligne conjecture,
non-commutative differential operators.
\item (Algebraic topology)
2-fold loop spaces on topological spaces 
carrying an action of the circle,
topological conformal field theories, 
Riemann surfaces, string topology. 
\item  (Mathematical physics)
BV quantization (gauge theory), BRST
cohomology, string theory, topological field theory, Renormalization theory.

\end{tenumerate}

 One can express the seven term relation by saying
 that $[\Delta,\mu] = \beta$ is a derivation
 for $\mu$, where $[f,g]$
 is the operadic commutator (\`a la Gerstenhaber)
 defined by
 \[ [f,g] = \sum_{i=1}^{\ari(f)} f\circ_i g 
 	  -(-1)^{|f||g|} \sum_{j=1}^{\ari(g)} g\circ_j f. \]
 This suggests defining $\beta = [\Delta,m]$, and presenting
 the BV-operad by quadratic-linear relations:
 \[ [\mu,\mu] = 0 ,\quad \Delta^2 =0 , \quad
  	[\Delta,\mu] = \beta, \quad
  	\beta\circ_1 \mu = \mu \circ_2 \beta +
  	 	\mu\circ_1 \beta (23).\]
  This presentation satisfies condition (1), but it
  \emph{does not} satisfy condition (2): one can
  deduce that $\beta$ is a Lie bracket of degree $-1$
 and that $\Delta$ is a derivation for $\beta$
  from the first three equations. In other words, one
  can deduce that $(\Delta,\beta)$ defines the
  datum of a dg Lie algebra purely from the
  fact that $\Delta^2=0$ and that $\mu$ is associative. 
  
  \begin{lemma}
  The BV-operad admits a quadratic-linear presentation
  satisfying conditions (1) and (2) 
  given by generators $\mu,\beta,\Delta$ of arities
  $2$, $2$ and $1$ and degrees $0$, $-1$ and $-1$,
  respectively. The operation $\mu$ is associative 
  commutative, $\beta$ is a Lie bracket, $\Delta$
  squares to zero, and 
 \[	\mathsf{Leib}(\Delta,\mu) = \beta, \quad
  	\mathsf{Leib}(\beta,\mu)=0, \quad
  	\mathsf{Leib}(\Delta,\beta) = 0.\]
  	  \end{lemma}
  	  
  	  \begin{theorem}
  	  The quadratic operad $q\BV$ is Koszul.
  	  \end{theorem}
  	  
  	  \begin{proof}
  	  We will use the distributive law criterion of
  	  Markl, adapted to the case the operads in his
  	  result have unary operators. One can first
  	  show that $\BV(n)$ and its
  	  quadratic counterpart both have dimension $2^n n!$
  	  by a Gr\"obner basis argument,
  	  and then that $q\BV$ is obtained from
  	  a distributive law between the quadratic 
  	  operads $\mathsf{Ger}$ and 
  	  $\mathsf{D} = \kk[\Delta]/(\Delta^2)$, in light of the
  	  relations
  	  \[\Delta(x_1x_2) = x_1\Delta(x_2) + \Delta(x_1)x_2, 
  	  \quad 
  	   \Delta[x_1,x_2] = [x_1,\Delta x_2] + [\Delta x_1,x_2]. \]
  	  One can prove this again by a dimension counting argument
  	  using the result above. With this at hand, we
  	  observe that $\mathsf{Ger}$ is Koszul, as it is
  	  in turn obtained from a distributive law between
  	  $\mathsf{Com}$ and $\sus\mathsf{Lie}$, which are
  	  both Koszul, and that $\mathsf{D}$ is
  	  Koszul (as any algebra with trivial multiplications
  	  is). We conclude that $q\BV$ is Koszul, and that we
  	  have isomorphisms of symmetric sequences
  	  \[ q\BV \cong \Com \circ \sus\Lie\circ \mathsf{D},
  	    	  \quad q\BV^\antishriek \cong
  	    	  T^c(\delta) \circ \Com^c \circ \sus^{-1}\Lie^c,\]
  	  which will be useful to describe the dg-cooperad
  	  $\BV^{\antishriek}$.
  	  \end{proof}

\subsection{The differential}  	  
  	 Let us note that a generic element
  	 of $\Com^c \circ \sus^{-1}\Lie^c$ consists of a
  	 corolla decorated by Lie words on a Lie bracket
  	 of degree $1$. Any Lie word can always be
  	 written uniquely as a linear combination of
  	 words in the form
  	 $\ell = [x_1, x_{\sigma 2} , \ldots , x_{\sigma n} ]
     $
     where $\sigma\in S_n$ fixes $1$, and we adopt the
     right bracketing convention:
     \[ [y_1,y_2,\ldots,y_n ] = [y_1,[y_2,\ldots,y_n]].\]
     We call $x_1$ the `head' of $\ell$. 
     We will then write an element of $\Com^c \circ \sus^{-1}\Lie^c$ generically by
  	 \[ \ell_1\odot \cdots  \odot \ell_n \]
	 where $\ell_i$ is a Lie word supported on $\pi_i$,
	 with `head' $x_j$ with $j_i = \min\pi_i$ and such that
	 $\min \pi_1 < \cdots < \min \pi_n$. 
	 	 
	 \begin{theorem}
	 A generic element of $q\BV^\antishriek$ is of the form
	     \[x =\delta^k\otimes \ell_1\odot \cdots  \odot \ell_n \]
	  and the differential $d$ of $q\BV^\antishriek$ is
	  \[ dx = \sum_{i=1}^n(-1)^{\varepsilon_i} \delta^{k-1} \otimes \ell_1\odot
	   \cdots \odot \ell_i^{(1)} \odot \ell_i^{(2)}
	    \odot \cdots \odot \ell_n,
	  	 \]	  	 
	  	 where $\ell \longmapsto 
	  	 	\sum \ell^{(1)}\otimes \ell^{(2)}$
	  	 is the binary component of the decomposition map in
	  	 $\sus^{-1}\Lie^c$.	  	 
	  	 \end{theorem}
  	  
%The operad Gerstenhaber, Gravity, Hypercomm, PreLie,
%Perm, Zinbiel, Dias, Leib, NAP (Livernet) and the 
%Connes--Kreimer Hopf algebra of
%renormalization. 
\subsection{Exercises}

%\section{Further contents}

%\subsection{Iterated integrals}
%$C^*_{\mathrm{dR}}(\Omega X)$, Chen iterated integrals,
%work of Getzler--Jones for $\Omega^2 X$. 

\bibliographystyle{alpha}
\bibliography{biblio}


\Addresses

\end{document}

	  	 
	  	
\subsection{The gravity operad and the hypercommutative operad}
  	  
  	 There is a unique square zero derivation $\Delta$ 
	 of $\mathsf{Ger}$
	 of degree $-1$ that sends the commutative
	 product $x_1x_2$ to the Lie bracket $[x_1,x_2]$,
	 which arises from the action of $S^1$ on $D_2$
	 by turning the whole configuration.
	 This derivation is acyclic, and $\mathsf{Grav}$
	 is, by definition, $\ker\Delta   =
	 \operatorname{im}\Delta  $.	 
	 	 
	 \begin{lemma}
	 Let $M$ be an $S^1$-space with a free action, and
	 let $\Delta : H_*(M) \longrightarrow H_{*+1}(M)$
	 be the square zero operator arising from the
	 fundamental class of $S^1$. Then
	 \[ \ker\Delta \cong \Sigma H_*(M/S^1) . \]
	 \end{lemma} 

	 The action of $S^1$ on little disks is free
	 (except in arity one) 
	 with quotient homotopy equivalent to the moduli
	 space of marked curves of genus zero, so 
	 	 	  \[ \ker\Delta =  
	 	 	  \Sigma H_*(D_2/S^1) = \Sigma H_*(\mathcal{M}_{0,\bullet+1}). \]
	 
	 \begin{theorem}
	 One can realize $\mathsf{Grav}$ as the suboperad
	 of $\mathsf{Ger}$ generated by
	 \[ \{x_1,\ldots,x_n\} =  
	 \sum_{i<j} \{x_i,x_j\}
	  x_1\cdots \widehat{x_i}\cdots \widehat{x_j} \cdots x_n, 
	  	\quad n\geqslant 2.\]
	 Similarly, one can realize it as the suboperad of	 
	 $\mathsf{BV}$ generated by 
	 \[ \{x_1,\ldots,x_n\} = 
	  \Delta(x_1\cdots x_n) - \sum_{i=1}^n x_1\cdots \Delta x_i
	   	\cdots x_n, \quad n\geqslant 2.\]
	   	We have that $\Sigma H_*(\mathcal{M}_{0,n+1}) \cong
	   	\mathsf{Grav}(n)$ for $n\geqslant 2$ where the
	   	operation $\{x_1,\ldots,x_n\}$ corresponds to the
	   	suspension of the generator of 
	   	$H_0(\mathcal{M}_{0,n+1})$.
	  \end{theorem}
	  
	 \begin{theorem}
	 The homology of the dg-cooperad 
	 $\mathsf{BV}^\antishriek$ (and hence
	 the bar homology of $\mathsf{BV}$) is
	 isomorphic to
	 \[ T(\delta) \oplus \sus^{-1}\mathsf{Grav}^*,\]
	 and the  
	 homotopy quotient of $\mathsf{BV}$ by
	 $\Delta$ is quasi-isomorphic to $\mathsf{Hycomm}$,
	 the Koszul dual of $\mathsf{Grav}$. These
	 two last operads are Koszul.
	 \end{theorem}



\subsection{The homotopy quotient}

\begin{definition} 
Let $f:\PP\to \mathcal Q$ be a morphism of operads, 
and factor $f$ into a cofibration
$i: \PP\longrightarrow \FF$ followed
by a trivial fibration $j:\FF\longrightarrow \QQ$.
We define the homotopy cofibre $C_f$ of $f$
as the quotient of $\FF$ by the ideal generated
by $\PP$. We write $\PP\sslash \QQ$ for
the homology of $C_f$.
\end{definition}

Note that $\PP\sslash \mathcal Q$ is independent 
of the choice of factorization of $f$.
Usually we can take $\FF$ of the form 
$(\PP \star \FF_\XX,d)$ where $\star$ is the
coproduct in the category of operads, 
so $C_f$ is
isomorphic to $(\FF_\XX,\bar{d})$. 
If $\alpha\in\PP$, we define the homotopy
quotient of $\PP$ as the homotopy cofibre of
the inclusion
$i : \PP_\alpha \longrightarrow \PP$
where $\PP_\alpha$ is the suboperad of $\PP$ 
generated by $\alpha$, and write it more
simply by $\PP\sslash \alpha$.  
In this case, to compute $\PP\sslash\alpha$, it suffices we compute a quasi-free
model $\FF = (\FF_\XX,d)$ of $\PP$ ---which in
particular we can assume contains an 
isomorphic copy of $\FF_\alpha$, and compute
the homology of the quotient $\FF/ (\FF_\alpha).$

This algebraic homotopy quotient is related
to the more geometrical homotopy quotient
with respect to the action of a topological
group $G$. Concretely, the exact functor 
$ t:\mathsf{Top} \longrightarrow \mathsf{Top}_G$
from the category of topological spaces to 
the category of topological spaces with a 
$G$-action, that sends a space $X$ to the
same space $tX$ with the trivial action, admits
a (non-Quillen-exact) left adjoint $F$ and
the homotopy quotient is obtained
as the left derived functor $\mathbb LF$ of
$F$, which we usually write $X\longmapsto X_{hG}$. At the level of homotopy categories, it
follows that we have an adjunction isomorphism
$[X_{hG},Y] \longrightarrow [X,tY]$.

Naturally, we may pass from the geometric to
the algebraic setting through the cohomology
functor, by considering the sphere $S^1$ whose
cohomology algebra coincides with 
$\Bbbk[\Delta]$. In this way, we can consider
the category $\mathsf{Op}_\Delta$ of dg-operads
under $\Bbbk[\Delta]$, and the exact functor
$t:\mathsf{Op} \longrightarrow \mathsf{Op}_\Delta$
that assigns an operad $\PP$ to the trivial
map $\Bbbk[\Delta]\to \PP$ sending $\Delta$ to $0$. The homotopy quotient functor $\PP\longmapsto \PP\sslash \Delta$ is the left adjoint, at the level of homotopy categories,
to $t$: we have an adjunction isomorphism
$[ \PP\sslash \Delta, \mathcal Q]
 	\longrightarrow [ \PP,t\mathcal Q]$. 
 	  