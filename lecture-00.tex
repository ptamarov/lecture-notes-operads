

\section{Motivation and history}\label{lecture:theintro}
 
\textbf{Goals.} The goal of this lecture is to
give a broad picture of the history and 
pre-history of operads, and some current trends,
and give a road-map for the course.  
  
Operads (topological operads, more precisely)
originally appeared as tools in algebraic
topology and homotopy theory, 
specifically in the study of iterated loop 
spaces (May, 1972 and Boardman and Vogt before).
They also appeared as \emph{comp algebras} in Gerstenhaber's 
work on Hochschild cohomology and topologically as 
Stasheff's `associahedra' for his homotopy
characterization of loop spaces (both in 1963).
The theory of operads, in particular topological
and algebraic, saw itself very much influenced by
homological algebra, category theory, algebraic
geometry, rational homotopy theory and mathematical
phyisics. Here we list a few examples:

\begin{tenumerate}
\item Stasheff~\cite{Stasheff1970,Stasheff1963,Stasheff1963a} and Sugawara~\cite{Sugawara55}
studied homotopy associative 
$H$-spaces, Stasheff implicitly discovers a topological
non-symmetric operad $K$ and a
recognition principle for infinite deloopings:
a connected topological space is an infinite
loop space if and only if it is acted upon
by this topological operad $K$. 

\item Boardmann--Vogt~\cite{Boardman1973,bams/1183530111} studied infinite loop spaces,
built a PROP that provides a version of an $E_\infty$-operad, 
and obtained a recognition principle for infinite loop spaces.

\item Kontsevich~\cite{Kontsevich2003} used $L_\infty$-algebras and
configuration spaces to prove his deformation
quantization theorem that every Poisson manifold
admits a deformation quantization.

\item The above is
implied by Kontsevich's the formality theorem: the
Lie algebra of polyvector fields is 
$L_\infty$-quasi-isomorphic to the Hochschild
complex, and $f_1 = \mathsf{HKR}$. 

\item Tamarkin~\cite{Tamarkin2003,Hinich2003} approached this result through 
the formality of the little disks operad $D_2$.
Proves that the Hochschild complex of a polynomial
algebra is \emph{intrinsically formal} as a 
Gerstenhaber algebra.

\item Manifold calculus describes the
homotopy type of embedding spaces as certain 
derived operadic module maps and to
produces their explicit deloopings
using little disk operads, due to 
many authors~\cite{Arone2007,Goodwillie1999,
Weiss1996,Weiss1999}. We point the reader
to~\cite{IdrissiPeccot} for a comprehensive
list of references.

\item Koszul duality~\cite{Ginzburg1994}
for algebraic 
operads and cousins allows to develop a robust 
homotopy theory of homotopy 
algebras~\cite{Vallette2020,Hinich1999}, cohomology
theory, deformation theory, Quillen homology, etc. Rational homotopy theory~\cite{Quillen1967,Felix2001,Sullivan1977}
motivated and heralded the development of the homotopy theory of and Koszul duality
theory for algebraic operads.

\item The study
of natural operations on the Hochschild complex
of an associative algebra~\cite{Hochschild1945} lead to a manifold of 
results beginning with the proof that there is
an action of the little disks operad $D_2$ on
it (Deligne's conjecture), and the ultimate version by Batanin--Markl~\cite{Batanin2014},
who proved that the operad of natural operations
on it has the homotopy type of $C_*(D_2)$.  
\end{tenumerate}

Operads are modelled by trees: planar or
non-planar, rooted or not, there exists a monad
of trees that defines them. This idea
 in fact extends
to more complicated situations, like that of
properads, dioperads or modular operads,
in the sense that relaxing these
graphs to allow for more structure 
produces other type of algebraic structures
of interest. The following table
gives the reader a ``taxonomy cheat sheet''
for operads and their kin; we will, for better
or worse, defer from diving into the curious
world that lies beyond operads, but encourage
the reader to do this for themselves (and find
out what ``wheeled structures'' are, and how
they fit in the table below).

\begin{center}
\begin{tabular}{@{}llccr@{}} \toprule
Type & Graph & Compositions & Due to & Ref.\\ \midrule
PROPs & Any graph & Any  & Adams--MacLane \\
Modular & Any graph & $\xi_{i,j}$, $\circ_{i,j}$  & Getzler--Kapranov & \cite{Getzler1998}\\ 
Properads & Connected graphs & Any & B. Vallette & \cite{Vallette2004}\\
Dioperads & Trees & ${}_i\circ_j$ & W. L. Gan & \cite{Gan2003}\\
Half-PROPs & Trees & $\circ_j$ , ${}_i\circ$ & 
 Kontsevich\footnotemark & \cite{Markl2007}\\ 
Cyclic operads & Trees &  $\circ_{i,j}$ & Getzler--Kapranov & \cite{Kajiura2008}\\ 
Symmetric operads & Rooted trees & $\circ_i$  & J. P. May & \cite{May1972}\\ 
\bottomrule
\end{tabular}
\end{center}
\footnotetext{The reference explains that ``These results are based on the useful notion of a 1/2 PROP introduced by Kontsevich in an e-mail message to the first author.''}
\subsection{Koszul duality}
 
Koszul duality was invented by Steward Priddy in
the seventies~\cite{Priddy1970}, with the objective of streamlining
computations of certain cohomology theories for
classes of algebras (notably, Lie and associative 
algebras). One of the reasons this was (and still is)
relevant is that such cohomology groups play a central
role in the computation of other more complicated invariants
of algebras and topological spaces: in particular,
the cohomology of the Steenrod algebra famously featured
in Adam's spectral sequence computing the stable homotopy
groups of spheres at each prime. In Priddy's own words:
\[
\claim{
The purpose of this paper is to construct resolutions for a 
large class of algebras which includes the Steenrod algebra
and the universal enveloping algebras. 
It is a basic problem of homological algebra to compute the 
cohomology algebras of various augmented algebras. Unfortunately, 
the canonical tool for attacking this problem ---the bar resolution--- is often intractable. In some instances,
however, one is able to find a simpler resolution.
}
\]
Priddy developed his theory for both ``inhomogeneous''
and ``homogeneous'' quadratic algebras ---those presented
in coordinates by quadratic equations in their variables---
and, while in the homogeneous case his formalism gave the
answer immediately, the inhomogenous case required an
additional step, which nonetheless simplified the existing
methods considerably.  

Although Koszul duality nowadays has a much broader meaning
and casts an immense net in modern day algebra, representation
theory, combinatorics, topology and geometry, 
among other areas of mathematics, in this lecture
series we will follow Priddy's motivation and see it as an
instance of a phenomenon in which certain algebraic objects
have very economical ---and thus computationally and theoretically
useful--- resolutions. An interested reader can 
consult~\cites{KellerKoszul2003,Positselski2011,
Sinha2010,MO329,holstein2021categorical} to obtain a broader
view of this phenomenon, and in particular find a wide
variety of answers to the question ``...but what \emph{exactly}
is Koszul duality?''.  
%
%With the risk of diverging 
%from our initial definition, a ``Koszul duality phenomenon'' 
%comes in the shape of a Quillen equivalence between 
%two model categories of algebraic objects of some kind,
%which allows to consider ``object-wise Koszul duality phenomenona''.
%The latter then induce equivalences between categories of
%``representations'' of such objects.

Naturally, one of the reasons why Koszul duality has cemented
itself in modern day mathematics is that it appears often:
algebraic structures of interest have an inclination to be
quadratic and, when in luck, Koszul. These can be anything
from Lie, commutative or associative algebras, to Feynmann
categories, dg categories, operads and their kin. In this
lecture series, we will focus on algebraic operads: our goal
is to introduce the reader to these objects in general
and to quadratic operads in particular, define what
it means for such operads to be Koszul, and explore
the consequences this property has on the operads
and its representations. We do this in Lectures~\ref{lecture:quadraticops}, \ref{lecture:KD1}, \ref{lecture:barcobar} and \ref{lecture:KD2}, and include further material
in the Appendix.

Although, as we
mentioned, we will take a rather old fashioned point of 
view and think of Koszul operads as those operads 
having a ``nice resolution'', we aim to give the reader
a modern outlook on the current methods available to
prove that an operad is Koszul, and some relatively 
new developments in the area from the last two (or
maybe three) decades:
the inhomogenous Koszul duality for (pr)operads due
to Galvez-Carillo--Tonks--Vallette, which
followed the original theory of 
Ginzburg--Kapranov~\cite{Ginzburg1994},
the use of filtered distributive laws 
of Dotsenko~\cite{Bremner2016}*{Section 6.3.5}
which followed the methods of Markl~\cite{MarklDistributive}, 
and the general theory stating Koszul operads give rise to good
notions of algebras up to homotopy, due to Vallette~\cite{Vallette2020},
following Hinich~\cite{Hinich2001} and Lefèvre-Hasegawa~\cite{LH03}.
Naturally, we will also focus on the classical developments,
and on the effective methods of Hoffbeck~\cite{Hoffbeck2009} and 
Dotsenko--Khoroshkin~\cite{Dotsenko2010}, which we mention again in
the next section.

\subsection{Gr\"obner bases}
Many algebraic structures that exhibit some kind of 
associativity can be described by making the following check-list:
they have a notion \emph{variable} (or alphabet) and a notion 
of \emph{monomial} in the given alphabet. In turn, they have
a notion of \emph{polynomial} and a notion of \emph{free algebra}:
its basis is given by such monomials.
At the same time, these structures have a notion of \emph{ideal} and
\emph{quotient algebra}, and hence of a \emph{presentation} of an object of that
type by \emph{generators and relations}.
At this point, a natural question arises, which was 
famously raised by Wolfgang Gr\"obner around 1964 to his student Bruno
Buchberger: \emph{How can one find a basis of a free algebra by
a set of relations?} 

In his PhD thesis, Buchberger developed the tool-kit to 
answer this question in a streamlined fashion for the case
of commutative (associative) algebras: the first step is
to have a notion of an admissible monomial order, that
effectively allows to do linear algebra on the free algebra.
With this at hand, one develops a notion of divisibility
and of leading terms of polynomials: with this, one can
perform long division and define normal forms. 
In tandem, one can define what a 
Gr\"obnber basis of an ideal is and, finally, long division 
can be used to build Bucherberger's algorithm, that allows
us to complete generators of ideals to Gr\"obner bases.
With this at hand, the membership problem is solved:
an element belong to an ideal if and only if the result
of long division of it using a Gr\"obner basis is zero,
and the normal forms with respect to a Gr\"obner basis give
a linear basis of the quotient algebra.

At this point, as in the following excerpt from~\cite{Bremner2016},
one has to make the key observation that this tool-kit works,
\emph{mutatis-mutandis}, for many different algebraic structures
exhibiting some kind of associativity, and not only for
those which Gr\"obner and Buchberger were originally concerned with:
\[\claim{
Both testing hypotheses and proving theorems about polynomial expressions of all those types often involves highly complex symbolic computations which can never be completed in a reasonable time unless one approaches them in an extremely structured way. [...] Our goal in this book is to present the solution 
to the problem of determining normal forms in a way that all the individual building blocks 
of that solution are clearly identified; this makes desired 
generalizations of the theory straightforward. We give complete proofs of key 
facts, many detailed examples, a large array of exercises, mostly coming from 
actual research questions, and references to further reading.}
\]

Lectures~\ref{lecture:shuffleops}-\ref{lecture:GB2} in these
notes are aimed at studying the analogue of the usual
theory of Gr\"obnber bases (for commutative algebras) as it was
developed for algebraic operads by Dotsenko and Khoroshkin, 
following the breakthrough paper of Hoffbeck~\cite{Hoffbeck2009}
giving a ``Poincar\'e--Birkhoff--Witt criterion'' for an operad
to be Koszul. We will not deviate too much from the exposition of the 
book~\cite{Bremner2016}, although we will incorporate to our
exposition new developments that happened after that book
was written, such as examples where we compute
Gr\"obner bases and count normal forms 
with the \textsf{Haskell Operad Calculator}
developed by Dotsenko and Heijltjes~\cite{Calculator}, along with new
monomial orders coming from~\cite{Dotsenko2020},
one of which we have implemented (along with F. Lebr\'on)
to this \textsf{Haskell} program. 

\subsection{Effective methods}

Checking whether an algebraic operad is
Koszul is potentially a complicated task,
and in Lectures~\ref{lecture:methods1} 
and~\ref{lecture:methods2}, we give the reader
a few methods to prove that an operad is Koszul,
hoping to convince them that the use of shuffle
operads and Gr\"obner basis theory is one of 
the most effective and streamlined ways to 
attack the problem. Naturally, not every
Koszul shuffle operad admits a quadratic
Gr\"obnber basis, so one should be judicious
when using this method. We do not consider,
for example, the partition poset criterion
of Vallette~\cite{Vallette2007}, but encourage
the reader to learn about it.

We begin Lecture~\ref{lecture:methods1} 
by considering quadratic monomial operads and 
showing they are Koszul, and follow up with 
the numerical criterion
of Ginzburg--Kapranov~\cite{Ginzburg1994}
to test if an operad is \emph{not} Koszul; we
provide two examples where the criterion detects
that an operad is not Koszul and
use the \textsf{Haskell Operad
Calculator} to replicate the results of Dzhumadil'daev
stating that the Novikov operad is not 
Koszul~\cite{Dzhumadildaev2009}.
In Lecture~\ref{lecture:methods2}, 
we consider the situation where one
can explicitly compute the homology of the
Koszul complex of an algebraic operad,
as in the case of the associative operad.
We also consider a recent computation
of the homology of a Koszul complex 
carried out by Dotsenko--Flynn-Connolly~\cite{Dotsenko2020Schur}
through filtration methods; although it is
not computing the Koszul homology of an operad,
in can be interpreted as computing a
relative version of it, which is very useful
for applications~\cite{DotsenkoTamaroff2020}. 
After this, we explain how to show, in the
spirit Hoffbeck's original article~\cite{Hoffbeck2009},
that a shuffle operad with a quadratic
Gr\"obner basis is Koszul; this result
single handedly deals with almost all 
quadratic algebraic operads we will introduce
in the lectures, with only a few exceptions
that the Ginzburg--Kapranov criterion can
discard. Along the way, we introduce filtration
methods that give the more general result that
a filtered quadratic operad whose associated
graded operad is Koszul is itself Koszul.
Finally, we survey the distributive law methods
of Dotsenko~\cite{Dotsenko2007}, Markl~\cite{MarklDistributive}
and Vallette~\cite{Vallette2004}, where one
has available a ``Diamond Lemma'' criterion,
which one can again effectively check using
rewriting methods. 

\newpage

\subsection{Exercises}
 
\textbf{A. Symmetric groups.} Operads are meant to encode
operations on objects \emph{along with their symmetries},
which is done through the representation theory of the
symmetric groups. The following exercises will remind
you of some basic facts about them.


\begin{question} Let $I = [n]$ so that $\Aut(I) = S_n$ is 
the symmetric group on $n$ letters. For each ordered 
partition $\pi = (\pi_1,\ldots,\pi_k)$, let $\lambda$
be the ordered partition of $n$ with $\lambda_i = \# \pi_i$
for $i\in \underline{k}$, where
we write $\underline{k} = \{1,\ldots,k\}$. Show that the permutations of
$\underline{n}$ that preserve $\pi$ determine a subgroup of $S_n$
isomorphic to $S_\lambda := S_{\lambda_1}\times
\cdots \times S_{\lambda_k}$. 
\end{question}


\begin{question} Consider the subgroup of $S_n$ corresponding
to the ordered partition of $\underline{n}$ given by $([1,k],[k+1,n])$, along with the 
inclusion $S_k\times S_{n-k} \hookrightarrow S_n$. Show that
a set of representatives for the cosets of this inclusion
in $S_n$ is given by the \emph{$(k,n-k)$-shuffles}, those
permutations $\sigma\in S_n$ that preserve the linear order in
$[1,k]$ and $[k+1,n]$. Conclude that there are exactly
$\binom nk$ shuffles of type $(k,n-k)$ on $\underline{n}$. Define
shuffles associated to other partitions of $n$.
\end{question}

\medskip
\textbf{B. Categories.} The language of categories and
functors permeates most of modern algebra and geometry,
and in particular is useful to work with operads and
other combinatorial structures defines by graphs. The
following will remind you of some important notions
we will use during the course.

\begin{question} A category $\mathcal C$ is the datum of
a set of objects $\operatorname{Ob}(\mathcal{C})$, and
for each $x,y\in \operatorname{Ob}(\mathcal{C})$ a
set $\mathcal C(x,y)$ of morphisms from $x$ to $y$.
Moreover, we require the existence of an associative
and unital composition law
\[-\circ -: \mathcal C(y,z) \times  \mathcal C(x,y) 
	\longrightarrow \mathcal C(x,z). \] 
	The latter means there are distinguished elements
	$1_x\in \mathcal{C}(x,x)$ for each object of $\mathcal{C}$
	that induce the identity $-\circ 1_x$ and $1_x\circ -$
	of any $\mathcal{C}(-,x)$ and $\mathcal{C}(x,-)$.
	Find examples of categories: sets, finite sets,
	rings, vector spaces, open subsets, posets, and 
	others.
\end{question}

\begin{question} A functor $F: \mathcal{C}_1\longrightarrow 
\mathcal{C}_2$ is a datum that assigns to each object
$x$ of the domain an object $F(x)$ of the codomain,
and to each morphism $f:x\to y$ a morphism $F(f)$ such
that $F(f\circ g) = F(f) \circ F(g)$ and $F(1_x) = 1_{Fx}$
for each pair of composable arrows $f$ and $g$ and each
object $x$ of $\mathcal C_1$. Find examples of functors
between the examples of categories you found above.

\end{question}
\begin{question} A monoidal category is a category $\mathcal C$ 
along with the datum of a bifunctor $\otimes :
\mathcal{C}\times \mathcal{C}\longrightarrow \mathcal{C}$
along with an associator and left and right units. 
A monoidal category is \emph{strict} if the associator and left
and right units are identities.
\begin{enumerate}
\item Expand on the details of these definitions. Define what
a braided monoidal category and what a symmetric monoidal category are.
\item Exhibit monoidal
structures the following categories: sets, vector spaces,
linear representations of a group $G$, topological spaces,
associative algebras, Lie algebras, and others.
\end{enumerate}
 \emph{Hint.} In the case of Lie algebras,
 consider the category of Lie groups with its
 canonical tensor product (the cartesian product) 
 and the functor $G\longmapsto T_e(G)$ to decide
 what the tensor product of two Lie algebras is.
\end{question}


\begin{question}
If $(\mathcal{V},\otimes)$ is a monoidal category, 
we say a category $\mathcal{C}$ 
is $\mathcal V$-enriched if each hom-set 
$\mathcal{C}(x,y)$ is an object of $\mathcal{V}$ and
there is a composition law 
\[-\circ -: \mathcal C(y,z) \otimes  \mathcal C(x,y) 
	\longrightarrow \mathcal C(x,z). \] 
which consists of morphisms in $\mathcal{V}$, and which is 
associative and unital. Note that an ordinary category is
just a category enriched over the category of sets.
A linear category is a category enriched over the
category of vector spaces, an additive category is a
category enriched over Abelian groups.
Expand on what this means. Find about Abelian
categories, and ponder over the difference: an additive
category is a category with structure, while an
Abelian category is a category with additional properties.
See~\cite{Kassel1995} for more on monoidal categories in
general.
\end{question}

\begin{question}\label{ex:skeleton}
A category $\mathcal D$ is skeletal if no two distinct
objects in it are isomorphic. We say that $\mathcal{D}$ is
the skeleton of a category $\mathcal{C}$ if:
\begin{tenumerate}
\item It is a full subcategory of $\mathcal{C}$: for each
pair of objects $x,y\in\mathcal{D}$, we have that $\mathcal{D}(x,y) = \mathcal{C}(x,y)$.
\item The inclusion of $\mathcal{D}$ in $\mathcal{C}$ is
essentially surjective: every object of $\mathcal{C}$ is
isomorphic to an object of $\mathcal{D}$.
\item $\mathcal{D}$ is skeletal.
\end{tenumerate}
Show that every small category admits a skeleton, and
compute the skeleton of the following categories: sets,
finite sets, finite dimensional vector spaces over
a field.
\end{question}
\medskip

\begin{question}
Suppose that $x$ is an object in a symmetric monoidal
category $(\mathcal{C},\tau)$. For each $n\in\NN$ and 
each $i\in \underline{n-1}$ define $\tau_i : x^{\otimes n} 
\longrightarrow x^{\otimes n}$ by
\[ \tau_i = 1^{i-1} \otimes \tau \otimes 1^{n-i-1}.\]
Show that the assignment
$(i,i+1)\in S_n\longmapsto \tau_i
	 \in\operatorname{Aut}( x^{\otimes n})$ 
	 is a group homomorphism. \emph{Note.} This
	 produces in particular a map $S_2\longrightarrow 
	 \operatorname{Aut}(x^{\otimes 2})$ that sends
	 the transposition $(12)\in S_2$ to the
	 flip $\tau_{x,x}:x\otimes x\longrightarrow 
	 x\otimes x$.  
\end{question}

\begin{question} A product and permutation category (abbreviated `PROP')
is a monoidal category $\mathcal{C}$ whose set of objects is $\mathbb N = 
\{0,1,2,\ldots\}$ and its tensor product is addition (in particular, it is strict and symmetric). Unravel the definitions:
\begin{tenumerate}
\item Use that $n = 1+\cdots + 1$ to show that $\mathcal{C}(m,n)$ is a right $S_n$-module.
\item Similarly, show that $\mathcal{C}(m,n)$ is also a left $S_m$-module.
\item Show these two actions are compatible (i.e. they commute).
\item Show that the product $+$ induces a \emph{horizontal} composition rule 
\[ \mathcal{C}(m,n) \times \mathcal{C}(m',n') 	\longrightarrow
 	\mathcal{C}(m+m',n+n'). \]
\item Interpret the usual categorical product as a \emph{vertical} composition rule 
\[ \mathcal{C}(n,k) \times \mathcal{C}(m,n) 	\longrightarrow
 	\mathcal{C}(m,k). \]
\end{tenumerate}
Consider the definition of a PROP enriched over a symmetric strict 
monoidal category, like $\mathsf{Vect}$, the category of vector spaces (these are called $\kk$-linear
PROPs). Define the category of PROPs. 
\end{question}

\emph{Note.} For 
each $n\in\mathrm{Ob}(\mathcal C)$ the
object $\mathcal{C}(n,n)$ is a monoid under composition
 that receives a map $S_n\longrightarrow \mathcal{C}(n,n)$.
 Under the interpretation above, the image of $\sigma$
  is equal to both the left and the right action of $S_n$ 
  on the identity map $n\to n$. In particular, the twist
  $\tau$ of $\mathcal{C}$ is equal to $(12)\mathrm{id}_2$,
  and may (or may not) be trivial.
  
  \medskip
  
\textbf{C. Graded spaces and complexes.} When studying
algebraic structures like operads, it will be necessary
to use some tools from homological algebra: graded spaces,
chain complexes, differentials, their homology, among
others. The following exercises are intended to familiarize
you with the elements of homological algebra, but we will
look at them in more detail during the course.

\begin{question} A $\mathbb Z$-graded vector space (usually 
just called a graded vector space) is a $\mathbb Z$-indexed 
sequence of vector spaces $n\in\mathbb Z\longmapsto V_n\in\mathsf{Vect}$. 
If $v\in V_n$ we say that $v$ is homogeneous of
degree $n$. Find out about the category of
graded vector spaces, specifically:
\begin{titemize}
\item What are its (degree zero) morphisms?
\item What are its (degree homogeneous) morphisms?
\item What is the tensor product of two graded spaces?
\item What is the natural isomorphism $V\otimes W\longrightarrow W\otimes V$?
\item How does the last item relate to the `Koszul sign
rule'?
\end{titemize}
%\emph{Note:} there is a subtle difference between
%graded spaces (as we just defined) and \emph{weight graded} spaces, 
%which we will define later. One can think of graded 
%spaces as a ``discrete diagram of spaces over the integers'',
%while a weight graded space (usually graded over the natural 
%numbers) is a vector space with the
%\emph{additional} datum of a direct sum
%decomposition indexed by $\mathbb N$. Informally,
%one can say that a graded space is like a (possibly infinite)
%row of houses,
%while a weight graded object is like a (possibly infinitely tall)
%house that has been 
%turned into an apartment building by subdividing it
%horizontally.
\end{question}

\begin{question} A differential graded (dg) vector space,
usually called a complex, is a pair $(V,d)$ where $V$ is 
a graded vector space $V$ and $d: V \to V$ is a homogeneous
map of degree $-1$ such that $d^2=0$. Repeat the previous
exercise replacing $\mathsf{gVect}$ with $\mathsf{Ch}$,
the category of complexes of vector spaces.
\end{question}

\begin{question} If $(V,d)$ is a complex, then $Z(V) = \ker d$ 
is called its space of cycles, and $B(V) = \operatorname{im} d$ 
is called
its space of boundaries. The quotient $Z(V)/B(V)$ is called
the homology of $V$, and is written $H(V)$. Show that a map 
of complexes $f:V\to W$ induces a map $Z(V) \to Z(W)$ and 
in turn a map $H(f) : H(V) \to H(W)$. 
\end{question}
 
\begin{question} (Leisure) Find a book on homological algebra
and read about the \emph{snake lemma} and the \emph{five lemma}.
If you are very motivated, read about double complexes and spectral
sequences.
\end{question}