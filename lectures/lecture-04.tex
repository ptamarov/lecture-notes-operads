
\section{Quadratic duals}\label{lecture:KD1}
\textbf{Goals.}
Give the definition of the Koszul dual
operad of a quadratic operad, and then compute
some Koszul duals.

\subsection{Differential graded sequences}


\textbf{Homologically graded $\Sigma$-modules.}
Recall that an homologically graded vector space is
the datum of a $\mathbb Z$-indexed sequence of vector
spaces $n\in\mathbb Z\longmapsto V_n\in\mathsf{Vect}$.
We call the spaces appearing in this sequence the \emph{graded (or
homogeneous) components of $V$}, and say that an element in 
one of these summands is \emph{homogeneous}. If
$v\in V_n$, we say that $v$ is \emph{homogeneous of
degree $n$} and write $|v|=n$. 

A map $f : V\longrightarrow W$ of graded vector spaces
is \emph{homogeneous of degree $n$} if $f(V_j)\subseteq W_{j+n}$ for
all $j\in \mathbb Z$. We write $\hom(V,W)$ for the
space of all homogeneous maps, which is itself a graded
vector space with $\hom(V,W)_n$ the space of all
graded maps of degree $n$ for each $n\in\mathbb Z$. 
In this way, we obtain the category $\mathsf{Vect}_\mathbb{Z}$
of graded vector spaces and graded maps. 

A \emph{differential graded (dg) vector space} is a pair 
$(V,d)$ where $V$ is a graded vector space and 
$d : V\longrightarrow V$ is a homogeneous map of degree 
$-1$ such that $d^2=0$. We usually will call $(V,d)$
a \emph{chain complex}. The collection of homogeneous
maps $\varphi:V\longrightarrow W$ is again a chain
complex, with differential
\[ d\varphi 
	= d_W\varphi - (-1)^{|\varphi|} \varphi d_V. \]
A homogeneous map of degree zero such that $d(\varphi)=0$
is called a \emph{chain map}.
It is
convenient to also consider \emph{cohomologically graded}
vector spaces, by formally inverting the order of $\mathbb{Z}$
and letting $V^n = V_{-n}$ for all $n\in\mathbb Z$. 
For more information, the reader can consult~\cite{Weibel1994,
Gelfand2003,Hilton1997,Cartan1956}.

\bigskip

\textbf{Monoidal structure.} If $V$ and $W$ are
graded vector spaces, we define their tensor product
by setting
\[ (V\otimes W)_n = \bigoplus_{i+j = n} V_i\otimes W_j \]
for all $n\in\mathbb Z$, and setting the symmetry map
\[\tau : V\otimes W \longrightarrow W\otimes V\]
to be $\tau(v\otimes w) = (-1)^{|v||w|}w\otimes v$
on homogeneous elements, and extending it linearly on all of
$V\otimes W$. This makes $\mathsf{Vect}_\mathbb{Z}$ into a
symmetric monoidal category with unit the graded vector
space with $V_0 = \kk$ and $V_n = 0$ for $n\neq 0$.
The tensor product of maps $f: V\longrightarrow W$
and $g : V'\longrightarrow W'$ acts
in such a way that $f\otimes g : V\otimes V'
\longrightarrow W\otimes W'$ is the map
\[ (f\otimes g)(v\otimes w)  = (-1)^{|g||v|} f(v)\otimes g(w).\]
In case $V$ and $W$ are in fact dg, their tensor product is
also dg with $d_{V\otimes W} = d_V\otimes 1+ 1\otimes d_W$. 

\begin{definition}
A (homologically) graded $\Sigma$-module $\XX$
is a $\Sigma$-module taking values in the category
of graded vector spaces. Similarly, a dg $\Sigma$-module
is one taking values in dg vector spaces.
\end{definition}

\textbf{The endomorphism operad functor on dg modules.}
Let us consider the most natural way to create dg modules
from dg vector spaces, as we did in the case of usual
vector spaces. Namely, we may as before consider the
\emph{endomorphism operad} of a dg vector space $V$
by setting, for each $n\geqslant 0$,
\[ \End_V(n) = \hom(V^{\otimes n},V) \]
where these consists of homogeneous maps of dg vector
spaces. In particular, each of these arity components is 
itself a dg vector space, and the (total or partial)
composition maps
of the resulting operad are maps of dg vector spaces.

Of particular importance to us will be the \emph{suspension}
operation on dg vector spaces. Let us write $s$ for the
unique dg vector space with $s_1 = \kk$ and zero
elsewhere, and similarly let us write $s^{-1}$ for the
unique dg vector space with $s^{-1} = \kk$
and zero elsewhere. The \emph{suspension} of the dg vector
space $V$ is the tensor product $s\otimes V$, which
we write more simply $sV$, and whose basis elements we
write $sv$ for $v\in V$. Thus $|sv| = |v|+1$ for all 
homogeneous $v\in V$. Similarly, we define the
\emph{desuspension} $s^{-1}V$.

\begin{note}
The differential of $sV$ is given by $d(sv) = -s dv$. Can you explain why this is so using
the Koszul sign rule?
\end{note}
 
The following lemma shows that $V\mapsto \End_V$ is 
monoidal for the \emph{Hadamard product} of operads on the
target (and the usual tensor product on the domain): 
 \begin{lemma}\label{lemma:hadamard}
The map $\Phi : \End_V\otimes \End_W\longrightarrow \End_{V\otimes W}$
that assigns $\varphi \otimes \psi \in \End_V(n)\otimes \End_W(n)$
to the map
\[ \Phi(\varphi,\psi)(v,w) = (-1)^\varepsilon \varphi(v)\otimes\psi(w)\]
where $\varepsilon = \sum_{i=1}^n (|w_1| +\cdots + |w_{i-1}|+|\psi|)|v_i|$
is an isomorphism of operads provided $V$ and $W$
are locally finite, that is, finite
dimensional in each degree.
 \end{lemma}
 
 \begin{proof}
 This is Exercise~\ref{ex:suspensions}
 \end{proof}

In particular, we see that $\End_{sV}$ is canonically isomorphic
with $\End_s\otimes \End_V$, and hence that algebra structures on $sV$
are related to algebra structures on $V$ through the operad $\End_s$.
We proceed to name it and study it in the following section.

\subsection{The Koszul dual}

\newcommand{\sus}{\mathscr{S}}
\textbf{Suspensions.} 
We call $\End_{s}$ the suspension operad
and write it $\sus$. Note that $\End_{s}(n)$ is
the sign representation of $\Sigma_n$ put in degree $1-n$.

\begin{proposition} For each $n\geqslant 1$ let us
we write $\nu_n$ for the unique map in $\End_s(n)$ 
that sends $s^n$ to $s$. Then for every $m\geqslant 1$
we have that
\[ \nu_n \circ_i \nu_m = (-1)^{(i-1)(m-1)} \nu_{m+n-1}. \]
In particular, the binary operation $\nu := \nu_2$ of degree
$-1$ generates $\End_s$,
and presents it as a quadratic operad subject to the 
anti-associativity relation
\[ \nu \circ_1\nu + \nu\circ_2 \nu = 0.\]
\end{proposition}
\begin{proof}
 This is Exercise~\ref{ex:suspensionoperad}.
\end{proof}

If $\PP$ is an operad, then the arity-wise tensor product
$\sus\otimes \PP$ is called the suspension of $\PP$
and we write it $\sus\PP$ or $\PP\{1\}$. Dually, we
write $\sus^{-1}$ for the desuspension operad
defined by $\End_{s^{-1}\kk}$. 

\begin{note} As we just observed,
the operad  $\sus\PP$ has the property that
$\sus\PP(sV) = s\PP(V)$, so that algebras over $\sus\PP$
are exactly those vector spaces $V$ such that $s^{-1}V$ is a
$\PP$-algebra. Equivalently, $sV$ is a $\sus\PP$-algebra
if and only if $V$ is a $\PP$-algebra. 
\end{note}

\textbf{Pairings.} We define a pairing between $\FF_\XX$ and
$\FF_{s^{-1}\sus^{-1}\XX^*}$ as follows (the appearance of
the suspensions will be evident later):
\[ \langle \Sigma\nu^* \circ_j \Sigma\mu*, 
	\rho \circ_i \tau  \rangle
   = \delta_{ij} (-1)^{\varepsilon}
   	\nu^*(\rho)\mu^*(\tau). \]
with $\varepsilon = \varepsilon_1+\varepsilon_2$,
where $\varepsilon_1 = (\ari(\nu)-1)(|\mu|+i-1)+|\nu||\mu|$ 
and $\varepsilon_2$ counts the total
number of inversions in the shuffle permutations
appearing in the two tree monomials.
If $\XX = \XX(2)$ is binary and has no homological degrees, 
this simplifies to
\[ \langle \Sigma\nu^* \circ_i \Sigma\mu*, 
	\rho \circ_i \tau  \rangle
   =  \begin{cases}
    	(-1)^\varepsilon \nu^*(\rho)\mu^*(\tau) & i=1 \\
    	-	\nu^*(\rho)\mu^*(\tau) & i = 2.
    	\end{cases} \] 
where $\varepsilon$ depends on the decoration
of the leaves (it is $1$ if both decorations
are equal, and is $-1$ if exactly one is the
shuffle $132$. 

\begin{definition}
The Koszul dual operad of a quadratic operad $\PP$ 
generated by $\XX$ subject to relations $\RR$, is
the operad $\PP^!$ generated by $s^{-1}\sus^{-1}\XX^*$ 
and subject to the orthogonal space of relations
$\RR^\perp$ according to the pairing above.
\end{definition} 

\begin{note}
Let $\PP$ be an operad. Then $\PP$ is quadratic if and
only if $\sus\PP$ is quadratic, and it is Koszul if and
only if $\sus\PP$ is Koszul. 
\end{note}

\textbf{Some examples.} Let us compute the Koszul duals of 
some of the quadratic operads we considered in 
Lecture~\ref{lecture:quadraticops}.
For simplicity, we will consider only those with binary 
generators of degree zero, though one can in the same way
carry out computations with generators of higher arities and
varying homological degrees.

\bigskip

\emph{The associative operad}. We saw previously that for
$\underline{\XX}$ consisting of a single operation
$x_1x_2$ with no symmetries, the
space $\FF_\XX(3)$ is twelve dimensional, spanned by
the $S_3$-orbits of $\alpha = x_1(x_2x_3)$ and $\beta =(x_1x_2)x_3$,
each of size six. We also noted that $\alpha-\beta$
spans a six dimensional submodule, complemented by the
orbit of $\alpha+\beta$. 

Using the pairing above, we see that
\[\langle \alpha,\alpha\rangle = 1,
	\quad \langle\beta,\beta\rangle = -1,
	\quad \langle \alpha,\beta\rangle = 0, \] 
from where it follows that the dual space to the associativity
relation is the corresponding associativity relation
$\alpha^* - \beta^*$ in $\XX^*$. In other words,
the associative operad is Koszul self-dual:
\[\mathsf{Ass}^! = \mathsf{Ass}.\]

It is important to note how the minus sign in our
definition of the pairing or, more generally, the
Koszul sign we have introduced, guaranteeing that
this pairing in equivariant, introduces the minus sign
in the dual of $\alpha+\beta$.

\bigskip

\emph{The commutative and Lie operads.}
We have computed that if $\XX(2)$ is the trivial representation
of $S_2$ spanned by some commutative operation $x_1x_2$,
then $\FF_\XX(3)$ is three dimensional, spanned by
$x_1(x_2x_3)$, $(x_1x_2)x_3$ and $(x_1x_3)x_2$.
Moreover, we verified that if we put
\[ \alpha = x_1(x_2x_3) -(x_1x_2)x_3, 	\quad 
     \beta =  x_1(x_2x_3) -(x_1x_3)x_2 \]
     then these two element span an $S_3$-submodule
     that is complemented by the $S_3$-submodule generated by
     \[ \gamma = x_1(x_2x_3) +(x_1x_2)x_3+  (x_1x_3)x_2.\]
 This is in fact an orthogonal complement as a direct computation
 shows, so we see that the orthogonal set of relations
 to the commutative associative relation is the dual of
 $\gamma$ for the dual antisymmetric operation $[x_1,x_2]$:
 this is exactly the Jacobi relation
 \[ 
 \gamma^* = -[x_1,[x_2,x_3]] +[[x_1,x_2],x_3]-[[x_1,x_3],x_2].
 \]
 Notice that there is an ``unexpected'' minus
 sign in the last term, coming from the fact
 the tree monomial $[[x_1,x_3],x_2]$
 has the odd permutation $132$
 at the top, and a minus sign appearing in the
 first term coming from a grafting at the second
 leaf.  It follows that the Koszul dual of 
 the commutative operad is the  Lie operad, 
 and conversely:
 \[ \mathsf{Com}^! = \mathsf{Lie}, \quad
  \mathsf{Lie}^! = \mathsf{Com}.
  	\]
With this at hand, one can compute that the Poisson operad 
is self-dual (Exercise~\ref{ex:duals}), one only needs to 
address the Leibniz relation.  

\bigskip

\emph{The pre-Lie and permutative operads. 
The Novikov operad.}
Recall the pre-Lie operad is generated by a single 
operation $x_1x_2$ with no symmetries, subject to 
the (right symmetric) pre-Lie relation
\[(x_1x_2)x_3 - x_1(x_2x_3) = (x_1x_3)x_2 - x_1(x_3x_2). \]
One can check that the $S_3$-orbit $V$ of 
this element is three dimensional,
so let us write $\alpha_1,\alpha_2$ and 
$\alpha_3$ for the translates of
this relation in $\FF_\XX(3)$. 

This orbit is complemented by the orbit $W$ of the 
associativity relation $(x_1x_2)x_3 - x_1(x_2x_3)$
 and the orbit $U$ of the right permutative relation
$(x_1x_2)x_3 - (x_1x_3)x_2$. 
The first is six dimensional, as we already
computed, while the second is three dimensional. 
A direct computation shows 
that $V^\perp$ identifies 
with the nine dimensional subspace 
$U^*\oplus W^*$. 

Thus, we see that the operad of (right)
pre-Lie algebras is Koszul dual to that
of (right) permutative algebras:
\[ \mathsf{PreLie}^! = \mathsf{Perm},\quad
 	\mathsf{Perm}^! = \mathsf{PreLie}.\]
A similar computation shows that the operad 
controlling Novikov algebras,
those right pre-Lie algebras whose product 
is \emph{left} permutative
\[ x_1(x_2x_3) = x_2(x_1x_3) \]
is almost Koszul self-dual: we have that $\mathsf{Nov}^! = 
\mathsf{Nov}^{\mathrm{op}}$, by which we mean the resulting
operad controls pre-Lie algebras whose associator is
left symmetric and whose operation is \emph{right} 
permutative.
\subsection{Exercises}

\begin{question}\label{ex:suspensions}
Show the map $\Phi_{V,W}$ of Lemma~\ref{lemma:hadamard}
is an isomorphism for $V$ and $W$ locally finite dimensional
dg symmetric sequences.
\end{question}

\begin{question}\label{ex:suspensionoperad}
Show that the suspension operad is binary quadratic
generated by a single operation $\nu$ of degree $-1$
that is ``anti-associative'', in the sense that
$\nu\circ_1\nu + \nu\circ_2\nu=0$. 
\end{question}
\begin{question}\label{ex:duals} Show that:
(1) $\mathsf{Ass}$ is Koszul
self dual, (2) $\mathsf{Com}$ and $\mathsf{Lie}$
are Koszul dual to each other, (3) $\mathsf{PreLie}$
and $\mathsf{Perm}$ are Koszul dual to each
other, (4) the Poisson operad is Koszul self-dual.
\end{question}

\begin{question} 
The operad $\mathsf{Nov}$ of Novikov algebras
is the quotient of the (right) pre-Lie operad by the 
left permutative relation
$x_1(x_2x_3) = x_2(x_1x_3)$.
Show that $\mathsf{Nov}$ is Koszul dual to its
``opposite'' operad $\mathsf{Nov}^\mathrm{op}$
controlling left pre-Lie algebras satisfying the
right permutative relation. 
\end{question}

\begin{question}
Show that:
\begin{tenumerate}
\item 
The Koszul dual of the operad controlling
totally associative $k$-ary algebras is the
operad controlling partially associative
$k$-ary algebras.
\item The Koszul dual of the operad controlling
commutative totally associative $k$-ary algebras
is the operad controlling $k$-ary Lie algebras. 
\end{tenumerate}
\end{question}

\begin{question}
Let $x_1x_2$ be a binary operation and consider
the two relations:
\[
  R = (x_1x_2)x_3 -
   	\sum_{\sigma \in S_3} 
   		\lambda_\sigma \sigma(x_1(x_2x_3)),
   		\qquad
   		S = x_1(x_2x_3) - 
   	\sum_{\sigma \in S_3} 
   		\lambda_\sigma \sigma^{-1}((x_1x_2)x_3).
 	\]
Show that the resulting quadratic operads $\FF(x_1x_2)/(R)$
and $\FF(x_1x_2)/(S)$ are Koszul dual to each other.
\end{question}

\begin{question}
Show that in the case of binary operads,
the bilinear form we constructed 
during the lectures is $S_3$-invariant.
\end{question}
