\section{Bar and cobar constructions}\label{lecture:barcobar}
\textbf{Goals.} Define the bar (and cobar) construction,
define the syzygy grading and give the first definition
of what it means for an operad to be Koszul. 

\subsection{The bar construction}

We have defined a functor on quadratic data $(\XX,\RR) \longmapsto 
(\XX^\vee,\RR^\perp)$ that gives us the Koszul dual
functor $\PP \longmapsto \PP^!$ on quadratic operads.
Let us explain how one can ``promote'' this construction
to a functor defined on all operads, and how our more
down-to-earth construction can be recovered from
this. Let us begin with a useful notion, which is
to an operad what a coassociative coalgebra is to 
an associative algebra.

\begin{definition}
A symmetric cooperad is a symmetric collection
$\CC$ endowed with maps of collections
$\Delta : \CC \longrightarrow \CC\circ_\Sigma\CC$
and $\varepsilon: \CC\longrightarrow \kk$
so that $\Delta$ is coassociative and counital
for $\varepsilon$. A cooperad $\CC$ is coaugmented
if there exists $\eta: \kk\longrightarrow \CC$
such that $\varepsilon\eta = 1_\kk$. In this case,
we write $\overline\CC$ for the cokernel of $\eta$.
\end{definition}

We warn the reader that, strictly speaking,
we should have defined cooperads using a variant
of the composition product $\circ_\Sigma$ that
takes $\Sigma$-invariants instead of coinvariants.
Since we will always work over a field of characteristic 
zero, we will not worry about this. At any rate,
unraveling the definitions, we see that
a cooperad is a symmetric sequence endowed with
equivariant decomposition maps of signature
\[
\Delta : \CC(n) \longrightarrow
\bigoplus_{n_1+\cdots+n_k = n} \CC(k)\otimes \CC(n_1)\otimes\cdots \CC(n_k)
\]
that are coassociative and counital. One can check (Exercise~\ref{ex:takeduals})
that if $\CC$ is a cooperad then its arity-wise dual $\CC^*$ 
is an operad, and that if $\PP$ is an arity-wise finite dimensional
reduced operad then its arity-wise dual $\PP^*$ is a cooperad. 

\begin{definition}
The reduced and half-reduced decomposition maps of a cooperad $\CC$ are the
maps $\overline{\Delta}$ and $\widetilde{\Delta}$
 uniquely determined by the equation
\[
\Delta(\nu) =  \overline{\Delta}(\nu) +
 (1;\nu) + (\nu ; 1,\ldots,1)=
 \widetilde{\Delta}(\nu) +
 (1;\nu)
\] 
for all arity-homogeneous $\nu\in\CC$. We say that
$\CC$ is \emph{conilpotent} if for each $\nu\in\CC$
the iteration of $\widetilde{\Delta}$ on $\nu$ defined by
$
\widetilde{\Delta}^n(\nu) = (1\circ \Delta)\widetilde{\Delta}^{n-1}(\nu)
$
stabilizes, in the sense that eventually all leaves are decorated by
the identity element of $\CC$.
\end{definition}



As in the case of algebras and coalgebras, where for a vector space
$V$ the underlying vector space of the free associative algebra
$TV$ on $V$ and the free \emph{conilpotent} coassociative coalgebra
coincide, the free \emph{conilpotent} cooperad over a symmetric
sequence is an object we have meet before.

\begin{definition}
The free conilpotent cooperad over a symmetric sequence $\XX$,
which we denote by $\FF_\XX^c$, has the same underlying symmetric
sequence as the free operad $\FF_\XX$, and its decomposition maps
are given by ``degrafing of trees''.
\end{definition}

We will look at the structure of the free conilpotent operad functor
$\FF^c$ in more detail in Exercise~\ref{ex:cofree}. 
As an example, if $x\in \XX(2)$ is a binary symmetric operation and 
we consider the fork
\[
\mathsymbol{0.75}{T=}\smallfork{}{}{}{1}{4}{2}{3}\mathsymbol{0.75}{,}
\]
we have that $\overline{\Delta}(T)$ is equal to the sum
\begin{align*}
{}& 
\smallleftc{}{}{}{}{} \hotimes \stick{1}  \hotimes \stick{4}
	\hotimes\smallb{}{2}{3} \hplus \\
	&
	\smallrightc{}{}{}{}{} \hotimes 
		\smallb{}{1}{4}   \hotimes \stick{2}
	\hotimes \stick{3}\hplus\\
	&\phantom{em}\smallb{}{}{} \hotimes 	
	\smallb{}{1}{4} \hotimes \smallb{}{2}{3} \mathsymbol{0.5}{.}
\end{align*}

Although both $\FF_\XX$ and $\FF_\XX^c$ satisfy universal properties
with respect to operad morphisms form $\XX$ into an operad or from
a conilpotent cooperad into $\XX$, it will be convenient for us to 
observe they both satisfy (and are uniquely characterized by) a
second universal property. To state it, we need to consider a natural
generalization of the ``Leibniz rule'' for derivatives in the case
of (co)operads.

\begin{definition}
Let $\PP$ be an operad. A derivation of $\PP$ is a map
$d : \PP\longrightarrow \PP$ with the property that
for any arity-homogeneous elements $\mu_0,\mu_1,\ldots,\mu_k\in\PP$
with $\ari{\mu_0} = k$, we have that 
\[
d\gamma(\mu_0;\mu_1,\ldots,\mu_k) = 
	\sum_{i=0}^k \gamma(\mu_0;\mu_1,\ldots,d\mu_i,\ldots,\mu_k).
	\]
If $\PP$ is homologically graded, we require that $d$ is degree homogeneous,
and then Koszul signs will appear in the display
above.
\end{definition}

The following is the universal property we were after.

\begin{lemma}
Let $\XX$ be a symmetric collection. The free operad $\FF_\XX$
on $\XX$ satisfies the following universal property: for every
map of symmetric sequences $f: \XX\longrightarrow \PP$ where $\PP$
is an operad, there exists a unique derivation $F: \FF_\XX
\longrightarrow \PP$ which coincides with $f$ on $\FF_\XX^{(1)}=\XX$.

Dually, the free conilpotent cooperad $\FF_\XX^c$ satisfies the following
universal property: for every
map of symmetric sequences $f: \CC\longrightarrow \XX$ where $\CC$
is a conilpotent cooperad, there exists a unique coderivation $F: \CC
\longrightarrow \FF_\XX^c$ which coincides with $f$ after
projecting onto $\FF_\XX^{c,(1)}=\XX$.
\end{lemma}

We use it to define one of the most important functors we will 
consider in this section, the \emph{bar construction}. To do this,
let us begin with an augmented operad $\PP$, and let us form
the cofree conilpotent cooperad $\FF_{s\overline{\PP}}^c$ on the
na\"ive suspension $s\overline{\PP}$ of its augmentation ideal.
This is for the moment a homologically graded conilpotent cooperad, which we will
make into a dg conilpotent cooperad as follows: the composition
map of $\PP$ provides us with a map of symmetric collections
\[
\FF_{s\overline{\PP}}^c \longrightarrow s\overline\PP
\]
that is zero in all tree monomials of weight different from two, and
on the latter, it is equal to the unique map
\[
\gamma' :  \FF_{s\overline{\PP}}^{c,(2)} 
 \longrightarrow s\overline{\PP}
\]
induced by the composition map of $\PP$. The universal property of $\FF^c$
guarantees there is a unique coderivation 
\[ \partial : \FF_{s\overline{\PP}}^c 
	\longrightarrow \FF_{s\overline{\PP}}^c\]
extending $\gamma'$. In Exercise~\ref{ex:zerosquare} we will see that
$\partial^2 = 0$, but for the moment notice that $\partial$ has degree $-1$,
as the map $\gamma'$ does: it erases one suspension sign from 
a single tree monomial with exactly one internal edge whose
two vertices are decorated by elements of $\overline{\PP}$,
by composing them along this unique internal edge.

\begin{figure}

\end{figure}

\begin{definition}
Let $\PP$ be an augmented operad. 
We call the dg cooperad $(\FF_{s\overline{\PP}}^c ,\partial)$
the bar construction of $\PP$ and write it $\B{\PP}$.
\end{definition}

Let us begin exploring the shape of this gadget. For starters,
since its underlying symmetric collection is equal to that of
a free operad, whenever $\PP$ is reduced ---which is running
assumption for us--- it has a basis of shuffle tree monomials
decorated by elements of $\PP$ that are \emph{not} the unit;
let us call these \emph{bar tree monomials}. A bar tree monomial
has homological degree $d$ precisely when it has $d$ internal
vertices, as the device of using $s\overline{\PP}$ instead of
$\overline\PP$ means precisely that we are turning the canonical
weight grading of $\FF$ into a homological grading. We will
use the notation $|T|$ for this homological degree, as usual,
and use the subscript notation $\B{\PP}_*$ when referring to
this grading. Thus, for example,  $\B{\PP}_2$ consists of
those bar tree monomials of homological degree two.

\begin{figure}
\[
\mathsymbol{0.5}{\Leftr{}{}{}{2}{4}{1}{3}}
\mathsymbol{1.5}{\leadsto}
\begin{tikzpicture}
	\begin{pgfonlayer}{main}
		\node [style=inner] (3) at (0, 1) {};
		\node [style=inner] (4) at (0, 2) {};
		\node [style=leaf] (5) at (-1.25, 2) {$1$};
		\node [style=leaf] (6) at (1.25, 2) {$3$};
		\node [style=leaf] (12) at (0, 0) {};
		\node [style=leaf] (13) at (-0.75, 3) {$2$};
		\node [style=leaf] (14) at (0.75, 3) {$4$};
	\end{pgfonlayer}
	\begin{pgfonlayer}{bg}
		\draw (3.center) to (5.center);
		\draw (4.center) to (3.center);
		\draw (3.center) to (6.center);
		\draw (3.center) to (12.center);
		\draw (13.center) to (4.center);
		\draw (14.center) to (4.center);
	\end{pgfonlayer}
\end{tikzpicture}
\hspace{0.5 em}
\mathsymbol{1.5}{-}
\hspace{-1 em}
\begin{tikzpicture}
	\begin{pgfonlayer}{main}
		\node [style=inner] (0) at (0, 1) {};
		\node [style=inner] (1) at (-1, 2) {};
		\node [style=leaf] (2) at (1, 2) {$3$};
		\node [style=leaf] (3) at (-2, 3) {$1$};
		\node [style=leaf] (4) at (-1, 3) {$2$};
		\node [style=leaf] (5) at (0, 3) {$4$};
		\node [style=leaf] (6) at (0, 0) {};
	\end{pgfonlayer}
	\begin{pgfonlayer}{bg}
		\draw (3.center) to (1.center);
		\draw (4.center) to (1.center);
		\draw (5.center) to (1.center);
		\draw (1.center) to (0.center);
		\draw (0.center) to (2.center);
		\draw (0.center) to (6.center);
	\end{pgfonlayer}
\end{tikzpicture}
\]
\caption{A boundary map in $\B{\Com}$, vertices are not decorated
as there is a unique $k$-ary variable in $\Com$ for each $k\in\NN$.}
\end{figure}
At the same time, we will usually take $\PP$ to be \emph{itself}
weight graded, so that bar tree monomials have a homological
degree, which keeps track of how many elements of $\PP$ appear
in a tree monomial, and a total weight, which simply is obtained
by adding up the weights of the elements of $\PP$ in them.
We will write $\lVert T\rVert$ for the weight of a tree monomial
and use the notation $\B{\PP}_{(*)}$ when referring to this
weight grading. Thus, for example, $\B{\PP}_{(2)}$ consists of
those bar tree monomials of total weight degree two, which
can then either have homological degree one and be decorated by
one weight two element of $\PP$, or have homological degree two
and be decorated by two weight one elements of $\PP$. Note
that in general $|T|\leqslant \lVert T\rVert$.

\begin{definition}
The syzygy degree of a bar tree monomial $T$ is the difference
$\lVert T\rVert - |T|$, and we write it $\syz{T}$. We call this
the syzygy grading of $\B{\PP}$, and we use the superscript
notation $\B{\PP}^*$ when referring this this cohomological
grading: the differential $d$ has degree $+1$ in the syzygy
grading.
\end{definition}

\begin{figure}
\[
\begin{tikzpicture}[scale = 0.8]
\tikzset{>=latex}

%Draw a nice grid
\draw[help lines, color=gray!30, dashed] 
	(-4,-1) grid (10.5,9.5);

%Draw bar components 
\foreach \x in {1,2,3,4}
	{     
	\pgfmathsetmacro\result{\x} 
	\foreach \y in {1,...,\result}
	  {
	  \node (\y\x) at (3*\x - 3*\y,2*\x) {$\B{\PP}_{\y,(\x)}$};
	  	  }
	}
    {
    \draw[->,>=stealth] (22) edge (12);
    \draw[->,>=stealth] (33) edge (23);
     \draw[->,>=stealth] (23) edge (13);
    \draw[->,>=stealth] (44) edge (34);
    \draw[->,>=stealth] (34) edge (24) ;
    \draw[->,>=stealth] (24)  edge (14);

    }
   
%Draw P antishriek

\foreach \x in {1,2,3,4}
	 {
	 \node (P\x) at (-3,2*\x) {$\PP^\antishriek_{(\x)}$} ;
	 \draw[->,>=stealth] (P\x) edge (\x\x);
 }
 
 %draw lonely unit
 
 \node at (-3,0) {$\kk$};
 
 %draw some "axes"
 
 \draw[->] (-4,-1) -- (10.5,-1) ;
  \draw[->] (-4,-1) -- (-4,9.5) ;
 \node (S) at (10.5,-0.5) {\scriptsize{syzygy}};
  \node (W) at (-4,10) {\scriptsize{weight}};
 \end{tikzpicture}	
\]
\caption{The bar construction of an augmented weight graded operad $\PP$. Horizontal arrows
are differentials, leftmost arrows are inclusions. Note
that the diagonals where $s+w$ is constant return the
usual homological degree $d$. }
\end{figure}

\begin{example} Let us consider the case where $\PP = \Com$ is the
commutative operad, in which case $\XX = s\overline{\PP}$ is a symmetric
sequence such that $\XX(0) = \XX(1) = 0$ and for which $\XX(n)$ is
one dimensional for each $n\geqslant 1$ and concentrated in
degree one and weight $n-1$. It follows that $\B{\Com}$ has basis consisting
of shuffle trees where the internal vertices have
all possible arities greater or equal to $2$, and in this case the
degree of a bar tree monomial is equal to the number of 
its internal vertices. Let us write $\Gamma_m$ for the element in
$\B{\Com}$ of degree $1$ and weight $m-1$ corresponding to the 
$m$-fold product in $\Com$.

If we write $T_i(A,B)$ for the shuffle tree in
Figure~\ref{fig:twolevel}, where $A = \{ i,j_1,\ldots,j_m\}$
and $B = \{1,\ldots,i-1,k_1,\ldots,k_n\}$, we see that
$\partial T_i(A,B) = (-1)^{i-1} \Gamma_m$ where $m\geqslant 3$,
so all of these elements are zero in homology. For example, if
we look at $\B{\Com}(3)$, we notice this complex has total 
dimension four, with the usual three shuffle tree monomials
\[ T_1(12,3), 
	\quad
		T_1(13,2),
			\quad
			\text{ and  }
		 	\quad T_2(1,23)
			\]
with three leaves in degree two (and weight two) and
one single tree monomial (the corolla $\Gamma_3$) in degree one.
Moreover, the computation above shows that generators for
the homology are
\[ 
T_1(12,3) + T_2(1,23),\qquad 
T_1(13,2) + T_2(1,23)
\]
Similarly, one can compute as in Exercise~\ref{ex:barcom} that
$\B{\Com}$ is of dimension $26$, and that the degree two cycles
are of dimension $9$ (all $10$ basis shuffle trees map to the only basis
elements of degree $1$). Moreover, one can check that all of these
cycles in degree two are boundaries. Since this complex has
$\chi = 6$ and we have just noted it has homology only in
degree three, we see this homology group is of this 
dimension. We will
see very soon that in
syzygy degree zero, the homology of $\B{\Com}(n)$ 
is of dimension $(n-1)!$.
\end{example}

\subsection{Koszul dual cooperad}

Let us begin by observing that, in the syzygy grading,
$H^0(\B{\PP})$ is computed as the kernel of a linear operator,
as all syzygy degrees are non-negative (so there are no boundaries
in degree zero). We can in fact try to study the shape of the bar
construction with a little more care, and determine precisely
what this homology group is as a \emph{cooperad} in case
$\PP$ is quadratic.

To do this, let us assume that $\PP$ is generated by 
$\XX$ (in weight one) subject to some quadratic relations $\RR$ in 
$\FF_\XX^{(2)}$. Then, we see that elements of $\B{\PP}^0$ 
(that is, those that have degree equal to weight) must
be precisely those of in $\FF^c_{s\XX}$. Note the difference!
While all of $\B{\PP}$ is given by a cofree cooperad on $\PP$,
the syzygy degree zero part only retains the elements 
that generate $\PP$. 

Having said this, if we decompose 
$\FF_{s\XX}^c$ into its tree monomial components, and
if we write $\mathcal V = \FF_\XX^{(2)} / \RR$,
we see
that the differential $\partial$ maps $\FF_{s\XX}^c$ to
$\FF_{s\XX\oplus s\mathcal V}^c$, since it at
most can introduce an element of $\PP$ of weight two,
and it maps to the symmetric subsequence
$\FF_{s\XX,s\mathcal V}^c$ where
\emph{exactly one} internal vertex is decorated by an 
element of $s\mathcal V$. 

\begin{definition}
Let $\XX$ be a symmetric sequence and let $\RR\subseteq \FF_\XX^{c,(2)}$.
The cofree conilpotent cooperad cogenerated by $\XX$ subject
to the correlations $\RR$ is the unique conilpotent cooperad $\FF^c(\XX,\RR)$
that is universal among those conilpotent cooperads
$\CC$ along with a map $\CC\longrightarrow \XX$ such
that the projection of the unique map $\CC \longrightarrow \FF_\XX^c$
to $\CC \longrightarrow \FF_\XX^{c,(2)}$ has image in
$\RR\subseteq \FF_\XX^{c,(2)}$. 
\end{definition}

If the reader finds the definition slightly confusing, let us remark
that the quotient operad $\FF(\XX,\RR) = \FF_\XX/ (\RR)$ is universal
among those operads $\PP$ along with a map $\XX\longrightarrow \PP$
with the property that the restriction of the unique map 
$\FF_\XX\longrightarrow \PP$ to $\FF_\XX^{(2)}$ \emph{factors}
through $\FF_\XX^{(2)}/\RR$: whereas the condition on the operad is
that $\RR$ is contained in the kernel of this map, the condition
on conilpotent operads is that $\RR$ is contained in the image. 

\begin{proposition}
If $\PP  = \FF(\XX,\RR)$ then the zeroth
syzygy homology groups $H^0(\B{\PP})$ are equal to $\FF^c(s\XX,s^2\RR)$.
More precisely, $\CC = H^0(\B{\PP})$ is a weight graded subcooperad,
with $\CC_{(1)} = s\XX$ and the projection $\CC \longrightarrow s\XX$
satisfies the universal property of the definition above for $s^2\RR$.
\end{proposition}

\begin{proof}
First, let us note that not only is $H^0(\B{\PP})$ equal to $s\XX$ in
weight $1$, but it is also equal to $s^2\RR$ in weight $2$, as this is
by definition the kernel of the composition
\[
 \B{\PP}_{2,(2)} = \FF^{(2)}_{s\XX} \subseteq \B{\PP}^0
 	\longrightarrow 
 	s\overline{\PP}_{(2)}  \subseteq   \B{\PP}^1 
\]
so that the projection $H^0(\B{\PP})\longrightarrow s\XX$ does induce a map
$H^0(\B{\PP}) \longrightarrow \FF^c_\XX$ with the desired properties. 
Suppose that $\CC$ is a cooperad with a map $f$ to $\XX$ for which
the projection $\pi\circ F: \CC \longrightarrow \FF_\XX^{c,(2)}$ lands
in $s^2\RR$. Since $H^0(\B{\PP})$ is a subcooperad of $\FF^c_{s\XX}$,
it suffices to show that the map $F: \CC\longrightarrow \FF^c_{s\XX}$
has image in $H^0(\B{\PP}) = \ker \partial^0$.

By definition, $F$ is obtained through $f$ by iteration of the decomposition
map of $\CC$: for each $n\geqslant 1$ the component $F: \CC\longrightarrow 
\FF^{c,(n)}_{s\XX}$ in weight $n$ is obtained using $n-1$ instances of the
decomposition map of $\CC$. By hypothesis, every use of the decomposition
map of $\CC$ always creates quadratic summands that are in $\RR$, so by
coassociativity, iteration of $\Delta$ will produce terms that can be written
to contain relations in any ``big vertex'' of $\FF_{s\XX}^c$ we prefer.
This guarantees that when we apply $\partial^0$ the result will be
zero (as this is an alternating sum of terms, each which annihilates 
an appropriate way of writing down the result of iterating $\Delta$)
so that $F$ indeed lands in $H^0(\B{\PP})$, like we wanted.
\end{proof}

The reader may compare to the case of associative algebras (that is,
when $\XX$ consists exclusively of arity one operations) in which case
the cofree conilpotent coalgebra $C$ with cogenerators $V$ and relations
$R\subseteq V^{\otimes 2}$ is such that for each $n\geqslant 2$,
\[
C_{(n)} = \bigcap_{i=1}^{n-1} V^{\otimes (i-1)}\otimes R\otimes 
					V^{\otimes(n-i-1)}.
\]

\begin{definition}
Let $\PP$ be the quadratic operad associated to the quadratic datum 
$(\XX,\RR)$. We call $H^0(\B{\PP})= \FF^c(s\XX,s^2\RR)$ the Koszul dual cooperad
to $\PP$, and write it $\PP^\antishriek$.
\end{definition}

Recall that we defined $\sus = \End_{s\kk}$, the suspension operad.
We will use $\sus^c$ to denote the endomorphism cooperad $\End_{s\kk}^c$,
which we can think simply as the arity-wise dual of $\sus^{-1}$.
The ``Koszul pairing'' we used when we defined the operad $\PP^!$
gives a hint to the following result.

\begin{lemma}\label{lemma:dualcooperad}
The dual of the cooperad  $\sus^c\otimes \PP^\emph{\antishriek}$
is isomorphic to $\PP^!$.
\end{lemma}

\begin{proof}
This is Exercise~\ref{ex:dualcooperad}. 
\end{proof}

In particular, this proves that the dual operad to $H^0(\B{\Com})$ is
equal to the Lie operad, which explains the computations done and suggested
above. We can now conclude this lecture by introducing one of the
central definitions in these notes.

\begin{definition}[First definition]
A weight graded (quadratic) operad $\PP$ is Koszul if
the inclusion of cooperads $\PP^\antishriek \longrightarrow \B{\PP}$
is a quasi-isomorphism. In other words, we say that $\PP$
is Koszul if the homology of its bar construction is
concentrated in syzygy degree zero.
\end{definition}

\subsection{Exercises}	

\begin{question}
All the results in this section have a corresponding
dual statement for conilpotent cooperads. State and
prove them. In particular, define for each weight graded
cooperad $\CC$ its cobar construction $\Omega(\CC)$,
and describe the syzygy grading and the differential.
\end{question}

\begin{question}\label{ex:cofree}
Consider a binary alphabet $\XX$ with a single operation,
and compute the decomposition map of $\FF^c_\XX$ in case
this operation is symmetric, antisymmetric or regular 
for tree monomials with four leaves.
\end{question}


\begin{question}\label{ex:conilpotent}
Show that a cooperad $\CC$ is conilpotent if and only if
there exists a symmetric sequence $\XX$ and an injective
map of cooperads $\CC \longrightarrow \FF_\XX^c$. 
\end{question}

\begin{question}\label{ex:takeduals}
Show that if $\CC$ is a cooperad then its arity-wise dual $\CC^*$ 
is an operad, and that if $\PP$ is an arity-wise finite dimensional
reduced operad then its arity-wise dual $\PP^*$ is a cooperad. 
Make sure to explain why the second set of hypotheses are needed.
\end{question}

\begin{question}\label{ex:dualcooperad}
Prove Lemma~\ref{lemma:dualcooperad}, which states
that the dual of the cooperad  $\sus^c\otimes \PP^{\antishriek}$
is isomorphic to $\PP^!$.
\end{question} 

\begin{question}\label{ex:barcom}
Compute the homology of the complex $\B{\Com}(4)$ explicitly.
Show it is concentrated in syzygy degree $0$, where it is
six dimensional, and give representatives for the 
cocycles giving a basis of it.
\end{question}

\begin{question}\label{ex:zerosquare}
Show that the differential $\partial$ of the bar construction of
an augmented operad $\PP$ squares to zero if and only if the composition
map of $\PP$ is associative. 
\end{question}

\begin{question}\label{ex:dgbar} Suppose that $\PP$ is an augmented dg
operad, and let us write $\partial_1$ for the differential of its
bar construction (considered as a non-dg operad). Recall
that $s\overline{\PP}$ gets the differential $-s d_\PP s^{-1}: 
s\overline{\PP}\longrightarrow s\overline{\PP}$.

\begin{tenumerate}
\item Show that in this case the differential
of $s\overline{\PP}$ induces a differential $\partial_2$ on $\FF_{s\PP}$. 
\item Show that $\partial_1\partial_2 + \partial_2\partial_1=0$
is equivalent to $d_\PP$ being a derivation for $\gamma_\PP$.
\item
Conclude that $\partial = \partial_1 + \partial_2$ is a differential
on $\FF_{s\overline{\PP}}$. 
\end{tenumerate}
We call the resulting cooperad the bar construction of
$\PP$ and write it $\B{\PP}$.
\end{question}


 