
\section{Methods to prove an operad is Koszul I}\label{lecture:methods1}

 \subsection{Monomial operads}
 
Let us fix a shuffle operad $\PP$ generated by some set of
variables $\XX$ subject to some relations $\RR$ determined
by tree polynomials, and suppose that $\prec$ is an admissible
monomial order on $\FF_\XX^\Sha$. With this information, one
can consider the operad $\PP_\mathrm{mon}$, generated by
the same variables $\XX$ but by the tree monomials 
$\leadm{\RR}$, sometimes denoted $\mathrm{in}_\prec(\RR)$
in the cases of associative (or associative commutative) algebras.
We call it the monomial operad associated to  $\PP$.

If we let $\mathcal N$ denote the subsequence of $\FF_\XX$ 
generated by the normal forms for this order with respect to $\RR$,
we have an isomorphism of collections $\PP_\mathrm{mon} \cong \mathcal N$
and, at the same time, a map of sequences
\[
\eta : \PP_\mathrm{mon} \longrightarrow \PP 
\]
obtained as the composition of the inclusion $\mathcal N\subseteq
\FF_\XX$ with the projection $\FF_\XX^\Sha\longrightarrow \PP$. One observation
we can begin to make is the following.

\begin{lemma} The composition $\PP_\mathrm{mon}\xrightarrow{\cong} \mathcal N \hookrightarrow 	\FF_\XX^\Sha \twoheadrightarrow \PP$
is an isomorphism of collections if, and only if, the set of relations 
$\RR$ is a Gr\"obnber basis for $\PP$. 
\end{lemma}

\begin{proof}
The normal forms generate $\PP$ if and only if this map is surjective, which simply
amounts to the statement that every tree polynomial can be reduced
modulo $\RR$ to a normal form: this is always true, since we have
available to us the long division algorithm. At the same time,
this map is injective if and only if every tree polynomial has a unique normal form
modulo $\RR$, which is precisely the condition that our set of
relations constitutes a Gr\"obner basis.
\end{proof}

Thus, one can think of monomial operads (shuffle, symmetric, or ns) 
as certain pivotal objects that have a simpler ``multiplication table''
than generic operads, and can be used as a first approximation to studying
these latter objects. More precisely, one can in general prove that the
homotopy type of an algebraic operad with a Gr\"obner basis can be obtained
by perturbing the homotopy type of the corresponding monomial operad,
which allows to make concrete informal claims such as the previous one.
Since we are interested in studying quadratic operads, 
let us begin by studying quadratic monomial operads and to do so,
let us introduce them proper.

\begin{definition}
Let $\XX$ be a set collection, and let $\RR$ be a set of tree monomials
in $\FF_\XX$. We call $\FF(\XX,\RR)$ the monomial operad associated to the
datum $(\XX,\RR)$. 
\end{definition}

Note that, just like in the case of associative algebras, one can apply
a linear change of coordinates to a monomial operad to obtain an operad
that is not presented by relations that are tree monomials. We will only
consider monomial operads with a given presentation or, what is the same,
``monomial data'' $(\XX,\RR)$ as above.

\begin{lemma}
Let $(\XX,\RR)$ be a shuffle quadratic monomial datum, and let us write
$(\XX^\vee,\RR^\perp)$ for the Koszul dual datum. 
\begin{tenumerate}
\item The shuffle tree monomials
not divisible by any element of $\RR$ form a linear basis of
$\PP = \FF(\XX,\RR)$.
\item  The set of shuffle tree monomials for which every
quadratic divisor belongs to $\RR$ forms a basis of $\PP^!$.
\end{tenumerate} 
\end{lemma}

\begin{proof}
This is a particular case of Exercise~\ref{ex:quadratic-GB-dual}.
\end{proof}

Let us now describe the right Koszul complex of a quadratic monomial operad
and show that quadratic monomial operads are Koszul.

\begin{theorem}
Quadratic monomial operads are Koszul.
\end{theorem}

\begin{proof}
Let $\PP$ be the monomial operad associated to a monomial datum $(\XX,\RR)$.
The right Koszul complex $\PP^\antishriek_\tau\circ\PP$ has a basis consisting
of all shuffle tree graftings of the form
\[ x = \gamma_{\pi}(T; T_1,\ldots,T_k) 
\]
where $T$ is a shuffle tree not divisible by any monomial in $\RR$ and
$T_1,\ldots,T_k$ are shuffle trees all whose quadratic divisors are in $\RR$.
The differential acts by first extracting a variable from $T$, that is,
applying $\Delta_{(1)}$ and then $\tau$ at the top, and then grafting this
variable with the corresponding trees at the top. If the resulting tree 
is divisible by a relation, it is zero in $\PP$, and this term vanishes,
and if not, it remains a basis element of $\PP$. Note that $dx$ will 
contain several summands. 

In parallel, let us consider the degree $+1$ map $h$ on $\PP^\antishriek\circ\PP$
that takes a grafting $x$ and, for the first $T_i$ that can be written as a
grafting $\gamma_{1,2}(v ; T',T'')$ for $v$ a variable, replaces
$T$ by the tree $\tilde{T}$ obtained by grafting $v$ on $T$, to obtain the element
\[
\gamma_{\tilde{\pi}}(\tilde{T}; T_1,\ldots,T_{i-1},T',T'',\ldots,T_k).
\]
We send this element to zero if $\tilde{T}$ is not in $\PP^\antishriek$
and to itself otherwise. Up to the correct choice
of signs, one can check that $hd + dh = 1-\iota\pi$, 
where $\pi$ is the projection onto $\kk$ in arity $1$ and $\iota$ the 
corresponding inclusion, so we obtain a contracting homotopy for our complex.
\end{proof}
\subsection{The numerical criterion}

\textbf{Hilbert and Poincar\'e series.} Let us begin
by presenting a numerical criterion that can give a \emph{negative}
answer to the question ``Is this operad Koszul?''. This
depends on certain invariants associated to chain complexes
and graded modules, which we now introduce.

\begin{definition}
Let $\XX$ be a locally finite reduced symmetric sequence
Its \emph{Hilbert series} is the formal power-series
\[
h_\XX(z) = \sum_{n\geqslant 1} \dim \XX(n) \frac{z^n}{n!}.
\]
If $\XX$ is weight graded (so that we may assume that each
weight component is locally finite) we can consider the two
variable Hilbert series
 \[
h_\XX(z,u) = \sum_{w\geqslant 0} h_{\XX^{(w)}}(z) u^w.
\]
\end{definition}

We will mainly be interested in the case of a weight graded operad $\PP$.
When $\PP$ is binary quadratic, the arity $n$ and the weight $w$ are related
by $n = w+1$ and then the two variable Hilbert series carries no new information.
For example, we can compute the following Hilbert series:
\begin{align*}
h_{\As}(z) &=  z + z^2 + z^3 + \cdots = \frac{z}{1-z} \\
h_{\Lie}(z) &=  z + \frac {z^2}2 + \frac{z^3} 3 + \cdots =-\log(1-z) \\
h_{\Com}(z) &=  z + \frac {z^2}2 + \frac{z^3} 6  + \cdots = \exp z -1 \\
h_{\As^-}(z) &=  z +  {z^2} + z^3.
\end{align*}

%Let $(C,d)$ be a chain complex. Its Euler characteristic (when defined)
%is the integer $\chi(C) = \sum_{n\in \mathbb Z} (-1)^n \dim C_n$. 

There is a related invariant, which we now define:
\begin{definition} Let $\PP$ be a weight graded operad, so that
$\B{\PP}$ is a weight graded dg cooperad. The Poincar\'e series
of $\PP$ is, by definition,
\[
p_{\PP}(z,u,t) = \sum_{w,d\geqslant 0,} \dim H_d(\B{\PP}^{(w)}(n)) t^d u^w z^n/n!.
\]
\end{definition}

We can now state and prove a numerical criterion to check if an
operad is \emph{not} Koszul.

\begin{theorem}
Let $\PP$ be a weight graded (non-dg) operad. Then we have the following
functional equation between the Hilbert and the Poincar\'e series of $\PP$:
\[
h_\PP( p_\PP(z,u,-1),u) = p_\PP(h_\PP(z,u),u,-1) = z. \]
Moreover, $\PP$ is Koszul (and in particular, quadratic) if and only
if $p_\PP(z,u,t) = h_{\PP^\eantishriek}(z,ut)$ in which case the previous
equation simplifies to
\[
h_\PP( h_{\PP^\eantishriek}(z,-u),u) = h_{\PP^\eantishriek}(h_\PP(z,u),-u) = z. 
\]
\end{theorem}

\begin{proof}
Let us begin by proving the first functional equation. In this case,
we can consider the two augmented bar complexes of Theorem~\ref{thm:augacyclic},
of the form
\[
\PP\circ_\pi \B{\PP},\qquad \B{\PP}\circ_\pi\PP. 
\]
That theorem states these two complexes are acyclic, which means their
(weight graded) Euler characteristics are equal to the Euler characteristic 
of the symmetric sequence $(0,\kk,0,\ldots)$ where $\kk$ is placed in arity one,
weight zero, and homological degree zero, so that its Euler characteristic is
$z$. On the other hand, the Euler characteristic of $\B{\PP}$ is
$p_\PP(z,u,-1)$ and that of $\PP$ is its Hilbert series (since $\PP$ carries no
homological degrees). Thus, we obtain that 
\[
z = \chi(\PP\circ\B{\PP}) = \chi(\PP)\circ \chi(\B{\PP}) = 
		h_\PP(p_\PP(z,u,-1,u)),
\]
while $z = p_\PP(h_\PP(z,u),u,-1)$ is proved in the same fashion. Suppose
now that $\PP$ is Koszul, which means that $\dim H_d(\B{\PP})_{(w)} = 0$
unless $w=d$. This is true if and only if the coefficient of $z^wu^d$ in
$p_\PP(z,u)$ is zero unless $w=d$, in which case we have that
$H_d(\B{\PP})_{(d)} \simeq  \PP^\antishriek_{(d)}$ and
\[
p_\PP(z,u,t) = \sum_{d\geqslant 0} \dim \PP^\antishriek_{(d)}(n) 
	\frac{z^n}{n!} (tu)^d = h_{\PP^\antishriek}(z,tu).
\]
Plugging this back into the equation relating the Hilbert and Poincar\'e
series for $\PP$ gives us the result.
\end{proof}

At this point it is useful to come back to the situation when $\PP$
is binary and quadratic, in which case the functional equation
simplifies even further. In this case, we notice that
$h_\PP(z,u) = u h_\PP(zu,1)$ so we obtain the equations  
\[
h_\PP(-h_{\PP^\antishriek}(-z)) = h_{\PP^\antishriek}(-h_\PP(-z)) = z. 
\]
Since $\PP^!$ and $\PP^\antishriek$ have the same Hilbert series (even
though one is homologically graded and the other is not, we are
ignoring the $t$ variable here) we will use the functional equation above
when with the operad $\PP^!$ most of the time.



\begin{example}
The quadratic operad $\As^-$ of anti-associative algebras is not Koszul.
Indeed, one can compute the first few terms of its sign-modified inverse of the 
Hilbert series to obtain
\[
z + z^2 + z^3  - 4 z^5 + \textrm{O}(z^{6}).
\]
Since the coefficient of $z^{5}$ is negative, we conclude that this operad cannot be Koszul. 
\end{example}

%
%\begin{example}
%The quadratic operad $\As^-$ of anti-associative algebras is not Koszul.
%Indeed, one can compute the first few terms of its sign-modified inverse of the 
%Hilbert series to obtain
%\[
%z + z^2 + \frac{3}{2}z^3  + \frac 52 z^4 + \frac{17}4 z^5  + 7z^6 + 
%	\frac{21}{2} z^7 + \frac{99}8 z^8
%+ \frac{55}{16} z^9 - \frac{715}{16} z^{10}  + \textrm{O}(z^{11}).
%\]
%Since the coefficient of $z^{10}$ is negative, we conclude that this operad
%cannot be Koszul. 
%\end{example}

\begin{example}
The quadratic operad $\mathsf{Nov}$ of Novikov algebras is not Koszul.
Indeed, A. Dzhumadil'daev computed that $\mathsf{Nov}$ has the same Hilbert
series $h(z)$ as its Koszul dual, and that 
 \[ h(-h(-z)) = z + \frac 16 z^5 + \mathrm{O}(z^6) \neq z.
 \] 
 Thus, the Novikov operad is not Koszul.
\end{example}


 
\subsection{Exercises}

\begin{question}
Use the Haskell Calculator to show that $\mathsf{Nov}$ has Hilbert series
\[
h_\mathsf{Nov}(z) = z +2 \frac{z^2}{2!} +6 \frac{z^3}{3!} + 20 \frac{z^4}{4!} + 70 \frac{z^5}{5!} +
\mathrm{O}(z^6).
\]
Show that $\mathsf{Nov}$ is Koszul dual to $\mathsf{Nov}^\mathrm{op}$
and conclude that $\mathsf{Nov}$ is not Koszul.
\end{question}
\begin{question}\label{ex:preLieHilbert}
Show (using Gr\"obner bases or otherwise) that the Hilbert series $h$
of the operad $\mathsf{PreLie}$ satisfies the equation
\[ h(z) = z\exp h(z) .\]
Use this to compute a few terms of the series, or directly
show that $\dim\mathsf{PreLie}(n) = n^{n-1}$. Compute
the Hilbert series of $\mathsf{Perm}$ and verify that
the numerical criterion holds.
\end{question}

\begin{figure}

\begin{verbatim}

Configuration:

actions:        normalise count 
count limit:    5
arity limit:    5
time limit:     none
output:         
field:          rationals
operad type:    unsigned shuffle operad
measure:        permutation reverse degree-lexicographic 
signature:      x(2) y(2)
theory:
  x(x(1 3) 2)  -  x(x(1 2) 3)
  y(1 x(2 3))  -  x(y(1 2) 3)
  y(1 y(2 3))  -  x(y(1 3) 2)
  x(x(1 2) 3)  -  x(1 x(2 3))  -  x(y(1 2) 3)  +  y(x(1 3) 2)
  x(x(1 3) 2)  -  x(1 y(2 3))  -  x(y(1 3) 2)  +  y(x(1 2) 3)
  y(1 x(2 3))  -  y(y(1 3) 2)  -  y(1 y(2 3))  +  y(y(1 2) 3)


Arity: 4   
Stable rewrite rules: 6   
Current critical pairs: 26   
Queued critical pairs: 0

Arity: 5   
Stable rewrite rules: 12   
Current critical pairs: 105   
Queued critical pairs: 68

Stopped at arity 5.

Counting normal forms:

  arity | normal forms
     3  |       6
     4  |      20
     5  |      70

*Main> 
\end{verbatim}
\caption{Counting normal forms in the Novikov operad.}
\end{figure}