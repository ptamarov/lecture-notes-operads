\section{Shuffle operads}\label{lecture:shuffleops}

\textbf{Goal.} Introduce shuffle operads
and prove that the free symmetric operad
on a reduced symmetric collection is
isomorphic, as a shuffle operad,
to the free shuffle operad on the
corresponding shuffle collection. 

\subsection{Shuffle operads}

Recall that the category of ns collections
on some category $\mathsf{C}$ 
consists of those pre-sheaves on
the category of finite ordered sets
and order preserving bijections with
values in $\mathsf{C}$: a ns collection
on $\mathsf{C}$ is simply a list of
objects of $\mathsf{C}$ indexed by
the non-negative integers (considered
as totally ordered sets of finite 
cardinality). 

\begin{definition}
An ordered partition $\pi$ of length $n$
of a finite totally order 
set is called \emph{shuffling} if
$\min \pi_i < \min \pi_{i+1}$ for
each $i\in \underline{n-1}$. Equivalently, 
a surjection $f:I\longrightarrow \underline{n}$
with $I$ a totally ordered set
is called \emph{shuffling} if
$\min f^{-1}(i) < \min f^{-1}(i+1)$
for each $i\in \underline{n-1}$.
\end{definition}

Although totally ordered
sets along with bijections form a rather
dull category, this category
admits a composition product, which we call
the \emph{shuffle composition product},
defined as follows, and which will turn
out to be crucial for our purposes.

\begin{definition}
For each pair of ns collections $\XX$
and $\YY$, we define the ns collection
$\XX\circ_{\Sha} \YY$ so that on each
totally order finite set we have that 
\[
(\XX\circ_{\Sha} \YY)(I)
	=
	 \bigoplus_{\substack{r\geqslant 1
	 	\\ f: I \longrightarrow \underline{r}}}
	 \XX(\underline{r})\otimes 
	 \YY(f^{-1}(1))
	 	\otimes
	 		\cdots
	 			\otimes
	 				\YY(f^{-1}(r))
\]
where the sum runs through all $r\geqslant 1$
and all possible 
shuffling surjections 
$f : I \longrightarrow \underline{r}$.
\end{definition}

One can prove that this product is
associative, in the same way that one
proves $\circ_\Sigma$ and $\circ_\mathrm{ns}$
are. In some way, the shuffle composition
product interpolates between the symmetric
composition product, which contains ``too
many'' summands, and the ns composition
product, which contains too few. We
leave the following proposition as an exercise.

\begin{proposition}
The category of ns collections along with
the shuffle composition product is 
monoidal with the same unit as that of the
ns composition product. \qed
\end{proposition}

Note that we can also define a shuffle
Cauchy product, by looking at shuffling
partitions of a finite ordered set that
have length two. Although we will not study
the resulting monoidal category here,
we remark it gives rise to interesting
monoids, usually known as shuffle algebras;
see~\cite{Bremner2016}*{Chapter 4} and
\cite{Mendez2010} for more on both
shuffle and twisted associative algebras.

\begin{definition}
A \emph{shuffle operad} is a monoid in the category of 
ns collections with the shuffle composition 
product. 
\end{definition}

Thus, a shuffle operad consists of the
datum of a ns sequence $\PP$ along with
shuffle composition maps, one for each
shuffle partition $\pi$ of a finite ordered
set $I$ of the form
\[
\gamma_\pi : \PP(\underline{r})\otimes
		\PP(\pi_1)\otimes\cdots\otimes\PP(\pi_r)
		 	\longrightarrow \PP(I)
\]
that satisfy suitable associativity and
unitality axioms. Precisely, let us pick
a finite totally ordered set $I$,
a shuffling partition $\pi$ of $I$,
and let us assume that we pick a shuffling
partition $\pi^{(i)}$ of each block of $\pi$.
There is a unique way to order the collection
of blocks of these to obtain a shuffling
partition $\pi'$ of $I$.
For each part $\pi_i$ of $\pi$
and each $(g_i;\vec{h}_i) \in \PP(\pi_i)\otimes
\PP[\pi^{(i)}]$, let us write
$f_i = \gamma_{\pi^{(i)}}(g_i;\vec{h}_i)$,
and let $\vec{h}$ be obtained for the
tuple $(\vec{h}_1,\ldots,\vec{h}_r)$
by reordering the entries according to $\pi'$.
Then
\[ 
\gamma_\pi(f ; f_1,\ldots,f_r) =
 \gamma_{\pi'}(\gamma_\pi(f;g_1,
 \ldots,g_r); \vec{h} ).
	\]
Moreover, for each finite set $I$,
if $\{I\}$ and $I$ denote the corresponding
partitions into one block and into singletons,
we have a fixed unit $1\in \PP(1)$ such that for
every $\nu\in\PP(I)$ we have
\[ \gamma_{\{I\}}(1;\nu) = \nu , 
\quad  \gamma_I(\nu ; 1,\ldots,1 ) = 
\nu.\]
Naturally, one can consider partial compositions
on a shuffle operad, but carefully noting that
for each $i$, there exist many different
shuffling partitions $\pi$ of the form
\[
 (1,\ldots,i-1,A,j_1,\ldots,j_s)
 	\]
 where $\min(A) = i$. Namely, for each
 $\underline{n}$ we need simply choose a subset 
 $A$ of $\underline{n}\smallsetminus \underline{i-1}$ that contains
 $i$, and this can be done by choosing a subset
 of $\underline{n}\smallsetminus \underline{i}$ and appending 
 $i$. 
\begin{definition}
An ideal of a shuffle operad $\PP$ is a
subcollection $\mathcal{I}$ such that
\[ \gamma_\pi(\nu_0;\nu_1,\ldots,\nu_r)\in 
\mathcal{I}\] if at least one of $\nu_i$
is in $\mathcal{I}$ for some $i\in \llbracket r\rrbracket$.
\end{definition}

As we will see later, ideals of shuffle operads
are slightly more refined than those in
symmetric operads. For example, the ideal 
generated by the left comb $(x_1x_2)x_3$
in a symmetric operad
automatically contains its two translates,
while in a shuffle operad, the three ideals
corresponding to these three possible shuffle
tree monomials are different. 

\subsection{Free shuffle operad}

Let us now give an explicit description
of the free shuffle operad on a 
collection. Since we have already defined
the free symmetric and non-symmetric operad on
a collection (of the appropriate kind), we
already have almost all the language necessary to 
define it.

\begin{definition}\label{def:canonicalplanarorder}
Let $\tau$ be a planar tree, which
we draw on the plane with the clockwise
orientation. Begin
at the left side of root edge, and traverse the 
``boundary'' of the tree in the clockwise
direction. This path will meet the vertices
of $\tau$ in some order, and we call this
total order the \emph{canonical planar order}
of its vertices.
\end{definition}

Observe that this
also orders the edges of $\tau$, and
the leaves (which are given the usual
left-to-right planar order). 

Now let $\XX$ be a ns collection and let
$T$ be a planar tree monomial with variables
in $\XX$, and let us pick a bijective labeling
$\mathsf{n} : L(\tau) 
\longrightarrow \underline{n}$ of the leaves of
$\tau$. This induces a
labelling of the vertices of $\tau$
inductively by inductively labelling
$v$ with the minimum label appearing 
among its set of children. 

\begin{definition}
We say a leaf labelling of a planar tree 
monomial $T$
is shuffling if the induced order on the
children of each of its vertices coincides with
the canonical planar order. A pair
$(T,\mathsf{n})$ where $\mathsf{n}$
is a shuffling leaf labeling is called
a shuffle tree monomial.
\end{definition}

We now define the ns collection 
$\mathrm{Tree}^\Sha_\XX$ so that for each
finite totally ordered set $I$ the set
$\mathrm{Tree}^\Sha_\XX(I)$ consists of those
shuffle tree monomials on $\XX$ with shuffling
labellings by $I$. We write $\FF^\Sha_\XX$
for the corresponding linear ns collection.

\begin{figure}[h]
\[ 
	\leftc{}{}{1}{2}{3}
 \quad
		\leftc{}{}{1}{3}{2} 
		\qquad
		\rightc{}{}{1}{2}{3}
		\]
		\caption{The three
		shuffle trees with three
		leaves on a binary generator.}\end{figure}

Suppose that $T$ and $T'$ are shuffle tree
monomials on $\underline{n}$ and $\underline{m}$, that $i\in \underline{n}$
and that we pick a shuffling partition $\pi$ of
$\underline{m+n-1}$ whose only non-singleton part
is of the form 
\[ \{i=j_1,j_2,\dots,j_m\}.\]
We define the tree monomial $T\circ_\pi T'$
by grating the tree $T'$ at the leaf of $T$
labelled by $i$, with its leaf labels
renumbered through the unique order
preserving bijection $j_i \longmapsto
i$, and we renumber the leaf labels
of $T$ distinct from $1,\ldots,i-1$ using
the remaining blocks of $\pi$. This
defines the ``partial shuffle composition''
of shuffle tree monomials.

\begin{figure}
\begin{tikzpicture}
	\begin{pgfonlayer}{main}
		\node [style=inner] (0) at (0, 0) {};
		\node [style=leaf] (1) at (-4, 2) {$1$};
		\node 				(2a) at (-3, 2) {$\cdots$};
		\node [style=leaf] (2) at (-2, 2) {$i-1$};
		\node [style=inner] (3) at (-0.5, 2) {};
		\node [style=leaf] (4) at (-2.25, 4) {$i$};
		\node [style=leaf] (4a) at (-1.25, 4) {$\cdots$};
		\node [style=leaf] (5) at (1, 4) {$j_m$};
		\node [style=leaf] (6) at (0, 4) {$j_{m-1}$};
		\node [style=leaf] (7) at (1, 2) {$k_1$};
		\node [style=leaf] (8) at (2, 2) {$k_2$};
		\node 				(8a) at (3, 2) {$\cdots$};
		\node [style=leaf] (9) at (4, 2) {$k_n$};
		\node [style=leaf] (10) at (0, -2) {};
	\end{pgfonlayer}
	\begin{pgfonlayer}{bg}
		\draw (0.center) to (1.center);
		\draw (2.center) to (0.center);
		\draw (0.center) to (3.center);
		\draw (3.center) to (4.center);
		\draw (3.center) to (6.center);
		\draw (3.center) to (5.center);
		\draw (7.center) to (0.center);
		\draw (0.center) to (8.center);
		\draw (0.center) to (9.center);
		\draw (0.center) to (10.center);
	\end{pgfonlayer}
\end{tikzpicture}

\caption{The two-level trees corresponding
 to partial compositions of shuffle operads}
 \label{fig:twolevel}
\end{figure}
We may as well define the ``total shuffle
composition'' of a tree $T_0$ with trees
$T_1,\ldots,T_n$ along a shuffling partition
$\pi = (\pi_1,\ldots,\pi_n)$ with $T_i$ having
as many leafs as $\pi_i$ for each $i\in \underline{n}$
Concretely, we consider for each such $i$
the unique order preserving bijection
between $\pi_i$ and the labels of $T_i$,
and graft $T_i$ at the input of $T_0$
labelled by $\min \pi_i$. 

\begin{proposition}
The shuffle composition of shuffle tree
monomials is again a shuffle tree
monomial.
\end{proposition} 

\begin{proof}
This is Exercise~\ref{ex:shufflecomp}. The
idea is to note that the local increasing
condition is not broken, and this is clear
on each $T_i$ since we simply relabelled their
leafs with an isomorphic totally ordered
set, 
while it is not broken on
$T_0$ since we grafted the $T_i$s using
a shuffling partition.
\end{proof}

With this at hand, we can state and prove the
main result in this section. 

\begin{proposition}
The ns collection
$\FF_\XX^\Sha$ with its corresponding
shuffle composition is the \emph{free shuffle
operad} generated by $\XX$, where the
inclusion $\XX\longrightarrow \FF_\XX^\Sha$
sends an element in $\XX$ to the corresponding
corolla with its unique shuffling leaf
labelling. \qed
\end{proposition}

\subsection{Forgetful functor}

Since every finite totally order 
set $I$ is in particular a finite set
$I^{\f}$
after forgetting the order,
we have a functor
$\XX \longmapsto \XX^{\f}$ that assigns
a symmetric collection $\XX$ to the ns
collection $\XX^{\f}$ such that
\[ \XX^{\f}(I) = \XX(I^{\f}) \]
for each finite order set $I$. 
We call this the \emph{forgetful functor}
from symmetric to ns collections. 
The following will be central in what
follows.

\begin{proposition}
The forgetful functor $\SMod \longrightarrow 
\nsMod$ is strong monoidal for the corresponding 
symmetric and shuffle composition products
when restricted
to \emph{reduced} collections, in the sense
that for each pair $\XX$ and $\YY$ with
$\YY$ reduced there is a natural isomorphism
\[
(\XX\circ_\Sigma \YY)^{\f} \longrightarrow
 \XX^{\f}\circ_\Sha \YY^{\f}.
\]
\end{proposition}

\begin{proof}
Let us begin by proving that if $\YY$ is a
reduced symmetric sequence then $\YY^{\otimes n}$
is a free $S_n$-module for every $n\geqslant 1$.
This is of course true for $n=1$. For $n>1$,
it suffices to exhibit an $S_n$-basis. 
For each finite totally ordered set $I$, let
us consider the components of $\YY^{\otimes n}(I^{\f})$,
and note that since $\YY$ is reduced they are of the form
\[
\YY(\pi_1)\otimes \cdots\otimes\YY(\pi_n)
\]
where $\pi$ is a partition of $I$ into $n$ blocks with
at least one element. For each such partition $\pi$
of $I$, there exists a unique permutation $\sigma\in S_n$
such that $(\sigma\pi)_i = \pi_{\sigma^{-1}(i)}$ is
shuffling, and this proves that $\YY^{\otimes n}(I^{\f})$ is
isomorphic to the free $S_n$-module generated
by $(\YY^{\f})^{\otimes_\Sha n}(I)$.
It follows that for each $n\geqslant 1$ we have a
natural isomorphism
\[ 
\XX(n)\otimes_{S_n}\YY^{\otimes n}(I^{\f})
 \longrightarrow 
  \XX^{\f}(n)\otimes (\YY^{\f})^{\otimes_\Sha n}(I)
\] 
which gives us the desired isomorphism
$
(\XX\circ_\Sigma \YY)^{\f} \longrightarrow
 \XX^{\f}\circ_\Sha \YY^{\f}.
$
\end{proof}

\begin{corollary}
For each reduced symmetric collection $\XX$, 
there is a natural isomorphism of shuffle operads
\[
(\FF_\XX^\Sigma)^{\f}
 \longrightarrow \FF_{\XX^{\f}}^\Sha.
\]
Moreover, if $I$ is an ideal in $\FF_\XX^\Sigma$
then $I^\f$ is an ideal in $\FF_{\XX^\f}^\Sha$ and the
resulting quotient shuffle operads are naturally
isomorphic via the induced map
\[
(\FF_\XX^\Sigma/ I )^{\f}
 \longrightarrow \FF_{\XX^{\f}}^\Sha/ I^\f.
\]
\end{corollary}

In particular, shuffle tree monomials on $\XX^{\f}$, when
considered with their non-planar tree structure,
give us a basis of the free symmetric operad on 
$\XX$, and we can study any presentation of a symmetric
operad through the resulting presentation of the
corresponding shuffle operad. 

\subsection{Exercises}

\begin{question}\label{ex:shufflecomp}
 Show that the shuffle composition of shuffle tree
monomials is again a shuffle tree
monomial.
\end{question}
 
\begin{question}
Use the definition of shuffle trees
to compute a basis
of $\mathrm{Tree}_\XX^\Sha(4)$ in case $\XX$ consists
of a single symmetric or antisymmetric
operation. What happens if the operation
is not symmetric?
\end{question}

\begin{question} Explain how 
$\XX^f \circ_\Sha \YY^f$ fails
 to identify with  $(\XX\circ_\Sigma \YY)^f$ 
 in case $\YY$ is
 not reduced. 
\end{question}

\begin{question}
Go through the definition of the shuffle
compositions $\gamma_\pi$ for shuffle
tree monomials, and show that it maps
shuffle tree monomials to shuffle tree
monomials.
\end{question}

\begin{question} 
Give an example of a shuffle operad that
is not obtained from a symmetric operad
through the forgetful functor. \emph{Suggestion:}
ideals coming from symmetric operads must be
stable under the (now non-existent) group action.
Can you find a shuffle ideal that is ``not very
symmetric''?
\end{question}

\begin{question}
Write down a presentation of the following as
shuffle operads: the  
commutative operad, the
Lie operad, the
associative operad, and the
operad of $3$-ary totally commutative
associative algebras.
\end{question}

\begin{question}
Repeat the theme of the last four exercises
with any other (quadratic) operad
of your choice.
\end{question}