\section{Free operads and presentations}\label{lecture:freeops}

\textbf{Goals.} We will define the free operad functor
and with it algebraic operads
by generators and relations. With this at
hand we will define quadratic and quadratic-linear
presentations of operads. 

\subsection{Trees}

Operads and their kin are gadgets modeled after
combinatorial graph-like objects. Operads themselves are modeled after
rooted trees, so it is a good idea to have a concrete definition of 
what a rooted tree is. We will also consider planar rooted trees,
and trees with certain decorations, so it is a good idea to digest
the definitions carefully to later embellish them.

A rooted tree $\tau$ is the datum of a finite set $V(\tau)$
of vertices along with a partition $V(\tau) = 
\mathrm{Int}(\tau)\sqcup L(\tau) \cup R(\tau)$,
where the first block consists of the \emph{interior} vertices, the second 
of the leaves, and the third $R(\tau)$ is
a singleton, called the root of $\tau$. We also require there is 
a function $p :V(\tau)\smallsetminus R(\tau) \longrightarrow V(\tau)$,
called the \emph{parent function of $\tau$},
describing the edges of $\tau$, 
with the following properties: call a vertex $v\in V(\tau)$ a child 
of $w\in V(\tau)$ if $v\in p^{-1}(w)$. Then:
\begin{tenumerate}
\item The root $r\in R(\tau)$ has exactly one child.
\item The leaves of $\tau$ have no children.
\item For each non-root vertex $v$ there exist a unique sequence
$(v_0,v_1,\ldots,v_k)$ such that $p(v_{i-1}) = v_{i}$ for $i\in \underline{k}$
with $v_0 = v$ and $v_k = r$.
\end{tenumerate}
We will call a non-leaf vertex that has no children a \emph{stump}
(or an endpoint, or a cherry-top).
A tree is reduced if has no stumps. We say it is 
series-reduced if all of its non-root and
non-leaf vertices have at least two children. We will also call
the root the (unique) \emph{output vertex} of $\tau$, and the leaves
the \emph{input vertices} of $\tau$. 

A planar rooted tree is a rooted tree $\tau$
along with a linear order in each of the fibres
of the parent function $p$ of $\tau$. In short,
the children of each vertex are linearly ordered,
so we are effectively considering a drawing of
$\tau$ in the plane, where the clockwise
orientation gives us the order at each
vertex. 

Two rooted trees $\tau$ and $\tau'$ are isomorphic
if there exists a bijection $f : V(\tau) \longrightarrow 
V(\tau')$ that preserves the root, the input vertices and the
interior vertices, so that $p'\circ f = p$ where we
also write $f$ for the induced bijection $f : V(\tau)\smallsetminus r \longrightarrow 
V(\tau')\smallsetminus r'$. Two planar rooted trees
are isomorphic if in addition $f$ respects the linear order
at each vertex.

For example, consider the rooted tree $\tau$ with
$V = \{1,2,3\}\cup \{4,5\}\cup \{0\}$, that is,
three leaves, two interior vertices and the root.
Then the choice of $p : [5] \to \llbracket 5 \rrbracket$ 
with $p(\{1,2\}) = 4$, $p(\{3,4\}) = 5$, $p(5) = 0$
gives a tree isomorphic to the one with 
with $p(\{1,2\}) = 3$, $p(\{3,4\}) = 5$, $p(5) = 0$.
On the other hand, if we consider the vertices linearly
ordered by their natural order, these two planar rooted trees 
are no longer isomorphic. 

\begin{definition}
For a finite set $I$, an $I$-labeled tree $T$
is a pair $(\tau,f)$ where $\tau$ is a 
reduced rooted tree, along with
a bijection $f : I \longrightarrow L(\tau)$.
Two $I$-labeled trees $T$ an $T'$ are isomorphic
if there exists a pair $(g,\sigma)$ where
$g$ is an isomorphism between $\tau$ and $\tau'$
and $\sigma$ is an automorphism of $I$ such that
$g\mid_{L(\tau)}\circ f = \sigma\circ f'$. 
\end{definition}

Suppose that $(\tau,f)$ is an $I$-tree and that
$(\tau',f')$ is a $J$-tree, and that $i\in I$. We define
$K=I\cup_i J = I\sqcup J \smallsetminus i$ and the
$K$-tree $\tau\circ_i \tau'$ as follows:
\begin{tenumerate}
\item Its leaves are $L(\tau\circ_i \tau') = L(\tau)\sqcup L(\tau')\smallsetminus f^{-1}(i)$.
\item Its internal vertices are $V(\tau)\sqcup V(\tau')$, with
root $r$. 
\item The parent function $q$ is defined by declaring that:
	\begin{titemize} 
	\item $q$ coincides with $p$ on $V(\tau)$, 
	\item $q(w) = p(f^{-1}(i))$ if
$w$ is the unique children of the root of $\tau'$, 
	\item  $q$
coincides with $p'$ on $V(\tau')\smallsetminus \{r',w\}$.
\end{titemize}
\item The leaf labeling is the unique bijection $L(\tau\circ_i \tau') \longrightarrow I\circ_i J$ extending $f$ and $f'$.
\end{tenumerate} 

\subsection{Tree monomials}
Let us now consider an 
(unbiased) reduced symmetric sequence $\XX$ which
we will think of as an \emph{alphabet}. A tree monomial in the
alphabet $\XX$ 
is a pair $(\tau,x)$ where $\tau$ is a reduced rooted
tree and $x : \mathrm{Int}(\tau) \longrightarrow \XX$ is a map
with the property that $x(v) \in \XX(p^{-1}(v))$. Observe that
reduced sequences and reduced trees correspond to each other, in the
sense that with this definition we can only decorate a stump
of $\tau$ with an element of $\XX(\varnothing)$. 

An $I$-labeled
$\XX$-tree $T$ is a triple $(\tau,x,f)$ where $(\tau,f)$ is $I$-labeled
and $(\tau,x)$ is an $\XX$-tree. We will say that $(\tau,x,f)$
is a (symmetric) tree monomial if $\XX$ is symmetric. If it
is just a collection, we will say that $(\tau,x,f)$ is a
ns tree monomial. In particular, if $T$ is an $I$-labeled
tree, and if $\sigma \in \Aut(I)$, there is another 
$I$-labeled tree $\sigma(T)=(\tau,f\sigma^{-1})$. 


Suppose that $T = (\tau,x,f)$ is a tree monomial on an
alphabet $\XX$, and let us pick a vertex $v$ of $\tau$
and a permutation $\sigma$ of the set $C = p^{-1}(v)$ of
children of $v$. We define the tree $\tau^\sigma$ as
follows: the datum defining $\tau$ remains unchanged
except $p$ is modified to $p^\sigma$ so that 
\[ p^\sigma(w) = 
\begin{cases}
 p(w) & \text{if $p^2(w)\neq v$} \\
 p(\sigma^{-1}(w')) & \text{if $p(w)=w'\in C$}.
\end{cases}
\]
Briefly, we just rearrange the vertices of $\tau$
that are children of $v$ using $\sigma$. With this at hand, we define
$T^\sigma$ to be the tree monomial with underlying 
tree $\tau^\sigma$ and with $x$ modified to $x^\sigma$ so that
\[ x^\sigma(w) = 
\begin{cases}
 \sigma x(v) & \text{if $v=w$,} \\
 x(\sigma^{-1}(w')) & \text{if $p(w)=w'\in C$}.
\end{cases}
\]
Note that it is possible some children of $v$ are
leaves, in which case the definitions make sense if
we think of leaves as decorated by the unit of
$\kk$.

\begin{example}
Let us consider the alphabet 
$\XX = \XX(2) = \{\ast \}$ where
the unique operation is antisymmetric.
Then we have the following equalities
of symmetric tree monomials:
\[ 
	\leftc{}{}{2}{1}{3}  \mathsymbol{1.5}{=}
\rightc{}{}{3}{1}{2}  
   \mathsymbol{1.5}{=-}
 {\leftc{}{}{1}{2}{3}}
  \mathsymbol{1.5}{.}\]
 
\end{example}


Let us now define for each $n\geqslant 1$ the
space $\FF_\XX(I)$ as the span of all tree monomials
$T = (\tau,f,x)$ on $\XX$ with leaves labeled by $I$,
modulo the subspace generated by all elements of the form
\[ R(T,v,\sigma) = T - T^\sigma \]
where $\sigma$ ranges through $\Aut(p^{-1}(v))$ as
$v$ ranges through the vertices of $\tau$. In case
all children of $v$ are leaves, this is saying that
the tree where $x_v$ is replace by $\sigma(x_v)$
is equal to the tree where the leaves of $T$ that are
children of $v$ are relabeled according to $\sigma$.
We also require that tree decorations behave like
tensors, so that $T = T_1+T_2$ if the decoration
of $T$ at a vertex $v$ is of the form $x_1 + x_2$
and for $i\in [2]$ the tree $T_i$ coincides with
$T$ except that it is decorated by $x_i$ at $v$.



\subsection{The free operad}
An algebraically inclined way to construct
(algebraic) operads is through generators and
relations. There is a forgetful functor
from the category of operads to the category
of collections. In general, it admits a left
adjoint, which is the free operad functor.

\begin{definition}
The \emph{free symmetric operad} on $\XX$ is the
symmetric sequence $\FF_\XX$ along with the composition
law obtained by grafting of trees. More precisely,
suppose that $T\in \FF_\XX(I)$ and that $T'\in\FF_\XX(J)$,
and that $i\in I$. We define $T'' =T\circ_i T' \in 
\FF_\XX(I\cup_I J)$ by taking its underlying labeled
tree to be $\tau\circ_i \tau'$, and by decorating it
in the unique way which extends the decorations of 
$T$ and $T'$.
\end{definition}

The following lemma shows that this indeed defines an operad.

\begin{lemma}
Tree grafting respects both $I$-tree 
isomorphisms and the relations $T\sim T^\sigma$
above, and hence is well defined on $\FF_\XX$.
\end{lemma}

\begin{proof}
This is Exercise~\ref{ex:grafting}.
\end{proof}

The functor $\XX\longmapsto \FF_\XX$ is in fact a \emph{monad},
the monad of rooted trees,
which gives us the definition of an operad alluded to 
in Lecture~\ref{lecture:theintro}. The
advantage of this `monadic approach' is its 
flexibility, which allow us to define other
operad like structures, like the ones 
mentioned in the Lecture~\ref{lecture:theintro}.
In this direction, a curious reader 
can consider the following 
equivalent definition; see also Exercise~\ref{ex:colim}.

\begin{definition}
The free operad generated by a symmetric
collection $X$ is defined inductively by
letting  $\FF_{0,X}=\kk$ be spanned
by the `twig' (tree with no vertices and one edge)
in arity zero and by
\[ \FF_{n+1,\XX} = \kk\oplus (\XX\circ  \FF_{n,\XX} ) \]
for $n\geqslant 0$. There are inclusions
 $i_n : \FF_{n,\XX} \longrightarrow \FF_{n+1,\XX}$
  for $n\geqslant 0$ defined inductively by
  $i_0 : \kk \longrightarrow \kk\oplus\XX$
 the inclusion in the first coordinate and
 $i_{n+1} = 1\oplus (1\circ i_n)$ for $n\geqslant 0$.
These are all split innjections, and
we define $\FF_\XX = \varinjlim_n \FF_{n,\XX}$.
\end{definition}

The composition maps are defined by induction,
and one shows these satisfy all necessary
constraints by induction;
see~\cite{Vallette2008} for details on
this approach, and conceptual explanation 
for the shape of the ``free object'' in categories
of operads, dioperads, properads and PROPs.
Intuitively, the previous definition says that an element of
$\FF_\XX$ is either the twig, or corolla
with $n$ vertices decorated by $\XX$, whose leaves
have on them an element of $\FF_\XX$. The
final shape of $\FF_\XX$ will however
depend on the symmetric structure of $ \XX$. 
 \subsection{Exercises}


 
 \begin{question}
Let $\XX$ be a collection such that $\underline{\XX} = 
\XX(2)$. Compute a basis of tree monomials 
for the free operad over
$\XX$ in case $\XX(2)$ is:
\begin{tenumerate}
\item The regular representation of $S_2$.
\item The sign representation of $S_2$.
\item The trivial representation of $S_2$. 
\end{tenumerate} 
In all cases, decompose the $S_3$-module $\mathcal{F}_\XX(3)$
into irreducible representations.
\end{question}


\begin{question}\label{ex:grafting}
Show that tree grafting respects both $I$-tree
isomorphism and the relation $T\sim T^\sigma$,
and hence descends to $\FF_\XX$.
\end{question}

\begin{question}
Suppose that $\XX$ is an alphabet (in sets) that is
finite in each arity and such that $\XX(n) = \varnothing$
for $n=0,1$. Show that $\mathcal{F}_\XX$ is finite
in each arity. 
\end{question}
 
\begin{question}
Define non-symmetric tree monomials over a ns alphabet
$\XX$ and thus define the free \emph{non-symmetric}
operad over a collection $\XX$.
\end{question}

\begin{question}\label{ex:colim}
Read the statement and proof of \emph{Theorem 5.4.2} in
\cite{Loday2012} that the colimit construction briefly
described in the lecture notes does give the free
operad on a symmetric collection.
\end{question}

\begin{question}
Consider the map from ns collections to symmetric sequences
that assigns $\XX$ to $\Sigma\otimes \XX$ such that
$(\Sigma\times \XX)(n) = \Sigma_n\times \XX(n)$ with its
corresponding symmetric group action. What is the relation
between the free ns operad on $\XX$ and the free symmetric
operad on $\Sigma\times \XX$?
\end{question}

\begin{question}
Let $V$ be an $S_2$-module, and let $\XX$ be the
symmetric collection with $\XX(2) = V$ and zero
everywhere else. Show that $\mathcal{F}_\XX(3)$
consists of three copies of $V^{\otimes 2}$ and
describe explicitly the action of $S_3$ on it.
\end{question}

\begin{question}
Show that the construction of the free operad we carried
out during Lecture~\ref{lecture:freeops} indeed defines the free
operad on $\XX$ where $i:\XX \longrightarrow \mathcal F_\XX$
sends an element $x\in \XX(I)$ to the corolla whose unique
internal vertex is labeled by $x$ (and whose
leaves are labeled by $I$). That is, show that for each
morphism of sequences $f: \XX\to \PP$ where $\PP$ is an operad,
there exists a unique morphisms of operads $F: \FF_\XX\longrightarrow
\PP$ such that $F\circ i  =f$.
\end{question}

% \begin{question}
% Follow the lecture notes and read about 
% weight gradings and the canonical weight
% grading on a free operad. 
% \end{question}